\ifx\master\undefined\input{settings/autocompile}\fi
\chapter{The Compact Muon Solenoid Experiment}
\label{ch:detector}
%

The Compact Muon Solenoid (CMS) Experiment is a ``general purpose'' particle
detector designed to measure collision events at the Large Hadron Collider
(LHC), a proton--proton synchrotron located at the CERN laboratory in Geneva,
Switzerland.  The design goals of the CMS experiment are~\cite{CMSExperiment},
in order of priority:
\begin{itemize}
  \item Good muon identification and momentum resolution over a wide range of
    momenta and angles, good dimuon mass resolution ($\approx 1\%$ at 100~\GeV),
    and the ability to determine unambiguously the charge of muons with $p <
    1~\TeV$;
  \item Good charged-particle momentum resolution and reconstruction efficiency
    in the inner tracker. Efficient triggering and offline tagging of $\tau$'s and
    $b$--jets, requiring pixel detectors close to the interaction region;
  \item Good electromagnetic energy resolution, good diphoton and dielectron
    mass resolution ($\approx 1\%$ at 100~\GeV), wide geometric coverage,
    $\pi^0$ rejection, and efficient photon and lepton isolation at high
    luminosities;
  \item Good missing--transverse--energy and dijet--mass resolution, requiring
    hadron calorimeters with a large hermetic geometric coverage and with fine
    lateral segmentation.
\end{itemize}
The detector uses a hermetic design that maximizes the
solid--angle of the fiducial region to capture as much information about the
collisions as possible.  The general geometry of the detector is cylindrical.
At cutaway diagram of the detector is shown in Figure~\ref{fig:AllCMSCutaways}.
Each of the sub--detector components consists of ``barrel'' and ``endcap''
components.  As its name suggests, the detector is centered around a four Tesla
superconducting solenoid magnet.  The individual sub--detectors of CMS are
arranged in a manner that permits identification of different species of
particles.  The central (closest to interaction point) sub--detector are the
charged particle tracking systems (the ``tracker'').  The tracker is designed to
be a \emph{non--destructive} instrument, which means that ideally that the
momentum of particles are unchanged after passing through it.  Outside of the
tracker is the electromagnetic and hadronic calorimeters, which are abbreviated
ECAL and HCAL, respectively.  The calorimeter is a \emph{destructive} detector,
and is designed such that visible incident particles are completely absorbed.
The outer layers of CMS are designed to measure muons, the
one\footnote{Neutrinos of course fulfill this requirement as well, but are so
weakly interacting that they are effectively invisible.} species of particle
that is immune to the effects of the calorimeter.  The arrangement of
destructive and non--destructive sub--detectors facilitates the identification
of different types of particles.  This concept is illustrated in
Figure~\ref{fig:SubdetectorID}.
\begin{figure}
  \centering
  %\subfigure[]{\includegraphics[width=0.97\textwidth]{detector_chapter/figures/cms_complete_labelled.pdf}
  \subfigure[]{\includegraphics[width=0.97\textwidth]{detector_chapter/figures/cms_complete_labelled.png}
  \label{fig:CMSDetector}
  }
  \subfigure[]{\includegraphics[width=0.97\textwidth]{detector_chapter/figures/CMS_Slice.png}
  \label{fig:SubdetectorID}
  }
  \caption[Schematic drawings of the CMS
  detector]{Figure~\subref{fig:CMSDetector}, top, shows a schematic drawing of the CMS
  detector.  The individual sub--detectors are labeled.  Two humans are shown in
  the foreground for scale. Figure~\subref{fig:SubdetectorID} shows a radial
  cross section of the detector and demonstrates how
  the (non--)destructiveness of different sub--detectors facilitates particle
  identification.} 
  \label{fig:AllCMSCutaways}
\end{figure}
In this chapter we give an brief overview of the LHC machine, and then describe
the individual sub--detector systems of CMS. 

\section{The Large Hadron Collider}
The Large Hadron Collider is a proton--proton synchrotron, with a design
collision energy of 14~\TeV.  At the time of this writing (and for the
foreseeable future), the LHC is the world's largest and highest energy particle
accelerator. A synchrotron is a machine that accelerates beams of charged
particles by using magnets to steer them in a circle through radio--frequency
resonating cavities which accelerate the particles. As the LHC is a collider,
there are two beams that are accelerated in opposite directions.  The maximum
beam energy of a synchrotron is determined by its radius and the maximum
strength of the magnetic fields used to bend the path of the beam.  The dipole
magnets used by the LHC to steer the particles are superconducting
niobium--titanium.  To maintain them in a superconducting state, they are cooled
using superfluid liquid helium to 1.9~Kelvin.  To store the beam at the
injection energy of 450~\GeV, the magnetic dipole fields must be maintained at
$1/2$~Tesla.  As the energy of each beam energy is increased to its (design)
maximum of 7~\TeV, the dipole fields are ramped to a maximum field of over 8
Tesla.
\section{Solenoid Magnet}
The four Tesla field of the CMS solenoid magnet is a critical factor in ability
of CMS to precisely measure collisions at the LHC\@. The momentum of charged
particles is measured in the detector by examining the curvature of the
particles path as it travels through the magnetic field.  The radius of
curvature $r$ of a charged particle in a magnetic field is given by
\begin{equation}
  r = \frac{p_\perp}{|q| B},
  \label{eq:LarmorRadius}
\end{equation}
where $q$ is the charge of the particle, $B$ is the magnetic field, and
$p_\perp$ is the component of the particle's relativistic momentum perpendicular
to the direction of the magnetic field.  From Equation~\ref{eq:LarmorRadius}, it
is evident that the ability to measure high momentum charged particles (a
critical goal of CMS) requires a high magnetic field.  Even at very high
particle energies where the resolution becomes poor, the strength of the
magnetic field is still very important for identifying the bending direction of
the particle; the direction corresponds to the particle's electric charge.
Furthermore, the homogeneity of the magnetic field is important to minimize
systematic errors in the measurement of tracks.

The CMS solenoid is extremely large.  The radial bore of the magnet is
6.3~meters; the magnet is 12.5~meters in length and weighs 220~tons.  The large
bore of the magnet allows the tracker and calorimeter systems to be located
inside the solenoid.  The internal windings of solenoid is arranged in four
layers to increase the total field strength.  The nominal current at full field
of the solenoid is 19.14~kA.  The solenoid itself is surrounded by an iron
return yoke with a total mass of 10,000~tons.  The return yoke surrounding the
solenoid minimizes the fringing field.  The muon detector system is interspersed
inside the yoke, and takes advantage of the field in the yoke to measure the
momentum and charge of muons.

\section{Charged Particle Tracking Systems}
\section{Electromagnetic Calorimeter}
\section{Hadronic Calorimeter}
\section{Muon System}
\section{Trigger System}
\section{Particle Flow Reconstruction Algorithm}
\ifx\master\undefined\input{settings/autocompile}\fi
