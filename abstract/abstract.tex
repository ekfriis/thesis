\ifx\master\undefined\input{settings/autocompile}\fi
\prelimheaders
\begin{abstract}
%
This thesis describes a search for the Higgs boson, a new particle predicted by
a theory called the Minimal Supersymmetric Model (MSSM).  The Standard Model of
particle physics, the MSSM, and Higgs phenomenology are introduced briefly. The
search presented in this thesis uses a single final state configuration, in
which the Higgs boson decays to two tau leptons, where one tau decays to a muon
and neutrinos,  and the other decays to pions and a single neutrino.  Two new
methods are introduced in this analysis, the Tau Neural Classifier tau
identification algorithm, and the Secondary Vertex fit tau pair mass
reconstruction method.  Both methods are discussed in detail.  The analysis uses
the 2010 dataset from the Compact Muon Solenoid (CMS) experiment, which contains
36~\pbinv of integrated luminosity at a center of mass energy of 7~\TeV.  In
total, 573 events are selected in the analysis; this value is compatible with
the Standard Model expectation.  No excess of signal events is observed, and we
set an upper limit on cross section times branching ratio of a Higgs boson.   We
finally interpret this limit in the parameter space of the MSSM.
\end{abstract}
%
\ifx\master\undefined\input{settings/autocompile}\fi
