This thesis describes a search for the Higgs boson, a new particle predicted by
a theory called the minimal supersymmetric extension to the standard model
(MSSM).  The standard model of particle physics, the MSSM, and Higgs boson
phenomenology are introduced briefly. The search presented in this thesis uses a
single final state configuration, in which the Higgs boson decays to two tau
leptons, with one tau decaying to a muon and neutrinos,  and the other decaying to
pions and a single neutrino.  Two new methods are introduced in this analysis,
the Tau Neural Classifier tau identification algorithm, and the Secondary Vertex
fit tau pair mass reconstruction method.  Both methods are discussed in detail.
The analysis uses the 2010 dataset from the Compact Muon Solenoid (CMS)
experiment, which contains 36~\pbinv of integrated luminosity at a center of
mass energy of 7~\TeV.  In total, 573 events are selected in the analysis.  We
fit the observed tau pair mass spectrum and measure the composition of the
events. The result is compatible with the standard model expectation.  No excess
of signal events is observed, and we set an upper limit on the cross section times
branching ratio of a Higgs boson.  This limit is interpreted in the parameter
space of the MSSM.
