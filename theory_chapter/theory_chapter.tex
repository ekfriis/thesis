\ifx\master\undefined\input{settings/autocompile}\fi

\chapter{The Standard Model and Beyond}
\label{ch:theory}

\section{The Standard Model}

The Standard Model (SM) is a ``theory of almost everything'' that describes the
interactions of elementary particles.  The Standard Model is a \emph{quantum
field theory}, first appearing in its modern form in the middle of the 20th
century.  The model is the synthesis of the independent theories of
electromagnetism, and the weak and strong nuclear forces.  Each of these
theories was used to describe different phenomena, which each have extremely
different strengths and act at different scales.  The interaction of light and
matter is described by Quantum Electrodynamics~(QED), a relativistic field
extension of the theory of electromagnetism.  The physics of radioactivity and
nuclear decay was described by the Fermi theory of weak interactions and the
forces that strong nuclear force binds the nuclei of atoms was described by
Yukawa.  An overview of these theories will be presented in this chapter.  

The feature that united the disparate theories into the Standard Model was the
application of the principle of \emph{local gauge invariance}. The principle of
gauge invariance first found success in QED, which predicted electromagnetic
phenomenon with astounding accuracy.  Local gauge invariance is now believed to
a fundamental feature of nature that underpins all theories of elementary
particles.  Furthermore, the development of the complete Standard Model as it
is known today was precipitated by  Glashow, Weinberg, and Salam's
discovery~\cite{ref:GlashowSymmetryBreaking} of the potential to unify QED with
the weak force using spontaneous symmetry breaking, and the development of the
Higgs mechanism~\cite{ref:HiggsMechanism}, which breaks the electroweak symmetry
using the clever addition of an additional scalar particle (the Higgs boson).

\subsection{Quantum Electrodynamics and Gauge Invariance}

The theory of QED is a modern extension of Maxwell's theory of
electromagnetism, describing the interaction of matter with light.  The
development of QED is a result of efforts to develop a quantum mechanical
formulation of electromagnetism compatible with the theory of Special Relativity.
QED is a \emph{gauge} theory, which means that the physical observables are
invariant under local gauge transformations.  Requiring local gauge
invariance gives rise to a ``gauge'' field, which can be interpreted as 
particles that are exchanged during an interaction.  

In the following, we first describe the Dirac equation for a free electron,
which is the relativistic extension of the Schroedinger equation for spin~$1/2$
particles.  We then show that requiring the corresponding Lagrangian of the free
charged particle to be invariant under local gauge transformations creates an
effective gauge boson field.  This ``gauge field'' creates terms in the
Lagrangian that represent interactions between the particles.

% Dirac equation comes from pg. 215 of Griffiths
The Dirac equation is the equation of motion of a free spin~$1/2$ particle of
mass~$m$ and is derived from the energy--momentum relationship of relativity
\begin{equation}
  p^{\mu}p_\mu - m^2c^2 = 0
  \label{eq:EnergyPRelationship}
\end{equation}.
Dirac sought to express this relationship in the framework of quantum mechanics
by applying the transformation
\begin{equation}
  p_\mu \to i \hbar \partial_\mu 
  \label{eq:QuantizeMomentumTransformation}
\end{equation}
to equation Equation~\ref{eq:EnergyPRelationship}, but with the requirement that
the resulting equation be first order in time.\footnote{A detailed
discussion of this topic is available in~\cite{Griffiths:IntroParticle}.}
To achieve this, Dirac factorized Equation~\ref{eq:EnergyPRelationship} into 
\begin{equation}
  (\gamma^\kappa p_\kappa + mc)(\gamma^\mu p_\mu - mc) = 0,
  \label{eq:DiracEquation}
\end{equation}
where $\gamma^\mu$ is a set of four $4\times4$ matrices referred to as the Dirac
matrices.  The equation of motion is obtained by choosing either term (they are
equivalent) from the
left hand side of Equation~\ref{eq:DiracEquation} and
making the substitution in Equation~\ref{eq:QuantizeMomentumTransformation}.
\begin{equation}
  i \hbar \gamma^\mu \partial_\mu \psi - mc \psi = 0
  \label{eq:DiracEquation}
\end{equation}
The solutions, $\psi$, to the Dirac equation are called ``Dirac Spinors,'' and
represent the quantum mechanical state of spin~$1/2$ particles.

The Lagrangian corresponding to the Dirac equation~(\ref{eq:DiracEquation}) is
\begin{equation}
  \mathcal{L} = \overline \psi (i\hbar c\gamma^\mu \partial_\mu - mc^2) \psi,
  \label{eq:FreeQEDLagrangian}
\end{equation}
where $\psi$ is the spinor field of the particle in question, $\hbar$ is Planck's
constant, $c$ the speed of light, and $\gamma^\mu$ are the Dirac matrices.  As
$\overline\psi$ is the Hermitian conjugate of $\psi$, the Lagrangian is invariant
under the global gauge transformation 
\begin{equation}
  \psi' \to e^{i\theta}\psi
  \label{eq:U1GaugeTransform}
\end{equation}.
The Lagrangian is invariant under \emph{local} gauge translations if $\theta$
can be defined differently at each point in space, \ie if $\theta = \theta(x)$
in equation~\ref{eq:U1GaugeTransform}.  However, as the derivative operator
$\partial_\mu$ in equation~\ref{eq:FreeQEDLagrangian} does not commute with
$\theta(x)$, the Lagrangian must be modified to satisfy local gauge invariance.
This modification is accomplished with the use of a ``gauge covariant
derivative.''  By making the replacement 
\begin{equation} 
  \partial_\mu \to D_\mu = \partial_\mu - \frac{ie}{\hbar}A^\mu      
\end{equation}
in equation~\ref{eq:FreeQEDLagrangian}, where \FIX{don't think this is right}
$A^\mu = \partial^\mu \theta(x)$ and $e$ is the electric charge, the Lagrangian
becomes locally gauge invariant:
\begin{equation}
  \mathcal{L} = \overline \psi (i\hbar c\gamma^\mu D_\mu - mc^2) \psi 
  \label{eq:LocalQEDLagrangian}
\end{equation}
The difference between
the locally (\ref{eq:LocalQEDLagrangian}) and the globally 
(\ref{eq:FreeQEDLagrangian}) gauge invariant Lagrangians is then
\begin{equation}
  \mathcal{L}_{int} = \frac{e}{\hbar}\overline\psi\gamma^\mu\psi A_\mu 
\end{equation}.
This term can be interpreted as the coupling between the particle and the gauge
boson (force carrier) fields.  The coupling is proportional to the constant $e$,
which is associated with the electric charge.  This is consistent with the
experimental observation that particles with zero electric charge do not
interact electromagnetically with each other.  In this interpretation, the
electromagnetic force between two charged particles is caused by the exchange of
gauge bosons (photons).  The existence of this ``minimal coupling'' is
\emph{required} if the Lagrangian is to satisfy local gauge invariance. The
addition of a term with the gauge Field Strength Tensor to represent the kinetic
term of the gauge (photon) field yields the QED Lagrangian:
\begin{equation}
  \mathcal{L}_{QED} = \overline \psi (i\hbar c\gamma^\mu D_\mu - mc^2) \psi -
  \frac{1}{4\mu_0}F_{\mu\nu}F^{\mu\nu}
\end{equation}

The gauge symmetry group of QED is $U(1)$, the unitary group of degree 1.  This
symmetry can be visualized as a rotation of a two--dimensional unit vector. (The
application of the gauge transformation $e^{i\theta}$ rotates a number in the
complex plane.)  In gauge theory that the symmetry group
defines the behavior of the gauge bosons and thus the interactions of the
theory.  

\subsection{The Weak Interactions}

The theory of Weak Interactions was created to describe the physics of
radioactive decay.  The first formulation of the theory was done by
Fermi~\cite{ref:FermiWeakInteration} to explain the phenomenon of the $\beta$
decay of neutrons. The initial theory was a four-fermion ``contact'' theory.  In
a contact theory, all four fermions come involved in the $\beta$--decay are
connected at a single vertex.  The Fermi theory Hamiltonian for the
$\beta$--decay of a proton is then~\cite{Morii:SMandBSM}
\begin{equation}
  H = \frac{G_\beta}{\sqrt{2}}
  \left[\overline \psi_p \gamma_\mu (1 - g_A\gamma_5)\psi_n\right]
  \left[[\overline \psi_e \gamma^\mu (1 - \gamma_5) \psi_\nu\right] + h.c., 
 \label{eq:FermiTheoryHamiltonian}
\end{equation}
where $G_\beta$ is the Fermi constant and $g_A$ is the relative fraction of the
interaction with axially Lorentz structure.  The value of $g_A$ was determined
experimentally to be $1.26$.  One of the most notable things discovered about
the weak force is that weak interactions violate parity; that is, the physics of
the interaction change (or become disallowed) under inversion of the spatial
coordinates.  This is evidenced by the $(1-\gamma_5)$ term in
Equation~\ref{eq:FermiTheoryHamiltonian}.  This term is the ``helicity
operator''; the left and right ``handed'' helicity states are eigenstates states
of this term.
\begin{eqnarray}
  h = (1 - \gamma_5)/2 \nonumber \\
  h\psi_R = \frac{1}{2}\psi_R \nonumber \\ 
  h\psi_L = -\frac{1}{2}\psi_L \nonumber
\end{eqnarray}
It is observed that only left--handed neutrinos (or right--handed
anti--neutrinos) \fixme{check handedness is correct} participate in the weak
interaction.

The Fermi interaction can describe both nuclear $\beta$ decay ($p \to n + e^+ +
\overline{\nu_e}$) as well as the decay of a muon into an electron ($\mu \to
\nu_\mu + e + \overline{\nu_e}$).  However, the contact interaction form of
Fermi's theory is not complete.  When applied to scattering processes, the
interaction violates unitarity: the calculated cross section grows with the
center of mass energy, so that for some energy the probability for an
interaction is greater than one.  Furthermore, the techniques successfully used
to ``renormalize''\footnote{Renormalization of quantum field theories is a broad
topic beyond the scope of this thesis.  Briefly, the process involves
``absorbing'' infinite divergences that occur in higher--order interactions into
physical observables~\cite{Griffiths:IntroParticle}.} QED fail when applied to
the Fermi interaction.

The first attempt to solve the problems with the Fermi theory was made by
introducing an intermediate weak boson.  The contact interaction is replaced by
a massive propagator, the $W^\pm$ bosons.  The decay of a muon to an electron
and two neutrinos then proceeds as pictured in
Figure~\ref{fig:MuonDecayFeynmanDiagram} with an amplitude given
by~\cite{Morii:SMandBSM}
\begin{equation}
  M = - \left[\frac{g}{\sqrt{2}} \overline u_{\nu_\mu} \gamma_\rho \frac{1 -
  \gamma_5}{2} u_\mu\right] \frac{-g^{\rho\sigma} + \frac{q^\rho
  q^\sigma}{M_W^2}}{q^2 - M_W^2} 
  \left[\frac{g}{\sqrt{2}} \overline u_{\nu_e} \gamma_\rho \frac{1 -
  \gamma_5}{2} u_e\right]
  \label{eq:WeakPropagator}
\end{equation}
The presence of the large mass of gauge boson in the denominator is the reason
why the contact interaction original formulated by Fermi is effectively
describes of low--energy weak phenomenon.  When the momentum transfer $q$ is
small compared to $M_W$, the effect of the propagator is an effective constant.
In the low energy limit, the full propagator in equation~\ref{eq:WeakPropagator}
is equivalent to the Fermi contact interaction
in~\ref{eq:FermiContactInteraction}.

The ``charge'' of the theory of interaction is called weak isospin.  Similar to
the electric charge of QED, only particles with non--zero weak isospin couple to
the gauge bosons of the weak force, and weak isospin is conserved in weak
interactions.   The weak force is \emph{chiral}; the ``handedness'' of the
particles affects how they interact with each other.  In fact, the weak
interaction violates parity: the interaction is not invariant under spatial
inversion.  Left--handed fermions carry weak isospin $T = 1/2$, and form a
doublet under the $SU(2)$ gauge group.  The right--handed fermions have no weak
isospin, and are singlets under the gauge group.  

\subsection{Quantum Chromodynamics}

\subsection{Electroweak Unification}
In the early 1960s Glashow published a series of papers describing how the
electromagnetic and weak forces could be unified into a common ``electroweak''
force.  The fact that at low energy the electromagnetic and weak forces appear
to be separate phenomena is due to the fact that the symmetry of the electroweak
gauge group is spontaneously broken.

\subsection{The Higgs mechanism}


\section{Searches for the Higgs boson}
\subsection{Higgs boson phenomenology}
\label{sec:SMHiggsPhenom}
\subsection{Results from LEP and Tevatron}

\section{Beyond the Standard Model}
\label{sec:BSM}
\subsection{The Hierarchy Problem}
\subsection{Supersymmetry}
\subsection{The Minimal Supersymmetric Model and $\tau$ leptons}
\label{sec:MSSMAndTaus}

\section{The physics of the \taul}
As discussed in sections~\ref{sec:SMHiggsPhenom} and~\ref{sec:MSSMAndTaus}, the
$\tau$ lepton is an important probe of Higgs physics.  The \taul has some
unusual properties which make it particularly challenging at hadron colliders.
With a mass of 1.78~\GeVcc, the \taul is heaviest of the leptons.  The nominal
decay distance $c\tau$ of the \taul is 87~\micron, which in practice means that
the $\tau$ will always decay before reaching the first layer of the detector.
Tau decays can be effectively classified into two types. ``Leptonic'' decays
consist of a $\tau$ decaying to a light lepton ($\ell = e, \mu$) and two
neutrinos $\tau^+ \to \ell^+ \nu_\tau \overline{ \nu_{\ell}}$.  ``Hadronic'' decays 
consist of a low--multiplicity collimated group of hadrons, typically \Pgppm and \Pgpz
mesons.  The hadronic decays of the \taul compose approximately $65\%$ of the
\taul branching fraction, with the remainder shared approximately equally by the
leptonic decays.  The branching fractions for the leptonic and most common hadronic decays
are shown in table~\ref{tab:TauDecayModes}.
\begin{table}
   \centering
   \begin{tabular}{lcrr}
      Visible Decay Products  & Resonance & Mass (\MeVcc) &
      Fraction~\cite{PDG} \\
      \hline
      \hline
      \multicolumn{4}{c}{Leptonic modes modes} \\
      \hline
      $e^- \nu_\tau \overline \nu_e$             & -      & 0.5  & 17.8\% \\
      $\mu^-\nu_\tau \overline \nu_\mu$          & -      & 105  & 17.4\% \\
      \hline
      \multicolumn{4}{c}{Hadronic modes} \\
      \hline
      $\pi^{-} \nu_\tau$                    & -      & 135  & 10.9\% \\
      $\pi^{-}\pi^0 \nu_\tau$               & $\rho$ & 770  & 25.5\% \\
      $\pi^{-}\pi^0\pi^0 \nu_\tau$          & $a1$   & 1200 & 9.3\% \\
      $\pi^{-}\pi^{-}\pi^{+} \nu_\tau$      & $a1$   & 1200 & 9.0\% \\
      $\pi^{-}\pi^{-}\pi^{+}\pi^0 \nu_\tau$ & $a1$   & 1200 & 4.5\% \\
      \hline
      Total                                 &        &      & 94.4\% \\
      \hline
   \end{tabular}
   \label{tab:decay_modes}
   \caption{Resonances and branching ratios of the dominant decay modes of
   the \taul.  The decay products listed correspond to a negatively
   charged \taul; the table is identical under charge conjugation.}
\end{table}

\ifx\master\undefined\input{settings/autocompile}\fi
