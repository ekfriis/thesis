\ifx\master\undefined\input{settings/autocompile}\fi

\chapter{The Standard Model and Beyond}
\label{ch:theory}

\section{The Standard Model}

The Standard Model (SM) is a ``theory of almost everything'' that describes the
interactions of elementary particles.  The theory is the synthesis of quantum
electrodynamics, the theory of weak nuclear interactions, and the strong nuclear 
force.  There are three types of elementary particles in the SM: quarks,
leptons, and gauge bosons.  

Prior to the development of the SM, three different theories theories of
particle physics where used to describe different phenomena, which are
discussed in turn.   The interaction of light and matter, is described by
Quantum Electrodynamics~(QED), a relativistic field extension of the theory of
electromagnetism.  The physics of radioactivity and nuclear decay was described
by the theory of weak interactions.  Finally, the forces that bind together the
nuclei of atoms was described by the strong nuclear force. 

The development of the complete Standard Model as it is known today was
precipitated by two theoretical developments:  Glashow's
discovery~\cite{ref:GlashowSymmetryBreaking} of the potential to unify QED with
the weak force using spontaneous symmetry breaking, and the development of the
Higgs mechanism~\cite{ref:HiggsMechanism}, which breaks the electroweak symmetry
using the clever addition of an additional scalar particle (the Higgs boson).

\subsection{Quantum Electrodynamics}

The theory of QED is a relativistic formulation of Maxwell's theory of
electromagnetism, describing the interaction of matter with light.  The
development of QED is a result of efforts to develop a quantum mechanical
formulation of electromagnetism compatible with the theory of Special Relativity.
QED is a \emph{gauge} theory, which means that the physical observables are
invariant under local gauge transformations.  Requiring local gauge
invariance gives rise to a ``gauge'' field, which can be interpreted as 
particles that are exchanged during an interaction.  

In the following, we can show that requiring the relativistic Lagrangian of a
free charged particle to be invariant under local gauge transformations creates
an effective gauge boson field.  This ``gauge field'' creates terms in the
Lagrangian that represent interactions between the particles.

The Dirac equation \FIX{what is this} of a free spin~$1/2$ particle of mass~$m$
is
%http://en.wikipedia.org/wiki/Gauge_field#An_example:_Electrodynamics
\begin{equation}
  \mathcal{L} = \bar \psi (i\hbar c\gamma^\mu \partial_\mu - mc^2) \psi 
  \label{eq:FreeQEDLagrangian}
\end{equation}
where $\psi$ is the field of the particle in question, $\hbar$ is Planck's
constant, $c$ the speed of light, and $\gamma^\mu$ are the Dirac matrices.  As
$\bar\psi$ is the Hermitian conjugate of $\psi$, the Lagrangian is invariant
under the global gauge transformation 
\begin{equation}
  \psi' \to e^{i\theta}\psi
  \label{eq:U1GaugeTransform}
\end{equation}.
The Lagrangian is invariant under \emph{local} gauge translations if $\theta$
can be defined differently at each point in space, \ie if $\theta = \theta(x)$
in equation~\ref{eq:U1GaugeTransform}.  However, as the derivative operator
$\partial_\mu$ in equation~\ref{eq:FreeQEDLagrangian} does not commute with
$\theta(x)$, the Lagrangian must be modified to satisfy local gauge invariance.
This modification is accomplished with the use of a ``gauge covariant
derivative.''  By making the replacement 
\begin{equation} 
  \partial_\mu \to D_\mu = \partial_\mu - \frac{ie}{\hbar}A^\mu      
\end{equation}
in equation~\ref{eq:FreeQEDLagrangian}, where \FIX{don't think this is right}
$A^\mu = \partial^\mu \theta(x)$ and $e$ is the electric charge, the Lagrangian
becomes locally gauge invariant:
\begin{equation}
  \mathcal{L} = \bar \psi (i\hbar c\gamma^\mu D_\mu - mc^2) \psi 
  \label{eq:LocalQEDLagrangian}
\end{equation}
The difference between
the locally (equation~\ref{eq:LocalQEDLagrangian}) and globally 
(equation~\ref{eq:FreeQEDLagrangian}) gauge invariant Lagrangians is then
\begin{equation}
  \mathcal{L}_{int} = \frac{e}{\hbar}\bar\psi\gamma^\mu\psi A_\mu 
\end{equation}.
This term can be interpreted as the coupling between the particle and the gauge
boson (force carrier) fields.  The coupling is proportional to the constant
$e$, which is associated with the electric charge.  This is consistent with the
experimental observation that particles with zero electric charge do not
interact electromagnetically with each other.  In this interpretation, the
electromagnetic force between two charged particles is caused by the exchange
of gauge bosons (photons).  The existence of this coupling is required if the
Lagrangian is to satisfy local gauge invariance. The addition of a term with
the gauge Field Strength Tensor yields the QED Lagrangian:
\begin{equation}
  \mathcal{L}_{QED} = \bar \psi (i\hbar c\gamma^\mu D_\mu - mc^2) \psi -
  \frac{1}{4\mu_0}F_{\mu\nu}F^{\mu\nu}
\end{equation}

\subsection{The Weak Interactions}

The theory of Weak Interactions was created to describe the physics of
radioactivity.  The first formulation of the theory was done by
Fermi~\cite{ref:FermiWeakInteration} to explain the phenomenon of the $beta$ decay
of radioactive elements.  The initial theory was a four-fermion ``contact''
theory.  In a contact theory, all four fermions come from a single vertex.
The Lagrangian for the Fermi interactions is:
\begin{equation}
 find me 
\end{equation}

However, the contact interaction form of Fermi's theory is not complete.  At 
higher energies the theory suffers from \FIX{what?}.  The theory can 
be completed using the gauge theory formalism.  In this case, the 


\subsection{Quantum Chromodynamics}

\subsection{Electroweak Unification}

\subsection{The Higgs mechanism}

\section{Searches for the Higgs boson}
\subsection{Higgs boson phenomenology}
\subsection{Searches at LEP}
\subsection{Searches at the Tevatron}
\subsection{Searches at the Tevatron}

\section{Beyond the Standard Model}
\subsection{The Hierarchy Problem}
\subsection{Supersymmetry}
\subsection{The Minimal Supersymmetric Model and $\tau$ leptons}

\section{$\tau$ leptons as a probe of new physics}

\ifx\master\undefined\input{settings/autocompile}\fi
