\ifx\master\undefined% AUTOCOMPILE
% Allows for individual chapters to be compiled.

% Usage:
% \ifx\master\undefined% AUTOCOMPILE
% Allows for individual chapters to be compiled.

% Usage:
% \ifx\master\undefined% AUTOCOMPILE
% Allows for individual chapters to be compiled.

% Usage:
% \ifx\master\undefined\input{../settings/autocompile}\fi
% (place at start and end of chapter file)

\ifx\noprelim\undefined
    % first time included
    % input preamble files
    \input{../settings/phdsetup}

    \begin{document}
    \def\noprelim{}
\else
    % already included once
    % input post files

    \singlespacing
    \bibliographystyle{../bibliography/expanded}
    \bibliography{../bibliography/references}

    \end{document}
\fi\fi
% (place at start and end of chapter file)

\ifx\noprelim\undefined
    % first time included
    % input preamble files
    % [ USER VARIABLES ]

\def\PHDTITLE {Extensions of the Theory of Computational Mechanics}
\def\PHDAUTHOR{Evan Klose Friis}
\def\PHDSCHOOL{University of California, Davis}

\def\PHDMONTH {June}
\def\PHDYEAR  {2011}
\def\PHDDEPT {Physics}

\def\BSSCHOOL {University of California at San Diego}
\def\BSYEAR   {2005}

\def\PHDCOMMITTEEA{Professor John Conway}
\def\PHDCOMMITTEEB{Professor Robin Erbacher}
\def\PHDCOMMITTEEC{Professor Mani Tripathi}

% [ GLOBAL SETUP ]

\documentclass[letterpaper,oneside,11pt]{report}

\usepackage{calc}
\usepackage{breakcites}
\usepackage[newcommands]{ragged2e}

\usepackage[pdftex]{graphicx}
\usepackage{epstopdf}

%\usepackage{tikz}
%\usetikzlibrary{positioning} % [right=of ...]
%\usetikzlibrary{fit} % [fit= ...]

%\pgfdeclarelayer{background layer}
%\pgfdeclarelayer{foreground layer}
%\pgfsetlayers{background layer,main,foreground layer}

%\newenvironment{wrap}{\noindent\begin{minipage}[t]{\linewidth}\vspace{-0.5\normalbaselineskip}\centering}{\vspace{0.5\normalbaselineskip}\end{minipage}}

%% [Venn diagram environment]
%\newenvironment{venn2}
%{\begin{tikzpicture} [every pin/.style={text=black, text opacity=1.0, pin distance=0.5cm, pin edge={black!60, semithick}},
%% define a new style 'venn'
%venn/.style={circle, draw=black!60, semithick, minimum size = 4cm}]
%
%% create circle and give it external (pin) label
%\node[venn] (X) at (-1,0) [pin={150:$H[X]$}] {};
%\node[venn] (Y) at (1,0) [pin={30:$H[Y]$}] {};
%
%% place labels of the atoms by hand
%\node at (-1.9,0) {$H[X|Y]$};
%\node at (1.9,0) {$H[Y|X]$};
%\node at (0,0) {$I[X;Y]$};}
%{\end{tikzpicture}}

%\newcommand{\wrapmath}[1]{\begin{wrap}\begin{tikzpicture}[every node/.style={inner ysep=0ex, inner xsep=0em}]\node[] {$\displaystyle\begin{aligned} #1\end{aligned}$};\end{tikzpicture}\end{wrap}}

\renewenvironment{abstract}{\chapter*{Abstract}}{}
\renewcommand{\bibname}{Bibliography}
\renewcommand{\contentsname}{Table of Contents}

\makeatletter
\renewcommand{\@biblabel}[1]{\textsc{#1}}
\makeatother

% [ FONT SETTINGS ]

\usepackage[tbtags, intlimits, namelimits]{amsmath}
\usepackage[adobe-utopia]{mathdesign}

\DeclareSymbolFont{pazomath}{OMS}{zplm}{m}{n}
\DeclareSymbolFontAlphabet{\mathcal}{pazomath}
\SetMathAlphabet\mathcal{bold}{OMS}{zplm}{b}{n}

\SetSymbolFont{largesymbols}{normal}{OMX}{zplm}{m}{n}
\SetSymbolFont{largesymbols}{bold}{OMX}{zplm}{m}{n}
\SetSymbolFont{symbols}{normal}{OMS}{zplm}{m}{n}
\SetSymbolFont{symbols}{bold}{OMS}{zplm}{b}{n}

\renewcommand{\sfdefault}{phv}
\renewcommand{\ttdefault}{fvm}

\widowpenalty 8000
\clubpenalty  8000

% [ PAGE LAYOUT ]
\usepackage{geometry}
\geometry{lmargin = 1.5in}
\geometry{rmargin = 1.0in}
\geometry{tmargin = 1.0in}
\geometry{bmargin = 1.0in}

% [ PDF SETTINGS ]

\usepackage[final]{hyperref}
\hypersetup{breaklinks  = true}
\hypersetup{colorlinks  = true}
\hypersetup{linktocpage = false}
\hypersetup{linkcolor   = blue}
\hypersetup{citecolor   = green}
\hypersetup{urlcolor    = black}
\hypersetup{plainpages  = false}
\hypersetup{pageanchor  = true}
\hypersetup{pdfauthor   = {\PHDAUTHOR}}
\hypersetup{pdftitle    = {\PHDTITLE}}
\hypersetup{pdfsubject  = {Dissertation, \PHDSCHOOL}}
\urlstyle{same}

% [ LETTER SPACING ]

\usepackage[final]{microtype}
\microtypesetup{protrusion=compatibility}
\microtypesetup{expansion=false}

\newcommand{\upper}[1]{\MakeUppercase{#1}}
\let\lsscshape\scshape

\ifcase\pdfoutput\else\microtypesetup{letterspace=15}
\renewcommand{\scshape}{\lsscshape\lsstyle}
\renewcommand{\upper}[1]{\textls[50]{\MakeUppercase{#1}}}\fi

% [ LINE SPACING ]

\usepackage[doublespacing]{setspace}
\renewcommand{\displayskipstretch}{0.0}

\setlength{\parskip   }{0em}
\setlength{\parindent }{2em}

% [ TABLE FORMATTING ]

\usepackage{booktabs}
\setlength{\heavyrulewidth}{1.5\arrayrulewidth}
\setlength{\lightrulewidth}{1.0\arrayrulewidth}
\setlength{\doublerulesep }{2.0\arrayrulewidth}

% [ SECTION FORMATTING ]

\usepackage[largestsep,nobottomtitles*]{titlesec}
\renewcommand{\bottomtitlespace}{0.75in}

\titleformat{\chapter}[display]{\bfseries\huge\singlespacing}{\filleft\textsc{\LARGE \chaptertitlename\ \thechapter}}{-0.2ex}{\titlerule[3pt]\vspace{0.2ex}}[]

\titleformat{\section}{\LARGE}{\S\thesection\hspace{0.5em}}{0ex}{}
\titleformat{\subsection}{\Large}{\S\thesubsection\hspace{0.5em}}{0ex}{}
\titleformat{\subsubsection}{\large}{\thesubsubsection\hspace{0.5em}}{0ex}{}

\titlespacing*{\chapter}{0em}{6ex}{4ex plus 2ex minus 0ex}
\titlespacing*{\section}{0em}{2ex plus 3ex minus 1ex}{0.5ex plus 0.5ex minus 0.5ex}
\titlespacing*{\subsection}{0ex}{2ex plus 3ex minus 1ex}{0ex}
\titlespacing*{\subsubsection}{0ex}{2ex plus 0ex minus 1ex}{0ex}

% [ HEADER SETTINGS ]

\usepackage{fancyhdr}

\setlength{\headheight}{\normalbaselineskip}
\setlength{\footskip  }{0.5in}
\setlength{\headsep   }{0.5in-\headheight}

\fancyheadoffset[R]{0.5in}
\renewcommand{\headrulewidth}{0pt}
\renewcommand{\footrulewidth}{0pt}

\newcommand{\pagebox}{\parbox[r][\headheight][t]{0.5in}{\hspace\fill\thepage}}

\newcommand{\prelimheaders}{\ifx\prelim\undefined\renewcommand{\thepage}{\textit{\roman{page}}}\fancypagestyle{plain}{\fancyhf{}\fancyfoot[L]{\makebox[\textwidth-0.5in]{\thepage}}}\pagestyle{plain}\def\prelim{}\fi}

\newcommand{\normalheaders}{\renewcommand{\thepage}{\arabic{page}}\fancypagestyle{plain}{\fancyhf{}\fancyhead[R]{\pagebox}}\pagestyle{plain}}

\normalheaders{}

% [ CUSTOM COMMANDS ]

\newcommand{\signaturebox}[1]{\multicolumn{1}{p{4in}}{\vspace{3ex}}\\\midrule #1\\}

%\input{../includes/cmechabbrev}

% [some math stuff - maybe stick in sep file]
\usepackage{amsthm}
\usepackage{amscd}
\theoremstyle{plain}    \newtheorem{Lem}{Lemma}
\theoremstyle{plain}    \newtheorem*{ProLem}{Proof}
\theoremstyle{plain} 	\newtheorem{Cor}{Corollary}
\theoremstyle{plain} 	\newtheorem*{ProCor}{Proof}
\theoremstyle{plain} 	\newtheorem{The}{Theorem}
\theoremstyle{plain} 	\newtheorem*{ProThe}{Proof}
\theoremstyle{plain} 	\newtheorem{Prop}{Proposition}
\theoremstyle{plain} 	\newtheorem*{ProProp}{Proof}
\theoremstyle{plain} 	\newtheorem*{Conj}{Conjecture}
\theoremstyle{plain}	\newtheorem*{Rem}{Remark}
\theoremstyle{plain}	\newtheorem*{Def}{Definition} 
\theoremstyle{plain}	\newtheorem*{Not}{Notation}

% [uniform figure scaling - maybe this is not a good idea]
\def\figscale{.7}
\def\lscale{1.0}

% [FIX ME! - red makes it easier to spot]
\newcommand{\FIX}[1]{\textbf{\textcolor{red}{#1}}}


    \begin{document}
    \def\noprelim{}
\else
    % already included once
    % input post files

    \singlespacing
    \bibliographystyle{../bibliography/expanded}
    \bibliography{../bibliography/references}

    \end{document}
\fi\fi
% (place at start and end of chapter file)

\ifx\noprelim\undefined
    % first time included
    % input preamble files
    % [ USER VARIABLES ]

\def\PHDTITLE {Extensions of the Theory of Computational Mechanics}
\def\PHDAUTHOR{Evan Klose Friis}
\def\PHDSCHOOL{University of California, Davis}

\def\PHDMONTH {June}
\def\PHDYEAR  {2011}
\def\PHDDEPT {Physics}

\def\BSSCHOOL {University of California at San Diego}
\def\BSYEAR   {2005}

\def\PHDCOMMITTEEA{Professor John Conway}
\def\PHDCOMMITTEEB{Professor Robin Erbacher}
\def\PHDCOMMITTEEC{Professor Mani Tripathi}

% [ GLOBAL SETUP ]

\documentclass[letterpaper,oneside,11pt]{report}

\usepackage{calc}
\usepackage{breakcites}
\usepackage[newcommands]{ragged2e}

\usepackage[pdftex]{graphicx}
\usepackage{epstopdf}

%\usepackage{tikz}
%\usetikzlibrary{positioning} % [right=of ...]
%\usetikzlibrary{fit} % [fit= ...]

%\pgfdeclarelayer{background layer}
%\pgfdeclarelayer{foreground layer}
%\pgfsetlayers{background layer,main,foreground layer}

%\newenvironment{wrap}{\noindent\begin{minipage}[t]{\linewidth}\vspace{-0.5\normalbaselineskip}\centering}{\vspace{0.5\normalbaselineskip}\end{minipage}}

%% [Venn diagram environment]
%\newenvironment{venn2}
%{\begin{tikzpicture} [every pin/.style={text=black, text opacity=1.0, pin distance=0.5cm, pin edge={black!60, semithick}},
%% define a new style 'venn'
%venn/.style={circle, draw=black!60, semithick, minimum size = 4cm}]
%
%% create circle and give it external (pin) label
%\node[venn] (X) at (-1,0) [pin={150:$H[X]$}] {};
%\node[venn] (Y) at (1,0) [pin={30:$H[Y]$}] {};
%
%% place labels of the atoms by hand
%\node at (-1.9,0) {$H[X|Y]$};
%\node at (1.9,0) {$H[Y|X]$};
%\node at (0,0) {$I[X;Y]$};}
%{\end{tikzpicture}}

%\newcommand{\wrapmath}[1]{\begin{wrap}\begin{tikzpicture}[every node/.style={inner ysep=0ex, inner xsep=0em}]\node[] {$\displaystyle\begin{aligned} #1\end{aligned}$};\end{tikzpicture}\end{wrap}}

\renewenvironment{abstract}{\chapter*{Abstract}}{}
\renewcommand{\bibname}{Bibliography}
\renewcommand{\contentsname}{Table of Contents}

\makeatletter
\renewcommand{\@biblabel}[1]{\textsc{#1}}
\makeatother

% [ FONT SETTINGS ]

\usepackage[tbtags, intlimits, namelimits]{amsmath}
\usepackage[adobe-utopia]{mathdesign}

\DeclareSymbolFont{pazomath}{OMS}{zplm}{m}{n}
\DeclareSymbolFontAlphabet{\mathcal}{pazomath}
\SetMathAlphabet\mathcal{bold}{OMS}{zplm}{b}{n}

\SetSymbolFont{largesymbols}{normal}{OMX}{zplm}{m}{n}
\SetSymbolFont{largesymbols}{bold}{OMX}{zplm}{m}{n}
\SetSymbolFont{symbols}{normal}{OMS}{zplm}{m}{n}
\SetSymbolFont{symbols}{bold}{OMS}{zplm}{b}{n}

\renewcommand{\sfdefault}{phv}
\renewcommand{\ttdefault}{fvm}

\widowpenalty 8000
\clubpenalty  8000

% [ PAGE LAYOUT ]
\usepackage{geometry}
\geometry{lmargin = 1.5in}
\geometry{rmargin = 1.0in}
\geometry{tmargin = 1.0in}
\geometry{bmargin = 1.0in}

% [ PDF SETTINGS ]

\usepackage[final]{hyperref}
\hypersetup{breaklinks  = true}
\hypersetup{colorlinks  = true}
\hypersetup{linktocpage = false}
\hypersetup{linkcolor   = blue}
\hypersetup{citecolor   = green}
\hypersetup{urlcolor    = black}
\hypersetup{plainpages  = false}
\hypersetup{pageanchor  = true}
\hypersetup{pdfauthor   = {\PHDAUTHOR}}
\hypersetup{pdftitle    = {\PHDTITLE}}
\hypersetup{pdfsubject  = {Dissertation, \PHDSCHOOL}}
\urlstyle{same}

% [ LETTER SPACING ]

\usepackage[final]{microtype}
\microtypesetup{protrusion=compatibility}
\microtypesetup{expansion=false}

\newcommand{\upper}[1]{\MakeUppercase{#1}}
\let\lsscshape\scshape

\ifcase\pdfoutput\else\microtypesetup{letterspace=15}
\renewcommand{\scshape}{\lsscshape\lsstyle}
\renewcommand{\upper}[1]{\textls[50]{\MakeUppercase{#1}}}\fi

% [ LINE SPACING ]

\usepackage[doublespacing]{setspace}
\renewcommand{\displayskipstretch}{0.0}

\setlength{\parskip   }{0em}
\setlength{\parindent }{2em}

% [ TABLE FORMATTING ]

\usepackage{booktabs}
\setlength{\heavyrulewidth}{1.5\arrayrulewidth}
\setlength{\lightrulewidth}{1.0\arrayrulewidth}
\setlength{\doublerulesep }{2.0\arrayrulewidth}

% [ SECTION FORMATTING ]

\usepackage[largestsep,nobottomtitles*]{titlesec}
\renewcommand{\bottomtitlespace}{0.75in}

\titleformat{\chapter}[display]{\bfseries\huge\singlespacing}{\filleft\textsc{\LARGE \chaptertitlename\ \thechapter}}{-0.2ex}{\titlerule[3pt]\vspace{0.2ex}}[]

\titleformat{\section}{\LARGE}{\S\thesection\hspace{0.5em}}{0ex}{}
\titleformat{\subsection}{\Large}{\S\thesubsection\hspace{0.5em}}{0ex}{}
\titleformat{\subsubsection}{\large}{\thesubsubsection\hspace{0.5em}}{0ex}{}

\titlespacing*{\chapter}{0em}{6ex}{4ex plus 2ex minus 0ex}
\titlespacing*{\section}{0em}{2ex plus 3ex minus 1ex}{0.5ex plus 0.5ex minus 0.5ex}
\titlespacing*{\subsection}{0ex}{2ex plus 3ex minus 1ex}{0ex}
\titlespacing*{\subsubsection}{0ex}{2ex plus 0ex minus 1ex}{0ex}

% [ HEADER SETTINGS ]

\usepackage{fancyhdr}

\setlength{\headheight}{\normalbaselineskip}
\setlength{\footskip  }{0.5in}
\setlength{\headsep   }{0.5in-\headheight}

\fancyheadoffset[R]{0.5in}
\renewcommand{\headrulewidth}{0pt}
\renewcommand{\footrulewidth}{0pt}

\newcommand{\pagebox}{\parbox[r][\headheight][t]{0.5in}{\hspace\fill\thepage}}

\newcommand{\prelimheaders}{\ifx\prelim\undefined\renewcommand{\thepage}{\textit{\roman{page}}}\fancypagestyle{plain}{\fancyhf{}\fancyfoot[L]{\makebox[\textwidth-0.5in]{\thepage}}}\pagestyle{plain}\def\prelim{}\fi}

\newcommand{\normalheaders}{\renewcommand{\thepage}{\arabic{page}}\fancypagestyle{plain}{\fancyhf{}\fancyhead[R]{\pagebox}}\pagestyle{plain}}

\normalheaders{}

% [ CUSTOM COMMANDS ]

\newcommand{\signaturebox}[1]{\multicolumn{1}{p{4in}}{\vspace{3ex}}\\\midrule #1\\}

%%%% macros fro standard references
%\eqref provided by amsmath
\newcommand{\figref}[1]{Fig.~\ref{#1}}
\newcommand{\tableref}[1]{Table~\ref{#1}}
\newcommand{\refcite}[1]{Ref.~\cite{#1}}

% Abbreviations from CMPPSS:

\newcommand{\eM}     {\mbox{$\epsilon$-machine}}
\newcommand{\eMs}    {\mbox{$\epsilon$-machines}}
\newcommand{\EM}     {\mbox{$\epsilon$-Machine}}
\newcommand{\EMs}    {\mbox{$\epsilon$-Machines}}
\newcommand{\eT}     {\mbox{$\epsilon$-transducer}}
\newcommand{\eTs}    {\mbox{$\epsilon$-transducers}}
\newcommand{\ET}     {\mbox{$\epsilon$-Transducer}}
\newcommand{\ETs}    {\mbox{$\epsilon$-Transducers}}

% Processes and sequences

\newcommand{\Process}{\mathcal{P}}

\newcommand{\ProbMach}{\Prob_{\mathrm{M}}}
\newcommand{\Lmax}   { {L_{\mathrm{max}}}}
\newcommand{\MeasAlphabet}	{\mathcal{A}}
% Original
%\newcommand{\MeasSymbol}   { {S} }
%\newcommand{\meassymbol}   { {s} }
% New symbol
\newcommand{\MeasSymbol}   { {X} }
\newcommand{\meassymbol}   { {x} }
\newcommand{\BiInfinity}	{ \overleftrightarrow {\MeasSymbol} }
\newcommand{\biinfinity}	{ \overleftrightarrow {\meassymbol} }
\newcommand{\Past}	{ \overleftarrow {\MeasSymbol} }
\newcommand{\past}	{ {\overleftarrow {\meassymbol}} }
\newcommand{\pastprime}	{ {\past}^{\prime}}
\newcommand{\Future}	{ \overrightarrow{\MeasSymbol} }
\newcommand{\future}	{ \overrightarrow{\meassymbol} }
\newcommand{\futureprime}	{ {\future}^{\prime}}
\newcommand{\PastPrime}	{ {\Past}^{\prime}}
\newcommand{\FuturePrime}	{ {\overrightarrow{\meassymbol}}^\prime }
\newcommand{\PastDblPrime}	{ {\overleftarrow{\meassymbol}}^{\prime\prime} }
\newcommand{\FutureDblPrime}	{ {\overrightarrow{\meassymbol}}^{\prime\prime} }
\newcommand{\pastL}	{ {\overleftarrow {\meassymbol}}{}^L }
\newcommand{\PastL}	{ {\overleftarrow {\MeasSymbol}}{}^L }
\newcommand{\PastLt}	{ {\overleftarrow {\MeasSymbol}}_t^L }
\newcommand{\PastLLessOne}	{ {\overleftarrow {\MeasSymbol}}^{L-1} }
\newcommand{\futureL}	{ {\overrightarrow{\meassymbol}}{}^L }
\newcommand{\FutureL}	{ {\overrightarrow{\MeasSymbol}}{}^L }
\newcommand{\FutureLt}	{ {\overrightarrow{\MeasSymbol}}_t^L }
\newcommand{\FutureLLessOne}	{ {\overrightarrow{\MeasSymbol}}^{L-1} }
\newcommand{\pastLprime}	{ {\overleftarrow {\meassymbol}}^{L^\prime} }
\newcommand{\futureLprime}	{ {\overrightarrow{\meassymbol}}^{L^\prime} }
\newcommand{\AllPasts}	{ { \overleftarrow {\rm {\bf \MeasSymbol}} } }
\newcommand{\AllFutures}	{ \overrightarrow {\rm {\bf \MeasSymbol}} }
\newcommand{\FutureSet}	{ \overrightarrow{\bf \MeasSymbol}}

% Causal states and epsilon-machines
\newcommand{\CausalState}	{ \mathcal{S} }
\newcommand{\CausalStatePrime}	{ {\CausalState}^{\prime}}
\newcommand{\causalstate}	{ \sigma }
\newcommand{\CausalStateSet}	{ \boldsymbol{\CausalState} }
\newcommand{\AlternateState}	{ \mathcal{R} }
\newcommand{\AlternateStatePrime}	{ {\cal R}^{\prime} }
\newcommand{\alternatestate}	{ \rho }
\newcommand{\alternatestateprime}	{ {\rho^{\prime}} }
\newcommand{\AlternateStateSet}	{ \boldsymbol{\AlternateState} }
\newcommand{\PrescientState}	{ \widehat{\AlternateState} }
\newcommand{\prescientstate}	{ \widehat{\alternatestate} }
\newcommand{\PrescientStateSet}	{ \boldsymbol{\PrescientState}}
\newcommand{\CausalEquivalence}	{ {\sim}_{\epsilon} }
\newcommand{\CausalEquivalenceNot}	{ {\not \sim}_{\epsilon}}

\newcommand{\NonCausalEquivalence}	{ {\sim}_{\eta} }
\newcommand{\NextObservable}	{ {\overrightarrow {\MeasSymbol}}^1 }
\newcommand{\LastObservable}	{ {\overleftarrow {\MeasSymbol}}^1 }
%\newcommand{\Prob}		{ {\rm P}}
\newcommand{\Prob}      {\Pr} % use standard command
\newcommand{\ProbAnd}	{ {,\;} }
\newcommand{\LLimit}	{ {L \rightarrow \infty}}
\newcommand{\Cmu}		{C_\mu}
\newcommand{\hmu}		{h_\mu}
\newcommand{\EE}		{{\bf E}}
\newcommand{\Measurable}{{\bf \mu}}

% Process Crypticity
\newcommand{\PC}		{\chi}
\newcommand{\FuturePC}		{\PC^+}
\newcommand{\PastPC}		{\PC^-}
% Causal Irreversibility
\newcommand{\CI}		{\Xi}
\newcommand{\ReverseMap}	{r}
\newcommand{\ForwardMap}	{f}

% Abbreviations from IB:
% None that aren't already in CMPPSS

% Abbreviations from Extensive Estimation:
\newcommand{\EstCausalState}	{\widehat{\CausalState}}
\newcommand{\estcausalstate}	{\widehat{\causalstate}}
\newcommand{\EstCausalStateSet}	{\boldsymbol{\EstCausalState}}
\newcommand{\EstCausalFunc}	{\widehat{\epsilon}}
\newcommand{\EstCmu}		{\widehat{\Cmu}}
\newcommand{\PastLOne}	{{\Past}^{L+1}}
\newcommand{\pastLOne}	{{\past}^{L+1}}

% Abbreviations from $\epsilon$-Transducers:
\newcommand{\InAlphabet}	{ \mathcal{A}}
\newcommand{\insymbol}		{ a}
\newcommand{\OutAlphabet}	{ \mathcal{B}}
\newcommand{\outsymbol}		{ b}
\newcommand{\InputSimple}	{ X}
\newcommand{\inputsimple}	{ x}
\newcommand{\BottleneckVar}	{\tilde{\InputSimple}}
\newcommand{\bottleneckvar}	{\tilde{\inputsimple}}
\newcommand{\InputSpace}	{ \mathbf{\InputSimple}}
\newcommand{\InputBi}	{ \overleftrightarrow {\InputSimple} }
\newcommand{\inputbi}	{ \overleftrightarrow {\inputsimple} }
\newcommand{\InputPast}	{ \overleftarrow {\InputSimple} }
\newcommand{\inputpast}	{ \overleftarrow {\inputsimple} }
\newcommand{\InputFuture}	{ \overrightarrow {\InputSimple} }
\newcommand{\inputfuture}	{ \overrightarrow {\inputsimple} }
\newcommand{\NextInput}	{ {{\InputFuture}^{1}}}
\newcommand{\NextOutput}	{ {\OutputFuture}^{1}}
\newcommand{\OutputSimple}	{ Y}
\newcommand{\outputsimple}	{ y}
\newcommand{\OutputSpace}	{ \mathbf{\OutputSimple}}
\newcommand{\OutputBi}	{ \overleftrightarrow{\OutputSimple} }
\newcommand{\outputbi}	{ \overleftrightarrow{\outputsimple} }
\newcommand{\OutputPast}	{ \overleftarrow{\OutputSimple} }
\newcommand{\outputpast}	{ \overleftarrow{\outputsimple} }
\newcommand{\OutputFuture}	{ \overrightarrow{\OutputSimple} }
\newcommand{\outputfuture}	{ \overrightarrow{\outputsimple} }
\newcommand{\OutputL}	{ {\OutputFuture}^L}
\newcommand{\outputL}	{ {\outputfuture}^L}
\newcommand{\InputLLessOne}	{ {\InputFuture}^{L-1}}
\newcommand{\inputLlessone}	{ {\inputufutre}^{L-1}}
\newcommand{\OutputPastLLessOne}	{{\OutputPast}^{L-1}_{-1}}
\newcommand{\outputpastLlessone}	{{\outputpast}^{L-1}}
\newcommand{\OutputPastLessOne}	{{\OutputPast}_{-1}}
\newcommand{\outputpastlessone}	{{\outputpast}_{-1}}
\newcommand{\OutputPastL}	{{\OutputPast}^{L}}
\newcommand{\OutputLPlusOne}	{ {\OutputFuture}^{L+1}}
\newcommand{\outputLplusone}	{ {\outputfutre}^{L+1}}
\newcommand{\InputPastL}	{{\InputPast}^{L}}
\newcommand{\inputpastL}	{{\inputpast}^{L}}
\newcommand{\JointPast}	{{(\InputPast,\OutputPast)}}
\newcommand{\jointpast}	{{(\inputpast,\outputpast)}}
\newcommand{\jointpastone}	{{(\inputpast_1,\outputpast_1)}}
\newcommand{\jointpasttwo}	{{(\inputpast_2,\outputpast_2)}}
\newcommand{\jointpastprime} {{({\inputpast}^{\prime},{\outputpast}^{\prime})}}
\newcommand{\NextJoint}	{{(\NextInput,\NextOutput)}}
\newcommand{\nextjoint}	{{(\insymbol,\outsymbol)}}
\newcommand{\AllInputPasts}	{ { \overleftarrow {\rm \InputSpace}}}
\newcommand{\AllOutputPasts}	{ {\overleftarrow {\rm \OutputSpace}}}
\newcommand{\DetCausalState}	{ {{\cal S}_D }}
\newcommand{\detcausalstate}	{ {{\sigma}_D} }
\newcommand{\DetCausalStateSet}	{ \boldsymbol{{\CausalState}_D}}
\newcommand{\DetCausalEquivalence}	{ {\sim}_{{\epsilon}_{D}}}
\newcommand{\PrescientEquivalence}	{ {\sim}_{\widehat{\eta}}}
\newcommand{\FeedbackCausalState}	{ \mathcal{F}}
\newcommand{\feedbackcausalstate}	{ \phi}
\newcommand{\FeedbackCausalStateSet}	{ \mathbf{\FeedbackCausalState}}
\newcommand{\JointCausalState}		{ \mathcal{J}}
\newcommand{\JointCausalStateSet}	{ \mathbf{\JointCausalState}}
\newcommand{\UtilityFunctional}	{ {\mathcal{L}}}
\newcommand{\NatureState}	{ {\Omega}}
\newcommand{\naturestate}	{ {\omega}}
\newcommand{\NatureStateSpace}	{ {\mathbf{\NatureState}}}
\newcommand{\AnAction}	{ {A}}
\newcommand{\anaction}	{ {a}}
\newcommand{\ActionSpace}	{ {\mathbf{\AnAction}}}

% Abbreviations from RURO:
\newcommand{\InfoGain}[2] { \mathcal{D} \left( {#1} || {#2} \right) }

% Abbreviations from Upper Bound:
\newcommand{\lcm}	{{\rm lcm}}
% Double-check that this isn't in the math set already!

% Abbreviations from Emergence in Space
\newcommand{\ProcessAlphabet}	{\MeasAlphabet}
\newcommand{\ProbEst}			{ {\widehat{\Prob}_N}}
\newcommand{\STRegion}			{ {\mathrm K}}
\newcommand{\STRegionVariable}		{ K}
\newcommand{\stregionvariable}		{ k}
\newcommand{\GlobalPast}		{ \overleftarrow{G}} 
\newcommand{\globalpast}		{ \overleftarrow{g}} 
\newcommand{\GlobalFuture}		{ \overrightarrow{G}}
\newcommand{\globalfuture}		{ \overrightarrow{g}}
\newcommand{\GlobalState}		{ \mathcal{G}}
\newcommand{\globalstate}		{ \gamma}
\newcommand{\GlobalStateSet}		{ {\mathbf \GlobalState}}
\newcommand{\LocalPast}			{ \overleftarrow{L}} 
\newcommand{\localpast}			{ \overleftarrow{l}}
\newcommand{\AllLocalPasts}		{ \mathbf{\LocalPast}}
\newcommand{\LocalPastRegion}		{ \overleftarrow{\mathrm L}}
\newcommand{\LocalFuture}		{ \overrightarrow{L}}
\newcommand{\localfuture}		{ \overrightarrow{l}}
\newcommand{\LocalFutureRegion}		{ \overrightarrow{\mathrm L}}
\newcommand{\LocalState}		{ \mathcal{L}}
\newcommand{\localstate}		{ \lambda}
\newcommand{\LocalStateSet}		{ {\mathbf \LocalState}}
\newcommand{\PatchPast}			{ \overleftarrow{P}}
\newcommand{\patchpast}			{ \overleftarrow{p}}
\newcommand{\PatchPastRegion}		{ \overleftarrow{\mathrm P}}
\newcommand{\PatchFuture}		{ \overrightarrow{P}}
\newcommand{\patchfuture}		{ \overrightarrow{p}}
\newcommand{\PatchFutureRegion}		{ \overrightarrow{\mathrm P}}
\newcommand{\PatchState}		{ \mathcal{P}}
\newcommand{\patchstate}		{ \pi}
\newcommand{\PatchStateSet}		{ {\mathbf \PatchState}}
\newcommand{\LocalStatesInPatch}	{\vec{\LocalState}}
\newcommand{\localstatesinpatch}	{\vec{\localstate}}
\newcommand{\PointInstantX}		{ {\mathbf x}}
% Galles's original LaTeX for the cond. indep. symbol follows:
\newcommand{\compos}{\mbox{$~\underline{~\parallel~}~$}}
\newcommand{\ncompos}{\not\hspace{-.15in}\compos}
\newcommand{\indep}			{ \rotatebox{90}{$\models$}}
\newcommand{\nindep}	{\not\hspace{-.05in}\indep}
\newcommand{\LocalEE}	{{\EE}^{loc}}
\newcommand{\EEDensity}	{\overline{\LocalEE}}
\newcommand{\LocalCmu}	{{\Cmu}^{loc}}
\newcommand{\CmuDensity}	{\overline{\LocalCmu}}

%%%%%%%%%%% added by sasa
\newcommand{\FinPast}[1]	{ \overleftarrow {\MeasSymbol} \stackrel{{#1}}{}}
\newcommand{\finpast}[1]  	{ \overleftarrow {\meassymbol}  \stackrel{{#1}}{}}
\newcommand{\FinFuture}[1]		{ \overrightarrow{\MeasSymbol} \stackrel{{#1}}{}}
\newcommand{\finfuture}[1]		{ \overrightarrow{\meassymbol} \stackrel{{#1}}{}}

\newcommand{\Partition}	{ \AlternateState }
\newcommand{\partitionstate}	{ \alternatestate }
\newcommand{\PartitionSet}	{ \AlternateStateSet }
\newcommand{\Fdet}   { F_{\rm det} }

\newcommand{\Dkl}[2] { D_{\rm KL} \left( {#1} || {#2} \right) }

\newcommand{\Period}	{p}

% To take into account time direction
\newcommand{\forward}{+}
\newcommand{\reverse}{-}
%\newcommand{\forwardreverse}{\:\!\diamond} % \pm
\newcommand{\forwardreverse}{\pm} % \pm
\newcommand{\FutureProcess}	{ {\Process}^{\forward} }
\newcommand{\PastProcess}	{ {\Process}^{\reverse} }
\newcommand{\FutureCausalState}	{ {\CausalState}^{\forward} }
\newcommand{\futurecausalstate}	{ \sigma^{\forward} }
\newcommand{\altfuturecausalstate}	{ \sigma^{\forward\prime} }
\newcommand{\PastCausalState}	{ {\CausalState}^{\reverse} }
\newcommand{\pastcausalstate}	{ \sigma^{\reverse} }
\newcommand{\BiCausalState}		{ {\CausalState}^{\forwardreverse} }
\newcommand{\bicausalstate}		{ {\sigma}^{\forwardreverse} }
\newcommand{\FutureCausalStateSet}	{ {\CausalStateSet}^{\forward} }
\newcommand{\PastCausalStateSet}	{ {\CausalStateSet}^{\reverse} }
\newcommand{\BiCausalStateSet}	{ {\CausalStateSet}^{\forwardreverse} }
\newcommand{\eMachine}	{ M }
\newcommand{\FutureEM}	{ {\eMachine}^{\forward} }
\newcommand{\PastEM}	{ {\eMachine}^{\reverse} }
\newcommand{\BiEM}		{ {\eMachine}^{\forwardreverse} }
\newcommand{\BiEquiv}	{ {\sim}^{\forwardreverse} }
\newcommand{\Futurehmu}	{ h_\mu^{\forward} }
\newcommand{\Pasthmu}	{ h_\mu^{\reverse} }
\newcommand{\FutureCmu}	{ C_\mu^{\forward} }
\newcommand{\PastCmu}	{ C_\mu^{\reverse} }
\newcommand{\BiCmu}		{ C_\mu^{\forwardreverse} }
\newcommand{\FutureEps}	{ \epsilon^{\forward} }
\newcommand{\PastEps}	{ \epsilon^{\reverse} }
\newcommand{\BiEps}	{ \epsilon^{\forwardreverse} }
\newcommand{\FutureSim}	{ \sim^{\forward} }
\newcommand{\PastSim}	{ \sim^{\reverse} }
% Used arrows for awhile, more or less confusing?
%\newcommand{\FutureCausalState}	{ \overrightarrow{\CausalState} }
%\newcommand{\PastCausalState}	{ \overleftarrow{\CausalState} }
%\newcommand{\eMachine}	{ M }
%\newcommand{\FutureEM}	{ \overrightarrow{\eMachine} }
%\newcommand{\PastEM}	{ \overleftarrow{\eMachine} }
%\newcommand{\FutureCmu}	{ \overrightarrow{\Cmu} }
%\newcommand{\PastCmu}	{ \overleftarrow{\Cmu} }

%% time-reversing and mixed state presentation operators
\newcommand{\TR}{\mathcal{T}}
\newcommand{\MSP}{\mathcal{U}}
\newcommand{\one}{\mathbf{1}}

%% (cje)
%% Provide a command \ifpm which is true when \pm 
%% is meant to be understood as "+ or -". This is
%% different from the usage in TBA.
\newif\ifpm 
\edef\tempa{\forwardreverse}
\edef\tempb{\pm}
\ifx\tempa\tempb
   \pmfalse
\else
   \pmtrue  
\fi





% [some math stuff - maybe stick in sep file]
\usepackage{amsthm}
\usepackage{amscd}
\theoremstyle{plain}    \newtheorem{Lem}{Lemma}
\theoremstyle{plain}    \newtheorem*{ProLem}{Proof}
\theoremstyle{plain} 	\newtheorem{Cor}{Corollary}
\theoremstyle{plain} 	\newtheorem*{ProCor}{Proof}
\theoremstyle{plain} 	\newtheorem{The}{Theorem}
\theoremstyle{plain} 	\newtheorem*{ProThe}{Proof}
\theoremstyle{plain} 	\newtheorem{Prop}{Proposition}
\theoremstyle{plain} 	\newtheorem*{ProProp}{Proof}
\theoremstyle{plain} 	\newtheorem*{Conj}{Conjecture}
\theoremstyle{plain}	\newtheorem*{Rem}{Remark}
\theoremstyle{plain}	\newtheorem*{Def}{Definition} 
\theoremstyle{plain}	\newtheorem*{Not}{Notation}

% [uniform figure scaling - maybe this is not a good idea]
\def\figscale{.7}
\def\lscale{1.0}

% [FIX ME! - red makes it easier to spot]
\newcommand{\FIX}[1]{\textbf{\textcolor{red}{#1}}}


    \begin{document}
    \def\noprelim{}
\else
    % already included once
    % input post files

    \singlespacing
    \bibliographystyle{../bibliography/expanded}
    \bibliography{../bibliography/references}

    \end{document}
\fi\fi

\chapter{The Standard Model and Beyond}
\label{ch:theory}

\section{The Standard Model}

The Standard Model (SM) is a ``theory of almost everything'' that describes the
interactions of elementary particles.  The Standard Model is a \emph{quantum
field theory}, first appearing in its modern form in the middle of the 20th
century.  The model is the synthesis of the independent theories of
electromagnetism, and the weak and strong nuclear forces.  Each of these
theories was used to describe different phenomena, which each have extremely
different strengths and act at different scales.  The interaction of light and
matter is described by Quantum Electrodynamics~(QED), a relativistic field
extension of the theory of electromagnetism.  The physics of radioactivity and
nuclear decay was described by the Fermi theory of weak interactions and the
forces that strong nuclear force binds the nuclei of atoms was described by
Yukawa.  An overview of these theories will be presented in this chapter.  

The feature that united the disparate theories into the Standard Model was the
application of the principle of \emph{local gauge invariance}. The principle of
gauge invariance first found success in QED, which predicted electromagnetic
phenomenon with astounding accuracy.  Local gauge invariance is now believed to
a fundamental feature of nature that underpins all theories of elementary
particles.  Furthermore, the development of the complete Standard Model as it
is known today was precipitated by  Glashow, Weinberg, and Salam's
discovery~\cite{ref:GlashowSymmetryBreaking} of the potential to unify QED with
the weak force using spontaneous symmetry breaking, and the development of the
Higgs mechanism~\cite{ref:HiggsMechanism}, which breaks the electroweak symmetry
using the clever addition of an additional scalar particle (the Higgs boson).

\subsection{Quantum Electrodynamics and Gauge Invariance}

The theory of QED is a modern extension of Maxwell's theory of
electromagnetism, describing the interaction of matter with light.  The
development of QED is a result of efforts to develop a quantum mechanical
formulation of electromagnetism compatible with the theory of Special Relativity.
QED is a \emph{gauge} theory, which means that the physical observables are
invariant under local gauge transformations.  Requiring local gauge
invariance gives rise to a ``gauge'' field, which can be interpreted as 
particles that are exchanged during an interaction.  

In the following, we first describe the Dirac equation for a free electron,
which is the relativistic extension of the Schroedinger equation for spin~$1/2$
particles.  We then show that requiring the corresponding Lagrangian of the free
charged particle to be invariant under local gauge transformations creates an
effective gauge boson field.  This ``gauge field'' creates terms in the
Lagrangian that represent interactions between the particles.

% Dirac equation comes from pg. 215 of Griffiths
The Dirac equation is the equation of motion of a free spin~$1/2$ particle of
mass~$m$ and is derived from the energy--momentum relationship of relativity
\begin{equation}
  p^{\mu}p_\mu - m^2c^2 = 0
  \label{eq:EnergyPRelationship}
\end{equation}.
Dirac sought to express this relationship in the framework of quantum mechanics
by applying the transformation
\begin{equation}
  p_\mu \to i \hbar \partial_\mu 
  \label{eq:QuantizeMomentumTransformation}
\end{equation}
to equation Equation~\ref{eq:EnergyPRelationship}, but with the requirement that
the resulting equation be first order in time.\footnote{A detailed
discussion of this topic is available in~\cite{Griffiths:IntroParticle}.}
To achieve this, Dirac factorized Equation~\ref{eq:EnergyPRelationship} into 
\begin{equation}
  (\gamma^\kappa p_\kappa + mc)(\gamma^\mu p_\mu - mc) = 0,
  \label{eq:DiracEquation}
\end{equation}
where $\gamma^\mu$ is a set of four $4\times4$ matrices referred to as the Dirac
matrices.  The equation of motion is obtained by choosing either term (they are
equivalent) from the
left hand side of Equation~\ref{eq:DiracEquation} and
making the substitution in Equation~\ref{eq:QuantizeMomentumTransformation}.
\begin{equation}
  i \hbar \gamma^\mu \partial_\mu \psi - mc \psi = 0
  \label{eq:DiracEquation}
\end{equation}
The solutions, $\psi$, to the Dirac equation are called ``Dirac Spinors,'' and
represent the quantum mechanical state of spin~$1/2$ particles.

The Lagrangian corresponding to the Dirac equation~(\ref{eq:DiracEquation}) is
\begin{equation}
  \mathcal{L} = \overline \psi (i\hbar c\gamma^\mu \partial_\mu - mc^2) \psi,
  \label{eq:FreeQEDLagrangian}
\end{equation}
where $\psi$ is the spinor field of the particle in question, $\hbar$ is Planck's
constant, $c$ the speed of light, and $\gamma^\mu$ are the Dirac matrices.  As
$\overline\psi$ is the Hermitian conjugate of $\psi$, the Lagrangian is invariant
under the global gauge transformation 
\begin{equation}
  \psi' \to e^{i\theta}\psi
  \label{eq:U1GaugeTransform}
\end{equation}.
The Lagrangian is invariant under \emph{local} gauge translations if $\theta$
can be defined differently at each point in space, \ie if $\theta = \theta(x)$
in equation~\ref{eq:U1GaugeTransform}.  However, as the derivative operator
$\partial_\mu$ in equation~\ref{eq:FreeQEDLagrangian} does not commute with
$\theta(x)$, the Lagrangian must be modified to satisfy local gauge invariance.
This modification is accomplished with the use of a ``gauge covariant
derivative.''  By making the replacement 
\begin{equation} 
  \partial_\mu \to D_\mu = \partial_\mu - \frac{ie}{\hbar}A^\mu      
\end{equation}
in equation~\ref{eq:FreeQEDLagrangian}, where \FIX{don't think this is right}
$A^\mu = \partial^\mu \theta(x)$ and $e$ is the electric charge, the Lagrangian
becomes locally gauge invariant:
\begin{equation}
  \mathcal{L} = \overline \psi (i\hbar c\gamma^\mu D_\mu - mc^2) \psi 
  \label{eq:LocalQEDLagrangian}
\end{equation}
The difference between
the locally (equation~\ref{eq:LocalQEDLagrangian}) and globally 
(equation~\ref{eq:FreeQEDLagrangian}) gauge invariant Lagrangians is then
\begin{equation}
  \mathcal{L}_{int} = \frac{e}{\hbar}\overline\psi\gamma^\mu\psi A_\mu 
\end{equation}.
This term can be interpreted as the coupling between the particle and the gauge
boson (force carrier) fields.  The coupling is proportional to the constant
$e$, which is associated with the electric charge.  This is consistent with the
experimental observation that particles with zero electric charge do not
interact electromagnetically with each other.  In this interpretation, the
electromagnetic force between two charged particles is caused by the exchange
of gauge bosons (photons).  The existence of this coupling is required if the
Lagrangian is to satisfy local gauge invariance. The addition of a term with
the gauge Field Strength Tensor yields the QED Lagrangian:
\begin{equation}
  \mathcal{L}_{QED} = \overline \psi (i\hbar c\gamma^\mu D_\mu - mc^2) \psi -
  \frac{1}{4\mu_0}F_{\mu\nu}F^{\mu\nu}
\end{equation}

The gauge symmetry group of QED is $U(1)$, the unitary group of degree 1.  This
symmetry can be visualized as a rotation of a two--dimensional unit vector. (The
application of the gauge transformation $e^{i\theta}$ rotates a number in the
complex plane.)  In gauge theory that the symmetry group
defines the behavior of the gauge bosons and thus the interactions of the
theory.  

\subsection{The Weak Interactions}

The theory of Weak Interactions was created to describe the physics of
radioactivity.  The first formulation of the theory was done by
Fermi~\cite{ref:FermiWeakInteration} to explain the phenomenon of the $beta$
decay of radioactive elements.  The initial theory was a four-fermion
``contact'' theory.  In a contact theory, all four fermions come from a single
vertex.  The Lagrangian for the Fermi interactions is:
\begin{equation}
 find me 
 \label{eq:FermiContactInteraction}
\end{equation}
The Fermi interaction can describe both nuclear $\beta$ decay ($p \to n + e +
\overline{\nu_e}$) as well as the decay of a muon into an electron ($\mu \to \nu_\mu
+ e + \overline{\nu_e}$).

However, the contact interaction form of Fermi's theory is not complete.  At
higher energies the theory suffers from \FIX{what?}.  The theory is made
renormalizable by the using the gauge theory formalism.  The contact interaction
is replaced by massive propagators, the $W^\pm$ and $Z$ gauge bosons.  The term
in the $\beta$--decay Feynman diagram associated with the massive $Z$ boson
propagator is
\begin{equation}
  fidn me
  \label{eq:WeakPropagator}
\end{equation}
The presence of the large mass of gauge boson in the denominator is the reason
why the contact interaction original formulated by Fermi is effectively
describes of low--energy weak phenomenon.  When the momentum transfer $q$ is
small compared to $M_W$, the effect of the propagator is an effective constant.
In the low energy limit, the full propagator in equation~\ref{eq:WeakPropagator}
is equivalent to the Fermi contact interaction
in~\ref{eq:FermiContactInteraction}.

The ``charge'' of the theory of interaction is called weak isospin.  Similar to
the electric charge of QED, only particles with non--zero weak isospin couple to
the gauge bosons of the weak force, and weak isospin is conserved in weak
interactions.   The weak force is \emph{chiral}; the ``handedness'' of the
particles affects how they interact with each other.  In fact, the weak
interaction violates parity: the interaction is not invariant under spatial
inversion.  Left--handed fermions carry weak isospin $T = 1/2$, and form a
doublet under the $SU(2)$ gauge group.  The right--handed fermions have no weak
isospin, and are singlets under the gauge group.  

\subsection{Quantum Chromodynamics}

\subsection{Electroweak Unification}
In the early 1960s Glashow published a series of papers describing how the
electromagnetic and weak forces could be unified into a common ``electroweak''
force.  The fact that at low energy the electromagnetic and weak forces appear
to be separate phenomena is due to the fact that the symmetry of the electroweak
gauge group is spontaneously broken.

\subsection{The Higgs mechanism}


\section{Searches for the Higgs boson}
\subsection{Higgs boson phenomenology}
\label{sec:SMHiggsPhenom}
\subsection{Results from LEP and Tevatron}

\section{Beyond the Standard Model}
\label{sec:BSM}
\subsection{The Hierarchy Problem}
\subsection{Supersymmetry}
\subsection{The Minimal Supersymmetric Model and $\tau$ leptons}
\label{sec:MSSMAndTaus}

\section{The physics of the \taul}
As discussed in sections~\ref{sec:SMHiggsPhenom} and~\ref{sec:MSSMAndTaus}, the
$\tau$ lepton is an important probe of Higgs physics.  The \taul has some
unusual properties which make it particularly challenging at hadron colliders.
With a mass of 1.78~\GeVcc, the \taul is heaviest of the leptons.  The nominal
decay distance $c\tau$ of the \taul is 87~\micron, which in practice means that
the $\tau$ will always decay before reaching the first layer of the detector.
Tau decays can be effectively classified into two types. ``Leptonic'' decays
consist of a $\tau$ decaying to a light lepton ($\ell = e, \mu$) and two
neutrinos $\tau^+ \to \ell^+ \nu_\tau \overline{ \nu_{\ell}}$.  ``Hadronic'' decays 
consist of a low--multiplicity collimated group of hadrons, typically \Pgppm and \Pgpz
mesons.  The hadronic decays of the \taul compose approximately $65\%$ of the
\taul branching fraction, with the remainder shared approximately equally by the
leptonic decays.  The branching fractions for the leptonic and most common hadronic decays
are shown in table~\ref{tab:TauDecayModes}.
\begin{table}
   \centering
   \begin{tabular}{lcrr}
      Visible Decay Products  & Resonance & Mass (\MeVcc) &
      Fraction~\cite{PDG} \\
      \hline
      \hline
      \multicolumn{4}{c}{Leptonic modes modes} \\
      \hline
      $e^- \nu_\tau \overline \nu_e$             & -      & 0.5  & 17.8\% \\
      $\mu^-\nu_\tau \overline \nu_\mu$          & -      & 105  & 17.4\% \\
      \hline
      \multicolumn{4}{c}{Hadronic modes} \\
      \hline
      $\pi^{-} \nu_\tau$                    & -      & 135  & 10.9\% \\
      $\pi^{-}\pi^0 \nu_\tau$               & $\rho$ & 770  & 25.5\% \\
      $\pi^{-}\pi^0\pi^0 \nu_\tau$          & $a1$   & 1200 & 9.3\% \\
      $\pi^{-}\pi^{-}\pi^{+} \nu_\tau$      & $a1$   & 1200 & 9.0\% \\
      $\pi^{-}\pi^{-}\pi^{+}\pi^0 \nu_\tau$ & $a1$   & 1200 & 4.5\% \\
      \hline
      Total                                 &        &      & 94.4\% \\
      \hline
   \end{tabular}
   \label{tab:decay_modes}
   \caption{Resonances and branching ratios of the dominant decay modes of
   the \taul.  The decay products listed correspond to a negatively
   charged \taul; the table is identical under charge conjugation.}
\end{table}

\ifx\master\undefined% AUTOCOMPILE
% Allows for individual chapters to be compiled.

% Usage:
% \ifx\master\undefined% AUTOCOMPILE
% Allows for individual chapters to be compiled.

% Usage:
% \ifx\master\undefined% AUTOCOMPILE
% Allows for individual chapters to be compiled.

% Usage:
% \ifx\master\undefined\input{../settings/autocompile}\fi
% (place at start and end of chapter file)

\ifx\noprelim\undefined
    % first time included
    % input preamble files
    \input{../settings/phdsetup}

    \begin{document}
    \def\noprelim{}
\else
    % already included once
    % input post files

    \singlespacing
    \bibliographystyle{../bibliography/expanded}
    \bibliography{../bibliography/references}

    \end{document}
\fi\fi
% (place at start and end of chapter file)

\ifx\noprelim\undefined
    % first time included
    % input preamble files
    % [ USER VARIABLES ]

\def\PHDTITLE {Extensions of the Theory of Computational Mechanics}
\def\PHDAUTHOR{Evan Klose Friis}
\def\PHDSCHOOL{University of California, Davis}

\def\PHDMONTH {June}
\def\PHDYEAR  {2011}
\def\PHDDEPT {Physics}

\def\BSSCHOOL {University of California at San Diego}
\def\BSYEAR   {2005}

\def\PHDCOMMITTEEA{Professor John Conway}
\def\PHDCOMMITTEEB{Professor Robin Erbacher}
\def\PHDCOMMITTEEC{Professor Mani Tripathi}

% [ GLOBAL SETUP ]

\documentclass[letterpaper,oneside,11pt]{report}

\usepackage{calc}
\usepackage{breakcites}
\usepackage[newcommands]{ragged2e}

\usepackage[pdftex]{graphicx}
\usepackage{epstopdf}

%\usepackage{tikz}
%\usetikzlibrary{positioning} % [right=of ...]
%\usetikzlibrary{fit} % [fit= ...]

%\pgfdeclarelayer{background layer}
%\pgfdeclarelayer{foreground layer}
%\pgfsetlayers{background layer,main,foreground layer}

%\newenvironment{wrap}{\noindent\begin{minipage}[t]{\linewidth}\vspace{-0.5\normalbaselineskip}\centering}{\vspace{0.5\normalbaselineskip}\end{minipage}}

%% [Venn diagram environment]
%\newenvironment{venn2}
%{\begin{tikzpicture} [every pin/.style={text=black, text opacity=1.0, pin distance=0.5cm, pin edge={black!60, semithick}},
%% define a new style 'venn'
%venn/.style={circle, draw=black!60, semithick, minimum size = 4cm}]
%
%% create circle and give it external (pin) label
%\node[venn] (X) at (-1,0) [pin={150:$H[X]$}] {};
%\node[venn] (Y) at (1,0) [pin={30:$H[Y]$}] {};
%
%% place labels of the atoms by hand
%\node at (-1.9,0) {$H[X|Y]$};
%\node at (1.9,0) {$H[Y|X]$};
%\node at (0,0) {$I[X;Y]$};}
%{\end{tikzpicture}}

%\newcommand{\wrapmath}[1]{\begin{wrap}\begin{tikzpicture}[every node/.style={inner ysep=0ex, inner xsep=0em}]\node[] {$\displaystyle\begin{aligned} #1\end{aligned}$};\end{tikzpicture}\end{wrap}}

\renewenvironment{abstract}{\chapter*{Abstract}}{}
\renewcommand{\bibname}{Bibliography}
\renewcommand{\contentsname}{Table of Contents}

\makeatletter
\renewcommand{\@biblabel}[1]{\textsc{#1}}
\makeatother

% [ FONT SETTINGS ]

\usepackage[tbtags, intlimits, namelimits]{amsmath}
\usepackage[adobe-utopia]{mathdesign}

\DeclareSymbolFont{pazomath}{OMS}{zplm}{m}{n}
\DeclareSymbolFontAlphabet{\mathcal}{pazomath}
\SetMathAlphabet\mathcal{bold}{OMS}{zplm}{b}{n}

\SetSymbolFont{largesymbols}{normal}{OMX}{zplm}{m}{n}
\SetSymbolFont{largesymbols}{bold}{OMX}{zplm}{m}{n}
\SetSymbolFont{symbols}{normal}{OMS}{zplm}{m}{n}
\SetSymbolFont{symbols}{bold}{OMS}{zplm}{b}{n}

\renewcommand{\sfdefault}{phv}
\renewcommand{\ttdefault}{fvm}

\widowpenalty 8000
\clubpenalty  8000

% [ PAGE LAYOUT ]
\usepackage{geometry}
\geometry{lmargin = 1.5in}
\geometry{rmargin = 1.0in}
\geometry{tmargin = 1.0in}
\geometry{bmargin = 1.0in}

% [ PDF SETTINGS ]

\usepackage[final]{hyperref}
\hypersetup{breaklinks  = true}
\hypersetup{colorlinks  = true}
\hypersetup{linktocpage = false}
\hypersetup{linkcolor   = blue}
\hypersetup{citecolor   = green}
\hypersetup{urlcolor    = black}
\hypersetup{plainpages  = false}
\hypersetup{pageanchor  = true}
\hypersetup{pdfauthor   = {\PHDAUTHOR}}
\hypersetup{pdftitle    = {\PHDTITLE}}
\hypersetup{pdfsubject  = {Dissertation, \PHDSCHOOL}}
\urlstyle{same}

% [ LETTER SPACING ]

\usepackage[final]{microtype}
\microtypesetup{protrusion=compatibility}
\microtypesetup{expansion=false}

\newcommand{\upper}[1]{\MakeUppercase{#1}}
\let\lsscshape\scshape

\ifcase\pdfoutput\else\microtypesetup{letterspace=15}
\renewcommand{\scshape}{\lsscshape\lsstyle}
\renewcommand{\upper}[1]{\textls[50]{\MakeUppercase{#1}}}\fi

% [ LINE SPACING ]

\usepackage[doublespacing]{setspace}
\renewcommand{\displayskipstretch}{0.0}

\setlength{\parskip   }{0em}
\setlength{\parindent }{2em}

% [ TABLE FORMATTING ]

\usepackage{booktabs}
\setlength{\heavyrulewidth}{1.5\arrayrulewidth}
\setlength{\lightrulewidth}{1.0\arrayrulewidth}
\setlength{\doublerulesep }{2.0\arrayrulewidth}

% [ SECTION FORMATTING ]

\usepackage[largestsep,nobottomtitles*]{titlesec}
\renewcommand{\bottomtitlespace}{0.75in}

\titleformat{\chapter}[display]{\bfseries\huge\singlespacing}{\filleft\textsc{\LARGE \chaptertitlename\ \thechapter}}{-0.2ex}{\titlerule[3pt]\vspace{0.2ex}}[]

\titleformat{\section}{\LARGE}{\S\thesection\hspace{0.5em}}{0ex}{}
\titleformat{\subsection}{\Large}{\S\thesubsection\hspace{0.5em}}{0ex}{}
\titleformat{\subsubsection}{\large}{\thesubsubsection\hspace{0.5em}}{0ex}{}

\titlespacing*{\chapter}{0em}{6ex}{4ex plus 2ex minus 0ex}
\titlespacing*{\section}{0em}{2ex plus 3ex minus 1ex}{0.5ex plus 0.5ex minus 0.5ex}
\titlespacing*{\subsection}{0ex}{2ex plus 3ex minus 1ex}{0ex}
\titlespacing*{\subsubsection}{0ex}{2ex plus 0ex minus 1ex}{0ex}

% [ HEADER SETTINGS ]

\usepackage{fancyhdr}

\setlength{\headheight}{\normalbaselineskip}
\setlength{\footskip  }{0.5in}
\setlength{\headsep   }{0.5in-\headheight}

\fancyheadoffset[R]{0.5in}
\renewcommand{\headrulewidth}{0pt}
\renewcommand{\footrulewidth}{0pt}

\newcommand{\pagebox}{\parbox[r][\headheight][t]{0.5in}{\hspace\fill\thepage}}

\newcommand{\prelimheaders}{\ifx\prelim\undefined\renewcommand{\thepage}{\textit{\roman{page}}}\fancypagestyle{plain}{\fancyhf{}\fancyfoot[L]{\makebox[\textwidth-0.5in]{\thepage}}}\pagestyle{plain}\def\prelim{}\fi}

\newcommand{\normalheaders}{\renewcommand{\thepage}{\arabic{page}}\fancypagestyle{plain}{\fancyhf{}\fancyhead[R]{\pagebox}}\pagestyle{plain}}

\normalheaders{}

% [ CUSTOM COMMANDS ]

\newcommand{\signaturebox}[1]{\multicolumn{1}{p{4in}}{\vspace{3ex}}\\\midrule #1\\}

%\input{../includes/cmechabbrev}

% [some math stuff - maybe stick in sep file]
\usepackage{amsthm}
\usepackage{amscd}
\theoremstyle{plain}    \newtheorem{Lem}{Lemma}
\theoremstyle{plain}    \newtheorem*{ProLem}{Proof}
\theoremstyle{plain} 	\newtheorem{Cor}{Corollary}
\theoremstyle{plain} 	\newtheorem*{ProCor}{Proof}
\theoremstyle{plain} 	\newtheorem{The}{Theorem}
\theoremstyle{plain} 	\newtheorem*{ProThe}{Proof}
\theoremstyle{plain} 	\newtheorem{Prop}{Proposition}
\theoremstyle{plain} 	\newtheorem*{ProProp}{Proof}
\theoremstyle{plain} 	\newtheorem*{Conj}{Conjecture}
\theoremstyle{plain}	\newtheorem*{Rem}{Remark}
\theoremstyle{plain}	\newtheorem*{Def}{Definition} 
\theoremstyle{plain}	\newtheorem*{Not}{Notation}

% [uniform figure scaling - maybe this is not a good idea]
\def\figscale{.7}
\def\lscale{1.0}

% [FIX ME! - red makes it easier to spot]
\newcommand{\FIX}[1]{\textbf{\textcolor{red}{#1}}}


    \begin{document}
    \def\noprelim{}
\else
    % already included once
    % input post files

    \singlespacing
    \bibliographystyle{../bibliography/expanded}
    \bibliography{../bibliography/references}

    \end{document}
\fi\fi
% (place at start and end of chapter file)

\ifx\noprelim\undefined
    % first time included
    % input preamble files
    % [ USER VARIABLES ]

\def\PHDTITLE {Extensions of the Theory of Computational Mechanics}
\def\PHDAUTHOR{Evan Klose Friis}
\def\PHDSCHOOL{University of California, Davis}

\def\PHDMONTH {June}
\def\PHDYEAR  {2011}
\def\PHDDEPT {Physics}

\def\BSSCHOOL {University of California at San Diego}
\def\BSYEAR   {2005}

\def\PHDCOMMITTEEA{Professor John Conway}
\def\PHDCOMMITTEEB{Professor Robin Erbacher}
\def\PHDCOMMITTEEC{Professor Mani Tripathi}

% [ GLOBAL SETUP ]

\documentclass[letterpaper,oneside,11pt]{report}

\usepackage{calc}
\usepackage{breakcites}
\usepackage[newcommands]{ragged2e}

\usepackage[pdftex]{graphicx}
\usepackage{epstopdf}

%\usepackage{tikz}
%\usetikzlibrary{positioning} % [right=of ...]
%\usetikzlibrary{fit} % [fit= ...]

%\pgfdeclarelayer{background layer}
%\pgfdeclarelayer{foreground layer}
%\pgfsetlayers{background layer,main,foreground layer}

%\newenvironment{wrap}{\noindent\begin{minipage}[t]{\linewidth}\vspace{-0.5\normalbaselineskip}\centering}{\vspace{0.5\normalbaselineskip}\end{minipage}}

%% [Venn diagram environment]
%\newenvironment{venn2}
%{\begin{tikzpicture} [every pin/.style={text=black, text opacity=1.0, pin distance=0.5cm, pin edge={black!60, semithick}},
%% define a new style 'venn'
%venn/.style={circle, draw=black!60, semithick, minimum size = 4cm}]
%
%% create circle and give it external (pin) label
%\node[venn] (X) at (-1,0) [pin={150:$H[X]$}] {};
%\node[venn] (Y) at (1,0) [pin={30:$H[Y]$}] {};
%
%% place labels of the atoms by hand
%\node at (-1.9,0) {$H[X|Y]$};
%\node at (1.9,0) {$H[Y|X]$};
%\node at (0,0) {$I[X;Y]$};}
%{\end{tikzpicture}}

%\newcommand{\wrapmath}[1]{\begin{wrap}\begin{tikzpicture}[every node/.style={inner ysep=0ex, inner xsep=0em}]\node[] {$\displaystyle\begin{aligned} #1\end{aligned}$};\end{tikzpicture}\end{wrap}}

\renewenvironment{abstract}{\chapter*{Abstract}}{}
\renewcommand{\bibname}{Bibliography}
\renewcommand{\contentsname}{Table of Contents}

\makeatletter
\renewcommand{\@biblabel}[1]{\textsc{#1}}
\makeatother

% [ FONT SETTINGS ]

\usepackage[tbtags, intlimits, namelimits]{amsmath}
\usepackage[adobe-utopia]{mathdesign}

\DeclareSymbolFont{pazomath}{OMS}{zplm}{m}{n}
\DeclareSymbolFontAlphabet{\mathcal}{pazomath}
\SetMathAlphabet\mathcal{bold}{OMS}{zplm}{b}{n}

\SetSymbolFont{largesymbols}{normal}{OMX}{zplm}{m}{n}
\SetSymbolFont{largesymbols}{bold}{OMX}{zplm}{m}{n}
\SetSymbolFont{symbols}{normal}{OMS}{zplm}{m}{n}
\SetSymbolFont{symbols}{bold}{OMS}{zplm}{b}{n}

\renewcommand{\sfdefault}{phv}
\renewcommand{\ttdefault}{fvm}

\widowpenalty 8000
\clubpenalty  8000

% [ PAGE LAYOUT ]
\usepackage{geometry}
\geometry{lmargin = 1.5in}
\geometry{rmargin = 1.0in}
\geometry{tmargin = 1.0in}
\geometry{bmargin = 1.0in}

% [ PDF SETTINGS ]

\usepackage[final]{hyperref}
\hypersetup{breaklinks  = true}
\hypersetup{colorlinks  = true}
\hypersetup{linktocpage = false}
\hypersetup{linkcolor   = blue}
\hypersetup{citecolor   = green}
\hypersetup{urlcolor    = black}
\hypersetup{plainpages  = false}
\hypersetup{pageanchor  = true}
\hypersetup{pdfauthor   = {\PHDAUTHOR}}
\hypersetup{pdftitle    = {\PHDTITLE}}
\hypersetup{pdfsubject  = {Dissertation, \PHDSCHOOL}}
\urlstyle{same}

% [ LETTER SPACING ]

\usepackage[final]{microtype}
\microtypesetup{protrusion=compatibility}
\microtypesetup{expansion=false}

\newcommand{\upper}[1]{\MakeUppercase{#1}}
\let\lsscshape\scshape

\ifcase\pdfoutput\else\microtypesetup{letterspace=15}
\renewcommand{\scshape}{\lsscshape\lsstyle}
\renewcommand{\upper}[1]{\textls[50]{\MakeUppercase{#1}}}\fi

% [ LINE SPACING ]

\usepackage[doublespacing]{setspace}
\renewcommand{\displayskipstretch}{0.0}

\setlength{\parskip   }{0em}
\setlength{\parindent }{2em}

% [ TABLE FORMATTING ]

\usepackage{booktabs}
\setlength{\heavyrulewidth}{1.5\arrayrulewidth}
\setlength{\lightrulewidth}{1.0\arrayrulewidth}
\setlength{\doublerulesep }{2.0\arrayrulewidth}

% [ SECTION FORMATTING ]

\usepackage[largestsep,nobottomtitles*]{titlesec}
\renewcommand{\bottomtitlespace}{0.75in}

\titleformat{\chapter}[display]{\bfseries\huge\singlespacing}{\filleft\textsc{\LARGE \chaptertitlename\ \thechapter}}{-0.2ex}{\titlerule[3pt]\vspace{0.2ex}}[]

\titleformat{\section}{\LARGE}{\S\thesection\hspace{0.5em}}{0ex}{}
\titleformat{\subsection}{\Large}{\S\thesubsection\hspace{0.5em}}{0ex}{}
\titleformat{\subsubsection}{\large}{\thesubsubsection\hspace{0.5em}}{0ex}{}

\titlespacing*{\chapter}{0em}{6ex}{4ex plus 2ex minus 0ex}
\titlespacing*{\section}{0em}{2ex plus 3ex minus 1ex}{0.5ex plus 0.5ex minus 0.5ex}
\titlespacing*{\subsection}{0ex}{2ex plus 3ex minus 1ex}{0ex}
\titlespacing*{\subsubsection}{0ex}{2ex plus 0ex minus 1ex}{0ex}

% [ HEADER SETTINGS ]

\usepackage{fancyhdr}

\setlength{\headheight}{\normalbaselineskip}
\setlength{\footskip  }{0.5in}
\setlength{\headsep   }{0.5in-\headheight}

\fancyheadoffset[R]{0.5in}
\renewcommand{\headrulewidth}{0pt}
\renewcommand{\footrulewidth}{0pt}

\newcommand{\pagebox}{\parbox[r][\headheight][t]{0.5in}{\hspace\fill\thepage}}

\newcommand{\prelimheaders}{\ifx\prelim\undefined\renewcommand{\thepage}{\textit{\roman{page}}}\fancypagestyle{plain}{\fancyhf{}\fancyfoot[L]{\makebox[\textwidth-0.5in]{\thepage}}}\pagestyle{plain}\def\prelim{}\fi}

\newcommand{\normalheaders}{\renewcommand{\thepage}{\arabic{page}}\fancypagestyle{plain}{\fancyhf{}\fancyhead[R]{\pagebox}}\pagestyle{plain}}

\normalheaders{}

% [ CUSTOM COMMANDS ]

\newcommand{\signaturebox}[1]{\multicolumn{1}{p{4in}}{\vspace{3ex}}\\\midrule #1\\}

%%%% macros fro standard references
%\eqref provided by amsmath
\newcommand{\figref}[1]{Fig.~\ref{#1}}
\newcommand{\tableref}[1]{Table~\ref{#1}}
\newcommand{\refcite}[1]{Ref.~\cite{#1}}

% Abbreviations from CMPPSS:

\newcommand{\eM}     {\mbox{$\epsilon$-machine}}
\newcommand{\eMs}    {\mbox{$\epsilon$-machines}}
\newcommand{\EM}     {\mbox{$\epsilon$-Machine}}
\newcommand{\EMs}    {\mbox{$\epsilon$-Machines}}
\newcommand{\eT}     {\mbox{$\epsilon$-transducer}}
\newcommand{\eTs}    {\mbox{$\epsilon$-transducers}}
\newcommand{\ET}     {\mbox{$\epsilon$-Transducer}}
\newcommand{\ETs}    {\mbox{$\epsilon$-Transducers}}

% Processes and sequences

\newcommand{\Process}{\mathcal{P}}

\newcommand{\ProbMach}{\Prob_{\mathrm{M}}}
\newcommand{\Lmax}   { {L_{\mathrm{max}}}}
\newcommand{\MeasAlphabet}	{\mathcal{A}}
% Original
%\newcommand{\MeasSymbol}   { {S} }
%\newcommand{\meassymbol}   { {s} }
% New symbol
\newcommand{\MeasSymbol}   { {X} }
\newcommand{\meassymbol}   { {x} }
\newcommand{\BiInfinity}	{ \overleftrightarrow {\MeasSymbol} }
\newcommand{\biinfinity}	{ \overleftrightarrow {\meassymbol} }
\newcommand{\Past}	{ \overleftarrow {\MeasSymbol} }
\newcommand{\past}	{ {\overleftarrow {\meassymbol}} }
\newcommand{\pastprime}	{ {\past}^{\prime}}
\newcommand{\Future}	{ \overrightarrow{\MeasSymbol} }
\newcommand{\future}	{ \overrightarrow{\meassymbol} }
\newcommand{\futureprime}	{ {\future}^{\prime}}
\newcommand{\PastPrime}	{ {\Past}^{\prime}}
\newcommand{\FuturePrime}	{ {\overrightarrow{\meassymbol}}^\prime }
\newcommand{\PastDblPrime}	{ {\overleftarrow{\meassymbol}}^{\prime\prime} }
\newcommand{\FutureDblPrime}	{ {\overrightarrow{\meassymbol}}^{\prime\prime} }
\newcommand{\pastL}	{ {\overleftarrow {\meassymbol}}{}^L }
\newcommand{\PastL}	{ {\overleftarrow {\MeasSymbol}}{}^L }
\newcommand{\PastLt}	{ {\overleftarrow {\MeasSymbol}}_t^L }
\newcommand{\PastLLessOne}	{ {\overleftarrow {\MeasSymbol}}^{L-1} }
\newcommand{\futureL}	{ {\overrightarrow{\meassymbol}}{}^L }
\newcommand{\FutureL}	{ {\overrightarrow{\MeasSymbol}}{}^L }
\newcommand{\FutureLt}	{ {\overrightarrow{\MeasSymbol}}_t^L }
\newcommand{\FutureLLessOne}	{ {\overrightarrow{\MeasSymbol}}^{L-1} }
\newcommand{\pastLprime}	{ {\overleftarrow {\meassymbol}}^{L^\prime} }
\newcommand{\futureLprime}	{ {\overrightarrow{\meassymbol}}^{L^\prime} }
\newcommand{\AllPasts}	{ { \overleftarrow {\rm {\bf \MeasSymbol}} } }
\newcommand{\AllFutures}	{ \overrightarrow {\rm {\bf \MeasSymbol}} }
\newcommand{\FutureSet}	{ \overrightarrow{\bf \MeasSymbol}}

% Causal states and epsilon-machines
\newcommand{\CausalState}	{ \mathcal{S} }
\newcommand{\CausalStatePrime}	{ {\CausalState}^{\prime}}
\newcommand{\causalstate}	{ \sigma }
\newcommand{\CausalStateSet}	{ \boldsymbol{\CausalState} }
\newcommand{\AlternateState}	{ \mathcal{R} }
\newcommand{\AlternateStatePrime}	{ {\cal R}^{\prime} }
\newcommand{\alternatestate}	{ \rho }
\newcommand{\alternatestateprime}	{ {\rho^{\prime}} }
\newcommand{\AlternateStateSet}	{ \boldsymbol{\AlternateState} }
\newcommand{\PrescientState}	{ \widehat{\AlternateState} }
\newcommand{\prescientstate}	{ \widehat{\alternatestate} }
\newcommand{\PrescientStateSet}	{ \boldsymbol{\PrescientState}}
\newcommand{\CausalEquivalence}	{ {\sim}_{\epsilon} }
\newcommand{\CausalEquivalenceNot}	{ {\not \sim}_{\epsilon}}

\newcommand{\NonCausalEquivalence}	{ {\sim}_{\eta} }
\newcommand{\NextObservable}	{ {\overrightarrow {\MeasSymbol}}^1 }
\newcommand{\LastObservable}	{ {\overleftarrow {\MeasSymbol}}^1 }
%\newcommand{\Prob}		{ {\rm P}}
\newcommand{\Prob}      {\Pr} % use standard command
\newcommand{\ProbAnd}	{ {,\;} }
\newcommand{\LLimit}	{ {L \rightarrow \infty}}
\newcommand{\Cmu}		{C_\mu}
\newcommand{\hmu}		{h_\mu}
\newcommand{\EE}		{{\bf E}}
\newcommand{\Measurable}{{\bf \mu}}

% Process Crypticity
\newcommand{\PC}		{\chi}
\newcommand{\FuturePC}		{\PC^+}
\newcommand{\PastPC}		{\PC^-}
% Causal Irreversibility
\newcommand{\CI}		{\Xi}
\newcommand{\ReverseMap}	{r}
\newcommand{\ForwardMap}	{f}

% Abbreviations from IB:
% None that aren't already in CMPPSS

% Abbreviations from Extensive Estimation:
\newcommand{\EstCausalState}	{\widehat{\CausalState}}
\newcommand{\estcausalstate}	{\widehat{\causalstate}}
\newcommand{\EstCausalStateSet}	{\boldsymbol{\EstCausalState}}
\newcommand{\EstCausalFunc}	{\widehat{\epsilon}}
\newcommand{\EstCmu}		{\widehat{\Cmu}}
\newcommand{\PastLOne}	{{\Past}^{L+1}}
\newcommand{\pastLOne}	{{\past}^{L+1}}

% Abbreviations from $\epsilon$-Transducers:
\newcommand{\InAlphabet}	{ \mathcal{A}}
\newcommand{\insymbol}		{ a}
\newcommand{\OutAlphabet}	{ \mathcal{B}}
\newcommand{\outsymbol}		{ b}
\newcommand{\InputSimple}	{ X}
\newcommand{\inputsimple}	{ x}
\newcommand{\BottleneckVar}	{\tilde{\InputSimple}}
\newcommand{\bottleneckvar}	{\tilde{\inputsimple}}
\newcommand{\InputSpace}	{ \mathbf{\InputSimple}}
\newcommand{\InputBi}	{ \overleftrightarrow {\InputSimple} }
\newcommand{\inputbi}	{ \overleftrightarrow {\inputsimple} }
\newcommand{\InputPast}	{ \overleftarrow {\InputSimple} }
\newcommand{\inputpast}	{ \overleftarrow {\inputsimple} }
\newcommand{\InputFuture}	{ \overrightarrow {\InputSimple} }
\newcommand{\inputfuture}	{ \overrightarrow {\inputsimple} }
\newcommand{\NextInput}	{ {{\InputFuture}^{1}}}
\newcommand{\NextOutput}	{ {\OutputFuture}^{1}}
\newcommand{\OutputSimple}	{ Y}
\newcommand{\outputsimple}	{ y}
\newcommand{\OutputSpace}	{ \mathbf{\OutputSimple}}
\newcommand{\OutputBi}	{ \overleftrightarrow{\OutputSimple} }
\newcommand{\outputbi}	{ \overleftrightarrow{\outputsimple} }
\newcommand{\OutputPast}	{ \overleftarrow{\OutputSimple} }
\newcommand{\outputpast}	{ \overleftarrow{\outputsimple} }
\newcommand{\OutputFuture}	{ \overrightarrow{\OutputSimple} }
\newcommand{\outputfuture}	{ \overrightarrow{\outputsimple} }
\newcommand{\OutputL}	{ {\OutputFuture}^L}
\newcommand{\outputL}	{ {\outputfuture}^L}
\newcommand{\InputLLessOne}	{ {\InputFuture}^{L-1}}
\newcommand{\inputLlessone}	{ {\inputufutre}^{L-1}}
\newcommand{\OutputPastLLessOne}	{{\OutputPast}^{L-1}_{-1}}
\newcommand{\outputpastLlessone}	{{\outputpast}^{L-1}}
\newcommand{\OutputPastLessOne}	{{\OutputPast}_{-1}}
\newcommand{\outputpastlessone}	{{\outputpast}_{-1}}
\newcommand{\OutputPastL}	{{\OutputPast}^{L}}
\newcommand{\OutputLPlusOne}	{ {\OutputFuture}^{L+1}}
\newcommand{\outputLplusone}	{ {\outputfutre}^{L+1}}
\newcommand{\InputPastL}	{{\InputPast}^{L}}
\newcommand{\inputpastL}	{{\inputpast}^{L}}
\newcommand{\JointPast}	{{(\InputPast,\OutputPast)}}
\newcommand{\jointpast}	{{(\inputpast,\outputpast)}}
\newcommand{\jointpastone}	{{(\inputpast_1,\outputpast_1)}}
\newcommand{\jointpasttwo}	{{(\inputpast_2,\outputpast_2)}}
\newcommand{\jointpastprime} {{({\inputpast}^{\prime},{\outputpast}^{\prime})}}
\newcommand{\NextJoint}	{{(\NextInput,\NextOutput)}}
\newcommand{\nextjoint}	{{(\insymbol,\outsymbol)}}
\newcommand{\AllInputPasts}	{ { \overleftarrow {\rm \InputSpace}}}
\newcommand{\AllOutputPasts}	{ {\overleftarrow {\rm \OutputSpace}}}
\newcommand{\DetCausalState}	{ {{\cal S}_D }}
\newcommand{\detcausalstate}	{ {{\sigma}_D} }
\newcommand{\DetCausalStateSet}	{ \boldsymbol{{\CausalState}_D}}
\newcommand{\DetCausalEquivalence}	{ {\sim}_{{\epsilon}_{D}}}
\newcommand{\PrescientEquivalence}	{ {\sim}_{\widehat{\eta}}}
\newcommand{\FeedbackCausalState}	{ \mathcal{F}}
\newcommand{\feedbackcausalstate}	{ \phi}
\newcommand{\FeedbackCausalStateSet}	{ \mathbf{\FeedbackCausalState}}
\newcommand{\JointCausalState}		{ \mathcal{J}}
\newcommand{\JointCausalStateSet}	{ \mathbf{\JointCausalState}}
\newcommand{\UtilityFunctional}	{ {\mathcal{L}}}
\newcommand{\NatureState}	{ {\Omega}}
\newcommand{\naturestate}	{ {\omega}}
\newcommand{\NatureStateSpace}	{ {\mathbf{\NatureState}}}
\newcommand{\AnAction}	{ {A}}
\newcommand{\anaction}	{ {a}}
\newcommand{\ActionSpace}	{ {\mathbf{\AnAction}}}

% Abbreviations from RURO:
\newcommand{\InfoGain}[2] { \mathcal{D} \left( {#1} || {#2} \right) }

% Abbreviations from Upper Bound:
\newcommand{\lcm}	{{\rm lcm}}
% Double-check that this isn't in the math set already!

% Abbreviations from Emergence in Space
\newcommand{\ProcessAlphabet}	{\MeasAlphabet}
\newcommand{\ProbEst}			{ {\widehat{\Prob}_N}}
\newcommand{\STRegion}			{ {\mathrm K}}
\newcommand{\STRegionVariable}		{ K}
\newcommand{\stregionvariable}		{ k}
\newcommand{\GlobalPast}		{ \overleftarrow{G}} 
\newcommand{\globalpast}		{ \overleftarrow{g}} 
\newcommand{\GlobalFuture}		{ \overrightarrow{G}}
\newcommand{\globalfuture}		{ \overrightarrow{g}}
\newcommand{\GlobalState}		{ \mathcal{G}}
\newcommand{\globalstate}		{ \gamma}
\newcommand{\GlobalStateSet}		{ {\mathbf \GlobalState}}
\newcommand{\LocalPast}			{ \overleftarrow{L}} 
\newcommand{\localpast}			{ \overleftarrow{l}}
\newcommand{\AllLocalPasts}		{ \mathbf{\LocalPast}}
\newcommand{\LocalPastRegion}		{ \overleftarrow{\mathrm L}}
\newcommand{\LocalFuture}		{ \overrightarrow{L}}
\newcommand{\localfuture}		{ \overrightarrow{l}}
\newcommand{\LocalFutureRegion}		{ \overrightarrow{\mathrm L}}
\newcommand{\LocalState}		{ \mathcal{L}}
\newcommand{\localstate}		{ \lambda}
\newcommand{\LocalStateSet}		{ {\mathbf \LocalState}}
\newcommand{\PatchPast}			{ \overleftarrow{P}}
\newcommand{\patchpast}			{ \overleftarrow{p}}
\newcommand{\PatchPastRegion}		{ \overleftarrow{\mathrm P}}
\newcommand{\PatchFuture}		{ \overrightarrow{P}}
\newcommand{\patchfuture}		{ \overrightarrow{p}}
\newcommand{\PatchFutureRegion}		{ \overrightarrow{\mathrm P}}
\newcommand{\PatchState}		{ \mathcal{P}}
\newcommand{\patchstate}		{ \pi}
\newcommand{\PatchStateSet}		{ {\mathbf \PatchState}}
\newcommand{\LocalStatesInPatch}	{\vec{\LocalState}}
\newcommand{\localstatesinpatch}	{\vec{\localstate}}
\newcommand{\PointInstantX}		{ {\mathbf x}}
% Galles's original LaTeX for the cond. indep. symbol follows:
\newcommand{\compos}{\mbox{$~\underline{~\parallel~}~$}}
\newcommand{\ncompos}{\not\hspace{-.15in}\compos}
\newcommand{\indep}			{ \rotatebox{90}{$\models$}}
\newcommand{\nindep}	{\not\hspace{-.05in}\indep}
\newcommand{\LocalEE}	{{\EE}^{loc}}
\newcommand{\EEDensity}	{\overline{\LocalEE}}
\newcommand{\LocalCmu}	{{\Cmu}^{loc}}
\newcommand{\CmuDensity}	{\overline{\LocalCmu}}

%%%%%%%%%%% added by sasa
\newcommand{\FinPast}[1]	{ \overleftarrow {\MeasSymbol} \stackrel{{#1}}{}}
\newcommand{\finpast}[1]  	{ \overleftarrow {\meassymbol}  \stackrel{{#1}}{}}
\newcommand{\FinFuture}[1]		{ \overrightarrow{\MeasSymbol} \stackrel{{#1}}{}}
\newcommand{\finfuture}[1]		{ \overrightarrow{\meassymbol} \stackrel{{#1}}{}}

\newcommand{\Partition}	{ \AlternateState }
\newcommand{\partitionstate}	{ \alternatestate }
\newcommand{\PartitionSet}	{ \AlternateStateSet }
\newcommand{\Fdet}   { F_{\rm det} }

\newcommand{\Dkl}[2] { D_{\rm KL} \left( {#1} || {#2} \right) }

\newcommand{\Period}	{p}

% To take into account time direction
\newcommand{\forward}{+}
\newcommand{\reverse}{-}
%\newcommand{\forwardreverse}{\:\!\diamond} % \pm
\newcommand{\forwardreverse}{\pm} % \pm
\newcommand{\FutureProcess}	{ {\Process}^{\forward} }
\newcommand{\PastProcess}	{ {\Process}^{\reverse} }
\newcommand{\FutureCausalState}	{ {\CausalState}^{\forward} }
\newcommand{\futurecausalstate}	{ \sigma^{\forward} }
\newcommand{\altfuturecausalstate}	{ \sigma^{\forward\prime} }
\newcommand{\PastCausalState}	{ {\CausalState}^{\reverse} }
\newcommand{\pastcausalstate}	{ \sigma^{\reverse} }
\newcommand{\BiCausalState}		{ {\CausalState}^{\forwardreverse} }
\newcommand{\bicausalstate}		{ {\sigma}^{\forwardreverse} }
\newcommand{\FutureCausalStateSet}	{ {\CausalStateSet}^{\forward} }
\newcommand{\PastCausalStateSet}	{ {\CausalStateSet}^{\reverse} }
\newcommand{\BiCausalStateSet}	{ {\CausalStateSet}^{\forwardreverse} }
\newcommand{\eMachine}	{ M }
\newcommand{\FutureEM}	{ {\eMachine}^{\forward} }
\newcommand{\PastEM}	{ {\eMachine}^{\reverse} }
\newcommand{\BiEM}		{ {\eMachine}^{\forwardreverse} }
\newcommand{\BiEquiv}	{ {\sim}^{\forwardreverse} }
\newcommand{\Futurehmu}	{ h_\mu^{\forward} }
\newcommand{\Pasthmu}	{ h_\mu^{\reverse} }
\newcommand{\FutureCmu}	{ C_\mu^{\forward} }
\newcommand{\PastCmu}	{ C_\mu^{\reverse} }
\newcommand{\BiCmu}		{ C_\mu^{\forwardreverse} }
\newcommand{\FutureEps}	{ \epsilon^{\forward} }
\newcommand{\PastEps}	{ \epsilon^{\reverse} }
\newcommand{\BiEps}	{ \epsilon^{\forwardreverse} }
\newcommand{\FutureSim}	{ \sim^{\forward} }
\newcommand{\PastSim}	{ \sim^{\reverse} }
% Used arrows for awhile, more or less confusing?
%\newcommand{\FutureCausalState}	{ \overrightarrow{\CausalState} }
%\newcommand{\PastCausalState}	{ \overleftarrow{\CausalState} }
%\newcommand{\eMachine}	{ M }
%\newcommand{\FutureEM}	{ \overrightarrow{\eMachine} }
%\newcommand{\PastEM}	{ \overleftarrow{\eMachine} }
%\newcommand{\FutureCmu}	{ \overrightarrow{\Cmu} }
%\newcommand{\PastCmu}	{ \overleftarrow{\Cmu} }

%% time-reversing and mixed state presentation operators
\newcommand{\TR}{\mathcal{T}}
\newcommand{\MSP}{\mathcal{U}}
\newcommand{\one}{\mathbf{1}}

%% (cje)
%% Provide a command \ifpm which is true when \pm 
%% is meant to be understood as "+ or -". This is
%% different from the usage in TBA.
\newif\ifpm 
\edef\tempa{\forwardreverse}
\edef\tempb{\pm}
\ifx\tempa\tempb
   \pmfalse
\else
   \pmtrue  
\fi





% [some math stuff - maybe stick in sep file]
\usepackage{amsthm}
\usepackage{amscd}
\theoremstyle{plain}    \newtheorem{Lem}{Lemma}
\theoremstyle{plain}    \newtheorem*{ProLem}{Proof}
\theoremstyle{plain} 	\newtheorem{Cor}{Corollary}
\theoremstyle{plain} 	\newtheorem*{ProCor}{Proof}
\theoremstyle{plain} 	\newtheorem{The}{Theorem}
\theoremstyle{plain} 	\newtheorem*{ProThe}{Proof}
\theoremstyle{plain} 	\newtheorem{Prop}{Proposition}
\theoremstyle{plain} 	\newtheorem*{ProProp}{Proof}
\theoremstyle{plain} 	\newtheorem*{Conj}{Conjecture}
\theoremstyle{plain}	\newtheorem*{Rem}{Remark}
\theoremstyle{plain}	\newtheorem*{Def}{Definition} 
\theoremstyle{plain}	\newtheorem*{Not}{Notation}

% [uniform figure scaling - maybe this is not a good idea]
\def\figscale{.7}
\def\lscale{1.0}

% [FIX ME! - red makes it easier to spot]
\newcommand{\FIX}[1]{\textbf{\textcolor{red}{#1}}}


    \begin{document}
    \def\noprelim{}
\else
    % already included once
    % input post files

    \singlespacing
    \bibliographystyle{../bibliography/expanded}
    \bibliography{../bibliography/references}

    \end{document}
\fi\fi
