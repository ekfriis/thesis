\ifx\master\undefined% AUTOCOMPILE
% Allows for individual chapters to be compiled.

% Usage:
% \ifx\master\undefined% AUTOCOMPILE
% Allows for individual chapters to be compiled.

% Usage:
% \ifx\master\undefined% AUTOCOMPILE
% Allows for individual chapters to be compiled.

% Usage:
% \ifx\master\undefined\input{../settings/autocompile}\fi
% (place at start and end of chapter file)

\ifx\noprelim\undefined
    % first time included
    % input preamble files
    \input{../settings/phdsetup}

    \begin{document}
    \def\noprelim{}
\else
    % already included once
    % input post files

    \singlespacing
    \bibliographystyle{../bibliography/expanded}
    \bibliography{../bibliography/references}

    \end{document}
\fi\fi
% (place at start and end of chapter file)

\ifx\noprelim\undefined
    % first time included
    % input preamble files
    % [ USER VARIABLES ]

\def\PHDTITLE {Extensions of the Theory of Computational Mechanics}
\def\PHDAUTHOR{Evan Klose Friis}
\def\PHDSCHOOL{University of California, Davis}

\def\PHDMONTH {June}
\def\PHDYEAR  {2011}
\def\PHDDEPT {Physics}

\def\BSSCHOOL {University of California at San Diego}
\def\BSYEAR   {2005}

\def\PHDCOMMITTEEA{Professor John Conway}
\def\PHDCOMMITTEEB{Professor Robin Erbacher}
\def\PHDCOMMITTEEC{Professor Mani Tripathi}

% [ GLOBAL SETUP ]

\documentclass[letterpaper,oneside,11pt]{report}

\usepackage{calc}
\usepackage{breakcites}
\usepackage[newcommands]{ragged2e}

\usepackage[pdftex]{graphicx}
\usepackage{epstopdf}

%\usepackage{tikz}
%\usetikzlibrary{positioning} % [right=of ...]
%\usetikzlibrary{fit} % [fit= ...]

%\pgfdeclarelayer{background layer}
%\pgfdeclarelayer{foreground layer}
%\pgfsetlayers{background layer,main,foreground layer}

%\newenvironment{wrap}{\noindent\begin{minipage}[t]{\linewidth}\vspace{-0.5\normalbaselineskip}\centering}{\vspace{0.5\normalbaselineskip}\end{minipage}}

%% [Venn diagram environment]
%\newenvironment{venn2}
%{\begin{tikzpicture} [every pin/.style={text=black, text opacity=1.0, pin distance=0.5cm, pin edge={black!60, semithick}},
%% define a new style 'venn'
%venn/.style={circle, draw=black!60, semithick, minimum size = 4cm}]
%
%% create circle and give it external (pin) label
%\node[venn] (X) at (-1,0) [pin={150:$H[X]$}] {};
%\node[venn] (Y) at (1,0) [pin={30:$H[Y]$}] {};
%
%% place labels of the atoms by hand
%\node at (-1.9,0) {$H[X|Y]$};
%\node at (1.9,0) {$H[Y|X]$};
%\node at (0,0) {$I[X;Y]$};}
%{\end{tikzpicture}}

%\newcommand{\wrapmath}[1]{\begin{wrap}\begin{tikzpicture}[every node/.style={inner ysep=0ex, inner xsep=0em}]\node[] {$\displaystyle\begin{aligned} #1\end{aligned}$};\end{tikzpicture}\end{wrap}}

\renewenvironment{abstract}{\chapter*{Abstract}}{}
\renewcommand{\bibname}{Bibliography}
\renewcommand{\contentsname}{Table of Contents}

\makeatletter
\renewcommand{\@biblabel}[1]{\textsc{#1}}
\makeatother

% [ FONT SETTINGS ]

\usepackage[tbtags, intlimits, namelimits]{amsmath}
\usepackage[adobe-utopia]{mathdesign}

\DeclareSymbolFont{pazomath}{OMS}{zplm}{m}{n}
\DeclareSymbolFontAlphabet{\mathcal}{pazomath}
\SetMathAlphabet\mathcal{bold}{OMS}{zplm}{b}{n}

\SetSymbolFont{largesymbols}{normal}{OMX}{zplm}{m}{n}
\SetSymbolFont{largesymbols}{bold}{OMX}{zplm}{m}{n}
\SetSymbolFont{symbols}{normal}{OMS}{zplm}{m}{n}
\SetSymbolFont{symbols}{bold}{OMS}{zplm}{b}{n}

\renewcommand{\sfdefault}{phv}
\renewcommand{\ttdefault}{fvm}

\widowpenalty 8000
\clubpenalty  8000

% [ PAGE LAYOUT ]
\usepackage{geometry}
\geometry{lmargin = 1.5in}
\geometry{rmargin = 1.0in}
\geometry{tmargin = 1.0in}
\geometry{bmargin = 1.0in}

% [ PDF SETTINGS ]

\usepackage[final]{hyperref}
\hypersetup{breaklinks  = true}
\hypersetup{colorlinks  = true}
\hypersetup{linktocpage = false}
\hypersetup{linkcolor   = blue}
\hypersetup{citecolor   = green}
\hypersetup{urlcolor    = black}
\hypersetup{plainpages  = false}
\hypersetup{pageanchor  = true}
\hypersetup{pdfauthor   = {\PHDAUTHOR}}
\hypersetup{pdftitle    = {\PHDTITLE}}
\hypersetup{pdfsubject  = {Dissertation, \PHDSCHOOL}}
\urlstyle{same}

% [ LETTER SPACING ]

\usepackage[final]{microtype}
\microtypesetup{protrusion=compatibility}
\microtypesetup{expansion=false}

\newcommand{\upper}[1]{\MakeUppercase{#1}}
\let\lsscshape\scshape

\ifcase\pdfoutput\else\microtypesetup{letterspace=15}
\renewcommand{\scshape}{\lsscshape\lsstyle}
\renewcommand{\upper}[1]{\textls[50]{\MakeUppercase{#1}}}\fi

% [ LINE SPACING ]

\usepackage[doublespacing]{setspace}
\renewcommand{\displayskipstretch}{0.0}

\setlength{\parskip   }{0em}
\setlength{\parindent }{2em}

% [ TABLE FORMATTING ]

\usepackage{booktabs}
\setlength{\heavyrulewidth}{1.5\arrayrulewidth}
\setlength{\lightrulewidth}{1.0\arrayrulewidth}
\setlength{\doublerulesep }{2.0\arrayrulewidth}

% [ SECTION FORMATTING ]

\usepackage[largestsep,nobottomtitles*]{titlesec}
\renewcommand{\bottomtitlespace}{0.75in}

\titleformat{\chapter}[display]{\bfseries\huge\singlespacing}{\filleft\textsc{\LARGE \chaptertitlename\ \thechapter}}{-0.2ex}{\titlerule[3pt]\vspace{0.2ex}}[]

\titleformat{\section}{\LARGE}{\S\thesection\hspace{0.5em}}{0ex}{}
\titleformat{\subsection}{\Large}{\S\thesubsection\hspace{0.5em}}{0ex}{}
\titleformat{\subsubsection}{\large}{\thesubsubsection\hspace{0.5em}}{0ex}{}

\titlespacing*{\chapter}{0em}{6ex}{4ex plus 2ex minus 0ex}
\titlespacing*{\section}{0em}{2ex plus 3ex minus 1ex}{0.5ex plus 0.5ex minus 0.5ex}
\titlespacing*{\subsection}{0ex}{2ex plus 3ex minus 1ex}{0ex}
\titlespacing*{\subsubsection}{0ex}{2ex plus 0ex minus 1ex}{0ex}

% [ HEADER SETTINGS ]

\usepackage{fancyhdr}

\setlength{\headheight}{\normalbaselineskip}
\setlength{\footskip  }{0.5in}
\setlength{\headsep   }{0.5in-\headheight}

\fancyheadoffset[R]{0.5in}
\renewcommand{\headrulewidth}{0pt}
\renewcommand{\footrulewidth}{0pt}

\newcommand{\pagebox}{\parbox[r][\headheight][t]{0.5in}{\hspace\fill\thepage}}

\newcommand{\prelimheaders}{\ifx\prelim\undefined\renewcommand{\thepage}{\textit{\roman{page}}}\fancypagestyle{plain}{\fancyhf{}\fancyfoot[L]{\makebox[\textwidth-0.5in]{\thepage}}}\pagestyle{plain}\def\prelim{}\fi}

\newcommand{\normalheaders}{\renewcommand{\thepage}{\arabic{page}}\fancypagestyle{plain}{\fancyhf{}\fancyhead[R]{\pagebox}}\pagestyle{plain}}

\normalheaders{}

% [ CUSTOM COMMANDS ]

\newcommand{\signaturebox}[1]{\multicolumn{1}{p{4in}}{\vspace{3ex}}\\\midrule #1\\}

%\input{../includes/cmechabbrev}

% [some math stuff - maybe stick in sep file]
\usepackage{amsthm}
\usepackage{amscd}
\theoremstyle{plain}    \newtheorem{Lem}{Lemma}
\theoremstyle{plain}    \newtheorem*{ProLem}{Proof}
\theoremstyle{plain} 	\newtheorem{Cor}{Corollary}
\theoremstyle{plain} 	\newtheorem*{ProCor}{Proof}
\theoremstyle{plain} 	\newtheorem{The}{Theorem}
\theoremstyle{plain} 	\newtheorem*{ProThe}{Proof}
\theoremstyle{plain} 	\newtheorem{Prop}{Proposition}
\theoremstyle{plain} 	\newtheorem*{ProProp}{Proof}
\theoremstyle{plain} 	\newtheorem*{Conj}{Conjecture}
\theoremstyle{plain}	\newtheorem*{Rem}{Remark}
\theoremstyle{plain}	\newtheorem*{Def}{Definition} 
\theoremstyle{plain}	\newtheorem*{Not}{Notation}

% [uniform figure scaling - maybe this is not a good idea]
\def\figscale{.7}
\def\lscale{1.0}

% [FIX ME! - red makes it easier to spot]
\newcommand{\FIX}[1]{\textbf{\textcolor{red}{#1}}}


    \begin{document}
    \def\noprelim{}
\else
    % already included once
    % input post files

    \singlespacing
    \bibliographystyle{../bibliography/expanded}
    \bibliography{../bibliography/references}

    \end{document}
\fi\fi
% (place at start and end of chapter file)

\ifx\noprelim\undefined
    % first time included
    % input preamble files
    % [ USER VARIABLES ]

\def\PHDTITLE {Extensions of the Theory of Computational Mechanics}
\def\PHDAUTHOR{Evan Klose Friis}
\def\PHDSCHOOL{University of California, Davis}

\def\PHDMONTH {June}
\def\PHDYEAR  {2011}
\def\PHDDEPT {Physics}

\def\BSSCHOOL {University of California at San Diego}
\def\BSYEAR   {2005}

\def\PHDCOMMITTEEA{Professor John Conway}
\def\PHDCOMMITTEEB{Professor Robin Erbacher}
\def\PHDCOMMITTEEC{Professor Mani Tripathi}

% [ GLOBAL SETUP ]

\documentclass[letterpaper,oneside,11pt]{report}

\usepackage{calc}
\usepackage{breakcites}
\usepackage[newcommands]{ragged2e}

\usepackage[pdftex]{graphicx}
\usepackage{epstopdf}

%\usepackage{tikz}
%\usetikzlibrary{positioning} % [right=of ...]
%\usetikzlibrary{fit} % [fit= ...]

%\pgfdeclarelayer{background layer}
%\pgfdeclarelayer{foreground layer}
%\pgfsetlayers{background layer,main,foreground layer}

%\newenvironment{wrap}{\noindent\begin{minipage}[t]{\linewidth}\vspace{-0.5\normalbaselineskip}\centering}{\vspace{0.5\normalbaselineskip}\end{minipage}}

%% [Venn diagram environment]
%\newenvironment{venn2}
%{\begin{tikzpicture} [every pin/.style={text=black, text opacity=1.0, pin distance=0.5cm, pin edge={black!60, semithick}},
%% define a new style 'venn'
%venn/.style={circle, draw=black!60, semithick, minimum size = 4cm}]
%
%% create circle and give it external (pin) label
%\node[venn] (X) at (-1,0) [pin={150:$H[X]$}] {};
%\node[venn] (Y) at (1,0) [pin={30:$H[Y]$}] {};
%
%% place labels of the atoms by hand
%\node at (-1.9,0) {$H[X|Y]$};
%\node at (1.9,0) {$H[Y|X]$};
%\node at (0,0) {$I[X;Y]$};}
%{\end{tikzpicture}}

%\newcommand{\wrapmath}[1]{\begin{wrap}\begin{tikzpicture}[every node/.style={inner ysep=0ex, inner xsep=0em}]\node[] {$\displaystyle\begin{aligned} #1\end{aligned}$};\end{tikzpicture}\end{wrap}}

\renewenvironment{abstract}{\chapter*{Abstract}}{}
\renewcommand{\bibname}{Bibliography}
\renewcommand{\contentsname}{Table of Contents}

\makeatletter
\renewcommand{\@biblabel}[1]{\textsc{#1}}
\makeatother

% [ FONT SETTINGS ]

\usepackage[tbtags, intlimits, namelimits]{amsmath}
\usepackage[adobe-utopia]{mathdesign}

\DeclareSymbolFont{pazomath}{OMS}{zplm}{m}{n}
\DeclareSymbolFontAlphabet{\mathcal}{pazomath}
\SetMathAlphabet\mathcal{bold}{OMS}{zplm}{b}{n}

\SetSymbolFont{largesymbols}{normal}{OMX}{zplm}{m}{n}
\SetSymbolFont{largesymbols}{bold}{OMX}{zplm}{m}{n}
\SetSymbolFont{symbols}{normal}{OMS}{zplm}{m}{n}
\SetSymbolFont{symbols}{bold}{OMS}{zplm}{b}{n}

\renewcommand{\sfdefault}{phv}
\renewcommand{\ttdefault}{fvm}

\widowpenalty 8000
\clubpenalty  8000

% [ PAGE LAYOUT ]
\usepackage{geometry}
\geometry{lmargin = 1.5in}
\geometry{rmargin = 1.0in}
\geometry{tmargin = 1.0in}
\geometry{bmargin = 1.0in}

% [ PDF SETTINGS ]

\usepackage[final]{hyperref}
\hypersetup{breaklinks  = true}
\hypersetup{colorlinks  = true}
\hypersetup{linktocpage = false}
\hypersetup{linkcolor   = blue}
\hypersetup{citecolor   = green}
\hypersetup{urlcolor    = black}
\hypersetup{plainpages  = false}
\hypersetup{pageanchor  = true}
\hypersetup{pdfauthor   = {\PHDAUTHOR}}
\hypersetup{pdftitle    = {\PHDTITLE}}
\hypersetup{pdfsubject  = {Dissertation, \PHDSCHOOL}}
\urlstyle{same}

% [ LETTER SPACING ]

\usepackage[final]{microtype}
\microtypesetup{protrusion=compatibility}
\microtypesetup{expansion=false}

\newcommand{\upper}[1]{\MakeUppercase{#1}}
\let\lsscshape\scshape

\ifcase\pdfoutput\else\microtypesetup{letterspace=15}
\renewcommand{\scshape}{\lsscshape\lsstyle}
\renewcommand{\upper}[1]{\textls[50]{\MakeUppercase{#1}}}\fi

% [ LINE SPACING ]

\usepackage[doublespacing]{setspace}
\renewcommand{\displayskipstretch}{0.0}

\setlength{\parskip   }{0em}
\setlength{\parindent }{2em}

% [ TABLE FORMATTING ]

\usepackage{booktabs}
\setlength{\heavyrulewidth}{1.5\arrayrulewidth}
\setlength{\lightrulewidth}{1.0\arrayrulewidth}
\setlength{\doublerulesep }{2.0\arrayrulewidth}

% [ SECTION FORMATTING ]

\usepackage[largestsep,nobottomtitles*]{titlesec}
\renewcommand{\bottomtitlespace}{0.75in}

\titleformat{\chapter}[display]{\bfseries\huge\singlespacing}{\filleft\textsc{\LARGE \chaptertitlename\ \thechapter}}{-0.2ex}{\titlerule[3pt]\vspace{0.2ex}}[]

\titleformat{\section}{\LARGE}{\S\thesection\hspace{0.5em}}{0ex}{}
\titleformat{\subsection}{\Large}{\S\thesubsection\hspace{0.5em}}{0ex}{}
\titleformat{\subsubsection}{\large}{\thesubsubsection\hspace{0.5em}}{0ex}{}

\titlespacing*{\chapter}{0em}{6ex}{4ex plus 2ex minus 0ex}
\titlespacing*{\section}{0em}{2ex plus 3ex minus 1ex}{0.5ex plus 0.5ex minus 0.5ex}
\titlespacing*{\subsection}{0ex}{2ex plus 3ex minus 1ex}{0ex}
\titlespacing*{\subsubsection}{0ex}{2ex plus 0ex minus 1ex}{0ex}

% [ HEADER SETTINGS ]

\usepackage{fancyhdr}

\setlength{\headheight}{\normalbaselineskip}
\setlength{\footskip  }{0.5in}
\setlength{\headsep   }{0.5in-\headheight}

\fancyheadoffset[R]{0.5in}
\renewcommand{\headrulewidth}{0pt}
\renewcommand{\footrulewidth}{0pt}

\newcommand{\pagebox}{\parbox[r][\headheight][t]{0.5in}{\hspace\fill\thepage}}

\newcommand{\prelimheaders}{\ifx\prelim\undefined\renewcommand{\thepage}{\textit{\roman{page}}}\fancypagestyle{plain}{\fancyhf{}\fancyfoot[L]{\makebox[\textwidth-0.5in]{\thepage}}}\pagestyle{plain}\def\prelim{}\fi}

\newcommand{\normalheaders}{\renewcommand{\thepage}{\arabic{page}}\fancypagestyle{plain}{\fancyhf{}\fancyhead[R]{\pagebox}}\pagestyle{plain}}

\normalheaders{}

% [ CUSTOM COMMANDS ]

\newcommand{\signaturebox}[1]{\multicolumn{1}{p{4in}}{\vspace{3ex}}\\\midrule #1\\}

%%%% macros fro standard references
%\eqref provided by amsmath
\newcommand{\figref}[1]{Fig.~\ref{#1}}
\newcommand{\tableref}[1]{Table~\ref{#1}}
\newcommand{\refcite}[1]{Ref.~\cite{#1}}

% Abbreviations from CMPPSS:

\newcommand{\eM}     {\mbox{$\epsilon$-machine}}
\newcommand{\eMs}    {\mbox{$\epsilon$-machines}}
\newcommand{\EM}     {\mbox{$\epsilon$-Machine}}
\newcommand{\EMs}    {\mbox{$\epsilon$-Machines}}
\newcommand{\eT}     {\mbox{$\epsilon$-transducer}}
\newcommand{\eTs}    {\mbox{$\epsilon$-transducers}}
\newcommand{\ET}     {\mbox{$\epsilon$-Transducer}}
\newcommand{\ETs}    {\mbox{$\epsilon$-Transducers}}

% Processes and sequences

\newcommand{\Process}{\mathcal{P}}

\newcommand{\ProbMach}{\Prob_{\mathrm{M}}}
\newcommand{\Lmax}   { {L_{\mathrm{max}}}}
\newcommand{\MeasAlphabet}	{\mathcal{A}}
% Original
%\newcommand{\MeasSymbol}   { {S} }
%\newcommand{\meassymbol}   { {s} }
% New symbol
\newcommand{\MeasSymbol}   { {X} }
\newcommand{\meassymbol}   { {x} }
\newcommand{\BiInfinity}	{ \overleftrightarrow {\MeasSymbol} }
\newcommand{\biinfinity}	{ \overleftrightarrow {\meassymbol} }
\newcommand{\Past}	{ \overleftarrow {\MeasSymbol} }
\newcommand{\past}	{ {\overleftarrow {\meassymbol}} }
\newcommand{\pastprime}	{ {\past}^{\prime}}
\newcommand{\Future}	{ \overrightarrow{\MeasSymbol} }
\newcommand{\future}	{ \overrightarrow{\meassymbol} }
\newcommand{\futureprime}	{ {\future}^{\prime}}
\newcommand{\PastPrime}	{ {\Past}^{\prime}}
\newcommand{\FuturePrime}	{ {\overrightarrow{\meassymbol}}^\prime }
\newcommand{\PastDblPrime}	{ {\overleftarrow{\meassymbol}}^{\prime\prime} }
\newcommand{\FutureDblPrime}	{ {\overrightarrow{\meassymbol}}^{\prime\prime} }
\newcommand{\pastL}	{ {\overleftarrow {\meassymbol}}{}^L }
\newcommand{\PastL}	{ {\overleftarrow {\MeasSymbol}}{}^L }
\newcommand{\PastLt}	{ {\overleftarrow {\MeasSymbol}}_t^L }
\newcommand{\PastLLessOne}	{ {\overleftarrow {\MeasSymbol}}^{L-1} }
\newcommand{\futureL}	{ {\overrightarrow{\meassymbol}}{}^L }
\newcommand{\FutureL}	{ {\overrightarrow{\MeasSymbol}}{}^L }
\newcommand{\FutureLt}	{ {\overrightarrow{\MeasSymbol}}_t^L }
\newcommand{\FutureLLessOne}	{ {\overrightarrow{\MeasSymbol}}^{L-1} }
\newcommand{\pastLprime}	{ {\overleftarrow {\meassymbol}}^{L^\prime} }
\newcommand{\futureLprime}	{ {\overrightarrow{\meassymbol}}^{L^\prime} }
\newcommand{\AllPasts}	{ { \overleftarrow {\rm {\bf \MeasSymbol}} } }
\newcommand{\AllFutures}	{ \overrightarrow {\rm {\bf \MeasSymbol}} }
\newcommand{\FutureSet}	{ \overrightarrow{\bf \MeasSymbol}}

% Causal states and epsilon-machines
\newcommand{\CausalState}	{ \mathcal{S} }
\newcommand{\CausalStatePrime}	{ {\CausalState}^{\prime}}
\newcommand{\causalstate}	{ \sigma }
\newcommand{\CausalStateSet}	{ \boldsymbol{\CausalState} }
\newcommand{\AlternateState}	{ \mathcal{R} }
\newcommand{\AlternateStatePrime}	{ {\cal R}^{\prime} }
\newcommand{\alternatestate}	{ \rho }
\newcommand{\alternatestateprime}	{ {\rho^{\prime}} }
\newcommand{\AlternateStateSet}	{ \boldsymbol{\AlternateState} }
\newcommand{\PrescientState}	{ \widehat{\AlternateState} }
\newcommand{\prescientstate}	{ \widehat{\alternatestate} }
\newcommand{\PrescientStateSet}	{ \boldsymbol{\PrescientState}}
\newcommand{\CausalEquivalence}	{ {\sim}_{\epsilon} }
\newcommand{\CausalEquivalenceNot}	{ {\not \sim}_{\epsilon}}

\newcommand{\NonCausalEquivalence}	{ {\sim}_{\eta} }
\newcommand{\NextObservable}	{ {\overrightarrow {\MeasSymbol}}^1 }
\newcommand{\LastObservable}	{ {\overleftarrow {\MeasSymbol}}^1 }
%\newcommand{\Prob}		{ {\rm P}}
\newcommand{\Prob}      {\Pr} % use standard command
\newcommand{\ProbAnd}	{ {,\;} }
\newcommand{\LLimit}	{ {L \rightarrow \infty}}
\newcommand{\Cmu}		{C_\mu}
\newcommand{\hmu}		{h_\mu}
\newcommand{\EE}		{{\bf E}}
\newcommand{\Measurable}{{\bf \mu}}

% Process Crypticity
\newcommand{\PC}		{\chi}
\newcommand{\FuturePC}		{\PC^+}
\newcommand{\PastPC}		{\PC^-}
% Causal Irreversibility
\newcommand{\CI}		{\Xi}
\newcommand{\ReverseMap}	{r}
\newcommand{\ForwardMap}	{f}

% Abbreviations from IB:
% None that aren't already in CMPPSS

% Abbreviations from Extensive Estimation:
\newcommand{\EstCausalState}	{\widehat{\CausalState}}
\newcommand{\estcausalstate}	{\widehat{\causalstate}}
\newcommand{\EstCausalStateSet}	{\boldsymbol{\EstCausalState}}
\newcommand{\EstCausalFunc}	{\widehat{\epsilon}}
\newcommand{\EstCmu}		{\widehat{\Cmu}}
\newcommand{\PastLOne}	{{\Past}^{L+1}}
\newcommand{\pastLOne}	{{\past}^{L+1}}

% Abbreviations from $\epsilon$-Transducers:
\newcommand{\InAlphabet}	{ \mathcal{A}}
\newcommand{\insymbol}		{ a}
\newcommand{\OutAlphabet}	{ \mathcal{B}}
\newcommand{\outsymbol}		{ b}
\newcommand{\InputSimple}	{ X}
\newcommand{\inputsimple}	{ x}
\newcommand{\BottleneckVar}	{\tilde{\InputSimple}}
\newcommand{\bottleneckvar}	{\tilde{\inputsimple}}
\newcommand{\InputSpace}	{ \mathbf{\InputSimple}}
\newcommand{\InputBi}	{ \overleftrightarrow {\InputSimple} }
\newcommand{\inputbi}	{ \overleftrightarrow {\inputsimple} }
\newcommand{\InputPast}	{ \overleftarrow {\InputSimple} }
\newcommand{\inputpast}	{ \overleftarrow {\inputsimple} }
\newcommand{\InputFuture}	{ \overrightarrow {\InputSimple} }
\newcommand{\inputfuture}	{ \overrightarrow {\inputsimple} }
\newcommand{\NextInput}	{ {{\InputFuture}^{1}}}
\newcommand{\NextOutput}	{ {\OutputFuture}^{1}}
\newcommand{\OutputSimple}	{ Y}
\newcommand{\outputsimple}	{ y}
\newcommand{\OutputSpace}	{ \mathbf{\OutputSimple}}
\newcommand{\OutputBi}	{ \overleftrightarrow{\OutputSimple} }
\newcommand{\outputbi}	{ \overleftrightarrow{\outputsimple} }
\newcommand{\OutputPast}	{ \overleftarrow{\OutputSimple} }
\newcommand{\outputpast}	{ \overleftarrow{\outputsimple} }
\newcommand{\OutputFuture}	{ \overrightarrow{\OutputSimple} }
\newcommand{\outputfuture}	{ \overrightarrow{\outputsimple} }
\newcommand{\OutputL}	{ {\OutputFuture}^L}
\newcommand{\outputL}	{ {\outputfuture}^L}
\newcommand{\InputLLessOne}	{ {\InputFuture}^{L-1}}
\newcommand{\inputLlessone}	{ {\inputufutre}^{L-1}}
\newcommand{\OutputPastLLessOne}	{{\OutputPast}^{L-1}_{-1}}
\newcommand{\outputpastLlessone}	{{\outputpast}^{L-1}}
\newcommand{\OutputPastLessOne}	{{\OutputPast}_{-1}}
\newcommand{\outputpastlessone}	{{\outputpast}_{-1}}
\newcommand{\OutputPastL}	{{\OutputPast}^{L}}
\newcommand{\OutputLPlusOne}	{ {\OutputFuture}^{L+1}}
\newcommand{\outputLplusone}	{ {\outputfutre}^{L+1}}
\newcommand{\InputPastL}	{{\InputPast}^{L}}
\newcommand{\inputpastL}	{{\inputpast}^{L}}
\newcommand{\JointPast}	{{(\InputPast,\OutputPast)}}
\newcommand{\jointpast}	{{(\inputpast,\outputpast)}}
\newcommand{\jointpastone}	{{(\inputpast_1,\outputpast_1)}}
\newcommand{\jointpasttwo}	{{(\inputpast_2,\outputpast_2)}}
\newcommand{\jointpastprime} {{({\inputpast}^{\prime},{\outputpast}^{\prime})}}
\newcommand{\NextJoint}	{{(\NextInput,\NextOutput)}}
\newcommand{\nextjoint}	{{(\insymbol,\outsymbol)}}
\newcommand{\AllInputPasts}	{ { \overleftarrow {\rm \InputSpace}}}
\newcommand{\AllOutputPasts}	{ {\overleftarrow {\rm \OutputSpace}}}
\newcommand{\DetCausalState}	{ {{\cal S}_D }}
\newcommand{\detcausalstate}	{ {{\sigma}_D} }
\newcommand{\DetCausalStateSet}	{ \boldsymbol{{\CausalState}_D}}
\newcommand{\DetCausalEquivalence}	{ {\sim}_{{\epsilon}_{D}}}
\newcommand{\PrescientEquivalence}	{ {\sim}_{\widehat{\eta}}}
\newcommand{\FeedbackCausalState}	{ \mathcal{F}}
\newcommand{\feedbackcausalstate}	{ \phi}
\newcommand{\FeedbackCausalStateSet}	{ \mathbf{\FeedbackCausalState}}
\newcommand{\JointCausalState}		{ \mathcal{J}}
\newcommand{\JointCausalStateSet}	{ \mathbf{\JointCausalState}}
\newcommand{\UtilityFunctional}	{ {\mathcal{L}}}
\newcommand{\NatureState}	{ {\Omega}}
\newcommand{\naturestate}	{ {\omega}}
\newcommand{\NatureStateSpace}	{ {\mathbf{\NatureState}}}
\newcommand{\AnAction}	{ {A}}
\newcommand{\anaction}	{ {a}}
\newcommand{\ActionSpace}	{ {\mathbf{\AnAction}}}

% Abbreviations from RURO:
\newcommand{\InfoGain}[2] { \mathcal{D} \left( {#1} || {#2} \right) }

% Abbreviations from Upper Bound:
\newcommand{\lcm}	{{\rm lcm}}
% Double-check that this isn't in the math set already!

% Abbreviations from Emergence in Space
\newcommand{\ProcessAlphabet}	{\MeasAlphabet}
\newcommand{\ProbEst}			{ {\widehat{\Prob}_N}}
\newcommand{\STRegion}			{ {\mathrm K}}
\newcommand{\STRegionVariable}		{ K}
\newcommand{\stregionvariable}		{ k}
\newcommand{\GlobalPast}		{ \overleftarrow{G}} 
\newcommand{\globalpast}		{ \overleftarrow{g}} 
\newcommand{\GlobalFuture}		{ \overrightarrow{G}}
\newcommand{\globalfuture}		{ \overrightarrow{g}}
\newcommand{\GlobalState}		{ \mathcal{G}}
\newcommand{\globalstate}		{ \gamma}
\newcommand{\GlobalStateSet}		{ {\mathbf \GlobalState}}
\newcommand{\LocalPast}			{ \overleftarrow{L}} 
\newcommand{\localpast}			{ \overleftarrow{l}}
\newcommand{\AllLocalPasts}		{ \mathbf{\LocalPast}}
\newcommand{\LocalPastRegion}		{ \overleftarrow{\mathrm L}}
\newcommand{\LocalFuture}		{ \overrightarrow{L}}
\newcommand{\localfuture}		{ \overrightarrow{l}}
\newcommand{\LocalFutureRegion}		{ \overrightarrow{\mathrm L}}
\newcommand{\LocalState}		{ \mathcal{L}}
\newcommand{\localstate}		{ \lambda}
\newcommand{\LocalStateSet}		{ {\mathbf \LocalState}}
\newcommand{\PatchPast}			{ \overleftarrow{P}}
\newcommand{\patchpast}			{ \overleftarrow{p}}
\newcommand{\PatchPastRegion}		{ \overleftarrow{\mathrm P}}
\newcommand{\PatchFuture}		{ \overrightarrow{P}}
\newcommand{\patchfuture}		{ \overrightarrow{p}}
\newcommand{\PatchFutureRegion}		{ \overrightarrow{\mathrm P}}
\newcommand{\PatchState}		{ \mathcal{P}}
\newcommand{\patchstate}		{ \pi}
\newcommand{\PatchStateSet}		{ {\mathbf \PatchState}}
\newcommand{\LocalStatesInPatch}	{\vec{\LocalState}}
\newcommand{\localstatesinpatch}	{\vec{\localstate}}
\newcommand{\PointInstantX}		{ {\mathbf x}}
% Galles's original LaTeX for the cond. indep. symbol follows:
\newcommand{\compos}{\mbox{$~\underline{~\parallel~}~$}}
\newcommand{\ncompos}{\not\hspace{-.15in}\compos}
\newcommand{\indep}			{ \rotatebox{90}{$\models$}}
\newcommand{\nindep}	{\not\hspace{-.05in}\indep}
\newcommand{\LocalEE}	{{\EE}^{loc}}
\newcommand{\EEDensity}	{\overline{\LocalEE}}
\newcommand{\LocalCmu}	{{\Cmu}^{loc}}
\newcommand{\CmuDensity}	{\overline{\LocalCmu}}

%%%%%%%%%%% added by sasa
\newcommand{\FinPast}[1]	{ \overleftarrow {\MeasSymbol} \stackrel{{#1}}{}}
\newcommand{\finpast}[1]  	{ \overleftarrow {\meassymbol}  \stackrel{{#1}}{}}
\newcommand{\FinFuture}[1]		{ \overrightarrow{\MeasSymbol} \stackrel{{#1}}{}}
\newcommand{\finfuture}[1]		{ \overrightarrow{\meassymbol} \stackrel{{#1}}{}}

\newcommand{\Partition}	{ \AlternateState }
\newcommand{\partitionstate}	{ \alternatestate }
\newcommand{\PartitionSet}	{ \AlternateStateSet }
\newcommand{\Fdet}   { F_{\rm det} }

\newcommand{\Dkl}[2] { D_{\rm KL} \left( {#1} || {#2} \right) }

\newcommand{\Period}	{p}

% To take into account time direction
\newcommand{\forward}{+}
\newcommand{\reverse}{-}
%\newcommand{\forwardreverse}{\:\!\diamond} % \pm
\newcommand{\forwardreverse}{\pm} % \pm
\newcommand{\FutureProcess}	{ {\Process}^{\forward} }
\newcommand{\PastProcess}	{ {\Process}^{\reverse} }
\newcommand{\FutureCausalState}	{ {\CausalState}^{\forward} }
\newcommand{\futurecausalstate}	{ \sigma^{\forward} }
\newcommand{\altfuturecausalstate}	{ \sigma^{\forward\prime} }
\newcommand{\PastCausalState}	{ {\CausalState}^{\reverse} }
\newcommand{\pastcausalstate}	{ \sigma^{\reverse} }
\newcommand{\BiCausalState}		{ {\CausalState}^{\forwardreverse} }
\newcommand{\bicausalstate}		{ {\sigma}^{\forwardreverse} }
\newcommand{\FutureCausalStateSet}	{ {\CausalStateSet}^{\forward} }
\newcommand{\PastCausalStateSet}	{ {\CausalStateSet}^{\reverse} }
\newcommand{\BiCausalStateSet}	{ {\CausalStateSet}^{\forwardreverse} }
\newcommand{\eMachine}	{ M }
\newcommand{\FutureEM}	{ {\eMachine}^{\forward} }
\newcommand{\PastEM}	{ {\eMachine}^{\reverse} }
\newcommand{\BiEM}		{ {\eMachine}^{\forwardreverse} }
\newcommand{\BiEquiv}	{ {\sim}^{\forwardreverse} }
\newcommand{\Futurehmu}	{ h_\mu^{\forward} }
\newcommand{\Pasthmu}	{ h_\mu^{\reverse} }
\newcommand{\FutureCmu}	{ C_\mu^{\forward} }
\newcommand{\PastCmu}	{ C_\mu^{\reverse} }
\newcommand{\BiCmu}		{ C_\mu^{\forwardreverse} }
\newcommand{\FutureEps}	{ \epsilon^{\forward} }
\newcommand{\PastEps}	{ \epsilon^{\reverse} }
\newcommand{\BiEps}	{ \epsilon^{\forwardreverse} }
\newcommand{\FutureSim}	{ \sim^{\forward} }
\newcommand{\PastSim}	{ \sim^{\reverse} }
% Used arrows for awhile, more or less confusing?
%\newcommand{\FutureCausalState}	{ \overrightarrow{\CausalState} }
%\newcommand{\PastCausalState}	{ \overleftarrow{\CausalState} }
%\newcommand{\eMachine}	{ M }
%\newcommand{\FutureEM}	{ \overrightarrow{\eMachine} }
%\newcommand{\PastEM}	{ \overleftarrow{\eMachine} }
%\newcommand{\FutureCmu}	{ \overrightarrow{\Cmu} }
%\newcommand{\PastCmu}	{ \overleftarrow{\Cmu} }

%% time-reversing and mixed state presentation operators
\newcommand{\TR}{\mathcal{T}}
\newcommand{\MSP}{\mathcal{U}}
\newcommand{\one}{\mathbf{1}}

%% (cje)
%% Provide a command \ifpm which is true when \pm 
%% is meant to be understood as "+ or -". This is
%% different from the usage in TBA.
\newif\ifpm 
\edef\tempa{\forwardreverse}
\edef\tempb{\pm}
\ifx\tempa\tempb
   \pmfalse
\else
   \pmtrue  
\fi





% [some math stuff - maybe stick in sep file]
\usepackage{amsthm}
\usepackage{amscd}
\theoremstyle{plain}    \newtheorem{Lem}{Lemma}
\theoremstyle{plain}    \newtheorem*{ProLem}{Proof}
\theoremstyle{plain} 	\newtheorem{Cor}{Corollary}
\theoremstyle{plain} 	\newtheorem*{ProCor}{Proof}
\theoremstyle{plain} 	\newtheorem{The}{Theorem}
\theoremstyle{plain} 	\newtheorem*{ProThe}{Proof}
\theoremstyle{plain} 	\newtheorem{Prop}{Proposition}
\theoremstyle{plain} 	\newtheorem*{ProProp}{Proof}
\theoremstyle{plain} 	\newtheorem*{Conj}{Conjecture}
\theoremstyle{plain}	\newtheorem*{Rem}{Remark}
\theoremstyle{plain}	\newtheorem*{Def}{Definition} 
\theoremstyle{plain}	\newtheorem*{Not}{Notation}

% [uniform figure scaling - maybe this is not a good idea]
\def\figscale{.7}
\def\lscale{1.0}

% [FIX ME! - red makes it easier to spot]
\newcommand{\FIX}[1]{\textbf{\textcolor{red}{#1}}}


    \begin{document}
    \def\noprelim{}
\else
    % already included once
    % input post files

    \singlespacing
    \bibliographystyle{../bibliography/expanded}
    \bibliography{../bibliography/references}

    \end{document}
\fi\fi

\chapter{The Standard Model and Beyond}
\label{ch:theory}

\section{The Standard Model}

The Standard Model (SM) is a ``theory of almost everything'' that describes the
interactions of elementary particles.  The Standard Model is a \emph{quantum
field theory}, first appearing in its modern form in the middle of the 20th
century.  The model is the synthesis of the independent theories of
electromagnetism, and the weak and strong nuclear forces.  Each of these
theories was used to describe different phenomena, which each have extremely
different strengths and act at different scales.  The interaction of light and
matter is described by Quantum Electrodynamics~(QED), a relativistic field
extension of the theory of electromagnetism.  The physics of radioactivity and
nuclear decay was described by the Fermi theory of weak interactions and the
forces that strong nuclear force binds the nuclei of atoms was described by
Yukawa.  An overview of these theories will be presented in this chapter.  

The feature that united the disparate theories into the Standard Model was the
application of the principle of \emph{local gauge invariance}. The principle of
gauge invariance first found success in QED, which predicted electromagnetic
phenomenon with astounding accuracy.  Local gauge invariance is now believed to
a fundamental feature of nature that underpins all theories of elementary
particles.  Furthermore, the development of the complete Standard Model as it is
known today was precipitated by Goldstones's work on spontaneous symmetry
breaking~\cite{Goldstone:1961eq,PhysRev.127.965}, which produces an effective
Lagrangian with additional massless ``Goldstone'' bosons.  Higgs (and
others)~\cite{PhysRevLett.13.321, PhysRevLett.13.508,PhysRevLett.13.585}
developed these ideas into what is ultimately called the ``Higgs Mechanism,''
which uses a combination of new fields with broken symmetry to give mass to the
Goldstone bosons.

In the 1960s, Glashow~\cite{Glashow:1961tr}, Weinberg~\cite{Weinberg:1967tq}, 
and Salam~\cite{Salam:1968rm} developed the above ideas into the
electroweak model, which unified QED with the weak force using intermediate weak
bosons in a gauge theory whose symmetry was spontaneously broken using the Higgs
mechanism.  This unified theory has been incredibly experimentally successful
and is the foundation of modern particle theory.

\subsection{Quantum Electrodynamics and Gauge Invariance}
\label{sec:QEDandGaugeInvariance}

The theory of QED is a modern extension of Maxwell's theory of
electromagnetism, describing the interaction of matter with light.  The
development of QED is a result of efforts to develop a quantum mechanical
formulation of electromagnetism compatible with the theory of Special Relativity.
QED is a \emph{gauge} theory, which means that the physical observables are
invariant under local gauge transformations.  Requiring local gauge
invariance gives rise to a ``gauge'' field, which can be interpreted as 
particles that are exchanged during an interaction.  

In the following, we first describe the Dirac equation for a free electron,
which is the relativistic extension of the Schroedinger equation for spin~$1/2$
particles.  We then show that requiring the corresponding Lagrangian of the free
charged particle to be invariant under local gauge transformations creates an
effective gauge boson field.  This ``gauge field'' creates terms in the
Lagrangian that represent interactions between the particles.

% Dirac equation comes from pg. 215 of Griffiths
The Dirac equation is the equation of motion of a free spin~$1/2$ particle of
mass~$m$ and is derived from the energy--momentum relationship of relativity
\begin{equation}
  p^{\mu}p_\mu - m^2c^2 = 0
  \label{eq:EnergyPRelationship}
\end{equation}.
Dirac sought to express this relationship in the framework of quantum mechanics
by applying the transformation
\begin{equation}
  p_\mu \to i \hbar \partial_\mu 
  \label{eq:QuantizeMomentumTransformation}
\end{equation}
to equation Equation~\ref{eq:EnergyPRelationship}, but with the requirement that
the resulting equation be first order in time.\footnote{A detailed
discussion of this topic is available in~\cite{Griffiths:IntroParticle}.}
To achieve this, Dirac factorized Equation~\ref{eq:EnergyPRelationship} into 
\begin{equation}
  (\gamma^\kappa p_\kappa + mc)(\gamma^\mu p_\mu - mc) = 0,
  \label{eq:DiracEquation}
\end{equation}
where $\gamma^\mu$ is a set of four $4\times4$ matrices referred to as the Dirac
matrices.  The equation of motion is obtained by choosing either term (they are
equivalent) from the
left hand side of Equation~\ref{eq:DiracEquation} and
making the substitution in Equation~\ref{eq:QuantizeMomentumTransformation}.
\begin{equation}
  i \hbar \gamma^\mu \partial_\mu \psi - mc \psi = 0
  \label{eq:DiracEquation}
\end{equation}
The solutions, $\psi$, to the Dirac equation are called ``Dirac spinors,'' and
represent the quantum mechanical state of spin~$1/2$ particles.

The Lagrangian corresponding to the Dirac equation~(\ref{eq:DiracEquation}) is
\begin{equation}
  \mathcal{L} = \overline \psi (i\hbar c\gamma^\mu \partial_\mu - mc^2) \psi,
  \label{eq:FreeQEDLagrangian}
\end{equation}
where $\psi$ is the spinor field of the particle in question, $\hbar$ is Planck's
constant, $c$ the speed of light, and $\gamma^\mu$ are the Dirac matrices.  As
$\overline\psi$ is the Hermitian conjugate of $\psi$, the Lagrangian is invariant
under the global gauge transformation 
\begin{equation}
  \psi' \to e^{i\theta}\psi
  \label{eq:U1GaugeTransform}
\end{equation}.
The Lagrangian is invariant under \emph{local} gauge translations if $\theta$
can be defined differently at each point in space, \ie if $\theta = \theta(x)$
in equation~\ref{eq:U1GaugeTransform}.  However, as the derivative operator
$\partial_\mu$ in equation~\ref{eq:FreeQEDLagrangian} does not commute with
$\theta(x)$, the Lagrangian must be modified to satisfy local gauge invariance.
This modification is accomplished with the use of a ``gauge covariant
derivative.''  By making the replacement 
\begin{equation} 
  \partial_\mu \to D_\mu = \partial_\mu - \frac{ie}{\hbar}A^\mu      
\end{equation}
in equation~\ref{eq:FreeQEDLagrangian}, where \FIX{don't think this is right}
$A^\mu = \partial^\mu \theta(x)$ and $e$ is the electric charge, the Lagrangian
becomes locally gauge invariant:
\begin{equation}
  \mathcal{L} = \overline \psi (i\hbar c\gamma^\mu D_\mu - mc^2) \psi 
  \label{eq:LocalQEDLagrangian}
\end{equation}
The difference between
the locally (\ref{eq:LocalQEDLagrangian}) and the globally 
(\ref{eq:FreeQEDLagrangian}) gauge invariant Lagrangians is then
\begin{equation}
  \mathcal{L}_{int} = \frac{e}{\hbar}\overline\psi\gamma^\mu\psi A_\mu 
\end{equation}.
This term can be interpreted as the coupling between the particle and the gauge
boson (force carrier) fields.  The coupling is proportional to the constant $e$,
which is associated with the electric charge.  This is consistent with the
experimental observation that particles with zero electric charge do not
interact electromagnetically with each other.  In this interpretation, the
electromagnetic force between two charged particles is caused by the exchange of
gauge bosons (photons).  The existence of this ``minimal coupling'' is
\emph{required} if the Lagrangian is to satisfy local gauge invariance. The
addition of a term with the gauge Field Strength Tensor to represent the kinetic
term of the gauge (photon) field yields the QED Lagrangian:
\begin{equation}
  \mathcal{L}_{QED} = \overline \psi (i\hbar c\gamma^\mu D_\mu - mc^2) \psi -
  \frac{1}{4\mu_0}F_{\mu\nu}F^{\mu\nu}
\end{equation}

The gauge symmetry group of QED is $U(1)$, the unitary group of degree 1.  This
symmetry can be visualized as a rotation of a two--dimensional unit vector. (The
application of the gauge transformation $e^{i\theta}$ rotates a number in the
complex plane.)  In gauge theory that the symmetry group
defines the behavior of the gauge bosons and thus the interactions of the
theory.  

\subsection{The Weak Interactions}
\label{sec:WeakInteractions}
The theory of Weak Interactions was created to describe the physics of
radioactive decay.  The first formulation of the theory was done by
Fermi~\cite{ref:FermiWeakInteration} to explain the phenomenon of the $\beta$
decay of neutrons. The initial theory was a four-fermion ``contact'' theory.  In
a contact theory, all four fermions come involved in the $\beta$--decay are
connected at a single vertex.  The Fermi theory Hamiltonian for the
$\beta$--decay of a proton is then~\cite{Morii:SMandBSM}
\begin{equation}
  H = \frac{G_\beta}{\sqrt{2}}
  \left[\overline \psi_p \gamma_\mu (1 - g_A\gamma_5)\psi_n\right]
  \left[[\overline \psi_e \gamma^\mu (1 - \gamma_5) \psi_\nu\right] + h.c., 
 \label{eq:FermiTheoryHamiltonian}
\end{equation}
where $G_\beta$ is the Fermi constant and $g_A$ is the relative fraction of the
interaction with axially Lorentz structure.  The value of $g_A$ was determined
experimentally to be $1.26$.  One of the most notable things discovered about
the weak force is that weak interactions violate parity; that is, the physics of
the interaction change (or become disallowed) under inversion of the spatial
coordinates.  This is evidenced by the $(1-\gamma_5)$ term in
Equation~\ref{eq:FermiTheoryHamiltonian}.  This term is the ``helicity
operator''; the left and right ``handed'' helicity states are eigenstates states
of this term.
\begin{eqnarray}
  h = (1 - \gamma_5)/2 \nonumber \\
  h\psi_R = \frac{1}{2}\psi_R \nonumber \\ 
  h\psi_L = -\frac{1}{2}\psi_L \nonumber
\end{eqnarray}
It is observed that only left--handed neutrinos (or right--handed
anti--neutrinos) \fixme{check handedness is correct} participate in the weak
interaction.

The Fermi interaction can describe both nuclear $\beta$ decay ($p \to n + e^+ +
\overline{\nu_e}$) as well as the decay of a muon into an electron ($\mu \to
\nu_\mu + e + \overline{\nu_e}$, Figure~\ref{fig:MuonDecayContact}).
Furthermore, the coupling constant $G$ is found to be a \emph{universal}
constant in weak interactions, in that it is the same for interactions
regardless of the particle species participating in the interaction.  That is,
$G_\mu = G_e = G_F$.  Using an Hamiltonian analogous to
Equation~\ref{eq:FermiTheoryHamiltonian} for muon decay, the decay amplitude $M$
is found to be
\begin{equation}
  M = \frac{G_F}{\sqrt{2}}
  \left[\overline u_{\nu_\mu} \gamma_\rho \frac{1 - \gamma_5}{2} u_\mu\right]  
  \left[\overline u_{\nu_e} \gamma_\rho \frac{1 - \gamma_5}{2} u_e\right].
  \label{eq:ContactAmplitudeMuonDecay}
\end{equation}

However, the contact interaction form of
Fermi's theory is not complete.  When applied to scattering processes, the
interaction violates unitarity: the calculated cross section grows with the
center of mass energy, so that for some energy the probability for an
interaction is greater than one.  Furthermore, the techniques successfully used
to ``renormalize''\footnote{Renormalization of quantum field theories is a broad
topic beyond the scope of this thesis.  Briefly, the process involves
``absorbing'' infinite divergences that occur in higher--order interactions into
physical observables~\cite{Griffiths:IntroParticle}.} QED fail when applied to
the Fermi interaction.

\begin{figure}
  \centering
  \begin{fmffile}{muon_decay}
    \begin{fmfgraph*}(120,75)
    \fmfleft{i1}
    \fmfright{o1,o2,o3}
    %\fmftop{i2,d2,o2}
    \fmf{fermion}{i1,v1,o1}
    \fmf{fermion}{v1,o2}
    \fmf{fermion}{o3,v1}
    \fmflabel{$\mu^-$}{i1}
    \fmflabel{$\nu_\mu$}{o1}
    \fmflabel{$\overline \nu_e$}{o3}
    \fmflabel{$e^-$}{o2}
    \fmflabel{$G_F$}{v1}
    \end{fmfgraph*}
  \end{fmffile}
  \caption[Fermi contact interaction diagram]{Feynamn diagram of muon decay
  in Fermi contact interaction theory.}
  \label{fig:MuonDecayContact}
\end{figure}

\begin{figure}
  \centering
  \begin{fmffile}{muon_decay_propagator}
    \begin{fmfgraph*}(120,75)
    \fmfleft{i1}
    \fmfright{o1,o2,o3}
    %\fmftop{i2,d2,o2}
    \fmf{fermion}{i1,v1,o1}
    \fmf{photon, label=$W^-$}{v1,v2}
    \fmf{fermion}{v2,o2}
    \fmf{fermion}{v2,o3}
    \fmflabel{$\mu^-$}{i1}
    \fmflabel{$\nu_\mu$}{o1}
    \fmflabel{$\overline \nu_e$}{o3}
    \fmflabel{$e^-$}{o2}
    \end{fmfgraph*}
  \end{fmffile}
  \caption[Muon decaying through intermediate gauge boson]{Feynamn diagram of
  muon decay proceeding through an intermediate gauge boson $W^-$. }
  \label{fig:MuonDecayFeynmanDiagram}
\end{figure}

The first attempt to solve the problems with the Fermi theory was made by
introducing an intermediate weak boson~\cite{Glashow:1961tr}.  The contact
interaction is replaced by a massive propagator, the $W^\pm$ bosons.  The decay
of a muon to an electron and two neutrinos then proceeds as pictured in
Figure~\ref{fig:MuonDecayFeynmanDiagram} with an amplitude given
by~\cite{Morii:SMandBSM}
\begin{equation}
  M = - \left[\frac{g}{\sqrt{2}} \overline u_{\nu_\mu} \gamma_\rho \frac{1 -
  \gamma_5}{2} u_\mu\right] \frac{-g^{\rho\sigma} + \frac{q^\rho
  q^\sigma}{M_W^2}}{q^2 - M_W^2} 
  \left[\frac{g}{\sqrt{2}} \overline u_{\nu_e} \gamma_\rho \frac{1 -
  \gamma_5}{2} u_e\right]
  \label{eq:WeakPropagator}
\end{equation}
The presence of the large mass of gauge boson in the denominator of the central
term of Equation~\ref{eq:WeakPropagator} is the reason why the contact
interaction original formulated by Fermi is effectively describes of low--energy
weak phenomenon.  When the momentum transfer $q$ in the interaction is small
compared to $M_W$, the effect of the propagator is an effective constant.  In
the low energy limit, the full propagator in equation~\ref{eq:WeakPropagator} is
effectively equivalent to the Fermi contact interaction
in~\ref{eq:ContactAmplitudeMuonDecay} as
\begin{equation}
  \lim_{q/M_W \to 0}\frac{g^2}{8(q^2-M^2_W)} = \frac{G_F}{\sqrt{2}} 
  \label{eq:ContactVersusWeakBosonExchange}
\end{equation}
Unfortunately, the weak boson exchange model did not solve the problems of
unitarity and renormalizability in the weak interaction.   However, the form of
the boson--exchange propagator in
Equation~\ref{eq:ContactVersusWeakBosonExchange} suggests the observed
``weakness'' of the weak interactions is an artifact of the presence of the
massive propagator ($M_W$) and that the fundamental scale of the interaction $g$
is the same order of magnitude as that of QED, $g \approx e$.  This observation
lead to the unification of the electromagnetic and weak forces, which we
describe in the next sections.

\subsection{Spontaneous Symmetry Breaking}
\label{sec:SSB}
In the early 1960s Glashow, Weinberg, and Salam published a series of papers
describing how the electromagnetic and weak forces could be unified into a
common ``electroweak'' force.  The fact that at low energy the electromagnetic
and weak forces appear to be separate phenomena at low energy is due to the fact that the
symmetry of the electroweak gauge group is ``spontaneously broken.''  Modern
field theories (both the Standard Model and beyond) are predicated on the idea
that the all interactions are part of a single, unified symmetry group and the 
differences between various scales (electromagnetic, weak, etc.) at lower
energies are due to the unified symmetry being spontaneously broken.

A symmetry of a Lagrangian is spontaneously broken when the ground state, or
vacuum, is at a value which about which the Lagrangian is not symmetric.  In
quantum field theories, a particle is interpreted as quantized fluctuations of
its corresponding field about some constant (vacuum) ground state.  The
``effective'' Lagrangian that we observe in the (low energy) laboratory would be
the expansion of the Lagrangian about this stable point.  The effective
Lagrangian no longer obeys the original symmetry, which has been ``broken''.  We
give a brief example of the phenomenological effects of a spontaneously broken
symmetry in a toy model, following the treatment in~\cite{Morii:SMandBSM}.
\begin{equation}
  \mathcal{L} = 
  \frac{1}{2}\partial_\mu\phi_1\partial^\mu\phi_1 + 
  \frac{1}{2}\partial_\mu\phi_2\partial^\mu\phi_2 -
  V(\phi_1^2 + \phi_2^2)
  \label{eq:ToySSBLagrangian}
\end{equation}
The toy Lagrangian in Equation~\ref{eq:ToySSBLagrangian} has a global
$U(1)$\footnote{Technically, the symmetric transformation is 
\begin{equation}
  \left(\begin{array}{c} \phi_1 \\ \phi_2 \end{array}\right)
    \to
  \left(\begin{array}{c} \phi'_1 \\ \phi'_2 \end{array}\right)
    =
  \left(\begin{array}{cc} \cos\theta & -\sin\theta \\ 
    \sin\theta & \cos\theta \end{array}\right)
  \left(\begin{array}{c} \phi_1 \\ \phi_2 \end{array}\right),
    \nonumber
\end{equation} which is $O(2)$.  However, this transformation is equivalent to 
$U(1)$, as the two real fields $\phi_1$ and $\phi_2$ can be seen to correspond
to the real and imaginary parts of a complex field $\phi$ that does transform
according to $U(1)$.} symmetry and consists of two real-valued fields, $\phi_1$
and $\phi_2$.  The particle mass spectra of the theory is given by expanding the
field potential $V(\phi_1, \phi_2)$ about its minimum, $(\phi^{min}_1,
\phi^{min}_2)$.  The first three terms in the series are found by
\begin{eqnarray}
  V(\phi_1, \phi_2) &= &V(\phi^{min}_1, \phi^{min}_2) + 
  \sum_{a=1,2} \left(\frac {\partial V}{\partial \phi_a}\right)_{0} (\phi_a -
  \phi^{min}_a) \\
  &+ &\frac{1}{2} \sum_{a,b=1,2} \left(\frac {\partial^2 V}{\partial \phi_a \partial
  \phi_b}\right)_{0} (\phi_a - \phi^{min}_a)(\phi_b - \phi^{min}_b) + \ldots
  \label{eq:ExpandedPotential}
\end{eqnarray}
Since at the minimum the partial derivative of $V$ is zero with respect to all
fields, the second term in equation~\ref{eq:ExpandedPotential} is zero.  The
third term determines the masses of the particles in the theory.  Since a mass
term for a particle corresponding to a field $\phi_n$ in the Lagrangian appears
as $\frac{1}{2}m^2\phi_n\phi_n$, we can identify 
\begin{equation}
\left(\frac {\partial^2
V}{\partial \phi_a \partial \phi_b}\right)_{\phi^{min}}
\label{eq:MassMatrixTerms}
\end{equation}
as the $a$th row and $b$th column in the ``mass matrix''.  Off diagonal terms in
this matrix indicate mixing terms between the fields.  By diagonalizing the
matrix, the combinations of fields which correspond to the physical particles
(the ``mass eigenstates'') are found.  The $m^2$ of each particle is then the
corresponding entry in the diagonal of the mass matrix.

The particle spectra of the model depends heavily on the form of the potential.
An illustrative form (that is renormalizable and bounded from below) of a
possible configuration for the potential $V$ in
Equation~\ref{eq:ToySSBLagrangian} is 
\begin{equation}
  V(\phi_1^2 \phi_2^2) = \frac{m^2}{2}(\phi_1^2 + \phi_2^2) +
  \frac{\lambda}{4}(\phi_1^2+\phi_2^2)^2.
  \label{eq:SSBPotential}
\end{equation}
If the parameters $m^2$ and $\lambda$ are both positive, then the minimum of $V$
is at the origin $(\phi_1 = \phi_2 = 0)$.  In this case, the mass matrix term in
Equation~\ref{eq:ExpandedPotential} takes the form $\left(\frac {\partial^2
V}{\partial \phi_a \partial \phi_b}\right)_{0} = \frac{m^2}{2}\delta_{ab}$,
where $\delta_{ab}$ is the Kronecker delta function.  Therefore the mass matrix
is already diagonalized, and the $\phi_1$ and $\phi_2$ both correspond to
particles with mass $m$.  If the $m^2$ parameter in
Equation~\ref{eq:SSBPotential} is negative, the spectrum is dramatically
different.  After making the replacement $m^2 = -\mu^2 (\mu^2 > 0)$, the extrema
of $V$ are no longer unique.  The requirement of $\frac{\partial V}{\partial
\phi_i} = 0$ for all $i$ is satisfied in two cases:
\begin{eqnarray}
  (\phi^{min}_1, \phi^{min}_2) & = & (0,0) \label{eq:WignerPoint}\\
  (\phi^{min}_1)^2+ (\phi^{min}_2)^2 & = & \frac{\mu^2}{\lambda} = \nu^2. 
  \label{eq:NambuGoldstonePoint} 
\end{eqnarray}
If the vacuum state is defined at the point in Equation~\ref{eq:WignerPoint},
the symmetry is unbroken and the mass spectra is unchanged.  However, the system
is unstable at this point, as it is a local maximum.  The true global minimum is
defined as the set of points which satisfy
Equation~\ref{eq:NambuGoldstonePoint}, which form a continuous circle in
$\phi_1-\phi_2$ space (and is therefore infinitely degenerate).  We can choose
any point on the circle as the vacuum expectation value (VEV).  If the point
$(\phi_1^{min} = \nu, \phi_2^{min} = 0)$\footnote{The point chosen for the VEV
here is not arbitrary.  One can chose any point thats satisfies
Equation~\ref{eq:NambuGoldstonePoint} as the VEV\@.  However, after the mass
matrix is diagonalized, there will always be one physical field with a VEV$ =
\nu$ and one with a VEV$ = 0$.  Therefore the physical content of the theory
does not depend on the choice of VEV\@.} is chosen, evaluating
Equation~\ref{eq:MassMatrixTerms} yields the mass matrix \fixme{check matrix}
\begin{equation}
  \left(
  \begin{array}{cc}
    v^2 & 0 \\
    0 & 0 \\
  \end{array}
  \right).
  \nonumber
\end{equation}
Breaking the symmetry has changed the mass spectrum of the physical particles in
the model.  There is now a massive particle with $m=v^2$ and a massless
particle.  This massless particle is called the ``Goldstone boson.''
\fixme{check ref} Goldstone found~\cite{Goldstone:1961eq} that a massless
particle appears for each generator in the symmetry group that is broken.

\subsection{The Higgs Mechanism}
\label{sec:HiggsMech}

As in section~\ref{sec:QEDandGaugeInvariance}, extending the gauge symmetry
requirement to be \emph{locally} invariant creates interesting consequences for
models that have spontaneously broken symmetry.  This gives rise to the ``Higgs
Mechanism,'' which we overview here.  For simplicity we will again consider a
model with $U(1)$ symmetry.  The model is identical to the one presented in
section~\ref{sec:SSB}, with two exceptions. First, we express the two real
fields $\phi_1$ and $\phi_2$ as a single complex--valued field $\phi$.  Second,
the model is required to be locally $U(1)$ invariant, and so uses the
gauge--covariant derivatives, minimal coupling to the gauge field, and contains
the kinetic term for the gauge field, as discussed in
section~\ref{sec:QEDandGaugeInvariance}.  The unbroken Lagrangian is
\begin{eqnarray}
  \mathcal{L} &=& -\frac{1}{4} F_{\mu\nu}F^{\mu\nu} 
  + (D_\mu \phi^*)(D^\mu \phi) - V(\phi^*\phi) \label{eq:LocalInvariantU1} \\
  V(\phi^*\phi) &=& -\mu^2 \phi^*\phi + \lambda (\phi^*\phi)^2,
  \label{eq:PotentialLocalInvariantU1}
\end{eqnarray}
where $F_{\mu\nu}$ is related to the gauge field by $F_{\mu\nu} = \partial_\mu
A_\nu - \partial_\nu A_\mu$.  The Lagrangian is invariant under the local $U(1)$
gauge transformation
\begin{eqnarray}
  \phi \to \phi' &=& e^{-i \alpha(x)} \phi \nonumber \\
  A_\mu \to A_\mu' &=& A_\mu - \frac{1}{2} \partial_\mu \alpha(x) 
\end{eqnarray}

The potential is minimized when $\phi^*\phi = \frac{\mu^2}{2\lambda}$.  To
simplify the algebra, we can
re--parameterize the field into a real part $\eta(x)$ defined about $\nu$, the minimum of $V$, 
and a complex phase $\theta(x)/\nu$
\begin{equation}
  \phi(x) = \frac{1}{\sqrt 2}(\nu + \eta(x))e^{i \theta(x)/\nu}
  \label{eq:HiggsMechanismFieldParameterization}
\end{equation}
If the gauge transform is chosen to be $\alpha(x) = \theta(x)/\nu$, the fields
of are defined in the
so--called ``unitary gauge'' and have the special forms
\begin{eqnarray}
  \phi(x) \to \phi'(x) &=& \frac{1}{\sqrt 2}(\nu + \eta(x)) \nonumber \\
  A_\mu(x) \to B_\mu(x) &=& A_\mu(x) - \frac{1}{e \nu}\partial_\mu \theta(x)
  \label{eq:AfterUnitaryGaugeTransformation}
\end{eqnarray}
The kinetic term of the gauge field $F_{\mu\nu}$ is invariant under this
transformation.  If the gauge transformations of
Equation~\ref{eq:AfterUnitaryGaugeTransformation} are substituted into the
Lagrangian~(\ref{eq:LocalInvariantU1}) the effective Lagrangian at the minimum
of $V$ is
\begin{equation}
  \mathcal{L} = \frac{1}{2} \partial_\mu \eta \partial^\mu \eta -
  \mu^2 \eta^2 
  - \frac{1}{4} F_{\mu\nu} F^{\mu\nu} + \frac{1}{2} (e\nu)^2 B_\mu B^\mu 
  + \frac{1}{2} e^2 B_\mu B^\mu \eta (\eta + 2 \nu) - \lambda \nu \eta^3 -
  \frac{\lambda }{4}\eta^4
  \label{eq:HiggsMechanism}
\end{equation}
The breaking of the original symmetry has dramatically altered the physical
consequences of the model.  In its unbroken form, the model described by
Equation~\ref{eq:LocalInvariantU1} would produce two real massive particles and
one massless gauge boson mandated by local gauge invariance.  After symmetry
breaking, the effective Lagrangian in Equation~\ref{eq:HiggsMechanism} contains
a massive scalar $\eta$ with $m = \sqrt{2\mu^2}$ and a \emph{massive} gauge
boson $B_\mu$ with mass $m = \sqrt 2 e\nu$.  By acquiring a mass, the gauge
boson $B_\mu$ has acquired the degree of freedom (as it can now be
longitudinally polarized) previously associated to the second degree of freedom
in the scalar $\phi$ field.  This phenomenon, known as the ``Higgs Mechanism,''
is a simplified version of the techniques successfully used to unify the
electromagnetic and weak forces that we will discuss in the next section.  

\subsection{Electroweak Unification}

In the 1960s, the ideas of local gauge invariance in field theories, spontaneous
symmetry breaking, and the Higgs mechanism were combined by
Glashow~\cite{GlashowEW}, Weinberg~\cite{WeinbergEW} and Salam~\cite{SalamEW} to
form the unified theory of electroweak interactions, the nucleus of the Standard
Model.  This model successfully unified the electromagnetic and weak
interactions into a unified theory with a larger symmetry group.  The reason for
the empirically observed difference in scales between two interactions is due to
the larger, unified symmetry group being broken.  This broken symmetry creates
heavy gauge bosons via the Higgs mechanism, whose large mass decreases the
strength of ``weak'' interactions at low energy, as discussed in
Section~\ref{sec:WeakInteractions}.  The model successfully predicted the
existence and approximate masses of the weak force carriers, the $W^\pm$ and $Z$
bosons.  These particles were later observed~\cite{SPSWandZDiscovery} with the
predicted masses at the SPS experiment~\cite{SPSexperiment}. 

To provide a simple introduction to the mechanisms of the model, we will start
with a model that includes only one family of leptons, the electron $e$ and it's
associated neutrino $\nu_e$. Following once again the treatment
of~\cite{Morii:SMandBSM}, we describe the representation of the $e$ and $\nu_e$
in, and the choice of, the symmetry group of the model.  We then construct the
gauge invariant Lagrangian, with spontaneously broken symmetry, and examine the
particle content of the resulting model.

The form of the charged current $J_\mu(x) = \overline u_{\nu_e} \gamma_\rho
\frac{1 - \gamma_5}{2} u_e$ in the weak interaction amplitudes
(~\ref{eq:ContactAmplitudeMuonDecay} indicates that the left--handed electron
and neutrino 
(remember that the $(1-\gamma_5)$ kills any right--handed spinors) can be
combined into a doublet $L$ of $SU(2)$.  
\begin{equation}
  L = \frac{1 - \gamma_5}{2}
  \left(\begin{array}{c} \nu_e \\ e^- \end{array}\right)
    = \left(\begin{array}{c} \nu_e \\ e^- \end{array}\right)_L
      \label{eq:EWDoubletForm}
\end{equation}
Defining operators that operate on ``weak isospin,''
\begin{eqnarray}
  \tau^+ &=& \frac{\tau^1 + i \tau^2}{2} = 
  \left(\begin{array}{cc} 0 & 1 \\ 0 & 0 \end{array}\right) 
    \label{eq:Su2GeneratorPlus} \\
  \tau^- &=& \frac{\tau^1 - i \tau^2}{2} = 
  \left(\begin{array}{cc} 0 & 0 \\ 1 & 0 \end{array}\right),
    \label{eq:Su2GeneratorMinus}
\end{eqnarray}
where the $\tau^i$ are the Pauli matrices, the weak currents $J_\mu^\pm$ can be
written by combining
equations~\ref{eq:EWDoubletForm}--\ref{eq:Su2GeneratorMinus}
\begin{equation}
  J_\mu^\pm = \overline L \gamma_\mu \tau^\pm L.
  \label{eq:WeakCurrentInSu2Doublets}
\end{equation}
Since $\tau^1$, $\tau^2$, and $\tau^3$ are the generators of the $SU(2)$ group,
we can complete the group by adding a neutral current to the charged currents of
Equation~\ref{eq:WeakCurrentInSu2Doubles}.  The $\tau^3$ generator is diagonal,
so the charge of the current is zero and no mixing of the fields occurs:
\begin{eqnarray}
  J^3_\mu &= \overline L \gamma_\mu \frac{\tau^3}{2} L \nonumber \\
  &= \overline L \gamma_\mu  \frac{1}{2} 
  \left(\begin{array}{cc} 1 & 0 \\ 0 & -1 \end{array}\right) L \nonumber \\
    &=  \frac{1}{2} \overline \nu_e \gamma_\mu \nu_e - \frac{1}{2} \overline e_L
    \gamma_\mu e_L 
    \label{eq:EWNeutralCurrent}
\end{eqnarray}
Naively one might hope that the neutral current of
Equation~\ref{eq:EWNeutralCurrent} would correspond to the electromagnetic
(photon) current of QED\@.  However, this is impossible for two reasons.  First,
the right--handed component $e_R$ does not appear in the current, so this
interaction violates parity, a known symmetry of the electromagnetic
interactions.  Second, the current couples to neutrinos, which have no electric
charge.   Therefore, the ``charge'' corresponding to the $SU(2)$ gauge symmetry
generators $T^i = \int J_0^i(x)d^3x$ cannot be that of the QED\@, and the gauge
group must be enlarged to include an additional $U(1)$ symmetry.  The generator
of the new symmetry is must commute with the generators of the $SU(2)_L$ group.
The symmetry cannot be directly extended with $U(1)_{em}$ as the electromagnetic
charge $Q = \int (e^\dag_L e_L +  e^\dag_R e_R) d^3x$ does not commute with
$T^i$.  The solution is to introduce the ``weak hypercharge'' $\frac{Y}{2} = Q -
T^3$, which commutes the generators of $SU(2)_L$.  Thus the symmetry group of the
electroweak model is $SU(2)_L\times U(1)_Y$.

The $SU(2)_L\times U(1)_Y$ gauge invariant Lagrangian is written 
\begin{eqnarray}
  \mathcal{L} &=& 
  \overline L i \gamma^\mu (\partial_\mu - i g \frac{\vec \tau}{2} \cdot \vec
  {A_\mu} + \frac{i}{2}g'B_\mu)L \nonumber \\
  &+& \overline R i \gamma^\mu (\partial_\mu + \frac{i}{2}g'B_\mu)R \nonumber \\
  &+& -\frac{1}{4}F^i_{\mu\nu} F^{i\mu\nu} -\frac{1}{4}B_{\mu\nu} B^{\mu\nu}.  
  \label{eq:FermionAndGaugeLagrangianGWS}
\end{eqnarray}
As $R$ is a singlet in $SU(2)$, it does not couple to the $SU(2)$ gauge bosons
$A^i_\mu$.  For this Lagrangian to correspond to empirical observations at low
energy, the $SU(2)_L \times U(1)_Y$ must be broken.  As $U(1)_{em}$ symmetry 
is observed to be good symmetry at all scales the broken Lagrangian must
be invariant under $U(1)_{em}$.

To accomplish the symmetry breaking, we introduce a new $SU(2)$ doublet of
complex Higgs fields $\phi$ that have hypercharge $Y = 1$, and contribute
$\mathcal{L}_S$ to the Lagrangian:
\begin{eqnarray}
  \phi =& \left(\begin{array}{c}\phi^+ \\ \phi^0\end{array}\right) \\
  \mathcal{L}_S =& (D_\mu \phi)^\dag(D^\mu \phi) - V(\phi^\dag\phi),
\end{eqnarray}
where $D_\mu$ is the gauge covariant derivative containing couplings to both the
$SU(2)_L$ and $U(1)_Y$ gauge fields, and $V$ has a form analogous to $V$ in
Equation~\ref{eq:PotentialLocalInvariantU1}.  At this point we also add $SU(2)_L
\times U(1)_Y$ invariant ``Yukawa'' terms 
\begin{equation}
  \mathcal{L}_Y = -G_e(\overline L \phi R + \overline R \phi^\dag L) + h.c.)
  \label{eq:YukawaTerms}
\end{equation}
to the Lagrangian which couple the fermions ($L$ and $R$) to the Higgs field.
After symmetry breaking these terms will allow the fermions to acquire masses.
By choosing the $m^2$ and $\lambda$ parameters of $V$ appropriately, the new
$\phi$ field acquires a non--zero VEV and the symmetry is spontaneously broken.

At the minimum of $V$, the Higgs field satisfies $\phi^\dag\phi =
\frac{\nu^2}{2}$ and the Higgs fields has a VEV of
\begin{equation}
  \phi_{min} = \left(\begin{array}{c} 0 \\ v/\sqrt{2} \end{array}\right).
\end{equation}
The new symmetry of the model can be confirmed by looking at the action of
different symmetry generators on the VEV\@.  If the generator acting on the
vacuum state has a non--zero value, then the corresponding symmetry is broken.
It can then be seen that the generators of $SU(2)_L \times U(1)_Y$, $T^{+,-,3}$
and $Y$ are all broken.  The vacuum \emph{is} invariant under $Q$, the generator
of $U(1)_{em}$
\begin{equation}
  Q\phi_{min} = (T^3 + \frac{Y}{2})\left(\begin{array}{c} 0 \\ v/\sqrt{2}
  \end{array}\right) = 0 \nonumber.
\end{equation}

The gauge boson content of the electroweak interaction is obtained by
parameterizing the Higgs field in the magnitude--phase notation of
Equation~\ref{eq:HiggsMechanismFieldParameterization} and using the unitary
gauge (see Section~\ref{sec:HiggsMech}), where the gauge transformation is
chosen so Higgs field is real.  The Higgs scalar doublet is then 
\begin{equation}
  \phi' =  \left(\begin{array}{c} 0 \\ \frac{1}{\sqrt 2}(\nu +
    H(x))\end{array}\right) = 
    \frac{1}{\sqrt 2}(\nu + H(x))\chi.
    \label{eq:HiggsFieldParameterization}
\end{equation}
In the unit
The mass spectrum of the gauge bosons of the electroweak interaction (the photon, $W$,
and $Z$) is determined by the interaction of the gauge field terms in the
covariant derivative with the non--zero vacuum expectation value of the scalar
Higgs field $\phi$
\begin{equation}
  (D_\mu\phi)' = (\partial_\mu - i g \frac{\vec \tau}{2}\cdot \vec{A'_\mu}) -
  \frac{i}{2}g'B'_\mu)\frac{1}{\sqrt 2}(\nu + H)\chi.
  \nonumber
\end{equation}
The terms in the expansion of the kinetic term of the Higgs field that are
quadratic in $\nu^2$ and a gauge boson field give the mass associated to that
boson, and can be written as
\begin{equation}
  \mathcal{L}_{mass} = \frac{\nu^2}{8}(
  g^2 A'^1_\mu A'^{1\mu} + g^2 A'^2_\mu A'^{2\mu} + (g A'^3_\mu - g'B'_\mu)^2).
  \label{eq:GaugeBosonMassTerms}
\end{equation}
The $A'^1_\mu$ and $A'^2_\mu$ fields can be combined such that the first two
terms in Equation~\ref{eq:GaugeBosonMassTerms} into the mass term of the charged
$W^\pm$ boson
\begin{equation}
  W^\pm_\mu = \frac{A'^{1}_\mu \mp iA'^2_\mu}{2},
\end{equation}
which then has mass $M_W = \frac{1}{2}g\nu$.  The third term in
Equation~\ref{eq:GaugeBosonMassTerms} can be written in matrix form and then
diagonalized into mass eigenstates
\begin{eqnarray}
  &\frac{\nu^2}{8}(A'^3_\mu~B'_\mu) 
  \left(
  \begin{array}{cc}
    g^2 & -gg' \\
    -gg' & g'^2 
  \end{array}
  \right)
  \left(
  \begin{array}{c}
    A'^{3\mu} \\
    B'^\mu
  \end{array}
  \right) \\
  \to & 
  \frac{\nu^2}{8}(Z_\mu~A_\mu) 
  \left(
  \begin{array}{cc}
    g^2 + g'^2  & 0\\
    0 & 0 
  \end{array}
  \right)
  \left(
  \begin{array}{c}
    Z^\mu \\
    A^\mu
  \end{array}
  \right), \nonumber 
\end{eqnarray}
giving a massive $Z$ boson with $M_Z = \frac{\nu}{2}\sqrt{g^2 + g'^2}$ and a
massless photon.  The mass of the $Z$ is related to the mass of the $W^\pm$ by
the 
\begin{equation}
  M_Z \equiv \frac{M_W}{\cos \theta_W},
\end{equation}
where $\theta_W$ is the ``Weinberg angle,'' which must be determined from
experiment.  As the Fermi contact interaction of
Section~\ref{sec:WeakInteractions} is an effective theory of the weak sector,
the value of $G_F$ obtained from $\beta$ and muon decay experiments give clues
to the masses of the $W$ and $Z$.  
\begin{eqnarray}
  M_W = \frac{1}{2}(\frac{e^2}{\sqrt 2 G_F})^(1/2)\frac{1}{\sin\theta_W} \approx
  \frac{38 \GeV}{\sin\theta_W} > 37 \GeV \\ 
  M_Z \approx \frac{76 \GeV}{\sin
  2\theta_W} > 76\GeV.
\end{eqnarray}
The discovery of the $W$~\cite{UA1WDiscovery,UA2WDiscovery} and
$Z$~\cite{UA1ZDiscovery, UA2ZDiscovery} at the CERN SPS was a huge triumph for
the electroweak model.

The model that is presented in this section assumes only one species of leptons,
the electron and its associated neutrino.  The electroweak model is trivially
extended~\cite{Morii:SMandBSM} to include the other species ($\mu$, $\tau$) of
leptons and the three families of quarks. The masses of the fermions are
determined by the Yukawa terms in Equation~\ref{eq:YukawaTerms}.  Each particle
species has a Yukawa term relating the Higgs VEV to its mass that is not
constrained by the theory, and must be determined by experiment.

\subsection{Quantum Chromodynamics}

After electroweak unification, the Standard Model is completed by the theory of
Quantum Chromodynamics (QCD), which describes the interactions between quarks
and gluons.  QCD is a broad field and only a brief introduction to its
motivations and the phenomenology relevant to the analysis presented in this
thesis is contained in this section.  The existence of quarks as the composite
was first proposed by Gell--Man and Zweig to explain the spectroscopy of
hadrons.  QCD is an $SU(3)$ non--Abelian gauge theory which is invariant under
\emph{color} transformations.  Color is the charge of QCD and comes in three
types: red, green and blue.  The gauge boson that carries the force of QCD is
called the gluon, which is massless as the $SU(3)_c$ color symmetry is unbroken.

There are two marked differences between the photon of QED and the gluon of QCD.
First, the gluon itself carries a color charge, where is the electron is
neutral.  This has the consequence that a gluon can couple to other gluons.
Furthermore, it is found that no colored object exists in nature.  The corollary
of this is that it is believed to be impossible for a single quark or gluon to
be observed.  The mechanism that gives rise to effect is called ``color
confinement.''  The strength of the force between two interacting colored
objects increases with distance.  If two colored objects in a hadron are pulled
apart, the energy required to separate them will eventually be large enough to
produce new colored objects, resulting in two (or more) color--less hadrons.

The practical consequence of color confinement to a physicist at a high--energy
particle physics experiment is the production of quark and gluon ``jets,'' which
are high multiplicity sprays of particles observed in the detector.  In a
proton--proton collision, quarks and gluons can knocked off the incident
protons.  These quarks and gluons immediately ``hadronize,'' surrounding
themselves with additional hadrons, the majority of which are charged and
neutral pions.  Heavier quarks, such as the charm, beauty, and top quarks
undergo a flavor--changing weak decays, giving rise to 

\section{Searches for the Higgs boson}
\subsection{Higgs boson phenomenology}
\label{sec:SMHiggsPhenom}
\subsection{Results from LEP and Tevatron}

\section{Beyond the Standard Model}
\label{sec:BSM}
\subsection{The Hierarchy Problem}
\subsection{Supersymmetry}
\subsection{The Minimal Supersymmetric Model and $\tau$ leptons}
\label{sec:MSSMAndTaus}

\section{The physics of the \taul}
As discussed in sections~\ref{sec:SMHiggsPhenom} and~\ref{sec:MSSMAndTaus}, the
$\tau$ lepton is an important probe of Higgs physics.  The \taul has some
unusual properties which make it particularly challenging at hadron colliders.
With a mass of 1.78~\GeVcc, the \taul is heaviest of the leptons.  The nominal
decay distance $c\tau$ of the \taul is 87~\micron, which in practice means that
the $\tau$ will always decay before reaching the first layer of the detector.
Tau decays can be effectively classified into two types. ``Leptonic'' decays
consist of a $\tau$ decaying to a light lepton ($\ell = e, \mu$) and two
neutrinos $\tau^+ \to \ell^+ \nu_\tau \overline{ \nu_{\ell}}$.  ``Hadronic'' decays 
consist of a low--multiplicity collimated group of hadrons, typically \Pgppm and \Pgpz
mesons.  The hadronic decays of the \taul compose approximately $65\%$ of the
\taul branching fraction, with the remainder shared approximately equally by the
leptonic decays.  The branching fractions for the leptonic and most common hadronic decays
are shown in table~\ref{tab:TauDecayModes}.
\begin{table}
   \centering
   \begin{tabular}{lcrr}
      Visible Decay Products  & Resonance & Mass (\MeVcc) &
      Fraction~\cite{PDG} \\
      \hline
      \hline
      \multicolumn{4}{c}{Leptonic modes modes} \\
      \hline
      $e^- \nu_\tau \overline \nu_e$             & -      & 0.5  & 17.8\% \\
      $\mu^-\nu_\tau \overline \nu_\mu$          & -      & 105  & 17.4\% \\
      \hline
      \multicolumn{4}{c}{Hadronic modes} \\
      \hline
      $\pi^{-} \nu_\tau$                    & -      & 135  & 10.9\% \\
      $\pi^{-}\pi^0 \nu_\tau$               & $\rho$ & 770  & 25.5\% \\
      $\pi^{-}\pi^0\pi^0 \nu_\tau$          & $a1$   & 1200 & 9.3\% \\
      $\pi^{-}\pi^{-}\pi^{+} \nu_\tau$      & $a1$   & 1200 & 9.0\% \\
      $\pi^{-}\pi^{-}\pi^{+}\pi^0 \nu_\tau$ & $a1$   & 1200 & 4.5\% \\
      \hline
      Total                                 &        &      & 94.4\% \\
      \hline
   \end{tabular}
   \label{tab:decay_modes}
   \caption{Resonances and branching ratios of the dominant decay modes of
   the \taul.  The decay products listed correspond to a negatively
   charged \taul; the table is identical under charge conjugation.}
\end{table}

\ifx\master\undefined% AUTOCOMPILE
% Allows for individual chapters to be compiled.

% Usage:
% \ifx\master\undefined% AUTOCOMPILE
% Allows for individual chapters to be compiled.

% Usage:
% \ifx\master\undefined% AUTOCOMPILE
% Allows for individual chapters to be compiled.

% Usage:
% \ifx\master\undefined\input{../settings/autocompile}\fi
% (place at start and end of chapter file)

\ifx\noprelim\undefined
    % first time included
    % input preamble files
    \input{../settings/phdsetup}

    \begin{document}
    \def\noprelim{}
\else
    % already included once
    % input post files

    \singlespacing
    \bibliographystyle{../bibliography/expanded}
    \bibliography{../bibliography/references}

    \end{document}
\fi\fi
% (place at start and end of chapter file)

\ifx\noprelim\undefined
    % first time included
    % input preamble files
    % [ USER VARIABLES ]

\def\PHDTITLE {Extensions of the Theory of Computational Mechanics}
\def\PHDAUTHOR{Evan Klose Friis}
\def\PHDSCHOOL{University of California, Davis}

\def\PHDMONTH {June}
\def\PHDYEAR  {2011}
\def\PHDDEPT {Physics}

\def\BSSCHOOL {University of California at San Diego}
\def\BSYEAR   {2005}

\def\PHDCOMMITTEEA{Professor John Conway}
\def\PHDCOMMITTEEB{Professor Robin Erbacher}
\def\PHDCOMMITTEEC{Professor Mani Tripathi}

% [ GLOBAL SETUP ]

\documentclass[letterpaper,oneside,11pt]{report}

\usepackage{calc}
\usepackage{breakcites}
\usepackage[newcommands]{ragged2e}

\usepackage[pdftex]{graphicx}
\usepackage{epstopdf}

%\usepackage{tikz}
%\usetikzlibrary{positioning} % [right=of ...]
%\usetikzlibrary{fit} % [fit= ...]

%\pgfdeclarelayer{background layer}
%\pgfdeclarelayer{foreground layer}
%\pgfsetlayers{background layer,main,foreground layer}

%\newenvironment{wrap}{\noindent\begin{minipage}[t]{\linewidth}\vspace{-0.5\normalbaselineskip}\centering}{\vspace{0.5\normalbaselineskip}\end{minipage}}

%% [Venn diagram environment]
%\newenvironment{venn2}
%{\begin{tikzpicture} [every pin/.style={text=black, text opacity=1.0, pin distance=0.5cm, pin edge={black!60, semithick}},
%% define a new style 'venn'
%venn/.style={circle, draw=black!60, semithick, minimum size = 4cm}]
%
%% create circle and give it external (pin) label
%\node[venn] (X) at (-1,0) [pin={150:$H[X]$}] {};
%\node[venn] (Y) at (1,0) [pin={30:$H[Y]$}] {};
%
%% place labels of the atoms by hand
%\node at (-1.9,0) {$H[X|Y]$};
%\node at (1.9,0) {$H[Y|X]$};
%\node at (0,0) {$I[X;Y]$};}
%{\end{tikzpicture}}

%\newcommand{\wrapmath}[1]{\begin{wrap}\begin{tikzpicture}[every node/.style={inner ysep=0ex, inner xsep=0em}]\node[] {$\displaystyle\begin{aligned} #1\end{aligned}$};\end{tikzpicture}\end{wrap}}

\renewenvironment{abstract}{\chapter*{Abstract}}{}
\renewcommand{\bibname}{Bibliography}
\renewcommand{\contentsname}{Table of Contents}

\makeatletter
\renewcommand{\@biblabel}[1]{\textsc{#1}}
\makeatother

% [ FONT SETTINGS ]

\usepackage[tbtags, intlimits, namelimits]{amsmath}
\usepackage[adobe-utopia]{mathdesign}

\DeclareSymbolFont{pazomath}{OMS}{zplm}{m}{n}
\DeclareSymbolFontAlphabet{\mathcal}{pazomath}
\SetMathAlphabet\mathcal{bold}{OMS}{zplm}{b}{n}

\SetSymbolFont{largesymbols}{normal}{OMX}{zplm}{m}{n}
\SetSymbolFont{largesymbols}{bold}{OMX}{zplm}{m}{n}
\SetSymbolFont{symbols}{normal}{OMS}{zplm}{m}{n}
\SetSymbolFont{symbols}{bold}{OMS}{zplm}{b}{n}

\renewcommand{\sfdefault}{phv}
\renewcommand{\ttdefault}{fvm}

\widowpenalty 8000
\clubpenalty  8000

% [ PAGE LAYOUT ]
\usepackage{geometry}
\geometry{lmargin = 1.5in}
\geometry{rmargin = 1.0in}
\geometry{tmargin = 1.0in}
\geometry{bmargin = 1.0in}

% [ PDF SETTINGS ]

\usepackage[final]{hyperref}
\hypersetup{breaklinks  = true}
\hypersetup{colorlinks  = true}
\hypersetup{linktocpage = false}
\hypersetup{linkcolor   = blue}
\hypersetup{citecolor   = green}
\hypersetup{urlcolor    = black}
\hypersetup{plainpages  = false}
\hypersetup{pageanchor  = true}
\hypersetup{pdfauthor   = {\PHDAUTHOR}}
\hypersetup{pdftitle    = {\PHDTITLE}}
\hypersetup{pdfsubject  = {Dissertation, \PHDSCHOOL}}
\urlstyle{same}

% [ LETTER SPACING ]

\usepackage[final]{microtype}
\microtypesetup{protrusion=compatibility}
\microtypesetup{expansion=false}

\newcommand{\upper}[1]{\MakeUppercase{#1}}
\let\lsscshape\scshape

\ifcase\pdfoutput\else\microtypesetup{letterspace=15}
\renewcommand{\scshape}{\lsscshape\lsstyle}
\renewcommand{\upper}[1]{\textls[50]{\MakeUppercase{#1}}}\fi

% [ LINE SPACING ]

\usepackage[doublespacing]{setspace}
\renewcommand{\displayskipstretch}{0.0}

\setlength{\parskip   }{0em}
\setlength{\parindent }{2em}

% [ TABLE FORMATTING ]

\usepackage{booktabs}
\setlength{\heavyrulewidth}{1.5\arrayrulewidth}
\setlength{\lightrulewidth}{1.0\arrayrulewidth}
\setlength{\doublerulesep }{2.0\arrayrulewidth}

% [ SECTION FORMATTING ]

\usepackage[largestsep,nobottomtitles*]{titlesec}
\renewcommand{\bottomtitlespace}{0.75in}

\titleformat{\chapter}[display]{\bfseries\huge\singlespacing}{\filleft\textsc{\LARGE \chaptertitlename\ \thechapter}}{-0.2ex}{\titlerule[3pt]\vspace{0.2ex}}[]

\titleformat{\section}{\LARGE}{\S\thesection\hspace{0.5em}}{0ex}{}
\titleformat{\subsection}{\Large}{\S\thesubsection\hspace{0.5em}}{0ex}{}
\titleformat{\subsubsection}{\large}{\thesubsubsection\hspace{0.5em}}{0ex}{}

\titlespacing*{\chapter}{0em}{6ex}{4ex plus 2ex minus 0ex}
\titlespacing*{\section}{0em}{2ex plus 3ex minus 1ex}{0.5ex plus 0.5ex minus 0.5ex}
\titlespacing*{\subsection}{0ex}{2ex plus 3ex minus 1ex}{0ex}
\titlespacing*{\subsubsection}{0ex}{2ex plus 0ex minus 1ex}{0ex}

% [ HEADER SETTINGS ]

\usepackage{fancyhdr}

\setlength{\headheight}{\normalbaselineskip}
\setlength{\footskip  }{0.5in}
\setlength{\headsep   }{0.5in-\headheight}

\fancyheadoffset[R]{0.5in}
\renewcommand{\headrulewidth}{0pt}
\renewcommand{\footrulewidth}{0pt}

\newcommand{\pagebox}{\parbox[r][\headheight][t]{0.5in}{\hspace\fill\thepage}}

\newcommand{\prelimheaders}{\ifx\prelim\undefined\renewcommand{\thepage}{\textit{\roman{page}}}\fancypagestyle{plain}{\fancyhf{}\fancyfoot[L]{\makebox[\textwidth-0.5in]{\thepage}}}\pagestyle{plain}\def\prelim{}\fi}

\newcommand{\normalheaders}{\renewcommand{\thepage}{\arabic{page}}\fancypagestyle{plain}{\fancyhf{}\fancyhead[R]{\pagebox}}\pagestyle{plain}}

\normalheaders{}

% [ CUSTOM COMMANDS ]

\newcommand{\signaturebox}[1]{\multicolumn{1}{p{4in}}{\vspace{3ex}}\\\midrule #1\\}

%\input{../includes/cmechabbrev}

% [some math stuff - maybe stick in sep file]
\usepackage{amsthm}
\usepackage{amscd}
\theoremstyle{plain}    \newtheorem{Lem}{Lemma}
\theoremstyle{plain}    \newtheorem*{ProLem}{Proof}
\theoremstyle{plain} 	\newtheorem{Cor}{Corollary}
\theoremstyle{plain} 	\newtheorem*{ProCor}{Proof}
\theoremstyle{plain} 	\newtheorem{The}{Theorem}
\theoremstyle{plain} 	\newtheorem*{ProThe}{Proof}
\theoremstyle{plain} 	\newtheorem{Prop}{Proposition}
\theoremstyle{plain} 	\newtheorem*{ProProp}{Proof}
\theoremstyle{plain} 	\newtheorem*{Conj}{Conjecture}
\theoremstyle{plain}	\newtheorem*{Rem}{Remark}
\theoremstyle{plain}	\newtheorem*{Def}{Definition} 
\theoremstyle{plain}	\newtheorem*{Not}{Notation}

% [uniform figure scaling - maybe this is not a good idea]
\def\figscale{.7}
\def\lscale{1.0}

% [FIX ME! - red makes it easier to spot]
\newcommand{\FIX}[1]{\textbf{\textcolor{red}{#1}}}


    \begin{document}
    \def\noprelim{}
\else
    % already included once
    % input post files

    \singlespacing
    \bibliographystyle{../bibliography/expanded}
    \bibliography{../bibliography/references}

    \end{document}
\fi\fi
% (place at start and end of chapter file)

\ifx\noprelim\undefined
    % first time included
    % input preamble files
    % [ USER VARIABLES ]

\def\PHDTITLE {Extensions of the Theory of Computational Mechanics}
\def\PHDAUTHOR{Evan Klose Friis}
\def\PHDSCHOOL{University of California, Davis}

\def\PHDMONTH {June}
\def\PHDYEAR  {2011}
\def\PHDDEPT {Physics}

\def\BSSCHOOL {University of California at San Diego}
\def\BSYEAR   {2005}

\def\PHDCOMMITTEEA{Professor John Conway}
\def\PHDCOMMITTEEB{Professor Robin Erbacher}
\def\PHDCOMMITTEEC{Professor Mani Tripathi}

% [ GLOBAL SETUP ]

\documentclass[letterpaper,oneside,11pt]{report}

\usepackage{calc}
\usepackage{breakcites}
\usepackage[newcommands]{ragged2e}

\usepackage[pdftex]{graphicx}
\usepackage{epstopdf}

%\usepackage{tikz}
%\usetikzlibrary{positioning} % [right=of ...]
%\usetikzlibrary{fit} % [fit= ...]

%\pgfdeclarelayer{background layer}
%\pgfdeclarelayer{foreground layer}
%\pgfsetlayers{background layer,main,foreground layer}

%\newenvironment{wrap}{\noindent\begin{minipage}[t]{\linewidth}\vspace{-0.5\normalbaselineskip}\centering}{\vspace{0.5\normalbaselineskip}\end{minipage}}

%% [Venn diagram environment]
%\newenvironment{venn2}
%{\begin{tikzpicture} [every pin/.style={text=black, text opacity=1.0, pin distance=0.5cm, pin edge={black!60, semithick}},
%% define a new style 'venn'
%venn/.style={circle, draw=black!60, semithick, minimum size = 4cm}]
%
%% create circle and give it external (pin) label
%\node[venn] (X) at (-1,0) [pin={150:$H[X]$}] {};
%\node[venn] (Y) at (1,0) [pin={30:$H[Y]$}] {};
%
%% place labels of the atoms by hand
%\node at (-1.9,0) {$H[X|Y]$};
%\node at (1.9,0) {$H[Y|X]$};
%\node at (0,0) {$I[X;Y]$};}
%{\end{tikzpicture}}

%\newcommand{\wrapmath}[1]{\begin{wrap}\begin{tikzpicture}[every node/.style={inner ysep=0ex, inner xsep=0em}]\node[] {$\displaystyle\begin{aligned} #1\end{aligned}$};\end{tikzpicture}\end{wrap}}

\renewenvironment{abstract}{\chapter*{Abstract}}{}
\renewcommand{\bibname}{Bibliography}
\renewcommand{\contentsname}{Table of Contents}

\makeatletter
\renewcommand{\@biblabel}[1]{\textsc{#1}}
\makeatother

% [ FONT SETTINGS ]

\usepackage[tbtags, intlimits, namelimits]{amsmath}
\usepackage[adobe-utopia]{mathdesign}

\DeclareSymbolFont{pazomath}{OMS}{zplm}{m}{n}
\DeclareSymbolFontAlphabet{\mathcal}{pazomath}
\SetMathAlphabet\mathcal{bold}{OMS}{zplm}{b}{n}

\SetSymbolFont{largesymbols}{normal}{OMX}{zplm}{m}{n}
\SetSymbolFont{largesymbols}{bold}{OMX}{zplm}{m}{n}
\SetSymbolFont{symbols}{normal}{OMS}{zplm}{m}{n}
\SetSymbolFont{symbols}{bold}{OMS}{zplm}{b}{n}

\renewcommand{\sfdefault}{phv}
\renewcommand{\ttdefault}{fvm}

\widowpenalty 8000
\clubpenalty  8000

% [ PAGE LAYOUT ]
\usepackage{geometry}
\geometry{lmargin = 1.5in}
\geometry{rmargin = 1.0in}
\geometry{tmargin = 1.0in}
\geometry{bmargin = 1.0in}

% [ PDF SETTINGS ]

\usepackage[final]{hyperref}
\hypersetup{breaklinks  = true}
\hypersetup{colorlinks  = true}
\hypersetup{linktocpage = false}
\hypersetup{linkcolor   = blue}
\hypersetup{citecolor   = green}
\hypersetup{urlcolor    = black}
\hypersetup{plainpages  = false}
\hypersetup{pageanchor  = true}
\hypersetup{pdfauthor   = {\PHDAUTHOR}}
\hypersetup{pdftitle    = {\PHDTITLE}}
\hypersetup{pdfsubject  = {Dissertation, \PHDSCHOOL}}
\urlstyle{same}

% [ LETTER SPACING ]

\usepackage[final]{microtype}
\microtypesetup{protrusion=compatibility}
\microtypesetup{expansion=false}

\newcommand{\upper}[1]{\MakeUppercase{#1}}
\let\lsscshape\scshape

\ifcase\pdfoutput\else\microtypesetup{letterspace=15}
\renewcommand{\scshape}{\lsscshape\lsstyle}
\renewcommand{\upper}[1]{\textls[50]{\MakeUppercase{#1}}}\fi

% [ LINE SPACING ]

\usepackage[doublespacing]{setspace}
\renewcommand{\displayskipstretch}{0.0}

\setlength{\parskip   }{0em}
\setlength{\parindent }{2em}

% [ TABLE FORMATTING ]

\usepackage{booktabs}
\setlength{\heavyrulewidth}{1.5\arrayrulewidth}
\setlength{\lightrulewidth}{1.0\arrayrulewidth}
\setlength{\doublerulesep }{2.0\arrayrulewidth}

% [ SECTION FORMATTING ]

\usepackage[largestsep,nobottomtitles*]{titlesec}
\renewcommand{\bottomtitlespace}{0.75in}

\titleformat{\chapter}[display]{\bfseries\huge\singlespacing}{\filleft\textsc{\LARGE \chaptertitlename\ \thechapter}}{-0.2ex}{\titlerule[3pt]\vspace{0.2ex}}[]

\titleformat{\section}{\LARGE}{\S\thesection\hspace{0.5em}}{0ex}{}
\titleformat{\subsection}{\Large}{\S\thesubsection\hspace{0.5em}}{0ex}{}
\titleformat{\subsubsection}{\large}{\thesubsubsection\hspace{0.5em}}{0ex}{}

\titlespacing*{\chapter}{0em}{6ex}{4ex plus 2ex minus 0ex}
\titlespacing*{\section}{0em}{2ex plus 3ex minus 1ex}{0.5ex plus 0.5ex minus 0.5ex}
\titlespacing*{\subsection}{0ex}{2ex plus 3ex minus 1ex}{0ex}
\titlespacing*{\subsubsection}{0ex}{2ex plus 0ex minus 1ex}{0ex}

% [ HEADER SETTINGS ]

\usepackage{fancyhdr}

\setlength{\headheight}{\normalbaselineskip}
\setlength{\footskip  }{0.5in}
\setlength{\headsep   }{0.5in-\headheight}

\fancyheadoffset[R]{0.5in}
\renewcommand{\headrulewidth}{0pt}
\renewcommand{\footrulewidth}{0pt}

\newcommand{\pagebox}{\parbox[r][\headheight][t]{0.5in}{\hspace\fill\thepage}}

\newcommand{\prelimheaders}{\ifx\prelim\undefined\renewcommand{\thepage}{\textit{\roman{page}}}\fancypagestyle{plain}{\fancyhf{}\fancyfoot[L]{\makebox[\textwidth-0.5in]{\thepage}}}\pagestyle{plain}\def\prelim{}\fi}

\newcommand{\normalheaders}{\renewcommand{\thepage}{\arabic{page}}\fancypagestyle{plain}{\fancyhf{}\fancyhead[R]{\pagebox}}\pagestyle{plain}}

\normalheaders{}

% [ CUSTOM COMMANDS ]

\newcommand{\signaturebox}[1]{\multicolumn{1}{p{4in}}{\vspace{3ex}}\\\midrule #1\\}

%%%% macros fro standard references
%\eqref provided by amsmath
\newcommand{\figref}[1]{Fig.~\ref{#1}}
\newcommand{\tableref}[1]{Table~\ref{#1}}
\newcommand{\refcite}[1]{Ref.~\cite{#1}}

% Abbreviations from CMPPSS:

\newcommand{\eM}     {\mbox{$\epsilon$-machine}}
\newcommand{\eMs}    {\mbox{$\epsilon$-machines}}
\newcommand{\EM}     {\mbox{$\epsilon$-Machine}}
\newcommand{\EMs}    {\mbox{$\epsilon$-Machines}}
\newcommand{\eT}     {\mbox{$\epsilon$-transducer}}
\newcommand{\eTs}    {\mbox{$\epsilon$-transducers}}
\newcommand{\ET}     {\mbox{$\epsilon$-Transducer}}
\newcommand{\ETs}    {\mbox{$\epsilon$-Transducers}}

% Processes and sequences

\newcommand{\Process}{\mathcal{P}}

\newcommand{\ProbMach}{\Prob_{\mathrm{M}}}
\newcommand{\Lmax}   { {L_{\mathrm{max}}}}
\newcommand{\MeasAlphabet}	{\mathcal{A}}
% Original
%\newcommand{\MeasSymbol}   { {S} }
%\newcommand{\meassymbol}   { {s} }
% New symbol
\newcommand{\MeasSymbol}   { {X} }
\newcommand{\meassymbol}   { {x} }
\newcommand{\BiInfinity}	{ \overleftrightarrow {\MeasSymbol} }
\newcommand{\biinfinity}	{ \overleftrightarrow {\meassymbol} }
\newcommand{\Past}	{ \overleftarrow {\MeasSymbol} }
\newcommand{\past}	{ {\overleftarrow {\meassymbol}} }
\newcommand{\pastprime}	{ {\past}^{\prime}}
\newcommand{\Future}	{ \overrightarrow{\MeasSymbol} }
\newcommand{\future}	{ \overrightarrow{\meassymbol} }
\newcommand{\futureprime}	{ {\future}^{\prime}}
\newcommand{\PastPrime}	{ {\Past}^{\prime}}
\newcommand{\FuturePrime}	{ {\overrightarrow{\meassymbol}}^\prime }
\newcommand{\PastDblPrime}	{ {\overleftarrow{\meassymbol}}^{\prime\prime} }
\newcommand{\FutureDblPrime}	{ {\overrightarrow{\meassymbol}}^{\prime\prime} }
\newcommand{\pastL}	{ {\overleftarrow {\meassymbol}}{}^L }
\newcommand{\PastL}	{ {\overleftarrow {\MeasSymbol}}{}^L }
\newcommand{\PastLt}	{ {\overleftarrow {\MeasSymbol}}_t^L }
\newcommand{\PastLLessOne}	{ {\overleftarrow {\MeasSymbol}}^{L-1} }
\newcommand{\futureL}	{ {\overrightarrow{\meassymbol}}{}^L }
\newcommand{\FutureL}	{ {\overrightarrow{\MeasSymbol}}{}^L }
\newcommand{\FutureLt}	{ {\overrightarrow{\MeasSymbol}}_t^L }
\newcommand{\FutureLLessOne}	{ {\overrightarrow{\MeasSymbol}}^{L-1} }
\newcommand{\pastLprime}	{ {\overleftarrow {\meassymbol}}^{L^\prime} }
\newcommand{\futureLprime}	{ {\overrightarrow{\meassymbol}}^{L^\prime} }
\newcommand{\AllPasts}	{ { \overleftarrow {\rm {\bf \MeasSymbol}} } }
\newcommand{\AllFutures}	{ \overrightarrow {\rm {\bf \MeasSymbol}} }
\newcommand{\FutureSet}	{ \overrightarrow{\bf \MeasSymbol}}

% Causal states and epsilon-machines
\newcommand{\CausalState}	{ \mathcal{S} }
\newcommand{\CausalStatePrime}	{ {\CausalState}^{\prime}}
\newcommand{\causalstate}	{ \sigma }
\newcommand{\CausalStateSet}	{ \boldsymbol{\CausalState} }
\newcommand{\AlternateState}	{ \mathcal{R} }
\newcommand{\AlternateStatePrime}	{ {\cal R}^{\prime} }
\newcommand{\alternatestate}	{ \rho }
\newcommand{\alternatestateprime}	{ {\rho^{\prime}} }
\newcommand{\AlternateStateSet}	{ \boldsymbol{\AlternateState} }
\newcommand{\PrescientState}	{ \widehat{\AlternateState} }
\newcommand{\prescientstate}	{ \widehat{\alternatestate} }
\newcommand{\PrescientStateSet}	{ \boldsymbol{\PrescientState}}
\newcommand{\CausalEquivalence}	{ {\sim}_{\epsilon} }
\newcommand{\CausalEquivalenceNot}	{ {\not \sim}_{\epsilon}}

\newcommand{\NonCausalEquivalence}	{ {\sim}_{\eta} }
\newcommand{\NextObservable}	{ {\overrightarrow {\MeasSymbol}}^1 }
\newcommand{\LastObservable}	{ {\overleftarrow {\MeasSymbol}}^1 }
%\newcommand{\Prob}		{ {\rm P}}
\newcommand{\Prob}      {\Pr} % use standard command
\newcommand{\ProbAnd}	{ {,\;} }
\newcommand{\LLimit}	{ {L \rightarrow \infty}}
\newcommand{\Cmu}		{C_\mu}
\newcommand{\hmu}		{h_\mu}
\newcommand{\EE}		{{\bf E}}
\newcommand{\Measurable}{{\bf \mu}}

% Process Crypticity
\newcommand{\PC}		{\chi}
\newcommand{\FuturePC}		{\PC^+}
\newcommand{\PastPC}		{\PC^-}
% Causal Irreversibility
\newcommand{\CI}		{\Xi}
\newcommand{\ReverseMap}	{r}
\newcommand{\ForwardMap}	{f}

% Abbreviations from IB:
% None that aren't already in CMPPSS

% Abbreviations from Extensive Estimation:
\newcommand{\EstCausalState}	{\widehat{\CausalState}}
\newcommand{\estcausalstate}	{\widehat{\causalstate}}
\newcommand{\EstCausalStateSet}	{\boldsymbol{\EstCausalState}}
\newcommand{\EstCausalFunc}	{\widehat{\epsilon}}
\newcommand{\EstCmu}		{\widehat{\Cmu}}
\newcommand{\PastLOne}	{{\Past}^{L+1}}
\newcommand{\pastLOne}	{{\past}^{L+1}}

% Abbreviations from $\epsilon$-Transducers:
\newcommand{\InAlphabet}	{ \mathcal{A}}
\newcommand{\insymbol}		{ a}
\newcommand{\OutAlphabet}	{ \mathcal{B}}
\newcommand{\outsymbol}		{ b}
\newcommand{\InputSimple}	{ X}
\newcommand{\inputsimple}	{ x}
\newcommand{\BottleneckVar}	{\tilde{\InputSimple}}
\newcommand{\bottleneckvar}	{\tilde{\inputsimple}}
\newcommand{\InputSpace}	{ \mathbf{\InputSimple}}
\newcommand{\InputBi}	{ \overleftrightarrow {\InputSimple} }
\newcommand{\inputbi}	{ \overleftrightarrow {\inputsimple} }
\newcommand{\InputPast}	{ \overleftarrow {\InputSimple} }
\newcommand{\inputpast}	{ \overleftarrow {\inputsimple} }
\newcommand{\InputFuture}	{ \overrightarrow {\InputSimple} }
\newcommand{\inputfuture}	{ \overrightarrow {\inputsimple} }
\newcommand{\NextInput}	{ {{\InputFuture}^{1}}}
\newcommand{\NextOutput}	{ {\OutputFuture}^{1}}
\newcommand{\OutputSimple}	{ Y}
\newcommand{\outputsimple}	{ y}
\newcommand{\OutputSpace}	{ \mathbf{\OutputSimple}}
\newcommand{\OutputBi}	{ \overleftrightarrow{\OutputSimple} }
\newcommand{\outputbi}	{ \overleftrightarrow{\outputsimple} }
\newcommand{\OutputPast}	{ \overleftarrow{\OutputSimple} }
\newcommand{\outputpast}	{ \overleftarrow{\outputsimple} }
\newcommand{\OutputFuture}	{ \overrightarrow{\OutputSimple} }
\newcommand{\outputfuture}	{ \overrightarrow{\outputsimple} }
\newcommand{\OutputL}	{ {\OutputFuture}^L}
\newcommand{\outputL}	{ {\outputfuture}^L}
\newcommand{\InputLLessOne}	{ {\InputFuture}^{L-1}}
\newcommand{\inputLlessone}	{ {\inputufutre}^{L-1}}
\newcommand{\OutputPastLLessOne}	{{\OutputPast}^{L-1}_{-1}}
\newcommand{\outputpastLlessone}	{{\outputpast}^{L-1}}
\newcommand{\OutputPastLessOne}	{{\OutputPast}_{-1}}
\newcommand{\outputpastlessone}	{{\outputpast}_{-1}}
\newcommand{\OutputPastL}	{{\OutputPast}^{L}}
\newcommand{\OutputLPlusOne}	{ {\OutputFuture}^{L+1}}
\newcommand{\outputLplusone}	{ {\outputfutre}^{L+1}}
\newcommand{\InputPastL}	{{\InputPast}^{L}}
\newcommand{\inputpastL}	{{\inputpast}^{L}}
\newcommand{\JointPast}	{{(\InputPast,\OutputPast)}}
\newcommand{\jointpast}	{{(\inputpast,\outputpast)}}
\newcommand{\jointpastone}	{{(\inputpast_1,\outputpast_1)}}
\newcommand{\jointpasttwo}	{{(\inputpast_2,\outputpast_2)}}
\newcommand{\jointpastprime} {{({\inputpast}^{\prime},{\outputpast}^{\prime})}}
\newcommand{\NextJoint}	{{(\NextInput,\NextOutput)}}
\newcommand{\nextjoint}	{{(\insymbol,\outsymbol)}}
\newcommand{\AllInputPasts}	{ { \overleftarrow {\rm \InputSpace}}}
\newcommand{\AllOutputPasts}	{ {\overleftarrow {\rm \OutputSpace}}}
\newcommand{\DetCausalState}	{ {{\cal S}_D }}
\newcommand{\detcausalstate}	{ {{\sigma}_D} }
\newcommand{\DetCausalStateSet}	{ \boldsymbol{{\CausalState}_D}}
\newcommand{\DetCausalEquivalence}	{ {\sim}_{{\epsilon}_{D}}}
\newcommand{\PrescientEquivalence}	{ {\sim}_{\widehat{\eta}}}
\newcommand{\FeedbackCausalState}	{ \mathcal{F}}
\newcommand{\feedbackcausalstate}	{ \phi}
\newcommand{\FeedbackCausalStateSet}	{ \mathbf{\FeedbackCausalState}}
\newcommand{\JointCausalState}		{ \mathcal{J}}
\newcommand{\JointCausalStateSet}	{ \mathbf{\JointCausalState}}
\newcommand{\UtilityFunctional}	{ {\mathcal{L}}}
\newcommand{\NatureState}	{ {\Omega}}
\newcommand{\naturestate}	{ {\omega}}
\newcommand{\NatureStateSpace}	{ {\mathbf{\NatureState}}}
\newcommand{\AnAction}	{ {A}}
\newcommand{\anaction}	{ {a}}
\newcommand{\ActionSpace}	{ {\mathbf{\AnAction}}}

% Abbreviations from RURO:
\newcommand{\InfoGain}[2] { \mathcal{D} \left( {#1} || {#2} \right) }

% Abbreviations from Upper Bound:
\newcommand{\lcm}	{{\rm lcm}}
% Double-check that this isn't in the math set already!

% Abbreviations from Emergence in Space
\newcommand{\ProcessAlphabet}	{\MeasAlphabet}
\newcommand{\ProbEst}			{ {\widehat{\Prob}_N}}
\newcommand{\STRegion}			{ {\mathrm K}}
\newcommand{\STRegionVariable}		{ K}
\newcommand{\stregionvariable}		{ k}
\newcommand{\GlobalPast}		{ \overleftarrow{G}} 
\newcommand{\globalpast}		{ \overleftarrow{g}} 
\newcommand{\GlobalFuture}		{ \overrightarrow{G}}
\newcommand{\globalfuture}		{ \overrightarrow{g}}
\newcommand{\GlobalState}		{ \mathcal{G}}
\newcommand{\globalstate}		{ \gamma}
\newcommand{\GlobalStateSet}		{ {\mathbf \GlobalState}}
\newcommand{\LocalPast}			{ \overleftarrow{L}} 
\newcommand{\localpast}			{ \overleftarrow{l}}
\newcommand{\AllLocalPasts}		{ \mathbf{\LocalPast}}
\newcommand{\LocalPastRegion}		{ \overleftarrow{\mathrm L}}
\newcommand{\LocalFuture}		{ \overrightarrow{L}}
\newcommand{\localfuture}		{ \overrightarrow{l}}
\newcommand{\LocalFutureRegion}		{ \overrightarrow{\mathrm L}}
\newcommand{\LocalState}		{ \mathcal{L}}
\newcommand{\localstate}		{ \lambda}
\newcommand{\LocalStateSet}		{ {\mathbf \LocalState}}
\newcommand{\PatchPast}			{ \overleftarrow{P}}
\newcommand{\patchpast}			{ \overleftarrow{p}}
\newcommand{\PatchPastRegion}		{ \overleftarrow{\mathrm P}}
\newcommand{\PatchFuture}		{ \overrightarrow{P}}
\newcommand{\patchfuture}		{ \overrightarrow{p}}
\newcommand{\PatchFutureRegion}		{ \overrightarrow{\mathrm P}}
\newcommand{\PatchState}		{ \mathcal{P}}
\newcommand{\patchstate}		{ \pi}
\newcommand{\PatchStateSet}		{ {\mathbf \PatchState}}
\newcommand{\LocalStatesInPatch}	{\vec{\LocalState}}
\newcommand{\localstatesinpatch}	{\vec{\localstate}}
\newcommand{\PointInstantX}		{ {\mathbf x}}
% Galles's original LaTeX for the cond. indep. symbol follows:
\newcommand{\compos}{\mbox{$~\underline{~\parallel~}~$}}
\newcommand{\ncompos}{\not\hspace{-.15in}\compos}
\newcommand{\indep}			{ \rotatebox{90}{$\models$}}
\newcommand{\nindep}	{\not\hspace{-.05in}\indep}
\newcommand{\LocalEE}	{{\EE}^{loc}}
\newcommand{\EEDensity}	{\overline{\LocalEE}}
\newcommand{\LocalCmu}	{{\Cmu}^{loc}}
\newcommand{\CmuDensity}	{\overline{\LocalCmu}}

%%%%%%%%%%% added by sasa
\newcommand{\FinPast}[1]	{ \overleftarrow {\MeasSymbol} \stackrel{{#1}}{}}
\newcommand{\finpast}[1]  	{ \overleftarrow {\meassymbol}  \stackrel{{#1}}{}}
\newcommand{\FinFuture}[1]		{ \overrightarrow{\MeasSymbol} \stackrel{{#1}}{}}
\newcommand{\finfuture}[1]		{ \overrightarrow{\meassymbol} \stackrel{{#1}}{}}

\newcommand{\Partition}	{ \AlternateState }
\newcommand{\partitionstate}	{ \alternatestate }
\newcommand{\PartitionSet}	{ \AlternateStateSet }
\newcommand{\Fdet}   { F_{\rm det} }

\newcommand{\Dkl}[2] { D_{\rm KL} \left( {#1} || {#2} \right) }

\newcommand{\Period}	{p}

% To take into account time direction
\newcommand{\forward}{+}
\newcommand{\reverse}{-}
%\newcommand{\forwardreverse}{\:\!\diamond} % \pm
\newcommand{\forwardreverse}{\pm} % \pm
\newcommand{\FutureProcess}	{ {\Process}^{\forward} }
\newcommand{\PastProcess}	{ {\Process}^{\reverse} }
\newcommand{\FutureCausalState}	{ {\CausalState}^{\forward} }
\newcommand{\futurecausalstate}	{ \sigma^{\forward} }
\newcommand{\altfuturecausalstate}	{ \sigma^{\forward\prime} }
\newcommand{\PastCausalState}	{ {\CausalState}^{\reverse} }
\newcommand{\pastcausalstate}	{ \sigma^{\reverse} }
\newcommand{\BiCausalState}		{ {\CausalState}^{\forwardreverse} }
\newcommand{\bicausalstate}		{ {\sigma}^{\forwardreverse} }
\newcommand{\FutureCausalStateSet}	{ {\CausalStateSet}^{\forward} }
\newcommand{\PastCausalStateSet}	{ {\CausalStateSet}^{\reverse} }
\newcommand{\BiCausalStateSet}	{ {\CausalStateSet}^{\forwardreverse} }
\newcommand{\eMachine}	{ M }
\newcommand{\FutureEM}	{ {\eMachine}^{\forward} }
\newcommand{\PastEM}	{ {\eMachine}^{\reverse} }
\newcommand{\BiEM}		{ {\eMachine}^{\forwardreverse} }
\newcommand{\BiEquiv}	{ {\sim}^{\forwardreverse} }
\newcommand{\Futurehmu}	{ h_\mu^{\forward} }
\newcommand{\Pasthmu}	{ h_\mu^{\reverse} }
\newcommand{\FutureCmu}	{ C_\mu^{\forward} }
\newcommand{\PastCmu}	{ C_\mu^{\reverse} }
\newcommand{\BiCmu}		{ C_\mu^{\forwardreverse} }
\newcommand{\FutureEps}	{ \epsilon^{\forward} }
\newcommand{\PastEps}	{ \epsilon^{\reverse} }
\newcommand{\BiEps}	{ \epsilon^{\forwardreverse} }
\newcommand{\FutureSim}	{ \sim^{\forward} }
\newcommand{\PastSim}	{ \sim^{\reverse} }
% Used arrows for awhile, more or less confusing?
%\newcommand{\FutureCausalState}	{ \overrightarrow{\CausalState} }
%\newcommand{\PastCausalState}	{ \overleftarrow{\CausalState} }
%\newcommand{\eMachine}	{ M }
%\newcommand{\FutureEM}	{ \overrightarrow{\eMachine} }
%\newcommand{\PastEM}	{ \overleftarrow{\eMachine} }
%\newcommand{\FutureCmu}	{ \overrightarrow{\Cmu} }
%\newcommand{\PastCmu}	{ \overleftarrow{\Cmu} }

%% time-reversing and mixed state presentation operators
\newcommand{\TR}{\mathcal{T}}
\newcommand{\MSP}{\mathcal{U}}
\newcommand{\one}{\mathbf{1}}

%% (cje)
%% Provide a command \ifpm which is true when \pm 
%% is meant to be understood as "+ or -". This is
%% different from the usage in TBA.
\newif\ifpm 
\edef\tempa{\forwardreverse}
\edef\tempb{\pm}
\ifx\tempa\tempb
   \pmfalse
\else
   \pmtrue  
\fi





% [some math stuff - maybe stick in sep file]
\usepackage{amsthm}
\usepackage{amscd}
\theoremstyle{plain}    \newtheorem{Lem}{Lemma}
\theoremstyle{plain}    \newtheorem*{ProLem}{Proof}
\theoremstyle{plain} 	\newtheorem{Cor}{Corollary}
\theoremstyle{plain} 	\newtheorem*{ProCor}{Proof}
\theoremstyle{plain} 	\newtheorem{The}{Theorem}
\theoremstyle{plain} 	\newtheorem*{ProThe}{Proof}
\theoremstyle{plain} 	\newtheorem{Prop}{Proposition}
\theoremstyle{plain} 	\newtheorem*{ProProp}{Proof}
\theoremstyle{plain} 	\newtheorem*{Conj}{Conjecture}
\theoremstyle{plain}	\newtheorem*{Rem}{Remark}
\theoremstyle{plain}	\newtheorem*{Def}{Definition} 
\theoremstyle{plain}	\newtheorem*{Not}{Notation}

% [uniform figure scaling - maybe this is not a good idea]
\def\figscale{.7}
\def\lscale{1.0}

% [FIX ME! - red makes it easier to spot]
\newcommand{\FIX}[1]{\textbf{\textcolor{red}{#1}}}


    \begin{document}
    \def\noprelim{}
\else
    % already included once
    % input post files

    \singlespacing
    \bibliographystyle{../bibliography/expanded}
    \bibliography{../bibliography/references}

    \end{document}
\fi\fi
