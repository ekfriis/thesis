\ifx\master\undefined\input{../settings/autocompile}\fi

\prelimheaders

%%%%%%%%%%%%%%%%%%%%%%%%%%%%%%%%%%%%%%%%%%%%%%%%%%%%%%%%%%%%%%%%%%%%%
\chapter*{How To Read This Thesis}
\label{ch:0}

This thesis draws largely from our series of published papers \FIX{cite tba, pratisp, iacp}. Some additional material is drawn from \FIX{emim, iacplocs, ruro2, iacp2}.

The first chapter, Ch.~\ref{ch:compmech}, provides an introduction to the topic of computational mechanics. It begins with some philosophy and motivation, followed by definitions and some examples of the main constructs. Ultimately it sets the stage for the topics that the remaining chapters address. The reader already familiar with standard computational mechanics might just read the short philosophy section Sec.~\ref{sec:philosophy}.

The next chapter, Ch.~\ref{ch:PRATISP}, contains the main results of \FIX{tba,pratisp}. The topics include forward, reverse and bidirectional representations, closed form calculation of the excess entropy and the introduction of several new stochastic process quantifiers.

Following, Ch.~\ref{ch:IACP} presents the work published in \FIX{iacp}. This is a detailed exploration of crypticity, in particular the cryptic order---the here defined analog of Markov order.

Several appendices are provided which contain some calculational details, and additionally, a short tutorial on information diagrams. As information diagrams are used throughout, this is a good place to start for those not familiar. For students of information theory, this can provide a nice visual reference for several basic concepts.

\ifx\master\undefined\input{../settings/autocompile}\fi
