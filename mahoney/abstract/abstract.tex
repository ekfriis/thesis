\ifx\master\undefined\input{../settings/autocompile}\fi

\prelimheaders

\begin{abstract}
Computational mechanics is a theory that attempts to describe dynamical systems in a universal language. The central coordinating role in this theory is played by the \eM---the minimal unifilar representation that is an optimal predictor of the dynamical system.

Properties of the dynamical system are then simply properties of the \eM. The oldest of these is the entropy rate, or information generation rate, $\hmu$. A second measure, the first real mark of computational mechanics, is the statistical complexity, or information storage---$\Cmu$. A third is the amount of information transmitted from the infinite past to the infinite future, the excess entropy---$\EE$.

The fundamental quantities, $\hmu$ and $\Cmu$, are calculated in closed form directly from the \eM. Despite being straightforward to define, and having simple geometric relationships to common quantities, excess entropy has resisted a closed form calculation. This has made estimation of $\EE$, in some cases, quite problematic.

This thesis describes a novel technique for deriving $\EE$ directly from an $\eM$. In addition to providing an exact quantification of $\EE$, it allows for the calculation of $\EE$ as a function of parameterized classes of $\eMs$. As a theoretical by-product of this technique, several natural quantifiers of stochastic processes have emerged, along with their calculation method. These include: new quantifiers of information storage---reverse and bidirectional statistical complexity; information overhead---directional and bidirectional crypticity; and causal irreversibility.

Of these quantities, the crypticity is treated in most detail. In particular, a new correlation length scale---the cryptic order---is introduced as a natural analog to Markov order. The cryptic order describes the range over which the crypticity exists just as the Markov order describes the time extent of state or synchronization information.

The approaches contained in this thesis are part of a unified effort to delineate a comprehensive theoretical framework for the analysis and understanding of nonlinear dynamical systems. The novel constructs and calculation techniques described herein represent concrete steps in this direction. We anticipate that this work will have impact on any field concerning stochastic processes, both in algorithms and in conceptual framework.

\end{abstract}

\ifx\master\undefined\input{../settings/autocompile}\fi
