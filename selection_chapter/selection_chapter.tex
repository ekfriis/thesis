\ifx\master\undefined\input{settings/autocompile}\fi

\chapter{Analysis Selections}
\label{ch:selections}

Events in the muon plus tau--jet channel are selected by requiring a muon of
$P_{T}^{\mu} > 15$~\GeV within $ \left| \eta_{\mu} \right| < 2.1$ and a tau-jet
candidate of $P_{T}^{\tau\mbox{--jet}} > 20$~\GeV within $ \left|
\eta_{\tau\mbox{--jet}} \right| < 2.3$. The muon and tau--jet candidate are
required to be of opposite charge, as expected for the $\ZTT$ signal.  The muon
is required to be reconstructed as global muon and pass the VBTF muon
identification criteria~\cite{EWK-10-002}. Furthermore, the muon is required to
be isolated with respect to charged hadrons of $P_{T} > 1.0$~\GeV and neutral
electromagnetic objects of $P_{T} > 1.5$~\GeV (reconstructed by the
particle--flow algorithm) within a cone of size $\Delta R = 0.4$ around the muon
direction.  The tau-jet candidate is required to pass the ``medium'' TaNC tau
identification discriminator.

Additional event selection criteria are applied to reduce
contributions of background processes. $Z
\rightarrow \mu^{+} \mu^{-}$ background contributions are largely due
to muons which failed to get reconstructed as global muons and are
misidentified as tau--jet candidates. These muons are typically
isolated and have a large chance to pass the TaNC tau ID discriminators. 
In order to reject this background, a dedicated
discriminator against muons is applied~\cite{PFT-08-001}. Residual
muon background is supressed by rejecting tau-jet candidates which have
a track of $P_{T} > 15$~\GeV and for which the sum of energy deposits
in ECAL plus HCAL is below $0.25 \cdot P$ within a cylinder of
radiusof radius $15$~cm (ECAL) and $25$~cm (HCAL), respectively.
The $t\bar{t}$, W + jet and QCD multi--jet backgrounds are supressed
by cuts on the transverse mass and the Pzeta variable.
Contamination from $Z
\rightarrow \tau^{+} \tau^{-}$ events in which the reconstructed
tau--jet candidate is due to a $\tau \rightarrow e \nu \nu$ decay is
reduced by applying a dedicated tau ID discriminator against
electrons.

The complete set of event selection criteria applied in the muon +
tau--jet channel are summarized in table~\ref{tab:AHtoMuTauEventSelection}.

\begin{table}[t]
\begin{center}

\begin{tabular}{|l|c|}
\hline
\multicolumn{2}{|c|}{Requirement} \\
\hline
\multirow{2}{12mm}{Trigger} & HLT\_Mu9 for MC \\
                            & {\it cf.}\ table~\ref{tab:AHtoMuTauTriggers} for Data \\ 
\hline
\multirow{2}{17mm}{Vertex}  & reconstructed with beam--spot constraint: \\
                           & $-24 < z_{vtx} < +24$~cm, $\left| \rho \right| < 2$~cm, nDoF $> 4$ \\ 
\hline
\multirow{4}{10mm}{Muon}    & reconstructed as global Muon with: \\
                            & $P_{T} > 15$~\GeV, $\vert \eta \vert < 2.1$, VBTF
                            Muon ID passed, \\
                            & isolated within $\Delta R =0.4$ cone with respect to charged hadrons \\
                            & of $P_{T} > 1.0$~\GeV and neutral electromagnetic objects of $P_{T} > 1.5$~\GeV \\ 
\hline
\multirow{4}{23mm}{Tau--jet Candidate} & reconstructed by HPS + TaNC combined Tau ID algorithm \\
                            & TaNC ``medium'' Tau ID discriminator \\
                            & and discriminators against electrons and muons passed, \\
                            & calorimeter muon rejection passed \\
\hline
\multirow{2}{23mm}{Muon + Tau--jet} & charge(Muon) + charge(Tau--jet) = 0, \\
                            & $\Delta R$(Muon, Tau--jet)$ > 0.5$ \\
\hline
\multirow{2}{20mm}{Kinematics} & $M_{T}$(Muon-MET)$ < 40$~\GeV \\
                            & $P_{\zeta} - 1.5 \cdot P_{\zeta}^{vis} > -20$~\GeV \\
\hline
\end{tabular}
\end{center}
\begin{center}
\caption{\captiontext Event selection criteria applied in the muon + tau--jet channel.}
\label{tab:AHtoMuTauEventSelection}
\end{center}
\end{table}

The events are triggered by a combination of muon and muon + tau--jet
``cross--channel'' triggers. For the muon triggers, paths with lowest $P_{T}$
thresholds are used as long as the path remained unprescaled (see
table~\ref{tab:AHtoMuTauTriggers}). 
The muon + tau--jet ``cross--channel'' trigger paths increase the trigger efficiency for
events containing muons of transverse momenta close to the
$P_{T}^{\mu} > 15$~\GeV cut threshold.
The trigger efficiency is measured in data via the tag--and--probe
technique. Details of the muon trigger efficiency measurement are
given in section~\ref{sec:ZmumuTagAndProbe} of the appendix.
Monte Carlo simulated events are required to pass the
HLT\_Mu9 trigger path. Weights are applied to simulated events to account for the
difference between the simulated HLT\_Mu9 efficiency and the combined
efficiency of the set HLT\_Mu9, HLT\_IsoMu9, HLT\_Mu11, HLT\_IsoMu13,
HLT\_Mu15, HLT\_IsoMu9\_PFTau15 and HLT\_Mu11\_PFTau15 used to trigger
the data.

\begin{table}[t]
\begin{center}

\begin{tabular}{|l|c|}
\hline
Trigger path & run--range \\
\hline
HLT\_Mu9             & 132440 - 147116 \\
HLT\_IsoMu9          & 147196 - 148058 \\
HLT\_Mu11            & 147196 - 148058 \\
HLT\_Mu15            & 147196 - 149442 \\
HLT\_IsoMu13         & 148822 - 149182 \\
HLT\_IsoMu9\_PFTau15 & 148822 - 149182 \\
HLT\_Mu11\_PFTau15   & 148822 - 149182 \\
\hline
\end{tabular}
\end{center}
\begin{center}
\caption{\captiontext Muon and muon + tau--jet ``cross--channel''
  trigger paths utilized to trigger events in the muon + tau--jet channel in different data--taking periods.}
\label{tab:AHtoMuTauTriggers}
\end{center}
\end{table}

\ifx\master\undefined\input{settings/autocompile}\fi
