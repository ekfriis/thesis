\ifx\master\undefined% AUTOCOMPILE
% Allows for individual chapters to be compiled.

% Usage:
% \ifx\master\undefined% AUTOCOMPILE
% Allows for individual chapters to be compiled.

% Usage:
% \ifx\master\undefined% AUTOCOMPILE
% Allows for individual chapters to be compiled.

% Usage:
% \ifx\master\undefined\input{../settings/autocompile}\fi
% (place at start and end of chapter file)

\ifx\noprelim\undefined
    % first time included
    % input preamble files
    \input{../settings/phdsetup}

    \begin{document}
    \def\noprelim{}
\else
    % already included once
    % input post files

    \singlespacing
    \bibliographystyle{../bibliography/expanded}
    \bibliography{../bibliography/references}

    \end{document}
\fi\fi
% (place at start and end of chapter file)

\ifx\noprelim\undefined
    % first time included
    % input preamble files
    % [ USER VARIABLES ]

\def\PHDTITLE {Extensions of the Theory of Computational Mechanics}
\def\PHDAUTHOR{Evan Klose Friis}
\def\PHDSCHOOL{University of California, Davis}

\def\PHDMONTH {June}
\def\PHDYEAR  {2011}
\def\PHDDEPT {Physics}

\def\BSSCHOOL {University of California at San Diego}
\def\BSYEAR   {2005}

\def\PHDCOMMITTEEA{Professor John Conway}
\def\PHDCOMMITTEEB{Professor Robin Erbacher}
\def\PHDCOMMITTEEC{Professor Mani Tripathi}

% [ GLOBAL SETUP ]

\documentclass[letterpaper,oneside,11pt]{report}

\usepackage{calc}
\usepackage{breakcites}
\usepackage[newcommands]{ragged2e}

\usepackage[pdftex]{graphicx}
\usepackage{epstopdf}

%\usepackage{tikz}
%\usetikzlibrary{positioning} % [right=of ...]
%\usetikzlibrary{fit} % [fit= ...]

%\pgfdeclarelayer{background layer}
%\pgfdeclarelayer{foreground layer}
%\pgfsetlayers{background layer,main,foreground layer}

%\newenvironment{wrap}{\noindent\begin{minipage}[t]{\linewidth}\vspace{-0.5\normalbaselineskip}\centering}{\vspace{0.5\normalbaselineskip}\end{minipage}}

%% [Venn diagram environment]
%\newenvironment{venn2}
%{\begin{tikzpicture} [every pin/.style={text=black, text opacity=1.0, pin distance=0.5cm, pin edge={black!60, semithick}},
%% define a new style 'venn'
%venn/.style={circle, draw=black!60, semithick, minimum size = 4cm}]
%
%% create circle and give it external (pin) label
%\node[venn] (X) at (-1,0) [pin={150:$H[X]$}] {};
%\node[venn] (Y) at (1,0) [pin={30:$H[Y]$}] {};
%
%% place labels of the atoms by hand
%\node at (-1.9,0) {$H[X|Y]$};
%\node at (1.9,0) {$H[Y|X]$};
%\node at (0,0) {$I[X;Y]$};}
%{\end{tikzpicture}}

%\newcommand{\wrapmath}[1]{\begin{wrap}\begin{tikzpicture}[every node/.style={inner ysep=0ex, inner xsep=0em}]\node[] {$\displaystyle\begin{aligned} #1\end{aligned}$};\end{tikzpicture}\end{wrap}}

\renewenvironment{abstract}{\chapter*{Abstract}}{}
\renewcommand{\bibname}{Bibliography}
\renewcommand{\contentsname}{Table of Contents}

\makeatletter
\renewcommand{\@biblabel}[1]{\textsc{#1}}
\makeatother

% [ FONT SETTINGS ]

\usepackage[tbtags, intlimits, namelimits]{amsmath}
\usepackage[adobe-utopia]{mathdesign}

\DeclareSymbolFont{pazomath}{OMS}{zplm}{m}{n}
\DeclareSymbolFontAlphabet{\mathcal}{pazomath}
\SetMathAlphabet\mathcal{bold}{OMS}{zplm}{b}{n}

\SetSymbolFont{largesymbols}{normal}{OMX}{zplm}{m}{n}
\SetSymbolFont{largesymbols}{bold}{OMX}{zplm}{m}{n}
\SetSymbolFont{symbols}{normal}{OMS}{zplm}{m}{n}
\SetSymbolFont{symbols}{bold}{OMS}{zplm}{b}{n}

\renewcommand{\sfdefault}{phv}
\renewcommand{\ttdefault}{fvm}

\widowpenalty 8000
\clubpenalty  8000

% [ PAGE LAYOUT ]
\usepackage{geometry}
\geometry{lmargin = 1.5in}
\geometry{rmargin = 1.0in}
\geometry{tmargin = 1.0in}
\geometry{bmargin = 1.0in}

% [ PDF SETTINGS ]

\usepackage[final]{hyperref}
\hypersetup{breaklinks  = true}
\hypersetup{colorlinks  = true}
\hypersetup{linktocpage = false}
\hypersetup{linkcolor   = blue}
\hypersetup{citecolor   = green}
\hypersetup{urlcolor    = black}
\hypersetup{plainpages  = false}
\hypersetup{pageanchor  = true}
\hypersetup{pdfauthor   = {\PHDAUTHOR}}
\hypersetup{pdftitle    = {\PHDTITLE}}
\hypersetup{pdfsubject  = {Dissertation, \PHDSCHOOL}}
\urlstyle{same}

% [ LETTER SPACING ]

\usepackage[final]{microtype}
\microtypesetup{protrusion=compatibility}
\microtypesetup{expansion=false}

\newcommand{\upper}[1]{\MakeUppercase{#1}}
\let\lsscshape\scshape

\ifcase\pdfoutput\else\microtypesetup{letterspace=15}
\renewcommand{\scshape}{\lsscshape\lsstyle}
\renewcommand{\upper}[1]{\textls[50]{\MakeUppercase{#1}}}\fi

% [ LINE SPACING ]

\usepackage[doublespacing]{setspace}
\renewcommand{\displayskipstretch}{0.0}

\setlength{\parskip   }{0em}
\setlength{\parindent }{2em}

% [ TABLE FORMATTING ]

\usepackage{booktabs}
\setlength{\heavyrulewidth}{1.5\arrayrulewidth}
\setlength{\lightrulewidth}{1.0\arrayrulewidth}
\setlength{\doublerulesep }{2.0\arrayrulewidth}

% [ SECTION FORMATTING ]

\usepackage[largestsep,nobottomtitles*]{titlesec}
\renewcommand{\bottomtitlespace}{0.75in}

\titleformat{\chapter}[display]{\bfseries\huge\singlespacing}{\filleft\textsc{\LARGE \chaptertitlename\ \thechapter}}{-0.2ex}{\titlerule[3pt]\vspace{0.2ex}}[]

\titleformat{\section}{\LARGE}{\S\thesection\hspace{0.5em}}{0ex}{}
\titleformat{\subsection}{\Large}{\S\thesubsection\hspace{0.5em}}{0ex}{}
\titleformat{\subsubsection}{\large}{\thesubsubsection\hspace{0.5em}}{0ex}{}

\titlespacing*{\chapter}{0em}{6ex}{4ex plus 2ex minus 0ex}
\titlespacing*{\section}{0em}{2ex plus 3ex minus 1ex}{0.5ex plus 0.5ex minus 0.5ex}
\titlespacing*{\subsection}{0ex}{2ex plus 3ex minus 1ex}{0ex}
\titlespacing*{\subsubsection}{0ex}{2ex plus 0ex minus 1ex}{0ex}

% [ HEADER SETTINGS ]

\usepackage{fancyhdr}

\setlength{\headheight}{\normalbaselineskip}
\setlength{\footskip  }{0.5in}
\setlength{\headsep   }{0.5in-\headheight}

\fancyheadoffset[R]{0.5in}
\renewcommand{\headrulewidth}{0pt}
\renewcommand{\footrulewidth}{0pt}

\newcommand{\pagebox}{\parbox[r][\headheight][t]{0.5in}{\hspace\fill\thepage}}

\newcommand{\prelimheaders}{\ifx\prelim\undefined\renewcommand{\thepage}{\textit{\roman{page}}}\fancypagestyle{plain}{\fancyhf{}\fancyfoot[L]{\makebox[\textwidth-0.5in]{\thepage}}}\pagestyle{plain}\def\prelim{}\fi}

\newcommand{\normalheaders}{\renewcommand{\thepage}{\arabic{page}}\fancypagestyle{plain}{\fancyhf{}\fancyhead[R]{\pagebox}}\pagestyle{plain}}

\normalheaders{}

% [ CUSTOM COMMANDS ]

\newcommand{\signaturebox}[1]{\multicolumn{1}{p{4in}}{\vspace{3ex}}\\\midrule #1\\}

%\input{../includes/cmechabbrev}

% [some math stuff - maybe stick in sep file]
\usepackage{amsthm}
\usepackage{amscd}
\theoremstyle{plain}    \newtheorem{Lem}{Lemma}
\theoremstyle{plain}    \newtheorem*{ProLem}{Proof}
\theoremstyle{plain} 	\newtheorem{Cor}{Corollary}
\theoremstyle{plain} 	\newtheorem*{ProCor}{Proof}
\theoremstyle{plain} 	\newtheorem{The}{Theorem}
\theoremstyle{plain} 	\newtheorem*{ProThe}{Proof}
\theoremstyle{plain} 	\newtheorem{Prop}{Proposition}
\theoremstyle{plain} 	\newtheorem*{ProProp}{Proof}
\theoremstyle{plain} 	\newtheorem*{Conj}{Conjecture}
\theoremstyle{plain}	\newtheorem*{Rem}{Remark}
\theoremstyle{plain}	\newtheorem*{Def}{Definition} 
\theoremstyle{plain}	\newtheorem*{Not}{Notation}

% [uniform figure scaling - maybe this is not a good idea]
\def\figscale{.7}
\def\lscale{1.0}

% [FIX ME! - red makes it easier to spot]
\newcommand{\FIX}[1]{\textbf{\textcolor{red}{#1}}}


    \begin{document}
    \def\noprelim{}
\else
    % already included once
    % input post files

    \singlespacing
    \bibliographystyle{../bibliography/expanded}
    \bibliography{../bibliography/references}

    \end{document}
\fi\fi
% (place at start and end of chapter file)

\ifx\noprelim\undefined
    % first time included
    % input preamble files
    % [ USER VARIABLES ]

\def\PHDTITLE {Extensions of the Theory of Computational Mechanics}
\def\PHDAUTHOR{Evan Klose Friis}
\def\PHDSCHOOL{University of California, Davis}

\def\PHDMONTH {June}
\def\PHDYEAR  {2011}
\def\PHDDEPT {Physics}

\def\BSSCHOOL {University of California at San Diego}
\def\BSYEAR   {2005}

\def\PHDCOMMITTEEA{Professor John Conway}
\def\PHDCOMMITTEEB{Professor Robin Erbacher}
\def\PHDCOMMITTEEC{Professor Mani Tripathi}

% [ GLOBAL SETUP ]

\documentclass[letterpaper,oneside,11pt]{report}

\usepackage{calc}
\usepackage{breakcites}
\usepackage[newcommands]{ragged2e}

\usepackage[pdftex]{graphicx}
\usepackage{epstopdf}

%\usepackage{tikz}
%\usetikzlibrary{positioning} % [right=of ...]
%\usetikzlibrary{fit} % [fit= ...]

%\pgfdeclarelayer{background layer}
%\pgfdeclarelayer{foreground layer}
%\pgfsetlayers{background layer,main,foreground layer}

%\newenvironment{wrap}{\noindent\begin{minipage}[t]{\linewidth}\vspace{-0.5\normalbaselineskip}\centering}{\vspace{0.5\normalbaselineskip}\end{minipage}}

%% [Venn diagram environment]
%\newenvironment{venn2}
%{\begin{tikzpicture} [every pin/.style={text=black, text opacity=1.0, pin distance=0.5cm, pin edge={black!60, semithick}},
%% define a new style 'venn'
%venn/.style={circle, draw=black!60, semithick, minimum size = 4cm}]
%
%% create circle and give it external (pin) label
%\node[venn] (X) at (-1,0) [pin={150:$H[X]$}] {};
%\node[venn] (Y) at (1,0) [pin={30:$H[Y]$}] {};
%
%% place labels of the atoms by hand
%\node at (-1.9,0) {$H[X|Y]$};
%\node at (1.9,0) {$H[Y|X]$};
%\node at (0,0) {$I[X;Y]$};}
%{\end{tikzpicture}}

%\newcommand{\wrapmath}[1]{\begin{wrap}\begin{tikzpicture}[every node/.style={inner ysep=0ex, inner xsep=0em}]\node[] {$\displaystyle\begin{aligned} #1\end{aligned}$};\end{tikzpicture}\end{wrap}}

\renewenvironment{abstract}{\chapter*{Abstract}}{}
\renewcommand{\bibname}{Bibliography}
\renewcommand{\contentsname}{Table of Contents}

\makeatletter
\renewcommand{\@biblabel}[1]{\textsc{#1}}
\makeatother

% [ FONT SETTINGS ]

\usepackage[tbtags, intlimits, namelimits]{amsmath}
\usepackage[adobe-utopia]{mathdesign}

\DeclareSymbolFont{pazomath}{OMS}{zplm}{m}{n}
\DeclareSymbolFontAlphabet{\mathcal}{pazomath}
\SetMathAlphabet\mathcal{bold}{OMS}{zplm}{b}{n}

\SetSymbolFont{largesymbols}{normal}{OMX}{zplm}{m}{n}
\SetSymbolFont{largesymbols}{bold}{OMX}{zplm}{m}{n}
\SetSymbolFont{symbols}{normal}{OMS}{zplm}{m}{n}
\SetSymbolFont{symbols}{bold}{OMS}{zplm}{b}{n}

\renewcommand{\sfdefault}{phv}
\renewcommand{\ttdefault}{fvm}

\widowpenalty 8000
\clubpenalty  8000

% [ PAGE LAYOUT ]
\usepackage{geometry}
\geometry{lmargin = 1.5in}
\geometry{rmargin = 1.0in}
\geometry{tmargin = 1.0in}
\geometry{bmargin = 1.0in}

% [ PDF SETTINGS ]

\usepackage[final]{hyperref}
\hypersetup{breaklinks  = true}
\hypersetup{colorlinks  = true}
\hypersetup{linktocpage = false}
\hypersetup{linkcolor   = blue}
\hypersetup{citecolor   = green}
\hypersetup{urlcolor    = black}
\hypersetup{plainpages  = false}
\hypersetup{pageanchor  = true}
\hypersetup{pdfauthor   = {\PHDAUTHOR}}
\hypersetup{pdftitle    = {\PHDTITLE}}
\hypersetup{pdfsubject  = {Dissertation, \PHDSCHOOL}}
\urlstyle{same}

% [ LETTER SPACING ]

\usepackage[final]{microtype}
\microtypesetup{protrusion=compatibility}
\microtypesetup{expansion=false}

\newcommand{\upper}[1]{\MakeUppercase{#1}}
\let\lsscshape\scshape

\ifcase\pdfoutput\else\microtypesetup{letterspace=15}
\renewcommand{\scshape}{\lsscshape\lsstyle}
\renewcommand{\upper}[1]{\textls[50]{\MakeUppercase{#1}}}\fi

% [ LINE SPACING ]

\usepackage[doublespacing]{setspace}
\renewcommand{\displayskipstretch}{0.0}

\setlength{\parskip   }{0em}
\setlength{\parindent }{2em}

% [ TABLE FORMATTING ]

\usepackage{booktabs}
\setlength{\heavyrulewidth}{1.5\arrayrulewidth}
\setlength{\lightrulewidth}{1.0\arrayrulewidth}
\setlength{\doublerulesep }{2.0\arrayrulewidth}

% [ SECTION FORMATTING ]

\usepackage[largestsep,nobottomtitles*]{titlesec}
\renewcommand{\bottomtitlespace}{0.75in}

\titleformat{\chapter}[display]{\bfseries\huge\singlespacing}{\filleft\textsc{\LARGE \chaptertitlename\ \thechapter}}{-0.2ex}{\titlerule[3pt]\vspace{0.2ex}}[]

\titleformat{\section}{\LARGE}{\S\thesection\hspace{0.5em}}{0ex}{}
\titleformat{\subsection}{\Large}{\S\thesubsection\hspace{0.5em}}{0ex}{}
\titleformat{\subsubsection}{\large}{\thesubsubsection\hspace{0.5em}}{0ex}{}

\titlespacing*{\chapter}{0em}{6ex}{4ex plus 2ex minus 0ex}
\titlespacing*{\section}{0em}{2ex plus 3ex minus 1ex}{0.5ex plus 0.5ex minus 0.5ex}
\titlespacing*{\subsection}{0ex}{2ex plus 3ex minus 1ex}{0ex}
\titlespacing*{\subsubsection}{0ex}{2ex plus 0ex minus 1ex}{0ex}

% [ HEADER SETTINGS ]

\usepackage{fancyhdr}

\setlength{\headheight}{\normalbaselineskip}
\setlength{\footskip  }{0.5in}
\setlength{\headsep   }{0.5in-\headheight}

\fancyheadoffset[R]{0.5in}
\renewcommand{\headrulewidth}{0pt}
\renewcommand{\footrulewidth}{0pt}

\newcommand{\pagebox}{\parbox[r][\headheight][t]{0.5in}{\hspace\fill\thepage}}

\newcommand{\prelimheaders}{\ifx\prelim\undefined\renewcommand{\thepage}{\textit{\roman{page}}}\fancypagestyle{plain}{\fancyhf{}\fancyfoot[L]{\makebox[\textwidth-0.5in]{\thepage}}}\pagestyle{plain}\def\prelim{}\fi}

\newcommand{\normalheaders}{\renewcommand{\thepage}{\arabic{page}}\fancypagestyle{plain}{\fancyhf{}\fancyhead[R]{\pagebox}}\pagestyle{plain}}

\normalheaders{}

% [ CUSTOM COMMANDS ]

\newcommand{\signaturebox}[1]{\multicolumn{1}{p{4in}}{\vspace{3ex}}\\\midrule #1\\}

%%%% macros fro standard references
%\eqref provided by amsmath
\newcommand{\figref}[1]{Fig.~\ref{#1}}
\newcommand{\tableref}[1]{Table~\ref{#1}}
\newcommand{\refcite}[1]{Ref.~\cite{#1}}

% Abbreviations from CMPPSS:

\newcommand{\eM}     {\mbox{$\epsilon$-machine}}
\newcommand{\eMs}    {\mbox{$\epsilon$-machines}}
\newcommand{\EM}     {\mbox{$\epsilon$-Machine}}
\newcommand{\EMs}    {\mbox{$\epsilon$-Machines}}
\newcommand{\eT}     {\mbox{$\epsilon$-transducer}}
\newcommand{\eTs}    {\mbox{$\epsilon$-transducers}}
\newcommand{\ET}     {\mbox{$\epsilon$-Transducer}}
\newcommand{\ETs}    {\mbox{$\epsilon$-Transducers}}

% Processes and sequences

\newcommand{\Process}{\mathcal{P}}

\newcommand{\ProbMach}{\Prob_{\mathrm{M}}}
\newcommand{\Lmax}   { {L_{\mathrm{max}}}}
\newcommand{\MeasAlphabet}	{\mathcal{A}}
% Original
%\newcommand{\MeasSymbol}   { {S} }
%\newcommand{\meassymbol}   { {s} }
% New symbol
\newcommand{\MeasSymbol}   { {X} }
\newcommand{\meassymbol}   { {x} }
\newcommand{\BiInfinity}	{ \overleftrightarrow {\MeasSymbol} }
\newcommand{\biinfinity}	{ \overleftrightarrow {\meassymbol} }
\newcommand{\Past}	{ \overleftarrow {\MeasSymbol} }
\newcommand{\past}	{ {\overleftarrow {\meassymbol}} }
\newcommand{\pastprime}	{ {\past}^{\prime}}
\newcommand{\Future}	{ \overrightarrow{\MeasSymbol} }
\newcommand{\future}	{ \overrightarrow{\meassymbol} }
\newcommand{\futureprime}	{ {\future}^{\prime}}
\newcommand{\PastPrime}	{ {\Past}^{\prime}}
\newcommand{\FuturePrime}	{ {\overrightarrow{\meassymbol}}^\prime }
\newcommand{\PastDblPrime}	{ {\overleftarrow{\meassymbol}}^{\prime\prime} }
\newcommand{\FutureDblPrime}	{ {\overrightarrow{\meassymbol}}^{\prime\prime} }
\newcommand{\pastL}	{ {\overleftarrow {\meassymbol}}{}^L }
\newcommand{\PastL}	{ {\overleftarrow {\MeasSymbol}}{}^L }
\newcommand{\PastLt}	{ {\overleftarrow {\MeasSymbol}}_t^L }
\newcommand{\PastLLessOne}	{ {\overleftarrow {\MeasSymbol}}^{L-1} }
\newcommand{\futureL}	{ {\overrightarrow{\meassymbol}}{}^L }
\newcommand{\FutureL}	{ {\overrightarrow{\MeasSymbol}}{}^L }
\newcommand{\FutureLt}	{ {\overrightarrow{\MeasSymbol}}_t^L }
\newcommand{\FutureLLessOne}	{ {\overrightarrow{\MeasSymbol}}^{L-1} }
\newcommand{\pastLprime}	{ {\overleftarrow {\meassymbol}}^{L^\prime} }
\newcommand{\futureLprime}	{ {\overrightarrow{\meassymbol}}^{L^\prime} }
\newcommand{\AllPasts}	{ { \overleftarrow {\rm {\bf \MeasSymbol}} } }
\newcommand{\AllFutures}	{ \overrightarrow {\rm {\bf \MeasSymbol}} }
\newcommand{\FutureSet}	{ \overrightarrow{\bf \MeasSymbol}}

% Causal states and epsilon-machines
\newcommand{\CausalState}	{ \mathcal{S} }
\newcommand{\CausalStatePrime}	{ {\CausalState}^{\prime}}
\newcommand{\causalstate}	{ \sigma }
\newcommand{\CausalStateSet}	{ \boldsymbol{\CausalState} }
\newcommand{\AlternateState}	{ \mathcal{R} }
\newcommand{\AlternateStatePrime}	{ {\cal R}^{\prime} }
\newcommand{\alternatestate}	{ \rho }
\newcommand{\alternatestateprime}	{ {\rho^{\prime}} }
\newcommand{\AlternateStateSet}	{ \boldsymbol{\AlternateState} }
\newcommand{\PrescientState}	{ \widehat{\AlternateState} }
\newcommand{\prescientstate}	{ \widehat{\alternatestate} }
\newcommand{\PrescientStateSet}	{ \boldsymbol{\PrescientState}}
\newcommand{\CausalEquivalence}	{ {\sim}_{\epsilon} }
\newcommand{\CausalEquivalenceNot}	{ {\not \sim}_{\epsilon}}

\newcommand{\NonCausalEquivalence}	{ {\sim}_{\eta} }
\newcommand{\NextObservable}	{ {\overrightarrow {\MeasSymbol}}^1 }
\newcommand{\LastObservable}	{ {\overleftarrow {\MeasSymbol}}^1 }
%\newcommand{\Prob}		{ {\rm P}}
\newcommand{\Prob}      {\Pr} % use standard command
\newcommand{\ProbAnd}	{ {,\;} }
\newcommand{\LLimit}	{ {L \rightarrow \infty}}
\newcommand{\Cmu}		{C_\mu}
\newcommand{\hmu}		{h_\mu}
\newcommand{\EE}		{{\bf E}}
\newcommand{\Measurable}{{\bf \mu}}

% Process Crypticity
\newcommand{\PC}		{\chi}
\newcommand{\FuturePC}		{\PC^+}
\newcommand{\PastPC}		{\PC^-}
% Causal Irreversibility
\newcommand{\CI}		{\Xi}
\newcommand{\ReverseMap}	{r}
\newcommand{\ForwardMap}	{f}

% Abbreviations from IB:
% None that aren't already in CMPPSS

% Abbreviations from Extensive Estimation:
\newcommand{\EstCausalState}	{\widehat{\CausalState}}
\newcommand{\estcausalstate}	{\widehat{\causalstate}}
\newcommand{\EstCausalStateSet}	{\boldsymbol{\EstCausalState}}
\newcommand{\EstCausalFunc}	{\widehat{\epsilon}}
\newcommand{\EstCmu}		{\widehat{\Cmu}}
\newcommand{\PastLOne}	{{\Past}^{L+1}}
\newcommand{\pastLOne}	{{\past}^{L+1}}

% Abbreviations from $\epsilon$-Transducers:
\newcommand{\InAlphabet}	{ \mathcal{A}}
\newcommand{\insymbol}		{ a}
\newcommand{\OutAlphabet}	{ \mathcal{B}}
\newcommand{\outsymbol}		{ b}
\newcommand{\InputSimple}	{ X}
\newcommand{\inputsimple}	{ x}
\newcommand{\BottleneckVar}	{\tilde{\InputSimple}}
\newcommand{\bottleneckvar}	{\tilde{\inputsimple}}
\newcommand{\InputSpace}	{ \mathbf{\InputSimple}}
\newcommand{\InputBi}	{ \overleftrightarrow {\InputSimple} }
\newcommand{\inputbi}	{ \overleftrightarrow {\inputsimple} }
\newcommand{\InputPast}	{ \overleftarrow {\InputSimple} }
\newcommand{\inputpast}	{ \overleftarrow {\inputsimple} }
\newcommand{\InputFuture}	{ \overrightarrow {\InputSimple} }
\newcommand{\inputfuture}	{ \overrightarrow {\inputsimple} }
\newcommand{\NextInput}	{ {{\InputFuture}^{1}}}
\newcommand{\NextOutput}	{ {\OutputFuture}^{1}}
\newcommand{\OutputSimple}	{ Y}
\newcommand{\outputsimple}	{ y}
\newcommand{\OutputSpace}	{ \mathbf{\OutputSimple}}
\newcommand{\OutputBi}	{ \overleftrightarrow{\OutputSimple} }
\newcommand{\outputbi}	{ \overleftrightarrow{\outputsimple} }
\newcommand{\OutputPast}	{ \overleftarrow{\OutputSimple} }
\newcommand{\outputpast}	{ \overleftarrow{\outputsimple} }
\newcommand{\OutputFuture}	{ \overrightarrow{\OutputSimple} }
\newcommand{\outputfuture}	{ \overrightarrow{\outputsimple} }
\newcommand{\OutputL}	{ {\OutputFuture}^L}
\newcommand{\outputL}	{ {\outputfuture}^L}
\newcommand{\InputLLessOne}	{ {\InputFuture}^{L-1}}
\newcommand{\inputLlessone}	{ {\inputufutre}^{L-1}}
\newcommand{\OutputPastLLessOne}	{{\OutputPast}^{L-1}_{-1}}
\newcommand{\outputpastLlessone}	{{\outputpast}^{L-1}}
\newcommand{\OutputPastLessOne}	{{\OutputPast}_{-1}}
\newcommand{\outputpastlessone}	{{\outputpast}_{-1}}
\newcommand{\OutputPastL}	{{\OutputPast}^{L}}
\newcommand{\OutputLPlusOne}	{ {\OutputFuture}^{L+1}}
\newcommand{\outputLplusone}	{ {\outputfutre}^{L+1}}
\newcommand{\InputPastL}	{{\InputPast}^{L}}
\newcommand{\inputpastL}	{{\inputpast}^{L}}
\newcommand{\JointPast}	{{(\InputPast,\OutputPast)}}
\newcommand{\jointpast}	{{(\inputpast,\outputpast)}}
\newcommand{\jointpastone}	{{(\inputpast_1,\outputpast_1)}}
\newcommand{\jointpasttwo}	{{(\inputpast_2,\outputpast_2)}}
\newcommand{\jointpastprime} {{({\inputpast}^{\prime},{\outputpast}^{\prime})}}
\newcommand{\NextJoint}	{{(\NextInput,\NextOutput)}}
\newcommand{\nextjoint}	{{(\insymbol,\outsymbol)}}
\newcommand{\AllInputPasts}	{ { \overleftarrow {\rm \InputSpace}}}
\newcommand{\AllOutputPasts}	{ {\overleftarrow {\rm \OutputSpace}}}
\newcommand{\DetCausalState}	{ {{\cal S}_D }}
\newcommand{\detcausalstate}	{ {{\sigma}_D} }
\newcommand{\DetCausalStateSet}	{ \boldsymbol{{\CausalState}_D}}
\newcommand{\DetCausalEquivalence}	{ {\sim}_{{\epsilon}_{D}}}
\newcommand{\PrescientEquivalence}	{ {\sim}_{\widehat{\eta}}}
\newcommand{\FeedbackCausalState}	{ \mathcal{F}}
\newcommand{\feedbackcausalstate}	{ \phi}
\newcommand{\FeedbackCausalStateSet}	{ \mathbf{\FeedbackCausalState}}
\newcommand{\JointCausalState}		{ \mathcal{J}}
\newcommand{\JointCausalStateSet}	{ \mathbf{\JointCausalState}}
\newcommand{\UtilityFunctional}	{ {\mathcal{L}}}
\newcommand{\NatureState}	{ {\Omega}}
\newcommand{\naturestate}	{ {\omega}}
\newcommand{\NatureStateSpace}	{ {\mathbf{\NatureState}}}
\newcommand{\AnAction}	{ {A}}
\newcommand{\anaction}	{ {a}}
\newcommand{\ActionSpace}	{ {\mathbf{\AnAction}}}

% Abbreviations from RURO:
\newcommand{\InfoGain}[2] { \mathcal{D} \left( {#1} || {#2} \right) }

% Abbreviations from Upper Bound:
\newcommand{\lcm}	{{\rm lcm}}
% Double-check that this isn't in the math set already!

% Abbreviations from Emergence in Space
\newcommand{\ProcessAlphabet}	{\MeasAlphabet}
\newcommand{\ProbEst}			{ {\widehat{\Prob}_N}}
\newcommand{\STRegion}			{ {\mathrm K}}
\newcommand{\STRegionVariable}		{ K}
\newcommand{\stregionvariable}		{ k}
\newcommand{\GlobalPast}		{ \overleftarrow{G}} 
\newcommand{\globalpast}		{ \overleftarrow{g}} 
\newcommand{\GlobalFuture}		{ \overrightarrow{G}}
\newcommand{\globalfuture}		{ \overrightarrow{g}}
\newcommand{\GlobalState}		{ \mathcal{G}}
\newcommand{\globalstate}		{ \gamma}
\newcommand{\GlobalStateSet}		{ {\mathbf \GlobalState}}
\newcommand{\LocalPast}			{ \overleftarrow{L}} 
\newcommand{\localpast}			{ \overleftarrow{l}}
\newcommand{\AllLocalPasts}		{ \mathbf{\LocalPast}}
\newcommand{\LocalPastRegion}		{ \overleftarrow{\mathrm L}}
\newcommand{\LocalFuture}		{ \overrightarrow{L}}
\newcommand{\localfuture}		{ \overrightarrow{l}}
\newcommand{\LocalFutureRegion}		{ \overrightarrow{\mathrm L}}
\newcommand{\LocalState}		{ \mathcal{L}}
\newcommand{\localstate}		{ \lambda}
\newcommand{\LocalStateSet}		{ {\mathbf \LocalState}}
\newcommand{\PatchPast}			{ \overleftarrow{P}}
\newcommand{\patchpast}			{ \overleftarrow{p}}
\newcommand{\PatchPastRegion}		{ \overleftarrow{\mathrm P}}
\newcommand{\PatchFuture}		{ \overrightarrow{P}}
\newcommand{\patchfuture}		{ \overrightarrow{p}}
\newcommand{\PatchFutureRegion}		{ \overrightarrow{\mathrm P}}
\newcommand{\PatchState}		{ \mathcal{P}}
\newcommand{\patchstate}		{ \pi}
\newcommand{\PatchStateSet}		{ {\mathbf \PatchState}}
\newcommand{\LocalStatesInPatch}	{\vec{\LocalState}}
\newcommand{\localstatesinpatch}	{\vec{\localstate}}
\newcommand{\PointInstantX}		{ {\mathbf x}}
% Galles's original LaTeX for the cond. indep. symbol follows:
\newcommand{\compos}{\mbox{$~\underline{~\parallel~}~$}}
\newcommand{\ncompos}{\not\hspace{-.15in}\compos}
\newcommand{\indep}			{ \rotatebox{90}{$\models$}}
\newcommand{\nindep}	{\not\hspace{-.05in}\indep}
\newcommand{\LocalEE}	{{\EE}^{loc}}
\newcommand{\EEDensity}	{\overline{\LocalEE}}
\newcommand{\LocalCmu}	{{\Cmu}^{loc}}
\newcommand{\CmuDensity}	{\overline{\LocalCmu}}

%%%%%%%%%%% added by sasa
\newcommand{\FinPast}[1]	{ \overleftarrow {\MeasSymbol} \stackrel{{#1}}{}}
\newcommand{\finpast}[1]  	{ \overleftarrow {\meassymbol}  \stackrel{{#1}}{}}
\newcommand{\FinFuture}[1]		{ \overrightarrow{\MeasSymbol} \stackrel{{#1}}{}}
\newcommand{\finfuture}[1]		{ \overrightarrow{\meassymbol} \stackrel{{#1}}{}}

\newcommand{\Partition}	{ \AlternateState }
\newcommand{\partitionstate}	{ \alternatestate }
\newcommand{\PartitionSet}	{ \AlternateStateSet }
\newcommand{\Fdet}   { F_{\rm det} }

\newcommand{\Dkl}[2] { D_{\rm KL} \left( {#1} || {#2} \right) }

\newcommand{\Period}	{p}

% To take into account time direction
\newcommand{\forward}{+}
\newcommand{\reverse}{-}
%\newcommand{\forwardreverse}{\:\!\diamond} % \pm
\newcommand{\forwardreverse}{\pm} % \pm
\newcommand{\FutureProcess}	{ {\Process}^{\forward} }
\newcommand{\PastProcess}	{ {\Process}^{\reverse} }
\newcommand{\FutureCausalState}	{ {\CausalState}^{\forward} }
\newcommand{\futurecausalstate}	{ \sigma^{\forward} }
\newcommand{\altfuturecausalstate}	{ \sigma^{\forward\prime} }
\newcommand{\PastCausalState}	{ {\CausalState}^{\reverse} }
\newcommand{\pastcausalstate}	{ \sigma^{\reverse} }
\newcommand{\BiCausalState}		{ {\CausalState}^{\forwardreverse} }
\newcommand{\bicausalstate}		{ {\sigma}^{\forwardreverse} }
\newcommand{\FutureCausalStateSet}	{ {\CausalStateSet}^{\forward} }
\newcommand{\PastCausalStateSet}	{ {\CausalStateSet}^{\reverse} }
\newcommand{\BiCausalStateSet}	{ {\CausalStateSet}^{\forwardreverse} }
\newcommand{\eMachine}	{ M }
\newcommand{\FutureEM}	{ {\eMachine}^{\forward} }
\newcommand{\PastEM}	{ {\eMachine}^{\reverse} }
\newcommand{\BiEM}		{ {\eMachine}^{\forwardreverse} }
\newcommand{\BiEquiv}	{ {\sim}^{\forwardreverse} }
\newcommand{\Futurehmu}	{ h_\mu^{\forward} }
\newcommand{\Pasthmu}	{ h_\mu^{\reverse} }
\newcommand{\FutureCmu}	{ C_\mu^{\forward} }
\newcommand{\PastCmu}	{ C_\mu^{\reverse} }
\newcommand{\BiCmu}		{ C_\mu^{\forwardreverse} }
\newcommand{\FutureEps}	{ \epsilon^{\forward} }
\newcommand{\PastEps}	{ \epsilon^{\reverse} }
\newcommand{\BiEps}	{ \epsilon^{\forwardreverse} }
\newcommand{\FutureSim}	{ \sim^{\forward} }
\newcommand{\PastSim}	{ \sim^{\reverse} }
% Used arrows for awhile, more or less confusing?
%\newcommand{\FutureCausalState}	{ \overrightarrow{\CausalState} }
%\newcommand{\PastCausalState}	{ \overleftarrow{\CausalState} }
%\newcommand{\eMachine}	{ M }
%\newcommand{\FutureEM}	{ \overrightarrow{\eMachine} }
%\newcommand{\PastEM}	{ \overleftarrow{\eMachine} }
%\newcommand{\FutureCmu}	{ \overrightarrow{\Cmu} }
%\newcommand{\PastCmu}	{ \overleftarrow{\Cmu} }

%% time-reversing and mixed state presentation operators
\newcommand{\TR}{\mathcal{T}}
\newcommand{\MSP}{\mathcal{U}}
\newcommand{\one}{\mathbf{1}}

%% (cje)
%% Provide a command \ifpm which is true when \pm 
%% is meant to be understood as "+ or -". This is
%% different from the usage in TBA.
\newif\ifpm 
\edef\tempa{\forwardreverse}
\edef\tempb{\pm}
\ifx\tempa\tempb
   \pmfalse
\else
   \pmtrue  
\fi





% [some math stuff - maybe stick in sep file]
\usepackage{amsthm}
\usepackage{amscd}
\theoremstyle{plain}    \newtheorem{Lem}{Lemma}
\theoremstyle{plain}    \newtheorem*{ProLem}{Proof}
\theoremstyle{plain} 	\newtheorem{Cor}{Corollary}
\theoremstyle{plain} 	\newtheorem*{ProCor}{Proof}
\theoremstyle{plain} 	\newtheorem{The}{Theorem}
\theoremstyle{plain} 	\newtheorem*{ProThe}{Proof}
\theoremstyle{plain} 	\newtheorem{Prop}{Proposition}
\theoremstyle{plain} 	\newtheorem*{ProProp}{Proof}
\theoremstyle{plain} 	\newtheorem*{Conj}{Conjecture}
\theoremstyle{plain}	\newtheorem*{Rem}{Remark}
\theoremstyle{plain}	\newtheorem*{Def}{Definition} 
\theoremstyle{plain}	\newtheorem*{Not}{Notation}

% [uniform figure scaling - maybe this is not a good idea]
\def\figscale{.7}
\def\lscale{1.0}

% [FIX ME! - red makes it easier to spot]
\newcommand{\FIX}[1]{\textbf{\textcolor{red}{#1}}}


    \begin{document}
    \def\noprelim{}
\else
    % already included once
    % input post files

    \singlespacing
    \bibliographystyle{../bibliography/expanded}
    \bibliography{../bibliography/references}

    \end{document}
\fi\fi
%
\chapter{Analysis Selections} \label{ch:selections}
%
The selections applied to events in this analysis are designed to maximize the
significance of Higgs boson signal events in the final set of selected events.
The analysis presented in this thesis is an inclusive analysis, meaning that no
preference is given to any single Higgs boson production mechanism.  The
analysis looks specifically at the channel in which one tau decays to a muon and
the other decays to hadrons.  Therefore the first step in the analysis selection
is to find HLT selection that is highly efficient for the signal
and is not highly prescaled\footnote{If a trigger has high background rates, it
may exceed its rate budget with increasing luminosity.  When this happens, it is
generally ``prescaled,'' and some fraction of the events that pass this trigger
are randomly thrown it away to reduce the rate.  In general, it is better to use
an unprescaled trigger with lower efficiency than a prescaled trigger.}.  After
the trigger selection, events are required to contain at least a good muon and a
good tau.  Vetoes on extra leptons are applied to reduce backgrounds from
dimuon events.  Finally, kinematic and charge selections are applied to
the event to reduce \WpJets and QCD backgrounds.

\section{High Level Trigger}
Because only data which passes the HLT is recorded, it is critical that an
appropriate trigger path is found. The events in this analysis are
triggered by a combination of muon and muon + hadronic tau ``cross--channel''
triggers. For the muon triggers, the HLT paths with lowest \pt thresholds are used as
long as the path remained unprescaled (see Table~\ref{tab:AHtoMuTauTriggers}).
The \mbox{muon + tau-jet} ``cross--channel'' trigger paths increase the trigger
efficiency for events containing muons of transverse momenta close to the
$\pt^{\mu} > 15$~\GeVc cut threshold.  The trigger efficiency is measured in
data via the tag--and--probe technique. Details of the muon trigger efficiency
measurement are given in Section~\ref{sec:ZmumuTagAndProbe}.  Monte Carlo
simulated events are required to pass the HLT\_Mu9 trigger path. Weights are
applied to simulated events to account for the difference between the simulated
HLT\_Mu9 efficiency and the combined efficiency of the set HLT\_Mu9,
HLT\_IsoMu9, HLT\_Mu11, HLT\_IsoMu13, HLT\_Mu15, HLT\_IsoMu9\_PFTau15 and
HLT\_Mu11\_PFTau15 used to trigger the data.

\begin{table}[t]
\begin{center}

\begin{tabular}{|l|c|}
\hline
Trigger path & run--range \\
\hline
HLT\_Mu9             & 132440 - 147116 \\
HLT\_IsoMu9          & 147196 - 148058 \\
HLT\_Mu11            & 147196 - 148058 \\
HLT\_Mu15            & 147196 - 149442 \\
HLT\_IsoMu13         & 148822 - 149182 \\
HLT\_IsoMu9\_PFTau15 & 148822 - 149182 \\
HLT\_Mu11\_PFTau15   & 148822 - 149182 \\
\hline
\end{tabular}
\end{center}
\begin{center}
\caption[High Level Trigger paths used to select $\mu + \tau_h$ events]{\captiontext
Muon and muon + tau ``cross--channel'' trigger paths utilized to trigger
events in different data--taking periods.}
\label{tab:AHtoMuTauTriggers}
\end{center}
\end{table}

\section{Particle Identification}
\subsection{Muons}
\label{sec:MuonId}

Muon candidates are required to be reconstructed as global and as tracker muons,
meaning that a full track is reconstructed in the muon system and is well
matched to a track in the silicon strip and pixel trackers.
Additionally, they are required to pass the ``Vector Boson Task Force'' (VBTF)
muon identification criteria developed for the $Z \to \mu^{+} \mu^{-}$
cross--section measurement~\cite{CMS-PAS-EWK-10-002}, which consist of:
\begin{itemize}
\item $\geq 1$ pixel hits,
\item $\geq 10$ hits in silicon pixel and strip detectors,
\item $\geq 1$ hit(s) in muon system,
\item $\geq 2$ matched segments,
\item $\chi^{2}/DoF < 10$ for global track fit,
\item and an inner track transverse impact parameter $d_\text{IP} < 2~\milli\meter$
  with respect to the beamspot.
\end{itemize}

In order to reduce background contributions from muons originating from heavy
quark decays in QCD multi--jet events, muons are required to be isolated.
Isolation is computed as the \pt sum of charged and neutral hadrons plus photons
reconstructed by the CMS particle--flow algorithm~\cite{CMS-PAS-PFT-09-001}
within a cone of size $\Delta R_{iso} = 0.4$ around the muon direction divided
by the muon \pt.  The
innermost region of size $\Delta R_{veto} = 0.08$ ($0.05$) is excluded from the
computation of the isolation \pt sum with respect to neutral hadrons (photons),
in order to avoid energy deposits in the electromagnetic and hadronic
calorimeters which are due to the muon to enter the sum.  In order to reduce
pile--up effects, particles entering the isolation \pt sum are required to have
transverse momenta $\pt > 1.0$~\GeVc.  Charged particles are additionally
required to originate from the same vertex as the muon.  The muons are required
to be isolated with respect to charged hadrons of $\pt > 1.0~\GeVc$ and photons
of $\pt > 1.5~\GeVc$ as reconstructed by the particle--flow
algorithm~\cite{CMS-PAS-PFT-09-001} in a cone of size $\Delta R = 0.4$ around
the direction of the muon.  The distribution of the muon isolation discriminant
is shown in Figure~\ref{fig:MuonIsoControlPlot}.

\subsection{Hadronic Taus}

Hadronic decays of taus are identified by the \hpsTanc hybrid algorithm
described in Section~\ref{sec:TauId}.  The expected sensitivity of the search
was evaluated using each of the hadronic tau identification work points.  The
``loose'' working point, corresponding to an expected QCD fake--rate of about
1\%, was found to have the highest performance and is used in this analysis.  $Z
\to \MM$ background contributions are largely due to muons which failed to get
reconstructed as global muons (thus failing the muon identification requirement)
and are misidentified as tau candidates.  These muons are typically
isolated and have a large chance to pass the hadronic tau ID discriminators.  To
reject these events, hadronic taus are additionally required to pass an
anti--muon veto described in Section~\ref{sec:LightLeptonRejection}. 

\subsection{Missing Transverse Energy}

The missing transverse energy \MET, in the event is reconstructed based on the
vectorial momentum sum of particle candidates reconstructed by the
particle--flow algorithm~\cite{CMS-PAS-PFT-09-001, CMS-PAS-JME-10-005}.  In the
ideal case, the \MET corresponds to the vector sum of the transverse components
of all neutrinos in the event.  The \MET resolution in simulated $Z\to\MM$
events is found to be smaller (better) than in the data.   The reconstructed
\MET in the simulated events is ``smeared'' by a correction factor such that the
data and simulation are in agreement.  The ``Z--recoil'' \MET correction
procedure is described in Section~\ref{sec:ZRecoilCorr}.

% End particle ID section
\section{Event Selections}
The selections applied to the analysis are designed to reject large fractions of
the background while maintaining a high efficiency for identifying signal Higgs
boson events.  The backgrounds can be divided into two classifications: ``fake''
backgrounds, in which there is at least one misidentified hadronic tau decay,
and the irreducible $Z\to\TT$ background, which cannot\footnote{Due to the
differences in spin between the $Z$ (spin~1) and the Higgs boson (spin~0), in the
future it may be possible to separate the two using spin correlations of the two
tau decays.} be distinguished from the potential presence of a Higgs boson of
the same mass.  Strategies for dealing with the irreducible $Z$ background will
be discussed in the Chapter~\ref{ch:results}.  The different fake backgrounds,
their cross section, and the basic removal strategies are outlined in
Table~\ref{tab:FakeBackgrounds}.
\begin{table}[t]
\begin{center}
\begin{tabular}{|l|c|}
\hline
Background & Cross Section (\pb) \\
\hline
QCD Heavy Flavor &  84679\footnote{The QCD cross section given here is after a
requirement that at least one muon with $\pt > 15~\GeVc$ is produced in the
event.} \\
$W \to \mu \nu + \text{jets}$ & 10435 \\
$Z \to \mu \mu + \text{jets}$ & 1666  \\
$\ttbar + \text{jets}$ & 158 \\
\hline
\end{tabular}
\end{center}
\begin{center}
\caption[Analysis backgrounds that include fake taus]{The different backgrounds
to the analysis presented in this thesis that include misidentified hadronic
taus.} \label{tab:FakeBackgrounds}
\end{center}
\end{table}

Events are selected by requiring a muon of $\pt^{\mu} > 15$~\GeVc within $
\left| \eta_{\mu} \right| < 2.1$ and a tau-jet candidate of
$\pt^{\tau} > 20$~\GeVc within $|\eta_{\tau}| < 2.3$.
The $\eta$ requirement on the muon ensures that it is within the fiducial region
of the muon trigger system.  The $\eta$ requirement on the hadronic tau ensures
it is well within the fiducial region of the tracker ($|\eta| < 2.5$) and
minimizes exposure to large QCD backgrounds in the very forward region.

The muon and tau candidate are required to be of opposite charge, as the
Higgs boson is neutral and charge is conserved.  The muon is required to be pass the
identification criteria described in Section~\ref{sec:MuonId}.   The tau-jet
candidate is required to pass the ``loose'' TaNC tau identification
discriminator. 

Additional event selection criteria are applied to reduce contributions of
specific background processes. In order to reject \ZMM background, a dedicated
discriminator against muons is applied~\cite{CMS-PAS-PFT-08-001}. The remaining
dimuon background is suppressed by rejecting events which have a track of $\pt
> 15~\GeVc$ and for which the sum of energy deposits in ECAL plus HCAL is below
$0.25 \cdot P$ within a cylinder of of radius $15$~\centi\meter~(ECAL) and
$25$~\centi\meter~(HCAL), respectively.  Contamination from $Z \to \TT$ events
in which the reconstructed tau candidate is due to a $\tau \to e \nu \nu$
decay is reduced by applying a dedicated tau ID discriminator against electrons.

The $\ttbar$ and \WpJets backgrounds
are suppressed by cuts on the transverse mass of the $\mu-\MET$ system and the
\Pzeta variable.  The transverse mass ($M_T$) cut is defined as the quantity
\begin{equation}
  M_T = \pt^\mu \MET \sqrt{1 - \cos \Delta\phi},
  \nonumber
\end{equation}
where $\Delta\phi$ is the angle between the muon and the reconstructed
\VMET in the transverse plane.  The $M_T$ quantity is much higher in
events $W \to \mu \nu$ decays than in signal Higgs boson events.  In $W \to \mu \nu$
decays, the neutrino expected to be produced in the opposite to the muon in
azimuth.  In signal events, there are three neutrinos produced, with the
majority (two) of the neutrinos being associated to the $\tau \to \mu \nu \nu$
decay.  Accordingly, we expect that the \VMET is on average collinear with the
muon in signal events.  The $M_T$ distribution immediately before the $M_T$ cut is applied
is illustrated in Figure~\ref{fig:MTCutControlPlot} for the different background
sources and 2010 data.
\begin{figure}
  \centering
  \subfigure[]{\includegraphics*[height=0.47\textwidth,angle=90]{results_chapter/figures/replotted_results/plotZtoMuTauOS_inclusive_cutFlowControlPlots_muonPFCandidateRelIso_afterMuonVbTfId.pdf}
  \label{fig:MuonIsoControlPlot} }
  \subfigure[]{\includegraphics*[height=0.47\textwidth,angle=90]{results_chapter/figures/replotted_results/plotZtoMuTauOS_inclusive_cutFlowControlPlots_mtMuonMET_afterAntiOverlapVeto.pdf}
  \label{fig:MTCutControlPlot} } \caption[Distributions of $M_T$ and muon
  isolation discriminants] {Distributions of the muon
  isolation~\subref{fig:MuonIsoControlPlot} and
  $M_T$~\subref{fig:MTCutControlPlot} discriminant variables.  The muon
  isolation discriminant rejects the QCD background at a high rate.  The $M_T$
  cut is designed to reject \WpJets and \ttbarpJets backgrounds. The
  distributions shown are computed immediately before the corresponding
  selection is applied.} \label{fig:CutFlowControlPlots}
\end{figure}

The \Pzeta variable is another quantity with discriminant power against \WpJets
and \ttbar backgrounds.  The observable has been introduced in the search for
$H\to\TT$ events performed by the CDF collaboration~\cite{CDFrefPzeta}.  The
observable is motivated by the fact that in $\TT$ signal events the neutrinos
are produced nearly collinear with their associated visible decay products. It
is therefore expected that the direction of the missing transverse energy vector
in these events points in a direction between\footnote{In other words, the
projection of \VMET is positive along the bisector of the muon and hadronic tau
momenta.} the visible
$\tau$ decay products. This event topology is not preferred in \WpJets, 
\ttbar and QCD background events.  The observable is computed as the
difference of the projections:
\begin{eqnarray}
P_{\zeta} &=& \vec{P}_{T}^{vis_{1}} + \vec{P}_{T}^{vis_{2}} + \MET \nonumber \\
P_{\zeta}^{vis} &=& \vec{P}_{T}^{vis_{1}} + \vec{P}_{T}^{vis_{2}} \nonumber
%\label{eq:PZetaEq}
\end{eqnarray}
on the axis $\zeta$ bisecting the directions $\vec{P}_{T}^{vis_{1}}$ and
$\vec{P}_{T}^{vis_{2}}$ of the visible $\tau$ lepton decay products in the
transverse plane (see Figure~\ref{fig:PzetaDefinition} for an illustration).
The distribution of \Pzeta after the $M_T$ selection has been applied is shown
in Figure~\ref{fig:PzetaCut}.
\begin{figure}[t]
\begin{center}
  \subfigure[]{\includegraphics*[width=0.45\textwidth]{selection_chapter/figures/pzeta_graphic.pdf}
  \label{fig:PzetaDefinition} }
  \subfigure[]{\includegraphics*[height=0.45\textwidth,angle=90]{results_chapter/figures/replotted_results/plotZtoMuTauOS_inclusive_cutFlowControlPlots_PzetaDiff_afterMt1MET.pdf}
  \label{fig:PzetaCut} } \caption[Reconstruction and distribution of \Pzeta
  discriminant]{The vector quantities used in construction of the quantity
  $P_{\zeta} - 1.5 \cdot P_{\zeta}^{vis}$ are illustrated in
  Figure~\subref{fig:PzetaDefinition}.  Image credit:~\cite{CDFrefPzeta} The
  distribution of the \Pzeta variable in the different background sources and
  2010 data after the $M_T$ cut has been applied is shown at right
  in~\subref{fig:PzetaCut}.} \label{fig:PzetaPlots}
\end{center}
\end{figure} 
The complete set of event selection criteria applied are summarized in
Table~\ref{tab:AHtoMuTauEventSelection}.

\begin{sidewaystable}[t]
\begin{center}

\begin{tabular}{|l|c|}
\hline
\multicolumn{2}{|c|}{Requirement} \\
\hline
\multirow{2}{12mm}{Trigger} & HLT\_Mu9 for MC \\
                            & {\it cf.}\ table~\ref{tab:AHtoMuTauTriggers} for Data \\ 
\hline
\multirow{2}{17mm}{Vertex}  & reconstructed with beam--spot constraint: \\
                           & $-24 < z_{vtx} < +24$~cm, $\left| \rho \right| <
                           2~\centi\meter$, $N_\text{DOF} > 4$ \\ 
\hline
\multirow{4}{10mm}{Muon}    & reconstructed as global Muon with: \\
                            & $\pt > 15~\GeVc$, $\vert \eta \vert < 2.1$, VBTF
                            Muon ID passed, \\
                            & isolated within $\Delta R =0.4$ cone with respect to charged hadrons \\
                            & of $\pt > 1.0$~\GeVc and neutral electromagnetic
                            objects of $\et > 1.5$~\GeV \\ 
\hline
\multirow{4}{23mm}{Tau Candidate} & reconstructed by HPS + TaNC combined Tau ID algorithm \\
                            & TaNC ``medium'' Tau ID discriminator \\
                            & and discriminators against electrons and muons passed, \\
                            & calorimeter muon rejection passed \\
\hline
\multirow{2}{23mm}{Muon + Tau} & charge(Muon) + charge(Tau) = 0, \\
                            & $\Delta R$(Muon, Tau)$ > 0.5$ \\
\hline
\multirow{2}{20mm}{Kinematics} & $M_{T}$(Muon-MET)$ < 40$~\GeV \\
                            & $P_{\zeta} - 1.5 \cdot P_{\zeta}^{vis} > -20$~\GeV \\
\hline
\end{tabular}
\end{center}
\begin{center}
\caption[Event selection summary]{\captiontext Event selection criteria applied to select
$H\to\TT\to\mu\tau_{had}$ events.}
\label{tab:AHtoMuTauEventSelection}
\end{center}
\end{sidewaystable}

\ifx\master\undefined% AUTOCOMPILE
% Allows for individual chapters to be compiled.

% Usage:
% \ifx\master\undefined% AUTOCOMPILE
% Allows for individual chapters to be compiled.

% Usage:
% \ifx\master\undefined% AUTOCOMPILE
% Allows for individual chapters to be compiled.

% Usage:
% \ifx\master\undefined\input{../settings/autocompile}\fi
% (place at start and end of chapter file)

\ifx\noprelim\undefined
    % first time included
    % input preamble files
    \input{../settings/phdsetup}

    \begin{document}
    \def\noprelim{}
\else
    % already included once
    % input post files

    \singlespacing
    \bibliographystyle{../bibliography/expanded}
    \bibliography{../bibliography/references}

    \end{document}
\fi\fi
% (place at start and end of chapter file)

\ifx\noprelim\undefined
    % first time included
    % input preamble files
    % [ USER VARIABLES ]

\def\PHDTITLE {Extensions of the Theory of Computational Mechanics}
\def\PHDAUTHOR{Evan Klose Friis}
\def\PHDSCHOOL{University of California, Davis}

\def\PHDMONTH {June}
\def\PHDYEAR  {2011}
\def\PHDDEPT {Physics}

\def\BSSCHOOL {University of California at San Diego}
\def\BSYEAR   {2005}

\def\PHDCOMMITTEEA{Professor John Conway}
\def\PHDCOMMITTEEB{Professor Robin Erbacher}
\def\PHDCOMMITTEEC{Professor Mani Tripathi}

% [ GLOBAL SETUP ]

\documentclass[letterpaper,oneside,11pt]{report}

\usepackage{calc}
\usepackage{breakcites}
\usepackage[newcommands]{ragged2e}

\usepackage[pdftex]{graphicx}
\usepackage{epstopdf}

%\usepackage{tikz}
%\usetikzlibrary{positioning} % [right=of ...]
%\usetikzlibrary{fit} % [fit= ...]

%\pgfdeclarelayer{background layer}
%\pgfdeclarelayer{foreground layer}
%\pgfsetlayers{background layer,main,foreground layer}

%\newenvironment{wrap}{\noindent\begin{minipage}[t]{\linewidth}\vspace{-0.5\normalbaselineskip}\centering}{\vspace{0.5\normalbaselineskip}\end{minipage}}

%% [Venn diagram environment]
%\newenvironment{venn2}
%{\begin{tikzpicture} [every pin/.style={text=black, text opacity=1.0, pin distance=0.5cm, pin edge={black!60, semithick}},
%% define a new style 'venn'
%venn/.style={circle, draw=black!60, semithick, minimum size = 4cm}]
%
%% create circle and give it external (pin) label
%\node[venn] (X) at (-1,0) [pin={150:$H[X]$}] {};
%\node[venn] (Y) at (1,0) [pin={30:$H[Y]$}] {};
%
%% place labels of the atoms by hand
%\node at (-1.9,0) {$H[X|Y]$};
%\node at (1.9,0) {$H[Y|X]$};
%\node at (0,0) {$I[X;Y]$};}
%{\end{tikzpicture}}

%\newcommand{\wrapmath}[1]{\begin{wrap}\begin{tikzpicture}[every node/.style={inner ysep=0ex, inner xsep=0em}]\node[] {$\displaystyle\begin{aligned} #1\end{aligned}$};\end{tikzpicture}\end{wrap}}

\renewenvironment{abstract}{\chapter*{Abstract}}{}
\renewcommand{\bibname}{Bibliography}
\renewcommand{\contentsname}{Table of Contents}

\makeatletter
\renewcommand{\@biblabel}[1]{\textsc{#1}}
\makeatother

% [ FONT SETTINGS ]

\usepackage[tbtags, intlimits, namelimits]{amsmath}
\usepackage[adobe-utopia]{mathdesign}

\DeclareSymbolFont{pazomath}{OMS}{zplm}{m}{n}
\DeclareSymbolFontAlphabet{\mathcal}{pazomath}
\SetMathAlphabet\mathcal{bold}{OMS}{zplm}{b}{n}

\SetSymbolFont{largesymbols}{normal}{OMX}{zplm}{m}{n}
\SetSymbolFont{largesymbols}{bold}{OMX}{zplm}{m}{n}
\SetSymbolFont{symbols}{normal}{OMS}{zplm}{m}{n}
\SetSymbolFont{symbols}{bold}{OMS}{zplm}{b}{n}

\renewcommand{\sfdefault}{phv}
\renewcommand{\ttdefault}{fvm}

\widowpenalty 8000
\clubpenalty  8000

% [ PAGE LAYOUT ]
\usepackage{geometry}
\geometry{lmargin = 1.5in}
\geometry{rmargin = 1.0in}
\geometry{tmargin = 1.0in}
\geometry{bmargin = 1.0in}

% [ PDF SETTINGS ]

\usepackage[final]{hyperref}
\hypersetup{breaklinks  = true}
\hypersetup{colorlinks  = true}
\hypersetup{linktocpage = false}
\hypersetup{linkcolor   = blue}
\hypersetup{citecolor   = green}
\hypersetup{urlcolor    = black}
\hypersetup{plainpages  = false}
\hypersetup{pageanchor  = true}
\hypersetup{pdfauthor   = {\PHDAUTHOR}}
\hypersetup{pdftitle    = {\PHDTITLE}}
\hypersetup{pdfsubject  = {Dissertation, \PHDSCHOOL}}
\urlstyle{same}

% [ LETTER SPACING ]

\usepackage[final]{microtype}
\microtypesetup{protrusion=compatibility}
\microtypesetup{expansion=false}

\newcommand{\upper}[1]{\MakeUppercase{#1}}
\let\lsscshape\scshape

\ifcase\pdfoutput\else\microtypesetup{letterspace=15}
\renewcommand{\scshape}{\lsscshape\lsstyle}
\renewcommand{\upper}[1]{\textls[50]{\MakeUppercase{#1}}}\fi

% [ LINE SPACING ]

\usepackage[doublespacing]{setspace}
\renewcommand{\displayskipstretch}{0.0}

\setlength{\parskip   }{0em}
\setlength{\parindent }{2em}

% [ TABLE FORMATTING ]

\usepackage{booktabs}
\setlength{\heavyrulewidth}{1.5\arrayrulewidth}
\setlength{\lightrulewidth}{1.0\arrayrulewidth}
\setlength{\doublerulesep }{2.0\arrayrulewidth}

% [ SECTION FORMATTING ]

\usepackage[largestsep,nobottomtitles*]{titlesec}
\renewcommand{\bottomtitlespace}{0.75in}

\titleformat{\chapter}[display]{\bfseries\huge\singlespacing}{\filleft\textsc{\LARGE \chaptertitlename\ \thechapter}}{-0.2ex}{\titlerule[3pt]\vspace{0.2ex}}[]

\titleformat{\section}{\LARGE}{\S\thesection\hspace{0.5em}}{0ex}{}
\titleformat{\subsection}{\Large}{\S\thesubsection\hspace{0.5em}}{0ex}{}
\titleformat{\subsubsection}{\large}{\thesubsubsection\hspace{0.5em}}{0ex}{}

\titlespacing*{\chapter}{0em}{6ex}{4ex plus 2ex minus 0ex}
\titlespacing*{\section}{0em}{2ex plus 3ex minus 1ex}{0.5ex plus 0.5ex minus 0.5ex}
\titlespacing*{\subsection}{0ex}{2ex plus 3ex minus 1ex}{0ex}
\titlespacing*{\subsubsection}{0ex}{2ex plus 0ex minus 1ex}{0ex}

% [ HEADER SETTINGS ]

\usepackage{fancyhdr}

\setlength{\headheight}{\normalbaselineskip}
\setlength{\footskip  }{0.5in}
\setlength{\headsep   }{0.5in-\headheight}

\fancyheadoffset[R]{0.5in}
\renewcommand{\headrulewidth}{0pt}
\renewcommand{\footrulewidth}{0pt}

\newcommand{\pagebox}{\parbox[r][\headheight][t]{0.5in}{\hspace\fill\thepage}}

\newcommand{\prelimheaders}{\ifx\prelim\undefined\renewcommand{\thepage}{\textit{\roman{page}}}\fancypagestyle{plain}{\fancyhf{}\fancyfoot[L]{\makebox[\textwidth-0.5in]{\thepage}}}\pagestyle{plain}\def\prelim{}\fi}

\newcommand{\normalheaders}{\renewcommand{\thepage}{\arabic{page}}\fancypagestyle{plain}{\fancyhf{}\fancyhead[R]{\pagebox}}\pagestyle{plain}}

\normalheaders{}

% [ CUSTOM COMMANDS ]

\newcommand{\signaturebox}[1]{\multicolumn{1}{p{4in}}{\vspace{3ex}}\\\midrule #1\\}

%\input{../includes/cmechabbrev}

% [some math stuff - maybe stick in sep file]
\usepackage{amsthm}
\usepackage{amscd}
\theoremstyle{plain}    \newtheorem{Lem}{Lemma}
\theoremstyle{plain}    \newtheorem*{ProLem}{Proof}
\theoremstyle{plain} 	\newtheorem{Cor}{Corollary}
\theoremstyle{plain} 	\newtheorem*{ProCor}{Proof}
\theoremstyle{plain} 	\newtheorem{The}{Theorem}
\theoremstyle{plain} 	\newtheorem*{ProThe}{Proof}
\theoremstyle{plain} 	\newtheorem{Prop}{Proposition}
\theoremstyle{plain} 	\newtheorem*{ProProp}{Proof}
\theoremstyle{plain} 	\newtheorem*{Conj}{Conjecture}
\theoremstyle{plain}	\newtheorem*{Rem}{Remark}
\theoremstyle{plain}	\newtheorem*{Def}{Definition} 
\theoremstyle{plain}	\newtheorem*{Not}{Notation}

% [uniform figure scaling - maybe this is not a good idea]
\def\figscale{.7}
\def\lscale{1.0}

% [FIX ME! - red makes it easier to spot]
\newcommand{\FIX}[1]{\textbf{\textcolor{red}{#1}}}


    \begin{document}
    \def\noprelim{}
\else
    % already included once
    % input post files

    \singlespacing
    \bibliographystyle{../bibliography/expanded}
    \bibliography{../bibliography/references}

    \end{document}
\fi\fi
% (place at start and end of chapter file)

\ifx\noprelim\undefined
    % first time included
    % input preamble files
    % [ USER VARIABLES ]

\def\PHDTITLE {Extensions of the Theory of Computational Mechanics}
\def\PHDAUTHOR{Evan Klose Friis}
\def\PHDSCHOOL{University of California, Davis}

\def\PHDMONTH {June}
\def\PHDYEAR  {2011}
\def\PHDDEPT {Physics}

\def\BSSCHOOL {University of California at San Diego}
\def\BSYEAR   {2005}

\def\PHDCOMMITTEEA{Professor John Conway}
\def\PHDCOMMITTEEB{Professor Robin Erbacher}
\def\PHDCOMMITTEEC{Professor Mani Tripathi}

% [ GLOBAL SETUP ]

\documentclass[letterpaper,oneside,11pt]{report}

\usepackage{calc}
\usepackage{breakcites}
\usepackage[newcommands]{ragged2e}

\usepackage[pdftex]{graphicx}
\usepackage{epstopdf}

%\usepackage{tikz}
%\usetikzlibrary{positioning} % [right=of ...]
%\usetikzlibrary{fit} % [fit= ...]

%\pgfdeclarelayer{background layer}
%\pgfdeclarelayer{foreground layer}
%\pgfsetlayers{background layer,main,foreground layer}

%\newenvironment{wrap}{\noindent\begin{minipage}[t]{\linewidth}\vspace{-0.5\normalbaselineskip}\centering}{\vspace{0.5\normalbaselineskip}\end{minipage}}

%% [Venn diagram environment]
%\newenvironment{venn2}
%{\begin{tikzpicture} [every pin/.style={text=black, text opacity=1.0, pin distance=0.5cm, pin edge={black!60, semithick}},
%% define a new style 'venn'
%venn/.style={circle, draw=black!60, semithick, minimum size = 4cm}]
%
%% create circle and give it external (pin) label
%\node[venn] (X) at (-1,0) [pin={150:$H[X]$}] {};
%\node[venn] (Y) at (1,0) [pin={30:$H[Y]$}] {};
%
%% place labels of the atoms by hand
%\node at (-1.9,0) {$H[X|Y]$};
%\node at (1.9,0) {$H[Y|X]$};
%\node at (0,0) {$I[X;Y]$};}
%{\end{tikzpicture}}

%\newcommand{\wrapmath}[1]{\begin{wrap}\begin{tikzpicture}[every node/.style={inner ysep=0ex, inner xsep=0em}]\node[] {$\displaystyle\begin{aligned} #1\end{aligned}$};\end{tikzpicture}\end{wrap}}

\renewenvironment{abstract}{\chapter*{Abstract}}{}
\renewcommand{\bibname}{Bibliography}
\renewcommand{\contentsname}{Table of Contents}

\makeatletter
\renewcommand{\@biblabel}[1]{\textsc{#1}}
\makeatother

% [ FONT SETTINGS ]

\usepackage[tbtags, intlimits, namelimits]{amsmath}
\usepackage[adobe-utopia]{mathdesign}

\DeclareSymbolFont{pazomath}{OMS}{zplm}{m}{n}
\DeclareSymbolFontAlphabet{\mathcal}{pazomath}
\SetMathAlphabet\mathcal{bold}{OMS}{zplm}{b}{n}

\SetSymbolFont{largesymbols}{normal}{OMX}{zplm}{m}{n}
\SetSymbolFont{largesymbols}{bold}{OMX}{zplm}{m}{n}
\SetSymbolFont{symbols}{normal}{OMS}{zplm}{m}{n}
\SetSymbolFont{symbols}{bold}{OMS}{zplm}{b}{n}

\renewcommand{\sfdefault}{phv}
\renewcommand{\ttdefault}{fvm}

\widowpenalty 8000
\clubpenalty  8000

% [ PAGE LAYOUT ]
\usepackage{geometry}
\geometry{lmargin = 1.5in}
\geometry{rmargin = 1.0in}
\geometry{tmargin = 1.0in}
\geometry{bmargin = 1.0in}

% [ PDF SETTINGS ]

\usepackage[final]{hyperref}
\hypersetup{breaklinks  = true}
\hypersetup{colorlinks  = true}
\hypersetup{linktocpage = false}
\hypersetup{linkcolor   = blue}
\hypersetup{citecolor   = green}
\hypersetup{urlcolor    = black}
\hypersetup{plainpages  = false}
\hypersetup{pageanchor  = true}
\hypersetup{pdfauthor   = {\PHDAUTHOR}}
\hypersetup{pdftitle    = {\PHDTITLE}}
\hypersetup{pdfsubject  = {Dissertation, \PHDSCHOOL}}
\urlstyle{same}

% [ LETTER SPACING ]

\usepackage[final]{microtype}
\microtypesetup{protrusion=compatibility}
\microtypesetup{expansion=false}

\newcommand{\upper}[1]{\MakeUppercase{#1}}
\let\lsscshape\scshape

\ifcase\pdfoutput\else\microtypesetup{letterspace=15}
\renewcommand{\scshape}{\lsscshape\lsstyle}
\renewcommand{\upper}[1]{\textls[50]{\MakeUppercase{#1}}}\fi

% [ LINE SPACING ]

\usepackage[doublespacing]{setspace}
\renewcommand{\displayskipstretch}{0.0}

\setlength{\parskip   }{0em}
\setlength{\parindent }{2em}

% [ TABLE FORMATTING ]

\usepackage{booktabs}
\setlength{\heavyrulewidth}{1.5\arrayrulewidth}
\setlength{\lightrulewidth}{1.0\arrayrulewidth}
\setlength{\doublerulesep }{2.0\arrayrulewidth}

% [ SECTION FORMATTING ]

\usepackage[largestsep,nobottomtitles*]{titlesec}
\renewcommand{\bottomtitlespace}{0.75in}

\titleformat{\chapter}[display]{\bfseries\huge\singlespacing}{\filleft\textsc{\LARGE \chaptertitlename\ \thechapter}}{-0.2ex}{\titlerule[3pt]\vspace{0.2ex}}[]

\titleformat{\section}{\LARGE}{\S\thesection\hspace{0.5em}}{0ex}{}
\titleformat{\subsection}{\Large}{\S\thesubsection\hspace{0.5em}}{0ex}{}
\titleformat{\subsubsection}{\large}{\thesubsubsection\hspace{0.5em}}{0ex}{}

\titlespacing*{\chapter}{0em}{6ex}{4ex plus 2ex minus 0ex}
\titlespacing*{\section}{0em}{2ex plus 3ex minus 1ex}{0.5ex plus 0.5ex minus 0.5ex}
\titlespacing*{\subsection}{0ex}{2ex plus 3ex minus 1ex}{0ex}
\titlespacing*{\subsubsection}{0ex}{2ex plus 0ex minus 1ex}{0ex}

% [ HEADER SETTINGS ]

\usepackage{fancyhdr}

\setlength{\headheight}{\normalbaselineskip}
\setlength{\footskip  }{0.5in}
\setlength{\headsep   }{0.5in-\headheight}

\fancyheadoffset[R]{0.5in}
\renewcommand{\headrulewidth}{0pt}
\renewcommand{\footrulewidth}{0pt}

\newcommand{\pagebox}{\parbox[r][\headheight][t]{0.5in}{\hspace\fill\thepage}}

\newcommand{\prelimheaders}{\ifx\prelim\undefined\renewcommand{\thepage}{\textit{\roman{page}}}\fancypagestyle{plain}{\fancyhf{}\fancyfoot[L]{\makebox[\textwidth-0.5in]{\thepage}}}\pagestyle{plain}\def\prelim{}\fi}

\newcommand{\normalheaders}{\renewcommand{\thepage}{\arabic{page}}\fancypagestyle{plain}{\fancyhf{}\fancyhead[R]{\pagebox}}\pagestyle{plain}}

\normalheaders{}

% [ CUSTOM COMMANDS ]

\newcommand{\signaturebox}[1]{\multicolumn{1}{p{4in}}{\vspace{3ex}}\\\midrule #1\\}

%%%% macros fro standard references
%\eqref provided by amsmath
\newcommand{\figref}[1]{Fig.~\ref{#1}}
\newcommand{\tableref}[1]{Table~\ref{#1}}
\newcommand{\refcite}[1]{Ref.~\cite{#1}}

% Abbreviations from CMPPSS:

\newcommand{\eM}     {\mbox{$\epsilon$-machine}}
\newcommand{\eMs}    {\mbox{$\epsilon$-machines}}
\newcommand{\EM}     {\mbox{$\epsilon$-Machine}}
\newcommand{\EMs}    {\mbox{$\epsilon$-Machines}}
\newcommand{\eT}     {\mbox{$\epsilon$-transducer}}
\newcommand{\eTs}    {\mbox{$\epsilon$-transducers}}
\newcommand{\ET}     {\mbox{$\epsilon$-Transducer}}
\newcommand{\ETs}    {\mbox{$\epsilon$-Transducers}}

% Processes and sequences

\newcommand{\Process}{\mathcal{P}}

\newcommand{\ProbMach}{\Prob_{\mathrm{M}}}
\newcommand{\Lmax}   { {L_{\mathrm{max}}}}
\newcommand{\MeasAlphabet}	{\mathcal{A}}
% Original
%\newcommand{\MeasSymbol}   { {S} }
%\newcommand{\meassymbol}   { {s} }
% New symbol
\newcommand{\MeasSymbol}   { {X} }
\newcommand{\meassymbol}   { {x} }
\newcommand{\BiInfinity}	{ \overleftrightarrow {\MeasSymbol} }
\newcommand{\biinfinity}	{ \overleftrightarrow {\meassymbol} }
\newcommand{\Past}	{ \overleftarrow {\MeasSymbol} }
\newcommand{\past}	{ {\overleftarrow {\meassymbol}} }
\newcommand{\pastprime}	{ {\past}^{\prime}}
\newcommand{\Future}	{ \overrightarrow{\MeasSymbol} }
\newcommand{\future}	{ \overrightarrow{\meassymbol} }
\newcommand{\futureprime}	{ {\future}^{\prime}}
\newcommand{\PastPrime}	{ {\Past}^{\prime}}
\newcommand{\FuturePrime}	{ {\overrightarrow{\meassymbol}}^\prime }
\newcommand{\PastDblPrime}	{ {\overleftarrow{\meassymbol}}^{\prime\prime} }
\newcommand{\FutureDblPrime}	{ {\overrightarrow{\meassymbol}}^{\prime\prime} }
\newcommand{\pastL}	{ {\overleftarrow {\meassymbol}}{}^L }
\newcommand{\PastL}	{ {\overleftarrow {\MeasSymbol}}{}^L }
\newcommand{\PastLt}	{ {\overleftarrow {\MeasSymbol}}_t^L }
\newcommand{\PastLLessOne}	{ {\overleftarrow {\MeasSymbol}}^{L-1} }
\newcommand{\futureL}	{ {\overrightarrow{\meassymbol}}{}^L }
\newcommand{\FutureL}	{ {\overrightarrow{\MeasSymbol}}{}^L }
\newcommand{\FutureLt}	{ {\overrightarrow{\MeasSymbol}}_t^L }
\newcommand{\FutureLLessOne}	{ {\overrightarrow{\MeasSymbol}}^{L-1} }
\newcommand{\pastLprime}	{ {\overleftarrow {\meassymbol}}^{L^\prime} }
\newcommand{\futureLprime}	{ {\overrightarrow{\meassymbol}}^{L^\prime} }
\newcommand{\AllPasts}	{ { \overleftarrow {\rm {\bf \MeasSymbol}} } }
\newcommand{\AllFutures}	{ \overrightarrow {\rm {\bf \MeasSymbol}} }
\newcommand{\FutureSet}	{ \overrightarrow{\bf \MeasSymbol}}

% Causal states and epsilon-machines
\newcommand{\CausalState}	{ \mathcal{S} }
\newcommand{\CausalStatePrime}	{ {\CausalState}^{\prime}}
\newcommand{\causalstate}	{ \sigma }
\newcommand{\CausalStateSet}	{ \boldsymbol{\CausalState} }
\newcommand{\AlternateState}	{ \mathcal{R} }
\newcommand{\AlternateStatePrime}	{ {\cal R}^{\prime} }
\newcommand{\alternatestate}	{ \rho }
\newcommand{\alternatestateprime}	{ {\rho^{\prime}} }
\newcommand{\AlternateStateSet}	{ \boldsymbol{\AlternateState} }
\newcommand{\PrescientState}	{ \widehat{\AlternateState} }
\newcommand{\prescientstate}	{ \widehat{\alternatestate} }
\newcommand{\PrescientStateSet}	{ \boldsymbol{\PrescientState}}
\newcommand{\CausalEquivalence}	{ {\sim}_{\epsilon} }
\newcommand{\CausalEquivalenceNot}	{ {\not \sim}_{\epsilon}}

\newcommand{\NonCausalEquivalence}	{ {\sim}_{\eta} }
\newcommand{\NextObservable}	{ {\overrightarrow {\MeasSymbol}}^1 }
\newcommand{\LastObservable}	{ {\overleftarrow {\MeasSymbol}}^1 }
%\newcommand{\Prob}		{ {\rm P}}
\newcommand{\Prob}      {\Pr} % use standard command
\newcommand{\ProbAnd}	{ {,\;} }
\newcommand{\LLimit}	{ {L \rightarrow \infty}}
\newcommand{\Cmu}		{C_\mu}
\newcommand{\hmu}		{h_\mu}
\newcommand{\EE}		{{\bf E}}
\newcommand{\Measurable}{{\bf \mu}}

% Process Crypticity
\newcommand{\PC}		{\chi}
\newcommand{\FuturePC}		{\PC^+}
\newcommand{\PastPC}		{\PC^-}
% Causal Irreversibility
\newcommand{\CI}		{\Xi}
\newcommand{\ReverseMap}	{r}
\newcommand{\ForwardMap}	{f}

% Abbreviations from IB:
% None that aren't already in CMPPSS

% Abbreviations from Extensive Estimation:
\newcommand{\EstCausalState}	{\widehat{\CausalState}}
\newcommand{\estcausalstate}	{\widehat{\causalstate}}
\newcommand{\EstCausalStateSet}	{\boldsymbol{\EstCausalState}}
\newcommand{\EstCausalFunc}	{\widehat{\epsilon}}
\newcommand{\EstCmu}		{\widehat{\Cmu}}
\newcommand{\PastLOne}	{{\Past}^{L+1}}
\newcommand{\pastLOne}	{{\past}^{L+1}}

% Abbreviations from $\epsilon$-Transducers:
\newcommand{\InAlphabet}	{ \mathcal{A}}
\newcommand{\insymbol}		{ a}
\newcommand{\OutAlphabet}	{ \mathcal{B}}
\newcommand{\outsymbol}		{ b}
\newcommand{\InputSimple}	{ X}
\newcommand{\inputsimple}	{ x}
\newcommand{\BottleneckVar}	{\tilde{\InputSimple}}
\newcommand{\bottleneckvar}	{\tilde{\inputsimple}}
\newcommand{\InputSpace}	{ \mathbf{\InputSimple}}
\newcommand{\InputBi}	{ \overleftrightarrow {\InputSimple} }
\newcommand{\inputbi}	{ \overleftrightarrow {\inputsimple} }
\newcommand{\InputPast}	{ \overleftarrow {\InputSimple} }
\newcommand{\inputpast}	{ \overleftarrow {\inputsimple} }
\newcommand{\InputFuture}	{ \overrightarrow {\InputSimple} }
\newcommand{\inputfuture}	{ \overrightarrow {\inputsimple} }
\newcommand{\NextInput}	{ {{\InputFuture}^{1}}}
\newcommand{\NextOutput}	{ {\OutputFuture}^{1}}
\newcommand{\OutputSimple}	{ Y}
\newcommand{\outputsimple}	{ y}
\newcommand{\OutputSpace}	{ \mathbf{\OutputSimple}}
\newcommand{\OutputBi}	{ \overleftrightarrow{\OutputSimple} }
\newcommand{\outputbi}	{ \overleftrightarrow{\outputsimple} }
\newcommand{\OutputPast}	{ \overleftarrow{\OutputSimple} }
\newcommand{\outputpast}	{ \overleftarrow{\outputsimple} }
\newcommand{\OutputFuture}	{ \overrightarrow{\OutputSimple} }
\newcommand{\outputfuture}	{ \overrightarrow{\outputsimple} }
\newcommand{\OutputL}	{ {\OutputFuture}^L}
\newcommand{\outputL}	{ {\outputfuture}^L}
\newcommand{\InputLLessOne}	{ {\InputFuture}^{L-1}}
\newcommand{\inputLlessone}	{ {\inputufutre}^{L-1}}
\newcommand{\OutputPastLLessOne}	{{\OutputPast}^{L-1}_{-1}}
\newcommand{\outputpastLlessone}	{{\outputpast}^{L-1}}
\newcommand{\OutputPastLessOne}	{{\OutputPast}_{-1}}
\newcommand{\outputpastlessone}	{{\outputpast}_{-1}}
\newcommand{\OutputPastL}	{{\OutputPast}^{L}}
\newcommand{\OutputLPlusOne}	{ {\OutputFuture}^{L+1}}
\newcommand{\outputLplusone}	{ {\outputfutre}^{L+1}}
\newcommand{\InputPastL}	{{\InputPast}^{L}}
\newcommand{\inputpastL}	{{\inputpast}^{L}}
\newcommand{\JointPast}	{{(\InputPast,\OutputPast)}}
\newcommand{\jointpast}	{{(\inputpast,\outputpast)}}
\newcommand{\jointpastone}	{{(\inputpast_1,\outputpast_1)}}
\newcommand{\jointpasttwo}	{{(\inputpast_2,\outputpast_2)}}
\newcommand{\jointpastprime} {{({\inputpast}^{\prime},{\outputpast}^{\prime})}}
\newcommand{\NextJoint}	{{(\NextInput,\NextOutput)}}
\newcommand{\nextjoint}	{{(\insymbol,\outsymbol)}}
\newcommand{\AllInputPasts}	{ { \overleftarrow {\rm \InputSpace}}}
\newcommand{\AllOutputPasts}	{ {\overleftarrow {\rm \OutputSpace}}}
\newcommand{\DetCausalState}	{ {{\cal S}_D }}
\newcommand{\detcausalstate}	{ {{\sigma}_D} }
\newcommand{\DetCausalStateSet}	{ \boldsymbol{{\CausalState}_D}}
\newcommand{\DetCausalEquivalence}	{ {\sim}_{{\epsilon}_{D}}}
\newcommand{\PrescientEquivalence}	{ {\sim}_{\widehat{\eta}}}
\newcommand{\FeedbackCausalState}	{ \mathcal{F}}
\newcommand{\feedbackcausalstate}	{ \phi}
\newcommand{\FeedbackCausalStateSet}	{ \mathbf{\FeedbackCausalState}}
\newcommand{\JointCausalState}		{ \mathcal{J}}
\newcommand{\JointCausalStateSet}	{ \mathbf{\JointCausalState}}
\newcommand{\UtilityFunctional}	{ {\mathcal{L}}}
\newcommand{\NatureState}	{ {\Omega}}
\newcommand{\naturestate}	{ {\omega}}
\newcommand{\NatureStateSpace}	{ {\mathbf{\NatureState}}}
\newcommand{\AnAction}	{ {A}}
\newcommand{\anaction}	{ {a}}
\newcommand{\ActionSpace}	{ {\mathbf{\AnAction}}}

% Abbreviations from RURO:
\newcommand{\InfoGain}[2] { \mathcal{D} \left( {#1} || {#2} \right) }

% Abbreviations from Upper Bound:
\newcommand{\lcm}	{{\rm lcm}}
% Double-check that this isn't in the math set already!

% Abbreviations from Emergence in Space
\newcommand{\ProcessAlphabet}	{\MeasAlphabet}
\newcommand{\ProbEst}			{ {\widehat{\Prob}_N}}
\newcommand{\STRegion}			{ {\mathrm K}}
\newcommand{\STRegionVariable}		{ K}
\newcommand{\stregionvariable}		{ k}
\newcommand{\GlobalPast}		{ \overleftarrow{G}} 
\newcommand{\globalpast}		{ \overleftarrow{g}} 
\newcommand{\GlobalFuture}		{ \overrightarrow{G}}
\newcommand{\globalfuture}		{ \overrightarrow{g}}
\newcommand{\GlobalState}		{ \mathcal{G}}
\newcommand{\globalstate}		{ \gamma}
\newcommand{\GlobalStateSet}		{ {\mathbf \GlobalState}}
\newcommand{\LocalPast}			{ \overleftarrow{L}} 
\newcommand{\localpast}			{ \overleftarrow{l}}
\newcommand{\AllLocalPasts}		{ \mathbf{\LocalPast}}
\newcommand{\LocalPastRegion}		{ \overleftarrow{\mathrm L}}
\newcommand{\LocalFuture}		{ \overrightarrow{L}}
\newcommand{\localfuture}		{ \overrightarrow{l}}
\newcommand{\LocalFutureRegion}		{ \overrightarrow{\mathrm L}}
\newcommand{\LocalState}		{ \mathcal{L}}
\newcommand{\localstate}		{ \lambda}
\newcommand{\LocalStateSet}		{ {\mathbf \LocalState}}
\newcommand{\PatchPast}			{ \overleftarrow{P}}
\newcommand{\patchpast}			{ \overleftarrow{p}}
\newcommand{\PatchPastRegion}		{ \overleftarrow{\mathrm P}}
\newcommand{\PatchFuture}		{ \overrightarrow{P}}
\newcommand{\patchfuture}		{ \overrightarrow{p}}
\newcommand{\PatchFutureRegion}		{ \overrightarrow{\mathrm P}}
\newcommand{\PatchState}		{ \mathcal{P}}
\newcommand{\patchstate}		{ \pi}
\newcommand{\PatchStateSet}		{ {\mathbf \PatchState}}
\newcommand{\LocalStatesInPatch}	{\vec{\LocalState}}
\newcommand{\localstatesinpatch}	{\vec{\localstate}}
\newcommand{\PointInstantX}		{ {\mathbf x}}
% Galles's original LaTeX for the cond. indep. symbol follows:
\newcommand{\compos}{\mbox{$~\underline{~\parallel~}~$}}
\newcommand{\ncompos}{\not\hspace{-.15in}\compos}
\newcommand{\indep}			{ \rotatebox{90}{$\models$}}
\newcommand{\nindep}	{\not\hspace{-.05in}\indep}
\newcommand{\LocalEE}	{{\EE}^{loc}}
\newcommand{\EEDensity}	{\overline{\LocalEE}}
\newcommand{\LocalCmu}	{{\Cmu}^{loc}}
\newcommand{\CmuDensity}	{\overline{\LocalCmu}}

%%%%%%%%%%% added by sasa
\newcommand{\FinPast}[1]	{ \overleftarrow {\MeasSymbol} \stackrel{{#1}}{}}
\newcommand{\finpast}[1]  	{ \overleftarrow {\meassymbol}  \stackrel{{#1}}{}}
\newcommand{\FinFuture}[1]		{ \overrightarrow{\MeasSymbol} \stackrel{{#1}}{}}
\newcommand{\finfuture}[1]		{ \overrightarrow{\meassymbol} \stackrel{{#1}}{}}

\newcommand{\Partition}	{ \AlternateState }
\newcommand{\partitionstate}	{ \alternatestate }
\newcommand{\PartitionSet}	{ \AlternateStateSet }
\newcommand{\Fdet}   { F_{\rm det} }

\newcommand{\Dkl}[2] { D_{\rm KL} \left( {#1} || {#2} \right) }

\newcommand{\Period}	{p}

% To take into account time direction
\newcommand{\forward}{+}
\newcommand{\reverse}{-}
%\newcommand{\forwardreverse}{\:\!\diamond} % \pm
\newcommand{\forwardreverse}{\pm} % \pm
\newcommand{\FutureProcess}	{ {\Process}^{\forward} }
\newcommand{\PastProcess}	{ {\Process}^{\reverse} }
\newcommand{\FutureCausalState}	{ {\CausalState}^{\forward} }
\newcommand{\futurecausalstate}	{ \sigma^{\forward} }
\newcommand{\altfuturecausalstate}	{ \sigma^{\forward\prime} }
\newcommand{\PastCausalState}	{ {\CausalState}^{\reverse} }
\newcommand{\pastcausalstate}	{ \sigma^{\reverse} }
\newcommand{\BiCausalState}		{ {\CausalState}^{\forwardreverse} }
\newcommand{\bicausalstate}		{ {\sigma}^{\forwardreverse} }
\newcommand{\FutureCausalStateSet}	{ {\CausalStateSet}^{\forward} }
\newcommand{\PastCausalStateSet}	{ {\CausalStateSet}^{\reverse} }
\newcommand{\BiCausalStateSet}	{ {\CausalStateSet}^{\forwardreverse} }
\newcommand{\eMachine}	{ M }
\newcommand{\FutureEM}	{ {\eMachine}^{\forward} }
\newcommand{\PastEM}	{ {\eMachine}^{\reverse} }
\newcommand{\BiEM}		{ {\eMachine}^{\forwardreverse} }
\newcommand{\BiEquiv}	{ {\sim}^{\forwardreverse} }
\newcommand{\Futurehmu}	{ h_\mu^{\forward} }
\newcommand{\Pasthmu}	{ h_\mu^{\reverse} }
\newcommand{\FutureCmu}	{ C_\mu^{\forward} }
\newcommand{\PastCmu}	{ C_\mu^{\reverse} }
\newcommand{\BiCmu}		{ C_\mu^{\forwardreverse} }
\newcommand{\FutureEps}	{ \epsilon^{\forward} }
\newcommand{\PastEps}	{ \epsilon^{\reverse} }
\newcommand{\BiEps}	{ \epsilon^{\forwardreverse} }
\newcommand{\FutureSim}	{ \sim^{\forward} }
\newcommand{\PastSim}	{ \sim^{\reverse} }
% Used arrows for awhile, more or less confusing?
%\newcommand{\FutureCausalState}	{ \overrightarrow{\CausalState} }
%\newcommand{\PastCausalState}	{ \overleftarrow{\CausalState} }
%\newcommand{\eMachine}	{ M }
%\newcommand{\FutureEM}	{ \overrightarrow{\eMachine} }
%\newcommand{\PastEM}	{ \overleftarrow{\eMachine} }
%\newcommand{\FutureCmu}	{ \overrightarrow{\Cmu} }
%\newcommand{\PastCmu}	{ \overleftarrow{\Cmu} }

%% time-reversing and mixed state presentation operators
\newcommand{\TR}{\mathcal{T}}
\newcommand{\MSP}{\mathcal{U}}
\newcommand{\one}{\mathbf{1}}

%% (cje)
%% Provide a command \ifpm which is true when \pm 
%% is meant to be understood as "+ or -". This is
%% different from the usage in TBA.
\newif\ifpm 
\edef\tempa{\forwardreverse}
\edef\tempb{\pm}
\ifx\tempa\tempb
   \pmfalse
\else
   \pmtrue  
\fi





% [some math stuff - maybe stick in sep file]
\usepackage{amsthm}
\usepackage{amscd}
\theoremstyle{plain}    \newtheorem{Lem}{Lemma}
\theoremstyle{plain}    \newtheorem*{ProLem}{Proof}
\theoremstyle{plain} 	\newtheorem{Cor}{Corollary}
\theoremstyle{plain} 	\newtheorem*{ProCor}{Proof}
\theoremstyle{plain} 	\newtheorem{The}{Theorem}
\theoremstyle{plain} 	\newtheorem*{ProThe}{Proof}
\theoremstyle{plain} 	\newtheorem{Prop}{Proposition}
\theoremstyle{plain} 	\newtheorem*{ProProp}{Proof}
\theoremstyle{plain} 	\newtheorem*{Conj}{Conjecture}
\theoremstyle{plain}	\newtheorem*{Rem}{Remark}
\theoremstyle{plain}	\newtheorem*{Def}{Definition} 
\theoremstyle{plain}	\newtheorem*{Not}{Notation}

% [uniform figure scaling - maybe this is not a good idea]
\def\figscale{.7}
\def\lscale{1.0}

% [FIX ME! - red makes it easier to spot]
\newcommand{\FIX}[1]{\textbf{\textcolor{red}{#1}}}


    \begin{document}
    \def\noprelim{}
\else
    % already included once
    % input post files

    \singlespacing
    \bibliographystyle{../bibliography/expanded}
    \bibliography{../bibliography/references}

    \end{document}
\fi\fi
