\ifx\master\undefined\input{settings/autocompile}\fi
\chapter*{Introduction} 
\addcontentsline{toc}{chapter}{Introduction}

This thesis describes a search for the Higgs boson, a new particle predicted by
the standard model of particle physics.  The search is optimized for a
particular extension of the standard model, a theory called the Minimal
Supersymmetric Model (MSSM).  The analysis uses the 2010 dataset from the
Compact Muon Solenoid (CMS) experiment, which contains 36~\pbinv of integrated
luminosity at a center of mass energy of 7~\TeV.  The Higgs boson is
hypothesized to be the catalyst of electroweak symmetry breaking, the phenomenon
strongly believed to impart mass to particles that form our natural world.

Chapter~\ref{ch:theory} begins with an introduction to the standard model (SM) of
particle physics. Emphasis is given to electroweak symmetry
breaking and the Higgs mechanism, the theoretical phenomena that motivate the
presence of a Higgs boson.  The theoretical issues which motivated the
development of the MSSM are discussed, and a brief introduction is given.
Finally, the phenomenology of Higgs bosons in the SM and MSSM is discussed, with
an overview of Higgs searches performed at LEP and the Tevatron.

This thesis then documents the development of a complete search for MSSM Higgs
bosons at CMS\@.  The CMS experiment is introduced briefly in
Chapter~\ref{ch:detector}. Chapters~\ref{ch:tanc} and~\ref{ch:svfit} document in
detail two fundamental components of the search, an advanced tau identification
algorithm, and a novel method for reconstructing the neutrinos associated to tau
decays.  The development of these algorithms was motivated by the challenges of
this analysis, and precipitated significant improvements in the final result.
Finally, in \mbox{Chapters~\ref{ch:selections}-\ref{ch:systematics}}, we
describe the methods and results of the event selection, background estimation,
and the systematic uncertainties, and finally compute an upper limit on the
presence of an MSSM Higgs boson.

The studies presented herein were part of a larger effort at CMS to search for
an MSSM Higgs boson decaying to tau lepton pairs.  In addition to the
$\mu-\tau_h$ channel described in this thesis, final states with an electronic
and hadronic tau decay ($e-\tau$) and electronic and muonic ($e-\mu$) were
considered.  The combination of all three was used to set a limit on the
MSSM~\cite{HIG-10-002}.  This result has recently been accepted for publication
in Physical Review Letters B.  At the time of this writing, the CMS analysis
sets the worlds strongest limit on the MSSM Higgs boson using a direct search.
%
\ifx\master\undefined\input{settings/autocompile}\fi
