\ifx\master\undefined\input{settings/autocompile}\fi
\chapter{Monte Carlo Corrections} \label{ch:corrections} 
\section{Muon Identification Efficiency}
\label{sec:ZmumuTagAndProbe}
The identification efficiencies associated with the muon in the $\mu$--$\tau$
channel are measured in $Z\rightarrow\mu^+\mu^-$ events using the ``tag and
probe'' technique~\cite{EWK-10-002}.  $Z\rightarrow\mu^+\mu^-$ events are
selected from the \mbox{/Mu/Run2010B-PromptReco-v2/RECO} and
\mbox{/Mu/Run2010A-Sep17ReReco\_v2/RECO} 7TeV CMS datasets by requiring that the
events pass the ``loose'' Vector Boson Task Force (VBTF) event
selections~\cite{EWK-10-002}.  In the selected events, we define the ``tag''
muons as those that have transverse momentum greater than 15~GeV/$c$ and pass
the VBTF muon selection.  The tag muons are further required to pass the
``combined relative isolation'' described in the VBTF paper.  We finally require
that the tag muon be matched to an HLT object corresponding to the
run--dependent requirements listed in table~\ref{tab:AHtoMuTauTriggers}.  We
compare the events selected in data to simulated $Z\rightarrow\mu^+\mu^-$
events.  The selection of events and tag muon in the simulated sample is the
same as the data sample, with the notable exception that the only HLT
requirement applied in MC is that the tag muon is matched to an HLT\_Mu9 object.  

The efficiencies for the muon selections applied in this analysis are measured
using the ``probe'' objects.  We measure the following marginal efficiencies:
\begin{itemize}
\item Efficiency of global probe muons to satisfy VBTF muon identification
selections.
\item Efficiency of global probe muons passing the VBTF muon identification
selection to satisfy the isolation criteria described in
section~\ref{sec:MuonId}.
\item Efficiency of probe muons passing the offline analysis selection to
pass the HLT selection.
\end{itemize}

In each case, the invariant mass spectrum of the tau--probe pair is fitted with
a Crystal Ball function for the signal ($Z\rightarrow\mu^+\mu^-$) events and an
exponential for the background.  The fit is done for two cases; where the probe
fails the selection and the where it passes.  The signal yield $N$ is extracted
from each fit and the efficiency is computed as $N_{pass}/(N_{pass} + N_{fail}$.
Each efficiency is measured in both the data and the simulation. The results of
the measurements are shown in table~\ref{tab:muonTagAndProbeResults}. In the
final analysis, the simulated events are weighted by the fractional difference
to the measured values; the statistical uncertainty on the weight is taken as
the sum in quadrature of the statistical uncertainties for the data and
simulation efficiency measurements.

The correction for the trigger efficiency needs to take into account the
differences in the HLT selections applied during different operating periods
(see table~\ref{tab:AHtoMuTauTriggers}).  To determine the overall correction
factor, we measure the trigger efficiency in data for each of the operating
periods and compare it to the simualted efficiency of the HLT\_Mu9 selection.
The overall efficiency in data is taken as the average of the three periods,
weighted by integrated luminosity.

The efficiency of the ``cross--triggers'' used in the run--range period 148822 -
149182 (period C) cannot be measured in $Z\rightarrow\mu^+\mu^-$ events as they
require a reconstructed PFTau object at the trigger level.  A single muon trigger
(HLT\_Mu15) is also used in period C.  The contribution of the cross--triggers
is taken as a correction to the single muon trigger period C efficiency. The
``muon leg'' of the cross--triggers have the same requirements as the single
muon triggers used in the run--range 147196 - 148058 (period B).  The
``cross--trigger'' contribution is estimated as the difference between the
efficiency in period B and the single--muon period C efficiency multiplied by a
correction factor of $0.9 \pm 10\%$ to account for the $\tau$ leg efficiency.
In the case that the measured single--muon period C efficiency is larger than
the period B efficiency (due to statistical fluctuations and improvements in the
trigger system), the period B efficiency is increased by 2\%.

\begin{table}[t]
\begin{center}
%\tablesize
\begin{tabular}{|l|c|c|c|c|}
\hline
\multirow{2}{*}{Muon selection} &  \multicolumn{2}{|c|}{Efficiency} & \multirow{2}{*}{Ratio} & \multirow{2}{*}{Corection} \\ 
&  Data  &          Simulation &      &      \\ 
\hline
% cut                           data      mc     ratio     correction
VBTF identification &   $99.2^{+0.1}_{-0.1}$\%  &  $99.1^{+0.1}_{-0.1}$\% & $1.001^{+0.001}_{-0.001}$ & 1.0\\
Particle Isolation  &  $76.8^{+0.4}_{-0.4}$\% &  $78.3^{+0.3}_{-0.3}$\% & $0.981^{+0.006}_{-0.006}$ & 0.98 \\
Trigger             &   $95.0^{+0.5}_{-0.5}$\% & $96.5^{+0.1}_{-0.2}$\% & $0.984^{+0.006}_{-0.006}$ & 0.98 \\
\hline
\end{tabular}
\end{center}
\begin{center}
\caption[Muon trigger, identification, and isolation correction
factors]{Efficiency of the various global muon selections applied in the
analysis measured in data and simulated $Z\rightarrow\mu^+\mu^-$ events.  The
``correction'' column gives the event weight correction applied to the simulated
events in the final analysis.  The efficiency for each selection is the marginal
efficiency with respect to the selection in the row above it.  }

\label{tab:muonTagAndProbeResults}
\end{center}
\end{table}

\begin{figure}[t]
\begin{center}
%\includegraphics*[height=52mm, viewport=19 0 524
%396]{figures/evtReco_PzetaDefinition.pdf}
\includegraphics*[height=52mm]{corrections_chapter/figures/pt_iso_pt_scaled.pdf}
\includegraphics*[height=52mm]{corrections_chapter/figures/etatrig_iso_abseta_scaled.pdf}
\includegraphics*[height=52mm]{corrections_chapter/figures/pt_trigCompX_pt_scaled.pdf}
\includegraphics*[height=52mm]{corrections_chapter/figures/etatrig_trigCompX_abseta_scaled.pdf}
\caption[Muon isolation correction factors]{Ratio of muon isolation efficiency
measured in data compared to simulated $Z\rightarrow\mu^+\mu^-$ events.}
\label{fig:MuonIsoCorrVersusPt}
\end{center}
\end{figure} 


\ifx\master\undefined\input{settings/autocompile}\fi
