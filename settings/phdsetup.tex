% [ USER VARIABLES ]

\def\PHDTITLE {Searches for Higgs Bosons decaying to $\tau$ leptons at $\sqrt s=7$~TeV}
\def\PHDAUTHOR{Evan Klose Friis}
\def\PHDSCHOOL{University of California, Davis}

\def\PHDMONTH {June}
\def\PHDYEAR  {2011}
\def\PHDDEPT {Physics}

\def\BSSCHOOL {University of California at San Diego}
\def\BSYEAR   {2005}

\def\PHDCOMMITTEEA{Professor John Conway}
\def\PHDCOMMITTEEB{Professor Robin Erbacher}
\def\PHDCOMMITTEEC{Professor Mani Tripathi}

% [ GLOBAL SETUP ]

\documentclass[letterpaper,oneside,11pt,draft]{report}

\usepackage{calc}
\usepackage{breakcites}
\usepackage[newcommands]{ragged2e}

\usepackage[pdftex,final]{graphicx}
\DeclareGraphicsRule{*}{mps}{*}{}
\usepackage{epstopdf}
\usepackage{multirow}
\usepackage{fixme}
\usepackage{subfigure}

\usepackage{feynmp}
%\usepackage{tikz}
%\usetikzlibrary{positioning} % [right=of ...]
%\usetikzlibrary{fit} % [fit= ...]

%\pgfdeclarelayer{background layer}
%\pgfdeclarelayer{foreground layer}
%\pgfsetlayers{background layer,main,foreground layer}

%\newenvironment{wrap}{\noindent\begin{minipage}[t]{\linewidth}\vspace{-0.5\normalbaselineskip}\centering}{\vspace{0.5\normalbaselineskip}\end{minipage}}

%% [Venn diagram environment]
%\newenvironment{venn2}
%{\begin{tikzpicture} [every pin/.style={text=black, text opacity=1.0, pin distance=0.5cm, pin edge={black!60, semithick}},
%% define a new style 'venn'
%venn/.style={circle, draw=black!60, semithick, minimum size = 4cm}]
%
%% create circle and give it external (pin) label
%\node[venn] (X) at (-1,0) [pin={150:$H[X]$}] {};
%\node[venn] (Y) at (1,0) [pin={30:$H[Y]$}] {};
%
%% place labels of the atoms by hand
%\node at (-1.9,0) {$H[X|Y]$};
%\node at (1.9,0) {$H[Y|X]$};
%\node at (0,0) {$I[X;Y]$};}
%{\end{tikzpicture}}

%\newcommand{\wrapmath}[1]{\begin{wrap}\begin{tikzpicture}[every node/.style={inner ysep=0ex, inner xsep=0em}]\node[] {$\displaystyle\begin{aligned} #1\end{aligned}$};\end{tikzpicture}\end{wrap}}

\renewenvironment{abstract}{\chapter*{Abstract}}{}
\renewcommand{\bibname}{Bibliography}
\renewcommand{\contentsname}{Table of Contents}

\makeatletter
% This doesn't compile?
%\renewcommand{\@biblabel}[1]{\textsc{#1}}
\makeatother

% [ FONT SETTINGS ]

\usepackage[tbtags, intlimits, namelimits]{amsmath}
%\usepackage[adobe-utopia]{mathdesign}

%\DeclareSymbolFont{pazomath}{OMS}{zplm}{m}{n}
%\DeclareSymbolFontAlphabet{\mathcal}{pazomath}
%\SetMathAlphabet\mathcal{bold}{OMS}{zplm}{b}{n}

%\SetSymbolFont{largesymbols}{normal}{OMX}{zplm}{m}{n}
%\SetSymbolFont{largesymbols}{bold}{OMX}{zplm}{m}{n}
%\SetSymbolFont{symbols}{normal}{OMS}{zplm}{m}{n}
%\SetSymbolFont{symbols}{bold}{OMS}{zplm}{b}{n}

\renewcommand{\sfdefault}{phv}
\renewcommand{\ttdefault}{fvm}

\widowpenalty 8000
\clubpenalty  8000

% [ PAGE LAYOUT ]
\usepackage{geometry}
\geometry{lmargin = 1.5in}
\geometry{rmargin = 1.0in}
\geometry{tmargin = 1.0in}
\geometry{bmargin = 1.0in}

% [ PDF SETTINGS ]
\usepackage{color}
\definecolor{darkred}{rgb}{0.663, 0.059, 0.2}

\usepackage[final]{hyperref}
\hypersetup{breaklinks  = true}
\hypersetup{colorlinks  = true}
\hypersetup{linktocpage = false}
\hypersetup{linkcolor   = darkred}
\hypersetup{citecolor   = darkred}
\hypersetup{urlcolor    = black}
\hypersetup{plainpages  = false}
\hypersetup{pageanchor  = true}
\hypersetup{pdfauthor   = {\PHDAUTHOR}}
\hypersetup{pdftitle    = {\PHDTITLE}}
\hypersetup{pdfsubject  = {Dissertation, \PHDSCHOOL}}
\urlstyle{same}
%\usepackage{refcheck}

% [ LETTER SPACING ]

\usepackage[final]{microtype}
\microtypesetup{protrusion=compatibility}
\microtypesetup{expansion=false}

\newcommand{\upper}[1]{\MakeUppercase{#1}}
\let\lsscshape\scshape

\ifcase\pdfoutput\else\microtypesetup{letterspace=15}
\renewcommand{\scshape}{\lsscshape\lsstyle}
\renewcommand{\upper}[1]{\textls[50]{\MakeUppercase{#1}}}\fi

% [ LINE SPACING ]

\usepackage[doublespacing]{setspace}
\renewcommand{\displayskipstretch}{0.0}

\setlength{\parskip   }{0em}
\setlength{\parindent }{2em}

% [ TABLE FORMATTING ]

\usepackage{booktabs}
\setlength{\heavyrulewidth}{1.5\arrayrulewidth}
\setlength{\lightrulewidth}{1.0\arrayrulewidth}
\setlength{\doublerulesep }{2.0\arrayrulewidth}

% [ SECTION FORMATTING ]

\usepackage[largestsep,nobottomtitles*]{titlesec}
\renewcommand{\bottomtitlespace}{0.75in}

\titleformat{\chapter}[display]{\bfseries\huge\singlespacing}{\filleft\textsc{\LARGE \chaptertitlename\ \thechapter}}{-0.2ex}{\titlerule[3pt]\vspace{0.2ex}}[]

\titleformat{\section}{\LARGE}{\S\thesection\hspace{0.5em}}{0ex}{}
\titleformat{\subsection}{\Large}{\S\thesubsection\hspace{0.5em}}{0ex}{}
\titleformat{\subsubsection}{\large}{\thesubsubsection\hspace{0.5em}}{0ex}{}

\titlespacing*{\chapter}{0em}{6ex}{4ex plus 2ex minus 0ex}
\titlespacing*{\section}{0em}{2ex plus 3ex minus 1ex}{0.5ex plus 0.5ex minus 0.5ex}
\titlespacing*{\subsection}{0ex}{2ex plus 3ex minus 1ex}{0ex}
\titlespacing*{\subsubsection}{0ex}{2ex plus 0ex minus 1ex}{0ex}

% [ HEADER SETTINGS ]

\usepackage{fancyhdr}

\setlength{\headheight}{\normalbaselineskip}
\setlength{\footskip  }{0.5in}
\setlength{\headsep   }{0.5in-\headheight}

\fancyheadoffset[R]{0.5in}
\renewcommand{\headrulewidth}{0pt}
\renewcommand{\footrulewidth}{0pt}

\newcommand{\pagebox}{\parbox[r][\headheight][t]{0.5in}{\hspace\fill\thepage}}

\newcommand{\prelimheaders}{\ifx\prelim\undefined\renewcommand{\thepage}{\textit{\roman{page}}}\fancypagestyle{plain}{\fancyhf{}\fancyfoot[L]{\makebox[\textwidth-0.5in]{\thepage}}}\pagestyle{plain}\def\prelim{}\fi}

\newcommand{\normalheaders}{\renewcommand{\thepage}{\arabic{page}}\fancypagestyle{plain}{\fancyhf{}\fancyhead[R]{\pagebox}}\pagestyle{plain}}

\normalheaders{}

% [ CUSTOM COMMANDS ]

\newcommand{\signaturebox}[1]{\multicolumn{1}{p{4in}}{\vspace{3ex}}\\\midrule #1\\}

\usepackage{xspace}
\usepackage{includes/pennames}
%%%%%%%%%%%%%%%%%%%%%%%%%%%%%%%%%%%%%%%%%%%%%%%%%%%%%%%%%%%%%%%%%%%%
%
%  Common definitions
%
%  N.B. use of \providecommand rather than \newcommand means
%       that a definition is ignored if already specified
%
%                                              L. Taylor 18 Feb 2005
%%%%%%%%%%%%%%%%%%%%%%%%%%%%%%%%%%%%%%%%%%%%%%%%%%%%%%%%%%%%%%%%%%%%

% Some shorthand
% turn off italics
\newcommand {\etal}{\mbox{et al.}\xspace} %et al. - no preceding comma
\newcommand {\ie}{\mbox{i.e.}\xspace}     %i.e.
\newcommand {\eg}{\mbox{e.g.}\xspace}     %e.g.
\newcommand {\etc}{\mbox{etc.}\xspace}     %etc.
\newcommand {\vs}{\mbox{\sl vs.}\xspace}      %vs.
\newcommand {\mdash}{\ensuremath{\mathrm{-}}} % for use within formulas

% some terms whose definition we may change
\newcommand {\Lone}{Level-1\xspace} % Level-1 or L1 ?
\newcommand {\Ltwo}{Level-2\xspace}
\newcommand {\Lthree}{Level-3\xspace}

% Some software programs (alphabetized)
\providecommand{\ACERMC} {\textsc{AcerMC}\xspace}
\providecommand{\ALPGEN} {{\textsc{alpgen}}\xspace}
\providecommand{\CHARYBDIS} {{\textsc{charybdis}}\xspace}
\providecommand{\CMKIN} {\textsc{cmkin}\xspace}
\providecommand{\CMSIM} {{\textsc{cmsim}}\xspace}
\providecommand{\CMSSW} {{\textsc{cmssw}}\xspace}
\providecommand{\COBRA} {{\textsc{cobra}}\xspace}
\providecommand{\COCOA} {{\textsc{cocoa}}\xspace}
\providecommand{\COMPHEP} {\textsc{CompHEP}\xspace}
\providecommand{\EVTGEN} {{\textsc{evtgen}}\xspace}
\providecommand{\FAMOS} {{\textsc{famos}}\xspace}
\providecommand{\GARCON} {\textsc{garcon}\xspace}
\providecommand{\GARFIELD} {{\textsc{garfield}}\xspace}
\providecommand{\GEANE} {{\textsc{geane}}\xspace}
\providecommand{\GEANTfour} {{\textsc{geant4}}\xspace}
\providecommand{\GEANTthree} {{\textsc{geant3}}\xspace}
\providecommand{\GEANT} {{\textsc{geant}}\xspace}
\providecommand{\HDECAY} {\textsc{hdecay}\xspace}
\providecommand{\HERWIG} {{\textsc{herwig}}\xspace}
\providecommand{\HIGLU} {{\textsc{higlu}}\xspace}
\providecommand{\HIJING} {{\textsc{hijing}}\xspace}
\providecommand{\IGUANA} {\textsc{iguana}\xspace}
\providecommand{\ISAJET} {{\textsc{isajet}}\xspace}
\providecommand{\ISAPYTHIA} {{\textsc{isapythia}}\xspace}
\providecommand{\ISASUGRA} {{\textsc{isasugra}}\xspace}
\providecommand{\ISASUSY} {{\textsc{isasusy}}\xspace}
\providecommand{\ISAWIG} {{\textsc{isawig}}\xspace}
\providecommand{\MADGRAPH} {\textsc{MadGraph}\xspace}
\providecommand{\MCATNLO} {\textsc{mc@nlo}\xspace}
\providecommand{\MCFM} {\textsc{mcfm}\xspace}
\providecommand{\MILLEPEDE} {{\textsc{millepede}}\xspace}
\providecommand{\ORCA} {{\textsc{orca}}\xspace}
\providecommand{\OSCAR} {{\textsc{oscar}}\xspace}
\providecommand{\PHOTOS} {\textsc{photos}\xspace}
\providecommand{\PROSPINO} {\textsc{prospino}\xspace}
\providecommand{\PYTHIA} {{\textsc{pythia}}\xspace}
\providecommand{\SHERPA} {{\textsc{sherpa}}\xspace}
\providecommand{\TAUOLA} {\textsc{tauola}\xspace}
\providecommand{\TOPREX} {\textsc{TopReX}\xspace}
\providecommand{\XDAQ} {{\textsc{xdaq}}\xspace}


%  Experiments
\newcommand {\DZERO}{D\O\xspace}     %etc.


% Measurements and units...

\newcommand{\de}{\ensuremath{^\circ}}
\newcommand{\ten}[1]{\ensuremath{\times \text{10}^\text{#1}}}
\newcommand{\unit}[1]{\ensuremath{\text{\,#1}}\xspace}
\newcommand{\mum}{\ensuremath{\,\mu\text{m}}\xspace}
\newcommand{\micron}{\ensuremath{\,\mu\text{m}}\xspace}
\newcommand{\cm}{\ensuremath{\,\text{cm}}\xspace}
\newcommand{\mm}{\ensuremath{\,\text{mm}}\xspace}
\newcommand{\mus}{\ensuremath{\,\mu\text{s}}\xspace}
\newcommand{\keV}{\ensuremath{\,\text{ke\hspace{-.08em}V}}\xspace}
\newcommand{\MeV}{\ensuremath{\,\text{Me\hspace{-.08em}V}}\xspace}
\newcommand{\GeV}{\ensuremath{\,\text{Ge\hspace{-.08em}V}}\xspace}
\newcommand{\TeV}{\ensuremath{\,\text{Te\hspace{-.08em}V}}\xspace}
\newcommand{\PeV}{\ensuremath{\,\text{Pe\hspace{-.08em}V}}\xspace}
\newcommand{\keVc}{\ensuremath{{\,\text{ke\hspace{-.08em}V\hspace{-0.16em}/\hspace{-0.08em}}c}}\xspace}
\newcommand{\MeVc}{\ensuremath{{\,\text{Me\hspace{-.08em}V\hspace{-0.16em}/\hspace{-0.08em}}c}}\xspace}
\newcommand{\GeVc}{\ensuremath{{\,\text{Ge\hspace{-.08em}V\hspace{-0.16em}/\hspace{-0.08em}}c}}\xspace}
\newcommand{\TeVc}{\ensuremath{{\,\text{Te\hspace{-.08em}V\hspace{-0.16em}/\hspace{-0.08em}}c}}\xspace}
\newcommand{\keVcc}{\ensuremath{{\,\text{ke\hspace{-.08em}V\hspace{-0.16em}/\hspace{-0.08em}}c^\text{2}}}\xspace}
\newcommand{\MeVcc}{\ensuremath{{\,\text{Me\hspace{-.08em}V\hspace{-0.16em}/\hspace{-0.08em}}c^\text{2}}}\xspace}
\newcommand{\GeVcc}{\ensuremath{{\,\text{Ge\hspace{-.08em}V\hspace{-0.16em}/\hspace{-0.08em}}c^\text{2}}}\xspace}
\newcommand{\TeVcc}{\ensuremath{{\,\text{Te\hspace{-.08em}V\hspace{-0.16em}/\hspace{-0.08em}}c^\text{2}}}\xspace}

\newcommand{\pbinv} {\mbox{\ensuremath{\,\text{pb}^\text{$-$1}}}\xspace}
\newcommand{\fbinv} {\mbox{\ensuremath{\,\text{fb}^\text{$-$1}}}\xspace}
\newcommand{\nbinv} {\mbox{\ensuremath{\,\text{nb}^\text{$-$1}}}\xspace}
\newcommand{\percms}{\ensuremath{\,\text{cm}^\text{$-$2}\,\text{s}^\text{$-$1}}\xspace}
\newcommand{\lumi}{\ensuremath{\mathcal{L}}\xspace}
\newcommand{\Lumi}{\ensuremath{\mathcal{L}}\xspace}%both upper and lower
%
% Need a convention here:
\newcommand{\LvLow}  {\ensuremath{\mathcal{L}=\text{10}^\text{32}\,\text{cm}^\text{$-$2}\,\text{s}^\text{$-$1}}\xspace}
\newcommand{\LLow}   {\ensuremath{\mathcal{L}=\text{10}^\text{33}\,\text{cm}^\text{$-$2}\,\text{s}^\text{$-$1}}\xspace}
\newcommand{\lowlumi}{\ensuremath{\mathcal{L}=\text{2}\times \text{10}^\text{33}\,\text{cm}^\text{$-$2}\,\text{s}^\text{$-$1}}\xspace}
\newcommand{\LMed}   {\ensuremath{\mathcal{L}=\text{2}\times \text{10}^\text{33}\,\text{cm}^\text{$-$2}\,\text{s}^\text{$-$1}}\xspace}
\newcommand{\LHigh}  {\ensuremath{\mathcal{L}=\text{10}^\text{34}\,\text{cm}^\text{$-$2}\,\text{s}^\text{$-$1}}\xspace}
\newcommand{\hilumi} {\ensuremath{\mathcal{L}=\text{10}^\text{34}\,\text{cm}^\text{$-$2}\,\text{s}^\text{$-$1}}\xspace}

% Some usual physics terms


% SM (still to be classified)

\newcommand{\kt}{\ensuremath{k_{\mathrm{T}}}\xspace}
\newcommand{\BC}{\ensuremath{{B_{\mathrm{c}}}}\xspace}
\newcommand{\bbarc}{\ensuremath{{\overline{b}c}}\xspace}
\newcommand{\bbbar}{\ensuremath{{b\overline{b}}}\xspace}
\newcommand{\ccbar}{\ensuremath{{c\overline{c}}}\xspace}
\newcommand{\JPsi}{\ensuremath{{J}\hspace{-.08em}/\hspace{-.14em}\psi}\xspace}
\newcommand{\bspsiphi}{\ensuremath{B_s \to \JPsi\, \phi}\xspace}
%\newcommand{\ttbar}{\ensuremath{{t\overline{t}}}\xspace}
\newcommand{\AFB}{\ensuremath{A_\text{FB}}\xspace}
\newcommand{\EE}{\ensuremath{e^+e^-}\xspace}
\newcommand{\MM}{\ensuremath{\mu^+\mu^-}\xspace}
\newcommand{\TT}{\ensuremath{\tau^+\tau^-}\xspace}
\newcommand{\wangle}{\ensuremath{\sin^{2}\theta_{\text{eff}}^\text{lept}(M^2_\mathrm{Z})}\xspace}
\newcommand{\ttbar}{\ensuremath{{t\overline{t}}}\xspace}
\newcommand{\stat}{\ensuremath{\,\text{(stat.)}}\xspace}
\newcommand{\syst}{\ensuremath{\,\text{(syst.)}}\xspace}
% these moved to similar defs
%\newcommand{\Etmiss}{\ensuremath{E_{\mathrm{T}\!{\rm miss}}}}
%\newcommand{\VEtmiss}{\ensuremath{{\vec E}_{\mathrm{T}\!{\rm miss}}}}

%%%  E-gamma definitions
\newcommand{\HGG}{\ensuremath{\mathrm{H}\to\gamma\gamma}}
\newcommand{\gev}{\GeV}
\newcommand{\GAMJET}{\ensuremath{\gamma + \text{jet}}}
\newcommand{\PPTOJETS}{\ensuremath{\mathrm{pp}\to\text{jets}}}
\newcommand{\PPTOGG}{\ensuremath{\mathrm{pp}\to\gamma\gamma}}
\newcommand{\PPTOGAMJET}{\ensuremath{\mathrm{pp}\to\gamma +
\mathrm{jet}
}}
\newcommand{\MH}{\ensuremath{\mathrm{M_{\mathrm{H}}}}}
\newcommand{\RNINE}{\ensuremath{\mathrm{R}_\mathrm{9}}}
\newcommand{\DR}{\ensuremath{\Delta\mathrm{R}}}



% Physics symbols ...

\newcommand{\PT}{\ensuremath{p_{\mathrm{T}}}\xspace}
\newcommand{\pt}{\ensuremath{p_{\mathrm{T}}}\xspace}
\newcommand{\ET}{\ensuremath{E_{\mathrm{T}}}\xspace}
\newcommand{\HT}{\ensuremath{H_{\mathrm{T}}}\xspace}
\newcommand{\et}{\ensuremath{E_{\mathrm{T}}}\xspace}
\providecommand{\Em}{\ensuremath{E\hspace{-0.6em}/}\xspace}
\providecommand{\Pm}{\ensuremath{p\hspace{-0.5em}/}\xspace}
\providecommand{\PTm}{\ensuremath{{p}_\mathrm{T}\hspace{-1.02em}/}\xspace}
\providecommand{\PTslash}{\ensuremath{{p}_\mathrm{T}\hspace{-1.02em}/}\xspace}
\newcommand{\ETm}{\ensuremath{E_{\mathrm{T}}^{\text{miss}}}\xspace}
\providecommand{\ETslash}{\ensuremath{E_{\mathrm{T}}\hspace{-1.1em}/}\xspace}
\newcommand{\MET}{\ensuremath{E_{\mathrm{T}}^{\text{miss}}}\xspace}
\newcommand{\ETmiss}{\ensuremath{E_{\mathrm{T}}^{\text{miss}}}\xspace}
\newcommand{\VEtmiss}{\ensuremath{{\vec E}_{\mathrm{T}}^{\text{miss}}}\xspace}

% roman face derivative
\providecommand{\dd}[2]{\ensuremath{\frac{\mathrm{d} #1}{\mathrm{d} #2}}}
% Particle names which will track the italic/non-italic face convention
\providecommand{\zp}{\ensuremath{\mathrm{Z}^\prime}\xspace}
\providecommand{\JPsi}{\ensuremath{\mathrm{J}\hspace{-.08em}/\hspace{-.14em}\psi}\xspace}
\providecommand{\Z}{\ensuremath{\mathrm{Z}}\xspace}
\providecommand{\ttbar}{\ensuremath{\mathrm{t}\overline{\mathrm{t}}}\xspace}
% Extensions for missing names in PENNAMES
\newcommand{\cPgn}{\ensuremath{\nu}}
\newcommand{\cPJgy}{\JPsi}
\newcommand{\cPZ}{\Z}
\newcommand{\cPZpr}{\zp}
%%%%%%
% From Albert
%

\newcommand{\ga}{\ensuremath{\gtrsim}}
\newcommand{\la}{\ensuremath{\lesssim}}
%\def\ga{\mathrel{\rlap{\raise.6ex\hbox{$>$}}{\lower.6ex\hbox{$\sim$}}}}
%\def\la{\mathrel{\rlap{\raise.6ex\hbox{$<$}}{\lower.6ex\hbox{$\sim$}}}}
%
\newcommand{\swsq}{\ensuremath{\sin^2\theta_W}\xspace}
\newcommand{\cwsq}{\ensuremath{\cos^2\theta_W}\xspace}
\newcommand{\tanb}{\ensuremath{\tan\beta}\xspace}
\newcommand{\tanbsq}{\ensuremath{\tan^{2}\beta}\xspace}
\newcommand{\sidb}{\ensuremath{\sin 2\beta}\xspace}
\newcommand{\alpS}{\ensuremath{\alpha_S}\xspace}
\newcommand{\alpt}{\ensuremath{\tilde{\alpha}}\xspace}

\newcommand{\QL}{\ensuremath{Q_L}\xspace}
\newcommand{\sQ}{\ensuremath{\tilde{Q}}\xspace}
\newcommand{\sQL}{\ensuremath{\tilde{Q}_L}\xspace}
\newcommand{\ULC}{\ensuremath{U_L^C}\xspace}
\newcommand{\sUC}{\ensuremath{\tilde{U}^C}\xspace}
\newcommand{\sULC}{\ensuremath{\tilde{U}_L^C}\xspace}
\newcommand{\DLC}{\ensuremath{D_L^C}\xspace}
\newcommand{\sDC}{\ensuremath{\tilde{D}^C}\xspace}
\newcommand{\sDLC}{\ensuremath{\tilde{D}_L^C}\xspace}
\newcommand{\LL}{\ensuremath{L_L}\xspace}
\newcommand{\sL}{\ensuremath{\tilde{L}}\xspace}
\newcommand{\sLL}{\ensuremath{\tilde{L}_L}\xspace}
\newcommand{\ELC}{\ensuremath{E_L^C}\xspace}
\newcommand{\sEC}{\ensuremath{\tilde{E}^C}\xspace}
\newcommand{\sELC}{\ensuremath{\tilde{E}_L^C}\xspace}
\newcommand{\sEL}{\ensuremath{\tilde{E}_L}\xspace}
\newcommand{\sER}{\ensuremath{\tilde{E}_R}\xspace}
\newcommand{\sFer}{\ensuremath{\tilde{f}}\xspace}
\newcommand{\sQua}{\ensuremath{\tilde{q}}\xspace}
\newcommand{\sUp}{\ensuremath{\tilde{u}}\xspace}
\newcommand{\suL}{\ensuremath{\tilde{u}_L}\xspace}
\newcommand{\suR}{\ensuremath{\tilde{u}_R}\xspace}
\newcommand{\sDw}{\ensuremath{\tilde{d}}\xspace}
\newcommand{\sdL}{\ensuremath{\tilde{d}_L}\xspace}
\newcommand{\sdR}{\ensuremath{\tilde{d}_R}\xspace}
\newcommand{\sTop}{\ensuremath{\tilde{t}}\xspace}
\newcommand{\stL}{\ensuremath{\tilde{t}_L}\xspace}
\newcommand{\stR}{\ensuremath{\tilde{t}_R}\xspace}
\newcommand{\stone}{\ensuremath{\tilde{t}_1}\xspace}
\newcommand{\sttwo}{\ensuremath{\tilde{t}_2}\xspace}
\newcommand{\sBot}{\ensuremath{\tilde{b}}\xspace}
\newcommand{\sbL}{\ensuremath{\tilde{b}_L}\xspace}
\newcommand{\sbR}{\ensuremath{\tilde{b}_R}\xspace}
\newcommand{\sbone}{\ensuremath{\tilde{b}_1}\xspace}
\newcommand{\sbtwo}{\ensuremath{\tilde{b}_2}\xspace}
\newcommand{\sLep}{\ensuremath{\tilde{l}}\xspace}
\newcommand{\sLepC}{\ensuremath{\tilde{l}^C}\xspace}
\newcommand{\sEl}{\ensuremath{\tilde{e}}\xspace}
\newcommand{\sElC}{\ensuremath{\tilde{e}^C}\xspace}
\newcommand{\seL}{\ensuremath{\tilde{e}_L}\xspace}
\newcommand{\seR}{\ensuremath{\tilde{e}_R}\xspace}
\newcommand{\snL}{\ensuremath{\tilde{\nu}_L}\xspace}
\newcommand{\sMu}{\ensuremath{\tilde{\mu}}\xspace}
\newcommand{\sNu}{\ensuremath{\tilde{\nu}}\xspace}
\newcommand{\sTau}{\ensuremath{\tilde{\tau}}\xspace}
\newcommand{\Glu}{\ensuremath{g}\xspace}
\newcommand{\sGlu}{\ensuremath{\tilde{g}}\xspace}
\newcommand{\Wpm}{\ensuremath{W^{\pm}}\xspace}
\newcommand{\sWpm}{\ensuremath{\tilde{W}^{\pm}}\xspace}
\newcommand{\Wz}{\ensuremath{W^{0}}\xspace}
\newcommand{\sWz}{\ensuremath{\tilde{W}^{0}}\xspace}
\newcommand{\sWino}{\ensuremath{\tilde{W}}\xspace}
\newcommand{\Bz}{\ensuremath{B^{0}}\xspace}
\newcommand{\sBz}{\ensuremath{\tilde{B}^{0}}\xspace}
\newcommand{\sBino}{\ensuremath{\tilde{B}}\xspace}
\newcommand{\Zz}{\ensuremath{Z^{0}}\xspace}
\newcommand{\sZino}{\ensuremath{\tilde{Z}^{0}}\xspace}
\newcommand{\sGam}{\ensuremath{\tilde{\gamma}}\xspace}
\newcommand{\chiz}{\ensuremath{\tilde{\chi}^{0}}\xspace}
\newcommand{\chip}{\ensuremath{\tilde{\chi}^{+}}\xspace}
\newcommand{\chim}{\ensuremath{\tilde{\chi}^{-}}\xspace}
\newcommand{\chipm}{\ensuremath{\tilde{\chi}^{\pm}}\xspace}
\newcommand{\Hone}{\ensuremath{H_{d}}\xspace}
\newcommand{\sHone}{\ensuremath{\tilde{H}_{d}}\xspace}
\newcommand{\Htwo}{\ensuremath{H_{u}}\xspace}
\newcommand{\sHtwo}{\ensuremath{\tilde{H}_{u}}\xspace}
\newcommand{\sHig}{\ensuremath{\tilde{H}}\xspace}
\newcommand{\sHa}{\ensuremath{\tilde{H}_{a}}\xspace}
\newcommand{\sHb}{\ensuremath{\tilde{H}_{b}}\xspace}
\newcommand{\sHpm}{\ensuremath{\tilde{H}^{\pm}}\xspace}
\newcommand{\hz}{\ensuremath{h^{0}}\xspace}
\newcommand{\Hz}{\ensuremath{H^{0}}\xspace}
\newcommand{\Az}{\ensuremath{A^{0}}\xspace}
\newcommand{\Hpm}{\ensuremath{H^{\pm}}\xspace}
\newcommand{\sGra}{\ensuremath{\tilde{G}}\xspace}
%
\newcommand{\mtil}{\ensuremath{\tilde{m}}\xspace}
%
\newcommand{\rpv}{\ensuremath{\rlap{\kern.2em/}R}\xspace}
\newcommand{\LLE}{\ensuremath{LL\bar{E}}\xspace}
\newcommand{\LQD}{\ensuremath{LQ\bar{D}}\xspace}
\newcommand{\UDD}{\ensuremath{\overline{UDD}}\xspace}
\newcommand{\Lam}{\ensuremath{\lambda}\xspace}
\newcommand{\Lamp}{\ensuremath{\lambda'}\xspace}
\newcommand{\Lampp}{\ensuremath{\lambda''}\xspace}
%
\newcommand{\spinbd}[2]{\ensuremath{\bar{#1}_{\dot{#2}}}\xspace}

\newcommand{\MD}{\ensuremath{{M_\mathrm{D}}}\xspace}% ED mass
\newcommand{\Mpl}{\ensuremath{{M_\mathrm{Pl}}}\xspace}% Planck mass
\newcommand{\Rinv} {\ensuremath{{R}^{-1}}\xspace}



%%%%%%%%%%%%%%%%%%%%%%%%%%%%%%%%%%%%%%%%%%%%%%%%%%%%%%%%%%%%%%%%%%%%
%
% Hyphenations (only need to add here if you get a nasty word break)
%
\hyphenation{en-viron-men-tal}%    just an example

% Some usual physics terms
\newcommand{\WpJets}{\ensuremath{W + \text{jets}}\xspace}
\newcommand{\ttbarpJets}{\ensuremath{\ttbar + \text{jets}}\xspace}
\newcommand{\taul}{\ensuremath{\tau}~lepton\xspace}
\newcommand{\captiontext}{}
\newcommand{\ZTT}{\ensuremath{Z\to\TT}}
\newcommand{\HTT}[1]{\ensuremath{H(#1)\to\TT}}


% [some math stuff - maybe stick in sep file]
\usepackage{amsthm}
\usepackage{amscd}
\theoremstyle{plain}    \newtheorem{Lem}{Lemma}
\theoremstyle{plain}    \newtheorem*{ProLem}{Proof}
\theoremstyle{plain} 	\newtheorem{Cor}{Corollary}
\theoremstyle{plain} 	\newtheorem*{ProCor}{Proof}
\theoremstyle{plain} 	\newtheorem{The}{Theorem}
\theoremstyle{plain} 	\newtheorem*{ProThe}{Proof}
\theoremstyle{plain} 	\newtheorem{Prop}{Proposition}
\theoremstyle{plain} 	\newtheorem*{ProProp}{Proof}
\theoremstyle{plain} 	\newtheorem*{Conj}{Conjecture}
\theoremstyle{plain}	\newtheorem*{Rem}{Remark}
\theoremstyle{plain}	\newtheorem*{Def}{Definition} 
\theoremstyle{plain}	\newtheorem*{Not}{Notation}

% [uniform figure scaling - maybe this is not a good idea]
\def\figscale{.7}
\def\lscale{1.0}

% [FIX ME! - red makes it easier to spot]
\newcommand{\FIX}[1]{\textbf{\textcolor{red}{#1}}}
