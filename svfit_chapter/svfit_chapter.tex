\ifx\master\undefined\input{settings/autocompile}\fi

\newcommand{\mtau}{m_{\tau}}
\newcommand{\mnus}{m_{\nu\nu}}
\newcommand{\mvis}{m_{vis}}

\chapter{The Secondary Vertex Fit}
\label{ch:svfit}

The dominant background in the search for the Higgs $\to \TT$ signal is due to
Standard Model $\ZTT$ events.  The most ``natural'' observable to discriminate
between Higgs signal and $Z$ background would be the invariant mass of the
di--tau system, utilizing the fact that the $Z$ resonance is well known ($m_{Z}
= 91.1876 \pm 0.0021$~\GeVcc) and has a narrow width ($\Gamma_{Z} = 2.4952 \pm
0.0023$~\GeV)~\cite{PDG}.  The experimental complication in this approach is due
to the neutrinos produced in the tau lepton decays, which escape detection,
carrying away an unmeasured amount of energy, and making it difficult to
reconstruct the tau lepton four--vectors.

The conventional algorithm to reconstruct the di--tau invariant mass assumes
that the momenta of neutrinos produced in tau decays are collinear to the
momenta of the visible tau decay products~\cite{massRecoCollinearApprox}.  The
main disadvantage of the collinear approximation algorithm is that it fails to
find physical solutions for about half of the MSSM Higgs $\to \TT$
decays, in particular for events in which the Higgs boson has small $P_{T}$ and
the two tau leptons are back--to--back in azimuth.  Improvements to the
collinear approximation algorithm have recently been made which aim to recover
part of the events with unphysical solutions~\cite{improvedCollinearApprox}.
But even with these improvements, no physical solution is still found for a
large fraction of signal events.

A novel algorithm is presented in the following, which succeeds in finding a
physical solution for every event.  As an additional benefit, the new algorithm
is found to improve the di--tau invariant mass resolution, making it easier to
separate the Higgs signal from the $\ZTT$ background.

The novel Secondary Vertex fit (SVfit) algorithm for di--tau invariant mass
reconstruction that we present in the following utilizes a likelihood
maximization to fit a $\TT$ invariant mass hypothesis for each
event.  The likelihood is composed of separate terms which represent probability
densities of:
\begin{itemize}
\item tau decay kinematics
\item matching between the momenta of neutrinos produced in the tau decays and
  the reconstructed missing transverse momentum
\item a regularization ``$P_{T}$--balance'' term which accounts for the effects on the di--tau invariant mass
 of acceptance cuts on the visible tau decay products
\item the compatibility of tau decay parameters with the position of reconstructed tracks
 and the known tau lifetime of $c\tau = 87~\micron$~\cite{PDG}.
\end{itemize}
The likelihood is maximized as function of a set of parameters which fully describe the tau decay.

\section{Parametrization of tau decays}
\label{sec:svParameterization}

The decay of a tau of visible four--momentum $P_{vis}$ measured in the CMS
detector (``laboratory'') frame can be parametrized by three variables.  The
invisible (neutrino) momentum is fully determined by these parameters.

The ``opening--angle'' $\theta$ is defined as the angle between the boost
direction of the tau lepton and the momentum vector of the visible decay
products in the rest frame of the tau.  The azimuthal angle of the tau in the
lab frame is denoted as $\overline{\phi}$ (we denote quantities defined in the
laboratory frame by a overline).  A local coordinate system is defined such that the
$\bar{z}$--direction lies along the visible momentum and $\overline{\phi} = 0$ lies
in the plane spanned by the momentum vector of the visible decay products and
the proton beam direction. The third parameter, $m_{\nu\nu}$, denotes the
invariant mass of the invisible momentum system.

Given $\theta$, $\overline{\phi}$ and $m_{\nu\nu}$, the energy and direction of the
tau lepton can be computed by means of the following equations: The energy of
the visible decay products in the rest frame of the tau lepton is related to the
invariant mass of the neutrino system by:
\begin{equation}
E^{vis} = \frac{m_\tau^2 + \mvis^2 - \mnus^2}{2m_\tau} 
\label{eq:restFrameMomentumRelation}
\end{equation}
Note that for hadronic decays, $\mnus$ is a constant of value zero, as only a
single neutrino is produced.  Consequently, the magnitude of $P^{vis}$ depends
on the reconstructed mass of the visible decay products only and is a constant
during the SVfit.

The opening angle $\overline{\theta}$ between the tau lepton direction and the
visible momentum vector in the laboratory frame is determined by the rest frame
quantities via the (Lorentz invariant) component of the visible momentum
perpendicular to the tau lepton direction:
\begin{align}
P^{vis}_{\perp} &= \overline P^{vis}_{\perp} \nonumber \\
\Rightarrow \sin \overline \theta &= \frac{P^{vis} \sin \theta}{\overline P^{vis}} 
\label{eq:labFrameOpeningAngle} 
\end{align}

Substituting the parameters $\mnus$ and $\theta$ into
equations~\ref{eq:restFrameMomentumRelation} and~\ref{eq:labFrameOpeningAngle},
the energy of the tau is obtained by solving for the boost factor $\gamma$ in
the Lorentz transformation between tau rest frame and laboratory frame of the
visible momentum component parallel to the tau direction:
\begin{align}
\overline{P}^{vis} \cos \overline{\theta} &= \gamma \beta E^{vis} + \gamma P^{vis} \cos\theta \nonumber \\ 
\Rightarrow \gamma &= \frac{E^{vis}[{\left( E^{vis} \right)^2 + \left( \overline{P}^{vis} \cos \overline{\theta} \right)^2 
 - \left( P^{vis} \cos \theta \right)^2}]^{1/2} - P^{vis} \cos \theta \overline{P}^{vis} \cos \overline{\theta}}{\left( E^{vis} \right)^2 
 - \left( P^{vis} \cos \theta \right)^2}, \nonumber \\
E^{\tau} &= \gamma m_\tau \nonumber
%\label{eq:LorentzRelationship}
\end{align}

The energy of the tau lepton in the laboratory frame as function of the measured
visible momentum depends on two of the three parameters only - the rest frame
opening angle $\theta$ and the invariant mass $\mnus$ of the neutrino system.
The direction of the tau lepton momentum vector is not fully determined by
$\theta$ and $\mnus$, but is constrained to lie on the surface of a cone of
opening angle $\overline{\theta}$ (given by equation~\ref{eq:labFrameOpeningAngle}),
the axis of which is given by the visible momentum vector.  The tau lepton
four--vector is fully determined by the addition of the third parameter
$\overline{\phi}$, which describes the azimuthal angle of the tau lepton with respect
to the visible momentum vector.  The spatial coordinate system used is
illustrated in figure~\ref{fig:svFitDecayParDiagram}.

\begin{figure}[t]
\begin{center}
\includegraphics*[width=72mm]{svfit_chapter/figures/decay_parameters_graphic.pdf}
\caption{Illustration of the coordinate system used by the SVfit to describe
the decays of tau leptons.} \label{fig:svFitDecayParDiagram}
\end{center}
\end{figure} 

\section{Likelihood for tau decay}

The probability density functions for the tau decay kinematics are taken from
the kinematics review of the PDG~\cite{PDG}.  The likelihood is proportional to
the phase--space volume for two--body ($\tau \to \tau_{had} \nu$) and
three--body ($\tau \to e \nu\nu$ and $\tau \to \mu \nu\nu$) decays.  For
two--body decays the likelihood depends on the decay angle $\theta$ only:
\begin{equation*}
\mathrm{d}\Gamma \propto |\mathcal{M}|^2 \sin \theta d \theta
\label{eq:pdfKineHadronic}
\end{equation*}
For three--body decays, the likelihood depends on the invariant mass of the
neutrino system also:
\begin{equation}
\mathrm{d}\Gamma \propto |\mathcal{M}|^2 
\frac{((\mtau^2 - (\mnus + \mvis)^2)(\mtau^2 - (\mnus - \mvis)^2))^{1/2}}
{2 \mtau} 
\mnus
\mathrm{d}\mnus
\sin \theta d\theta 
\label{eq:pdfKineLeptonic}
\end{equation}
In the present implementation of the SVfit algorithm, the matrix element is
assumed to be constant, so that the likelihood depends on the phase--space
volume of the decay only~\footnote{The full matrix elements for tau decays may
be added in the future, including terms for the polarization of the tau lepton
pair, which is different in Higgs and $Z$ decays~\cite{tauDecayPolarization}.
}.


\subsection{Likelihood for reconstructed missing transverse momentum}

Momentum conservation in the plane perpendicular to the beam axis implies that
the vectorial sum of the momenta of all neutrinos produced in the decay of the
tau lepton pair matches the reconstructed missing transverse momentum.
Differences are possible due to the experimental resolution and finite $P_{T}$
of particles escaping detection in beam direction at high $\left| \eta \right|$.

The \MET resolution is measured in $Z \to \mu^{+} \mu^{-}$ events selected in
the $7$~TeV data collected by CMS in 2010.  Corrections are applied to Monte
Carlo simulated events to match the resolution measured in data.  The momentum
vectors of reconstructed \MET and neutrino momenta given by the fit parameters
are projected in direction parallel and perpendicular to the direction of the
$\tau^{+} \tau^{-}$ momentum vector.  For both components, a Gaussian
probability function is assumed.  The width and mean values of the Gaussian in
parallel (``$\parallel$'') and perpendicular (``$\perp$'') direction are:
\begin{align}
\sigma_\parallel &= \max \left( 7.54 \left( 1 - 0.00542 \cdot q_{T} \right), 5. \right) \nonumber\\
\mu_\parallel &= -0.96 \nonumber \\
\sigma_\perp &= \max \left( 6.85 \left( 1 - 0.00547 \cdot q_{T} \right), 5. \right) \nonumber \\
\mu_\perp &= 0.0, \nonumber
%\label{eq:METparams}
\end{align}
where $q_{T}$ denotes the transverse momentum of the tau lepton pair.


\subsection{Likelihood for tau lepton transverse momentum balance}

The tau lepton transverse momentum balance likelihood term represents the
probability $p \left( P^{\tau}_{T} | M_{\tau\tau} \right)$ for a tau to have a
certain $P_{T}$, given that the tau is produced in the decay of a resonance of
mass $M_{\tau\tau}$.  The likelihood is constructed by parametrizing the shape
of the tau lepton $P_{T}$ distribution in simulated Higgs $\to \tau^{+}
\tau^{-}$ events as a function of the Higgs mass.  The functional form of the
parametrization is taken to be the sum of two terms.  The first term, denoted by
$p^{*} \left( P_{T} | M \right)$, is derived by assuming an isotropic two--body
decay, that is
\begin{equation*}
\mathrm{d}p^{*} \propto \sin\theta \mathrm{d}\theta.
\end{equation*}
Performing a variable transformation from $\theta$ to $P_{T} \sim \frac{M}{2}
\sin\theta$, we obtain
\begin{align}
p^{*} \left( P_{T} | M \right) &= \frac{\mathrm{d}p}{\mathrm{d}P_T} = \frac{\mathrm{d}p}{\mathrm{d}\cos\theta}
\left| \frac{\mathrm{d}\cos\theta}{\mathrm{d}P_T} \right| \nonumber \\
&\propto \left| \frac{\mathrm{d}}{\mathrm{d}P_T} \sqrt{1 - \left(2 \frac{P_T}{M} \right)^2} \right| \nonumber \\
&= \frac{1}{\sqrt{\left( \frac{M}{2 P_{T}} \right)^2 - 1}}. 
\label{eq:ptBalanceTerm1}
\end{align}
The first term of the $P_{T}$--balance likelihood is taken as the convolution of
equation~\ref{eq:ptBalanceTerm1} with a Gaussian of width s.  The second term is
taken to be a Gamma distribution of scale parameter $\theta$ and shape
parameter~$k$, in order to account for tails in the $P_{T}$ distribution of the
tau lepton pair.  The complete functional form is thus given by
\begin{equation}
p \left( P_{T} | M \right) \propto \int_0^{\frac{M}{2}} \! p^* \left( P_{T}' | M \right) 
 e^{-\frac{\left( P_{T} - P_{T}' \right)^2}{2 s^2}} \; \mathrm{d}P_T' + a \Gamma \left( P_T, k, \theta \right).
\label{eq:ptBalanceLikelihood}
\end{equation}
Numerical values of the parameters s, $\theta$ and k are determined by fitting
function~\ref{eq:ptBalanceLikelihood} to the tau lepton $P_{T}$ distribution in
simulated Higgs $\to\TT$ events.  The relative weight a of the
two terms is also determined in the fit.  Replacing the integrand in
equation~\ref{eq:ptBalanceLikelihood} by its Taylor expansion, so that the
integration can be carried out analytically, keeping polynomial terms up to
fifth order, and assuming the fit parameters to depend at most linearly on the
Higgs mass, we obtain the following numerical values for the parameters:
\begin{align}
s &= 1.8 + 0.018 \cdot M_{\tau\tau} \nonumber \\
k &= 2.2 + 0.0364 \cdot M_{\tau\tau} \nonumber \\
\theta &= 6.74 + 0.02 \cdot M_{\tau\tau} \nonumber \\
a &= 0.48 - 0.0007 \cdot M_{\tau\tau}. \nonumber 
\end{align}

The motivation to add the $P_{T}$--balance likelihood to the SVfit is to add a
``regularization'' term which compensates for the effect of $P_{T}$ cuts applied
on the visible decay products of the two tau leptons.  In particular for tau
lepton pairs produced in decays of resonances of low mass, the visible $P_{T}$
cuts significantly affect the distribution of the visible momentum fraction $x =
\frac{E_{vis}}{E_{\tau}}$.  The effect is illustrated in
figures~\ref{fig:ptBalancePtVisCuts}
and~\ref{fig:ptBalancePtVisCutsCompareMasses}.  If no attempt would be made to
compensate for this effect,
equations~\ref{eq:pdfKineHadronic},~\ref{eq:pdfKineLeptonic} would yield
likelihood values that are too high at low $x$, resulting in the SVfit to
underestimate the energy of visible decay products (overestimate the energy of
neutrinos) produced in the tau decay, resulting in a significant tail of the
reconstructed mass distribution in the high mass region.  The $\tau^{+}
\tau^{-}$ invariant mass distribution reconstructed with and without the
$P_{T}$--balance likelihood term is shown in
figure~\ref{fig:ptBalanceImprovedMassResolution}.  A significant improvement in
resolution and in particular a significant reduction of the non--Gaussian tail
in the region of high masses is seen.

\begin{figure}[t]
\begin{center}
\includegraphics*[height=72mm, angle=90]{svfit_chapter/figures/pt_balance_effect.pdf}
\caption{\captiontext Distribution of di--tau invariant mass reconstructed by
the SVfit algorithm in simulated Higgs events with $M_{A} = 130$~\GeV$/c^2$.
The SVfit algorithm is run in two configurations, with~(blue) and without~(red)
the $P_{T}$--balance likelihood term included in the fit. 
  %The addition of the $P_{T}$ balance term improves the resolution and reduces
  %the non--Gaussian tail.
} \label{fig:ptBalanceImprovedMassResolution}
\end{center}
\end{figure} 

\begin{figure}[t]
\begin{center}
\includegraphics*[width=72mm]{svfit_chapter/figures/scuplting_A130_muon.pdf}
\includegraphics*[width=72mm]{svfit_chapter/figures/scuplting_A130_tau.pdf}
\caption{\captiontext Normalized distributions of the fraction of total tau
decay energy carried by the muon (left) and hadronic constituents (right) in
simulated Higgs events with \mbox{$M_{A} = 130~\GeVcc$}.  The distribution is shown
before (blue) and after (red) the requirement on the $P_T$ of the visible decay
products described in section~\ref{sec:EventSelectionMuTau}.  }
\label{fig:ptBalancePtVisCuts}
\end{center}
\end{figure} 

\begin{figure}[t]
\begin{center}
\includegraphics*[width=72mm]{svfit_chapter/figures/scuplting_Ztautau_powheg_muon.pdf}
\includegraphics*[width=72mm]{svfit_chapter/figures/scuplting_A200_muon.pdf}
\caption{\captiontext Normalized distributions of the fraction of total tau
decay energy carried by the muon in simulated $Z\to\TT$~(left) and Higgs events
with \mbox{$M_{A} = 200~\GeVcc$}~(right).  The distribution is shown before
(blue) and after (red) the requirement that the $P_T$ of the muon be greater
than 15~\GeVc.} \label{fig:ptBalancePtVisCutsCompareMasses}
\end{center}
\end{figure} 

\subsection{Secondary vertex information} 

The parametrization of the tau decay kinematics described in
section~\ref{sec:svParameterization} can be extended to describe the production
and decay of the tau.  As the flight direction of the tau is already fully
determined by the parameters $\theta$, $\overline{\phi}$ and $m_{\nu\nu}$, the
position of the secondary (decay) vertex is hence fully determined by addition
of a single parameter for the flight distance, $r$.  The tau lifetime $c\tau =
87$~$\mu$m is large enough to allow the displacement of the tau decay vertex
from the primary event vertex to be resolved by the CMS tracking detector.  The
resolution provided by the CMS tracking detector is utilized to improve the
resolution on the $\tau^{+} \tau^{-}$ invariant mass reconstructed by the SVfit
algorithm.  The likelihood term based on the secondary vertex information is
based on the compatibility of the decay vertex position with the reconstructed
tracks of charged tau decay products.  Perhaps surprisingly, it turns out that
the flight distance parameter $R$ is sufficiently constrained even for tau
decays into a single charged hadron, electron or muon.

The parameter $R$ can be constrained further by a term which represents the
probability for a tau lepton of momentum $P$ to travel a distance $d$ before
decaying:
\begin{equation*}
p \left( d | P \right) = \frac{m_\tau}{P c\tau}e^{-\frac{m_\tau d}{P c\tau}}
\end{equation*}

The likelihood terms for the secondary vertex fit have been implemented in the
SVfit algorithm.  In the analysis presented in this note, the decay vertex
information is not used, however, because of systematic effects arising from
tracker (mis--)alignment which are not yet fully understood.

\section{Performance}

Blah blah blah see figure~\ref{fig:SVversusCollinear}.
Blah blah blah see figure~\ref{fig:SVversusVis}.

\begin{figure}[t]
\begin{center}
\includegraphics*[width=72mm]{svfit_chapter/figures/sv_fit_approval_plots/sv_vs_coll_ZTT_normed.pdf}
\includegraphics*[width=72mm]{svfit_chapter/figures/sv_fit_approval_plots/sv_vs_coll_A120.pdf}
\caption[Comparison of SVfit with the Collinear Approximation
algorithm]{Comparison of the reconstructed tau pair mass spectrum in
\ZTT~(left) and MSSM \HTT{120}~(right) events after the selections described in
chapter~\ref{ch:selections}.  The mass spectrum reconstructed by the Secondary
Vertex fit is shown in blue, the result of the collinear approximation algorithm
is given in green.  In the left plot, both distributions are normalized to
unity, illustrating the improvement in resolution (shape) provided by the SVfit.
In the right plot, the distributions are normalized to an (arbitrary)
luminosity, illustrating the loss of events that occurs due to unphysical
solutions in the application of the collinear approximation.}
\label{fig:SVversusCollinear}
\end{center}
\end{figure} 

\begin{figure}[t]
\begin{center}
\includegraphics*[width=72mm]{svfit_chapter/figures/sv_fit_approval_plots/sv_vs_vis_A120.pdf}
\includegraphics*[width=72mm]{svfit_chapter/figures/sv_fit_approval_plots/sv_vs_vis_A300.pdf}
\caption[Comparison of SVfit with the visible mass observable]{Comparison of the
invariant mass of the muon and \tjet (the ``visible mass'') with the full \TT
mass reconstructed by the SVfit. The spectrum is shown for two simulated MSSM
Higgs samples, with $\ma = 120\GeVcc$~(left), and $\ma = 200\GeVcc$~(right).  To
illustrate that relative resolution of the SVfit is superior to that of the
visible mass, the visible mass is also shown scaled up such that the mean of the
two distributions are identical.} \label{fig:SVversusVis}
\end{center}
\end{figure} 

\ifx\master\undefined\input{settings/autocompile}\fi
