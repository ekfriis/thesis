\ifx\master\undefined% AUTOCOMPILE
% Allows for individual chapters to be compiled.

% Usage:
% \ifx\master\undefined% AUTOCOMPILE
% Allows for individual chapters to be compiled.

% Usage:
% \ifx\master\undefined% AUTOCOMPILE
% Allows for individual chapters to be compiled.

% Usage:
% \ifx\master\undefined\input{../settings/autocompile}\fi
% (place at start and end of chapter file)

\ifx\noprelim\undefined
    % first time included
    % input preamble files
    \input{../settings/phdsetup}

    \begin{document}
    \def\noprelim{}
\else
    % already included once
    % input post files

    \singlespacing
    \bibliographystyle{../bibliography/expanded}
    \bibliography{../bibliography/references}

    \end{document}
\fi\fi
% (place at start and end of chapter file)

\ifx\noprelim\undefined
    % first time included
    % input preamble files
    % [ USER VARIABLES ]

\def\PHDTITLE {Extensions of the Theory of Computational Mechanics}
\def\PHDAUTHOR{Evan Klose Friis}
\def\PHDSCHOOL{University of California, Davis}

\def\PHDMONTH {June}
\def\PHDYEAR  {2011}
\def\PHDDEPT {Physics}

\def\BSSCHOOL {University of California at San Diego}
\def\BSYEAR   {2005}

\def\PHDCOMMITTEEA{Professor John Conway}
\def\PHDCOMMITTEEB{Professor Robin Erbacher}
\def\PHDCOMMITTEEC{Professor Mani Tripathi}

% [ GLOBAL SETUP ]

\documentclass[letterpaper,oneside,11pt]{report}

\usepackage{calc}
\usepackage{breakcites}
\usepackage[newcommands]{ragged2e}

\usepackage[pdftex]{graphicx}
\usepackage{epstopdf}

%\usepackage{tikz}
%\usetikzlibrary{positioning} % [right=of ...]
%\usetikzlibrary{fit} % [fit= ...]

%\pgfdeclarelayer{background layer}
%\pgfdeclarelayer{foreground layer}
%\pgfsetlayers{background layer,main,foreground layer}

%\newenvironment{wrap}{\noindent\begin{minipage}[t]{\linewidth}\vspace{-0.5\normalbaselineskip}\centering}{\vspace{0.5\normalbaselineskip}\end{minipage}}

%% [Venn diagram environment]
%\newenvironment{venn2}
%{\begin{tikzpicture} [every pin/.style={text=black, text opacity=1.0, pin distance=0.5cm, pin edge={black!60, semithick}},
%% define a new style 'venn'
%venn/.style={circle, draw=black!60, semithick, minimum size = 4cm}]
%
%% create circle and give it external (pin) label
%\node[venn] (X) at (-1,0) [pin={150:$H[X]$}] {};
%\node[venn] (Y) at (1,0) [pin={30:$H[Y]$}] {};
%
%% place labels of the atoms by hand
%\node at (-1.9,0) {$H[X|Y]$};
%\node at (1.9,0) {$H[Y|X]$};
%\node at (0,0) {$I[X;Y]$};}
%{\end{tikzpicture}}

%\newcommand{\wrapmath}[1]{\begin{wrap}\begin{tikzpicture}[every node/.style={inner ysep=0ex, inner xsep=0em}]\node[] {$\displaystyle\begin{aligned} #1\end{aligned}$};\end{tikzpicture}\end{wrap}}

\renewenvironment{abstract}{\chapter*{Abstract}}{}
\renewcommand{\bibname}{Bibliography}
\renewcommand{\contentsname}{Table of Contents}

\makeatletter
\renewcommand{\@biblabel}[1]{\textsc{#1}}
\makeatother

% [ FONT SETTINGS ]

\usepackage[tbtags, intlimits, namelimits]{amsmath}
\usepackage[adobe-utopia]{mathdesign}

\DeclareSymbolFont{pazomath}{OMS}{zplm}{m}{n}
\DeclareSymbolFontAlphabet{\mathcal}{pazomath}
\SetMathAlphabet\mathcal{bold}{OMS}{zplm}{b}{n}

\SetSymbolFont{largesymbols}{normal}{OMX}{zplm}{m}{n}
\SetSymbolFont{largesymbols}{bold}{OMX}{zplm}{m}{n}
\SetSymbolFont{symbols}{normal}{OMS}{zplm}{m}{n}
\SetSymbolFont{symbols}{bold}{OMS}{zplm}{b}{n}

\renewcommand{\sfdefault}{phv}
\renewcommand{\ttdefault}{fvm}

\widowpenalty 8000
\clubpenalty  8000

% [ PAGE LAYOUT ]
\usepackage{geometry}
\geometry{lmargin = 1.5in}
\geometry{rmargin = 1.0in}
\geometry{tmargin = 1.0in}
\geometry{bmargin = 1.0in}

% [ PDF SETTINGS ]

\usepackage[final]{hyperref}
\hypersetup{breaklinks  = true}
\hypersetup{colorlinks  = true}
\hypersetup{linktocpage = false}
\hypersetup{linkcolor   = blue}
\hypersetup{citecolor   = green}
\hypersetup{urlcolor    = black}
\hypersetup{plainpages  = false}
\hypersetup{pageanchor  = true}
\hypersetup{pdfauthor   = {\PHDAUTHOR}}
\hypersetup{pdftitle    = {\PHDTITLE}}
\hypersetup{pdfsubject  = {Dissertation, \PHDSCHOOL}}
\urlstyle{same}

% [ LETTER SPACING ]

\usepackage[final]{microtype}
\microtypesetup{protrusion=compatibility}
\microtypesetup{expansion=false}

\newcommand{\upper}[1]{\MakeUppercase{#1}}
\let\lsscshape\scshape

\ifcase\pdfoutput\else\microtypesetup{letterspace=15}
\renewcommand{\scshape}{\lsscshape\lsstyle}
\renewcommand{\upper}[1]{\textls[50]{\MakeUppercase{#1}}}\fi

% [ LINE SPACING ]

\usepackage[doublespacing]{setspace}
\renewcommand{\displayskipstretch}{0.0}

\setlength{\parskip   }{0em}
\setlength{\parindent }{2em}

% [ TABLE FORMATTING ]

\usepackage{booktabs}
\setlength{\heavyrulewidth}{1.5\arrayrulewidth}
\setlength{\lightrulewidth}{1.0\arrayrulewidth}
\setlength{\doublerulesep }{2.0\arrayrulewidth}

% [ SECTION FORMATTING ]

\usepackage[largestsep,nobottomtitles*]{titlesec}
\renewcommand{\bottomtitlespace}{0.75in}

\titleformat{\chapter}[display]{\bfseries\huge\singlespacing}{\filleft\textsc{\LARGE \chaptertitlename\ \thechapter}}{-0.2ex}{\titlerule[3pt]\vspace{0.2ex}}[]

\titleformat{\section}{\LARGE}{\S\thesection\hspace{0.5em}}{0ex}{}
\titleformat{\subsection}{\Large}{\S\thesubsection\hspace{0.5em}}{0ex}{}
\titleformat{\subsubsection}{\large}{\thesubsubsection\hspace{0.5em}}{0ex}{}

\titlespacing*{\chapter}{0em}{6ex}{4ex plus 2ex minus 0ex}
\titlespacing*{\section}{0em}{2ex plus 3ex minus 1ex}{0.5ex plus 0.5ex minus 0.5ex}
\titlespacing*{\subsection}{0ex}{2ex plus 3ex minus 1ex}{0ex}
\titlespacing*{\subsubsection}{0ex}{2ex plus 0ex minus 1ex}{0ex}

% [ HEADER SETTINGS ]

\usepackage{fancyhdr}

\setlength{\headheight}{\normalbaselineskip}
\setlength{\footskip  }{0.5in}
\setlength{\headsep   }{0.5in-\headheight}

\fancyheadoffset[R]{0.5in}
\renewcommand{\headrulewidth}{0pt}
\renewcommand{\footrulewidth}{0pt}

\newcommand{\pagebox}{\parbox[r][\headheight][t]{0.5in}{\hspace\fill\thepage}}

\newcommand{\prelimheaders}{\ifx\prelim\undefined\renewcommand{\thepage}{\textit{\roman{page}}}\fancypagestyle{plain}{\fancyhf{}\fancyfoot[L]{\makebox[\textwidth-0.5in]{\thepage}}}\pagestyle{plain}\def\prelim{}\fi}

\newcommand{\normalheaders}{\renewcommand{\thepage}{\arabic{page}}\fancypagestyle{plain}{\fancyhf{}\fancyhead[R]{\pagebox}}\pagestyle{plain}}

\normalheaders{}

% [ CUSTOM COMMANDS ]

\newcommand{\signaturebox}[1]{\multicolumn{1}{p{4in}}{\vspace{3ex}}\\\midrule #1\\}

%\input{../includes/cmechabbrev}

% [some math stuff - maybe stick in sep file]
\usepackage{amsthm}
\usepackage{amscd}
\theoremstyle{plain}    \newtheorem{Lem}{Lemma}
\theoremstyle{plain}    \newtheorem*{ProLem}{Proof}
\theoremstyle{plain} 	\newtheorem{Cor}{Corollary}
\theoremstyle{plain} 	\newtheorem*{ProCor}{Proof}
\theoremstyle{plain} 	\newtheorem{The}{Theorem}
\theoremstyle{plain} 	\newtheorem*{ProThe}{Proof}
\theoremstyle{plain} 	\newtheorem{Prop}{Proposition}
\theoremstyle{plain} 	\newtheorem*{ProProp}{Proof}
\theoremstyle{plain} 	\newtheorem*{Conj}{Conjecture}
\theoremstyle{plain}	\newtheorem*{Rem}{Remark}
\theoremstyle{plain}	\newtheorem*{Def}{Definition} 
\theoremstyle{plain}	\newtheorem*{Not}{Notation}

% [uniform figure scaling - maybe this is not a good idea]
\def\figscale{.7}
\def\lscale{1.0}

% [FIX ME! - red makes it easier to spot]
\newcommand{\FIX}[1]{\textbf{\textcolor{red}{#1}}}


    \begin{document}
    \def\noprelim{}
\else
    % already included once
    % input post files

    \singlespacing
    \bibliographystyle{../bibliography/expanded}
    \bibliography{../bibliography/references}

    \end{document}
\fi\fi
% (place at start and end of chapter file)

\ifx\noprelim\undefined
    % first time included
    % input preamble files
    % [ USER VARIABLES ]

\def\PHDTITLE {Extensions of the Theory of Computational Mechanics}
\def\PHDAUTHOR{Evan Klose Friis}
\def\PHDSCHOOL{University of California, Davis}

\def\PHDMONTH {June}
\def\PHDYEAR  {2011}
\def\PHDDEPT {Physics}

\def\BSSCHOOL {University of California at San Diego}
\def\BSYEAR   {2005}

\def\PHDCOMMITTEEA{Professor John Conway}
\def\PHDCOMMITTEEB{Professor Robin Erbacher}
\def\PHDCOMMITTEEC{Professor Mani Tripathi}

% [ GLOBAL SETUP ]

\documentclass[letterpaper,oneside,11pt]{report}

\usepackage{calc}
\usepackage{breakcites}
\usepackage[newcommands]{ragged2e}

\usepackage[pdftex]{graphicx}
\usepackage{epstopdf}

%\usepackage{tikz}
%\usetikzlibrary{positioning} % [right=of ...]
%\usetikzlibrary{fit} % [fit= ...]

%\pgfdeclarelayer{background layer}
%\pgfdeclarelayer{foreground layer}
%\pgfsetlayers{background layer,main,foreground layer}

%\newenvironment{wrap}{\noindent\begin{minipage}[t]{\linewidth}\vspace{-0.5\normalbaselineskip}\centering}{\vspace{0.5\normalbaselineskip}\end{minipage}}

%% [Venn diagram environment]
%\newenvironment{venn2}
%{\begin{tikzpicture} [every pin/.style={text=black, text opacity=1.0, pin distance=0.5cm, pin edge={black!60, semithick}},
%% define a new style 'venn'
%venn/.style={circle, draw=black!60, semithick, minimum size = 4cm}]
%
%% create circle and give it external (pin) label
%\node[venn] (X) at (-1,0) [pin={150:$H[X]$}] {};
%\node[venn] (Y) at (1,0) [pin={30:$H[Y]$}] {};
%
%% place labels of the atoms by hand
%\node at (-1.9,0) {$H[X|Y]$};
%\node at (1.9,0) {$H[Y|X]$};
%\node at (0,0) {$I[X;Y]$};}
%{\end{tikzpicture}}

%\newcommand{\wrapmath}[1]{\begin{wrap}\begin{tikzpicture}[every node/.style={inner ysep=0ex, inner xsep=0em}]\node[] {$\displaystyle\begin{aligned} #1\end{aligned}$};\end{tikzpicture}\end{wrap}}

\renewenvironment{abstract}{\chapter*{Abstract}}{}
\renewcommand{\bibname}{Bibliography}
\renewcommand{\contentsname}{Table of Contents}

\makeatletter
\renewcommand{\@biblabel}[1]{\textsc{#1}}
\makeatother

% [ FONT SETTINGS ]

\usepackage[tbtags, intlimits, namelimits]{amsmath}
\usepackage[adobe-utopia]{mathdesign}

\DeclareSymbolFont{pazomath}{OMS}{zplm}{m}{n}
\DeclareSymbolFontAlphabet{\mathcal}{pazomath}
\SetMathAlphabet\mathcal{bold}{OMS}{zplm}{b}{n}

\SetSymbolFont{largesymbols}{normal}{OMX}{zplm}{m}{n}
\SetSymbolFont{largesymbols}{bold}{OMX}{zplm}{m}{n}
\SetSymbolFont{symbols}{normal}{OMS}{zplm}{m}{n}
\SetSymbolFont{symbols}{bold}{OMS}{zplm}{b}{n}

\renewcommand{\sfdefault}{phv}
\renewcommand{\ttdefault}{fvm}

\widowpenalty 8000
\clubpenalty  8000

% [ PAGE LAYOUT ]
\usepackage{geometry}
\geometry{lmargin = 1.5in}
\geometry{rmargin = 1.0in}
\geometry{tmargin = 1.0in}
\geometry{bmargin = 1.0in}

% [ PDF SETTINGS ]

\usepackage[final]{hyperref}
\hypersetup{breaklinks  = true}
\hypersetup{colorlinks  = true}
\hypersetup{linktocpage = false}
\hypersetup{linkcolor   = blue}
\hypersetup{citecolor   = green}
\hypersetup{urlcolor    = black}
\hypersetup{plainpages  = false}
\hypersetup{pageanchor  = true}
\hypersetup{pdfauthor   = {\PHDAUTHOR}}
\hypersetup{pdftitle    = {\PHDTITLE}}
\hypersetup{pdfsubject  = {Dissertation, \PHDSCHOOL}}
\urlstyle{same}

% [ LETTER SPACING ]

\usepackage[final]{microtype}
\microtypesetup{protrusion=compatibility}
\microtypesetup{expansion=false}

\newcommand{\upper}[1]{\MakeUppercase{#1}}
\let\lsscshape\scshape

\ifcase\pdfoutput\else\microtypesetup{letterspace=15}
\renewcommand{\scshape}{\lsscshape\lsstyle}
\renewcommand{\upper}[1]{\textls[50]{\MakeUppercase{#1}}}\fi

% [ LINE SPACING ]

\usepackage[doublespacing]{setspace}
\renewcommand{\displayskipstretch}{0.0}

\setlength{\parskip   }{0em}
\setlength{\parindent }{2em}

% [ TABLE FORMATTING ]

\usepackage{booktabs}
\setlength{\heavyrulewidth}{1.5\arrayrulewidth}
\setlength{\lightrulewidth}{1.0\arrayrulewidth}
\setlength{\doublerulesep }{2.0\arrayrulewidth}

% [ SECTION FORMATTING ]

\usepackage[largestsep,nobottomtitles*]{titlesec}
\renewcommand{\bottomtitlespace}{0.75in}

\titleformat{\chapter}[display]{\bfseries\huge\singlespacing}{\filleft\textsc{\LARGE \chaptertitlename\ \thechapter}}{-0.2ex}{\titlerule[3pt]\vspace{0.2ex}}[]

\titleformat{\section}{\LARGE}{\S\thesection\hspace{0.5em}}{0ex}{}
\titleformat{\subsection}{\Large}{\S\thesubsection\hspace{0.5em}}{0ex}{}
\titleformat{\subsubsection}{\large}{\thesubsubsection\hspace{0.5em}}{0ex}{}

\titlespacing*{\chapter}{0em}{6ex}{4ex plus 2ex minus 0ex}
\titlespacing*{\section}{0em}{2ex plus 3ex minus 1ex}{0.5ex plus 0.5ex minus 0.5ex}
\titlespacing*{\subsection}{0ex}{2ex plus 3ex minus 1ex}{0ex}
\titlespacing*{\subsubsection}{0ex}{2ex plus 0ex minus 1ex}{0ex}

% [ HEADER SETTINGS ]

\usepackage{fancyhdr}

\setlength{\headheight}{\normalbaselineskip}
\setlength{\footskip  }{0.5in}
\setlength{\headsep   }{0.5in-\headheight}

\fancyheadoffset[R]{0.5in}
\renewcommand{\headrulewidth}{0pt}
\renewcommand{\footrulewidth}{0pt}

\newcommand{\pagebox}{\parbox[r][\headheight][t]{0.5in}{\hspace\fill\thepage}}

\newcommand{\prelimheaders}{\ifx\prelim\undefined\renewcommand{\thepage}{\textit{\roman{page}}}\fancypagestyle{plain}{\fancyhf{}\fancyfoot[L]{\makebox[\textwidth-0.5in]{\thepage}}}\pagestyle{plain}\def\prelim{}\fi}

\newcommand{\normalheaders}{\renewcommand{\thepage}{\arabic{page}}\fancypagestyle{plain}{\fancyhf{}\fancyhead[R]{\pagebox}}\pagestyle{plain}}

\normalheaders{}

% [ CUSTOM COMMANDS ]

\newcommand{\signaturebox}[1]{\multicolumn{1}{p{4in}}{\vspace{3ex}}\\\midrule #1\\}

%%%% macros fro standard references
%\eqref provided by amsmath
\newcommand{\figref}[1]{Fig.~\ref{#1}}
\newcommand{\tableref}[1]{Table~\ref{#1}}
\newcommand{\refcite}[1]{Ref.~\cite{#1}}

% Abbreviations from CMPPSS:

\newcommand{\eM}     {\mbox{$\epsilon$-machine}}
\newcommand{\eMs}    {\mbox{$\epsilon$-machines}}
\newcommand{\EM}     {\mbox{$\epsilon$-Machine}}
\newcommand{\EMs}    {\mbox{$\epsilon$-Machines}}
\newcommand{\eT}     {\mbox{$\epsilon$-transducer}}
\newcommand{\eTs}    {\mbox{$\epsilon$-transducers}}
\newcommand{\ET}     {\mbox{$\epsilon$-Transducer}}
\newcommand{\ETs}    {\mbox{$\epsilon$-Transducers}}

% Processes and sequences

\newcommand{\Process}{\mathcal{P}}

\newcommand{\ProbMach}{\Prob_{\mathrm{M}}}
\newcommand{\Lmax}   { {L_{\mathrm{max}}}}
\newcommand{\MeasAlphabet}	{\mathcal{A}}
% Original
%\newcommand{\MeasSymbol}   { {S} }
%\newcommand{\meassymbol}   { {s} }
% New symbol
\newcommand{\MeasSymbol}   { {X} }
\newcommand{\meassymbol}   { {x} }
\newcommand{\BiInfinity}	{ \overleftrightarrow {\MeasSymbol} }
\newcommand{\biinfinity}	{ \overleftrightarrow {\meassymbol} }
\newcommand{\Past}	{ \overleftarrow {\MeasSymbol} }
\newcommand{\past}	{ {\overleftarrow {\meassymbol}} }
\newcommand{\pastprime}	{ {\past}^{\prime}}
\newcommand{\Future}	{ \overrightarrow{\MeasSymbol} }
\newcommand{\future}	{ \overrightarrow{\meassymbol} }
\newcommand{\futureprime}	{ {\future}^{\prime}}
\newcommand{\PastPrime}	{ {\Past}^{\prime}}
\newcommand{\FuturePrime}	{ {\overrightarrow{\meassymbol}}^\prime }
\newcommand{\PastDblPrime}	{ {\overleftarrow{\meassymbol}}^{\prime\prime} }
\newcommand{\FutureDblPrime}	{ {\overrightarrow{\meassymbol}}^{\prime\prime} }
\newcommand{\pastL}	{ {\overleftarrow {\meassymbol}}{}^L }
\newcommand{\PastL}	{ {\overleftarrow {\MeasSymbol}}{}^L }
\newcommand{\PastLt}	{ {\overleftarrow {\MeasSymbol}}_t^L }
\newcommand{\PastLLessOne}	{ {\overleftarrow {\MeasSymbol}}^{L-1} }
\newcommand{\futureL}	{ {\overrightarrow{\meassymbol}}{}^L }
\newcommand{\FutureL}	{ {\overrightarrow{\MeasSymbol}}{}^L }
\newcommand{\FutureLt}	{ {\overrightarrow{\MeasSymbol}}_t^L }
\newcommand{\FutureLLessOne}	{ {\overrightarrow{\MeasSymbol}}^{L-1} }
\newcommand{\pastLprime}	{ {\overleftarrow {\meassymbol}}^{L^\prime} }
\newcommand{\futureLprime}	{ {\overrightarrow{\meassymbol}}^{L^\prime} }
\newcommand{\AllPasts}	{ { \overleftarrow {\rm {\bf \MeasSymbol}} } }
\newcommand{\AllFutures}	{ \overrightarrow {\rm {\bf \MeasSymbol}} }
\newcommand{\FutureSet}	{ \overrightarrow{\bf \MeasSymbol}}

% Causal states and epsilon-machines
\newcommand{\CausalState}	{ \mathcal{S} }
\newcommand{\CausalStatePrime}	{ {\CausalState}^{\prime}}
\newcommand{\causalstate}	{ \sigma }
\newcommand{\CausalStateSet}	{ \boldsymbol{\CausalState} }
\newcommand{\AlternateState}	{ \mathcal{R} }
\newcommand{\AlternateStatePrime}	{ {\cal R}^{\prime} }
\newcommand{\alternatestate}	{ \rho }
\newcommand{\alternatestateprime}	{ {\rho^{\prime}} }
\newcommand{\AlternateStateSet}	{ \boldsymbol{\AlternateState} }
\newcommand{\PrescientState}	{ \widehat{\AlternateState} }
\newcommand{\prescientstate}	{ \widehat{\alternatestate} }
\newcommand{\PrescientStateSet}	{ \boldsymbol{\PrescientState}}
\newcommand{\CausalEquivalence}	{ {\sim}_{\epsilon} }
\newcommand{\CausalEquivalenceNot}	{ {\not \sim}_{\epsilon}}

\newcommand{\NonCausalEquivalence}	{ {\sim}_{\eta} }
\newcommand{\NextObservable}	{ {\overrightarrow {\MeasSymbol}}^1 }
\newcommand{\LastObservable}	{ {\overleftarrow {\MeasSymbol}}^1 }
%\newcommand{\Prob}		{ {\rm P}}
\newcommand{\Prob}      {\Pr} % use standard command
\newcommand{\ProbAnd}	{ {,\;} }
\newcommand{\LLimit}	{ {L \rightarrow \infty}}
\newcommand{\Cmu}		{C_\mu}
\newcommand{\hmu}		{h_\mu}
\newcommand{\EE}		{{\bf E}}
\newcommand{\Measurable}{{\bf \mu}}

% Process Crypticity
\newcommand{\PC}		{\chi}
\newcommand{\FuturePC}		{\PC^+}
\newcommand{\PastPC}		{\PC^-}
% Causal Irreversibility
\newcommand{\CI}		{\Xi}
\newcommand{\ReverseMap}	{r}
\newcommand{\ForwardMap}	{f}

% Abbreviations from IB:
% None that aren't already in CMPPSS

% Abbreviations from Extensive Estimation:
\newcommand{\EstCausalState}	{\widehat{\CausalState}}
\newcommand{\estcausalstate}	{\widehat{\causalstate}}
\newcommand{\EstCausalStateSet}	{\boldsymbol{\EstCausalState}}
\newcommand{\EstCausalFunc}	{\widehat{\epsilon}}
\newcommand{\EstCmu}		{\widehat{\Cmu}}
\newcommand{\PastLOne}	{{\Past}^{L+1}}
\newcommand{\pastLOne}	{{\past}^{L+1}}

% Abbreviations from $\epsilon$-Transducers:
\newcommand{\InAlphabet}	{ \mathcal{A}}
\newcommand{\insymbol}		{ a}
\newcommand{\OutAlphabet}	{ \mathcal{B}}
\newcommand{\outsymbol}		{ b}
\newcommand{\InputSimple}	{ X}
\newcommand{\inputsimple}	{ x}
\newcommand{\BottleneckVar}	{\tilde{\InputSimple}}
\newcommand{\bottleneckvar}	{\tilde{\inputsimple}}
\newcommand{\InputSpace}	{ \mathbf{\InputSimple}}
\newcommand{\InputBi}	{ \overleftrightarrow {\InputSimple} }
\newcommand{\inputbi}	{ \overleftrightarrow {\inputsimple} }
\newcommand{\InputPast}	{ \overleftarrow {\InputSimple} }
\newcommand{\inputpast}	{ \overleftarrow {\inputsimple} }
\newcommand{\InputFuture}	{ \overrightarrow {\InputSimple} }
\newcommand{\inputfuture}	{ \overrightarrow {\inputsimple} }
\newcommand{\NextInput}	{ {{\InputFuture}^{1}}}
\newcommand{\NextOutput}	{ {\OutputFuture}^{1}}
\newcommand{\OutputSimple}	{ Y}
\newcommand{\outputsimple}	{ y}
\newcommand{\OutputSpace}	{ \mathbf{\OutputSimple}}
\newcommand{\OutputBi}	{ \overleftrightarrow{\OutputSimple} }
\newcommand{\outputbi}	{ \overleftrightarrow{\outputsimple} }
\newcommand{\OutputPast}	{ \overleftarrow{\OutputSimple} }
\newcommand{\outputpast}	{ \overleftarrow{\outputsimple} }
\newcommand{\OutputFuture}	{ \overrightarrow{\OutputSimple} }
\newcommand{\outputfuture}	{ \overrightarrow{\outputsimple} }
\newcommand{\OutputL}	{ {\OutputFuture}^L}
\newcommand{\outputL}	{ {\outputfuture}^L}
\newcommand{\InputLLessOne}	{ {\InputFuture}^{L-1}}
\newcommand{\inputLlessone}	{ {\inputufutre}^{L-1}}
\newcommand{\OutputPastLLessOne}	{{\OutputPast}^{L-1}_{-1}}
\newcommand{\outputpastLlessone}	{{\outputpast}^{L-1}}
\newcommand{\OutputPastLessOne}	{{\OutputPast}_{-1}}
\newcommand{\outputpastlessone}	{{\outputpast}_{-1}}
\newcommand{\OutputPastL}	{{\OutputPast}^{L}}
\newcommand{\OutputLPlusOne}	{ {\OutputFuture}^{L+1}}
\newcommand{\outputLplusone}	{ {\outputfutre}^{L+1}}
\newcommand{\InputPastL}	{{\InputPast}^{L}}
\newcommand{\inputpastL}	{{\inputpast}^{L}}
\newcommand{\JointPast}	{{(\InputPast,\OutputPast)}}
\newcommand{\jointpast}	{{(\inputpast,\outputpast)}}
\newcommand{\jointpastone}	{{(\inputpast_1,\outputpast_1)}}
\newcommand{\jointpasttwo}	{{(\inputpast_2,\outputpast_2)}}
\newcommand{\jointpastprime} {{({\inputpast}^{\prime},{\outputpast}^{\prime})}}
\newcommand{\NextJoint}	{{(\NextInput,\NextOutput)}}
\newcommand{\nextjoint}	{{(\insymbol,\outsymbol)}}
\newcommand{\AllInputPasts}	{ { \overleftarrow {\rm \InputSpace}}}
\newcommand{\AllOutputPasts}	{ {\overleftarrow {\rm \OutputSpace}}}
\newcommand{\DetCausalState}	{ {{\cal S}_D }}
\newcommand{\detcausalstate}	{ {{\sigma}_D} }
\newcommand{\DetCausalStateSet}	{ \boldsymbol{{\CausalState}_D}}
\newcommand{\DetCausalEquivalence}	{ {\sim}_{{\epsilon}_{D}}}
\newcommand{\PrescientEquivalence}	{ {\sim}_{\widehat{\eta}}}
\newcommand{\FeedbackCausalState}	{ \mathcal{F}}
\newcommand{\feedbackcausalstate}	{ \phi}
\newcommand{\FeedbackCausalStateSet}	{ \mathbf{\FeedbackCausalState}}
\newcommand{\JointCausalState}		{ \mathcal{J}}
\newcommand{\JointCausalStateSet}	{ \mathbf{\JointCausalState}}
\newcommand{\UtilityFunctional}	{ {\mathcal{L}}}
\newcommand{\NatureState}	{ {\Omega}}
\newcommand{\naturestate}	{ {\omega}}
\newcommand{\NatureStateSpace}	{ {\mathbf{\NatureState}}}
\newcommand{\AnAction}	{ {A}}
\newcommand{\anaction}	{ {a}}
\newcommand{\ActionSpace}	{ {\mathbf{\AnAction}}}

% Abbreviations from RURO:
\newcommand{\InfoGain}[2] { \mathcal{D} \left( {#1} || {#2} \right) }

% Abbreviations from Upper Bound:
\newcommand{\lcm}	{{\rm lcm}}
% Double-check that this isn't in the math set already!

% Abbreviations from Emergence in Space
\newcommand{\ProcessAlphabet}	{\MeasAlphabet}
\newcommand{\ProbEst}			{ {\widehat{\Prob}_N}}
\newcommand{\STRegion}			{ {\mathrm K}}
\newcommand{\STRegionVariable}		{ K}
\newcommand{\stregionvariable}		{ k}
\newcommand{\GlobalPast}		{ \overleftarrow{G}} 
\newcommand{\globalpast}		{ \overleftarrow{g}} 
\newcommand{\GlobalFuture}		{ \overrightarrow{G}}
\newcommand{\globalfuture}		{ \overrightarrow{g}}
\newcommand{\GlobalState}		{ \mathcal{G}}
\newcommand{\globalstate}		{ \gamma}
\newcommand{\GlobalStateSet}		{ {\mathbf \GlobalState}}
\newcommand{\LocalPast}			{ \overleftarrow{L}} 
\newcommand{\localpast}			{ \overleftarrow{l}}
\newcommand{\AllLocalPasts}		{ \mathbf{\LocalPast}}
\newcommand{\LocalPastRegion}		{ \overleftarrow{\mathrm L}}
\newcommand{\LocalFuture}		{ \overrightarrow{L}}
\newcommand{\localfuture}		{ \overrightarrow{l}}
\newcommand{\LocalFutureRegion}		{ \overrightarrow{\mathrm L}}
\newcommand{\LocalState}		{ \mathcal{L}}
\newcommand{\localstate}		{ \lambda}
\newcommand{\LocalStateSet}		{ {\mathbf \LocalState}}
\newcommand{\PatchPast}			{ \overleftarrow{P}}
\newcommand{\patchpast}			{ \overleftarrow{p}}
\newcommand{\PatchPastRegion}		{ \overleftarrow{\mathrm P}}
\newcommand{\PatchFuture}		{ \overrightarrow{P}}
\newcommand{\patchfuture}		{ \overrightarrow{p}}
\newcommand{\PatchFutureRegion}		{ \overrightarrow{\mathrm P}}
\newcommand{\PatchState}		{ \mathcal{P}}
\newcommand{\patchstate}		{ \pi}
\newcommand{\PatchStateSet}		{ {\mathbf \PatchState}}
\newcommand{\LocalStatesInPatch}	{\vec{\LocalState}}
\newcommand{\localstatesinpatch}	{\vec{\localstate}}
\newcommand{\PointInstantX}		{ {\mathbf x}}
% Galles's original LaTeX for the cond. indep. symbol follows:
\newcommand{\compos}{\mbox{$~\underline{~\parallel~}~$}}
\newcommand{\ncompos}{\not\hspace{-.15in}\compos}
\newcommand{\indep}			{ \rotatebox{90}{$\models$}}
\newcommand{\nindep}	{\not\hspace{-.05in}\indep}
\newcommand{\LocalEE}	{{\EE}^{loc}}
\newcommand{\EEDensity}	{\overline{\LocalEE}}
\newcommand{\LocalCmu}	{{\Cmu}^{loc}}
\newcommand{\CmuDensity}	{\overline{\LocalCmu}}

%%%%%%%%%%% added by sasa
\newcommand{\FinPast}[1]	{ \overleftarrow {\MeasSymbol} \stackrel{{#1}}{}}
\newcommand{\finpast}[1]  	{ \overleftarrow {\meassymbol}  \stackrel{{#1}}{}}
\newcommand{\FinFuture}[1]		{ \overrightarrow{\MeasSymbol} \stackrel{{#1}}{}}
\newcommand{\finfuture}[1]		{ \overrightarrow{\meassymbol} \stackrel{{#1}}{}}

\newcommand{\Partition}	{ \AlternateState }
\newcommand{\partitionstate}	{ \alternatestate }
\newcommand{\PartitionSet}	{ \AlternateStateSet }
\newcommand{\Fdet}   { F_{\rm det} }

\newcommand{\Dkl}[2] { D_{\rm KL} \left( {#1} || {#2} \right) }

\newcommand{\Period}	{p}

% To take into account time direction
\newcommand{\forward}{+}
\newcommand{\reverse}{-}
%\newcommand{\forwardreverse}{\:\!\diamond} % \pm
\newcommand{\forwardreverse}{\pm} % \pm
\newcommand{\FutureProcess}	{ {\Process}^{\forward} }
\newcommand{\PastProcess}	{ {\Process}^{\reverse} }
\newcommand{\FutureCausalState}	{ {\CausalState}^{\forward} }
\newcommand{\futurecausalstate}	{ \sigma^{\forward} }
\newcommand{\altfuturecausalstate}	{ \sigma^{\forward\prime} }
\newcommand{\PastCausalState}	{ {\CausalState}^{\reverse} }
\newcommand{\pastcausalstate}	{ \sigma^{\reverse} }
\newcommand{\BiCausalState}		{ {\CausalState}^{\forwardreverse} }
\newcommand{\bicausalstate}		{ {\sigma}^{\forwardreverse} }
\newcommand{\FutureCausalStateSet}	{ {\CausalStateSet}^{\forward} }
\newcommand{\PastCausalStateSet}	{ {\CausalStateSet}^{\reverse} }
\newcommand{\BiCausalStateSet}	{ {\CausalStateSet}^{\forwardreverse} }
\newcommand{\eMachine}	{ M }
\newcommand{\FutureEM}	{ {\eMachine}^{\forward} }
\newcommand{\PastEM}	{ {\eMachine}^{\reverse} }
\newcommand{\BiEM}		{ {\eMachine}^{\forwardreverse} }
\newcommand{\BiEquiv}	{ {\sim}^{\forwardreverse} }
\newcommand{\Futurehmu}	{ h_\mu^{\forward} }
\newcommand{\Pasthmu}	{ h_\mu^{\reverse} }
\newcommand{\FutureCmu}	{ C_\mu^{\forward} }
\newcommand{\PastCmu}	{ C_\mu^{\reverse} }
\newcommand{\BiCmu}		{ C_\mu^{\forwardreverse} }
\newcommand{\FutureEps}	{ \epsilon^{\forward} }
\newcommand{\PastEps}	{ \epsilon^{\reverse} }
\newcommand{\BiEps}	{ \epsilon^{\forwardreverse} }
\newcommand{\FutureSim}	{ \sim^{\forward} }
\newcommand{\PastSim}	{ \sim^{\reverse} }
% Used arrows for awhile, more or less confusing?
%\newcommand{\FutureCausalState}	{ \overrightarrow{\CausalState} }
%\newcommand{\PastCausalState}	{ \overleftarrow{\CausalState} }
%\newcommand{\eMachine}	{ M }
%\newcommand{\FutureEM}	{ \overrightarrow{\eMachine} }
%\newcommand{\PastEM}	{ \overleftarrow{\eMachine} }
%\newcommand{\FutureCmu}	{ \overrightarrow{\Cmu} }
%\newcommand{\PastCmu}	{ \overleftarrow{\Cmu} }

%% time-reversing and mixed state presentation operators
\newcommand{\TR}{\mathcal{T}}
\newcommand{\MSP}{\mathcal{U}}
\newcommand{\one}{\mathbf{1}}

%% (cje)
%% Provide a command \ifpm which is true when \pm 
%% is meant to be understood as "+ or -". This is
%% different from the usage in TBA.
\newif\ifpm 
\edef\tempa{\forwardreverse}
\edef\tempb{\pm}
\ifx\tempa\tempb
   \pmfalse
\else
   \pmtrue  
\fi





% [some math stuff - maybe stick in sep file]
\usepackage{amsthm}
\usepackage{amscd}
\theoremstyle{plain}    \newtheorem{Lem}{Lemma}
\theoremstyle{plain}    \newtheorem*{ProLem}{Proof}
\theoremstyle{plain} 	\newtheorem{Cor}{Corollary}
\theoremstyle{plain} 	\newtheorem*{ProCor}{Proof}
\theoremstyle{plain} 	\newtheorem{The}{Theorem}
\theoremstyle{plain} 	\newtheorem*{ProThe}{Proof}
\theoremstyle{plain} 	\newtheorem{Prop}{Proposition}
\theoremstyle{plain} 	\newtheorem*{ProProp}{Proof}
\theoremstyle{plain} 	\newtheorem*{Conj}{Conjecture}
\theoremstyle{plain}	\newtheorem*{Rem}{Remark}
\theoremstyle{plain}	\newtheorem*{Def}{Definition} 
\theoremstyle{plain}	\newtheorem*{Not}{Notation}

% [uniform figure scaling - maybe this is not a good idea]
\def\figscale{.7}
\def\lscale{1.0}

% [FIX ME! - red makes it easier to spot]
\newcommand{\FIX}[1]{\textbf{\textcolor{red}{#1}}}


    \begin{document}
    \def\noprelim{}
\else
    % already included once
    % input post files

    \singlespacing
    \bibliographystyle{../bibliography/expanded}
    \bibliography{../bibliography/references}

    \end{document}
\fi\fi
%
\newcommand{\mtau}{m_{\tau}}
\newcommand{\mnus}{m_{\nu\nu}}
\newcommand{\mvis}{m_{vis}}
%
\chapter{Mass Reconstruction: The Secondary Vertex Fit}
\label{ch:svfit}
%
The dominant background in the search for a Higgs decaying to a $\TT$ pair are 
standard model $\ZTT$ events.  The most ``natural'' observable to discriminate
between a Higgs signal and the $Z$ background is the invariant mass of the
di--tau system, utilizing the fact that the $Z$ resonance is well known ($m_{Z}
= 91.1876 \pm 0.0021$~\GeVcc) and has a narrow width ($\Gamma_{Z} = 2.4952 \pm
0.0023$~\GeV)~\cite{PDG}.  The experimental complication in this approach is due
to the neutrinos produced in the tau lepton decays, which escape detection and
carry away an unmeasured amount of energy, making it difficult to
reconstruct the tau lepton four--vectors.  In this chapter we give an overview
of techniques used in previous literature~\cite{Rainwater:1998kj,CDFMSSMHiggs, 
CMS-PTDRII} to construct an observable related to the tau pair mass.  We then
introduce a new algorithm, called the Secondary Vertex (SV) fit.  The SVfit
reconstructs the ``full'' tau pair mass, and provides increased performance with
respect to techniques previously used in the literature.
%
\section{Existing Mass Reconstruction Algorithms}
%
The simplest observable elated to the \TT mass is one can construct that is
sensitive to new particle content is the invariant mass of the visible
(reconstructible) decay products associated with each tau decays.  This
quantity, referred in this document as the ``Visible Mass,'' has the advantages
of simplicity and lack of exposure to systematic errors associated with the
reconstruction of the $\MET$.  However, no attempt is made to reconstruct the
neutrinos in the event.  The reconstructed mass is thus systematically smaller
than mass of the resonance which produced the tau leptons.  The visible mass is
typically on the order of $1/2$ of the resonance mass, depending on the
kinematic requirements applied to the visible products of the tau decays. 

The Collinear Approximation is a technique previously
used~\cite{Rainwater:1998kj} to reconstruct the full \TT mass.  In an
event with two tau decays, there are a total of six\footnote{Technically, there
is an extra unknown for each leptonic tau decay, which has two associated
neutrinos.  This is a small effect compared to the overall resolution of the
collinear approximation.} unknowns associated with the
missing energy: the three components of the momentum of each neutrino.  The
Collinear Approximation makes the assumption that the neutrinos have the same
direction as their associated visible decay products.  This assumption reduces
the number of unknown quantities to two, corresponding to the total energy of
each neutrino.  These two unknowns can be solved for by using the two components
of the reconstructed missing transverse energy, which in the ideal case
corresponds to the transverse component of the vector sum of the two neutrino's
four momentum.  The characteristic equation of the Collinear Approximation is 
\begin{equation}
  \left( \begin{array}{c} \MEx \\ \MEy \end{array} \right) = 
    \left( \begin{array}{cc} 
      \cos\phi_1 & \cos\phi_2 \\
      \sin\phi_1 & \sin\phi_2 
    \end{array} \right) 
    \left(\begin{array}{c} E_1 \\ E_2 \end{array} \right)
      \label{eq:CollinearApproxCharacteristicEquation}
\end{equation}
where $(\MEx, \MEy)$ are the two components of the reconstructed missing
transverse energy, $\phi_{1(2)}$ is the azimuthal angle of the visible component
of the first (second) tau decay, and $E_{1(2)}$ is the reconstructed energy of
neutrino of the first (second) tau decay.  $E_1$ and $E_2$ can be extracted by
inverting the matrix on the right hand side of
Equation~\ref{eq:CollinearApproxCharacteristicEquation}.
\begin{equation}
  \left(\begin{array}{c} E_1 \\ E_2 \end{array} \right) = 
    \frac{1}{\sin(\phi_2 - \phi_1)} 
    \left( \begin{array}{cc} 
      \sin\phi_2 & -\cos\phi_2 \\
      -\sin\phi_1 & \cos\phi_1 
    \end{array} \right) 
  \left( \begin{array}{c} \MEx \\ \MEy \end{array} \right) 
    \label{eq:CollinearApproxInvertedEquation} 
\end{equation}

The Collinear Approximation suffers from two problems.  The approximation can
fail (yielding unphysical negative energies for the reconstructed neutrinos)
when the missing transverse energy is mis--measured.  The events with unphysical
solutions must be removed from the analysis, leading to a dramatic reduction in
acceptance (on the order of 50\% in this analysis).  Improvements to the
collinear approximation algorithm have recently been made which aim to recover
part of the events with unphysical solutions~\cite{improvedCollinearApprox}.
But even with these improvements, no physical solution is still found for a
large fraction of signal events.  Additionally, the method is numerically
sensitive when the two \taul are nearly back-to-back in azimuth.  In these cases
the $\sin(\phi_2 - \phi_1)^{-1}$ term in
Equation~\ref{eq:CollinearApproxInvertedEquation} is very large and small
mis--measurements of the missing transverse energy can produce a large tail on
the reconstructed mass.  This tail is particularly large for low--mass
resonances.  The large tail for high mass is predominantly due to the fact
(discussed in Section~\ref{sec:ptBalance}) that the kinematic
requirements\footnote{The kinematic requirements on the visible decay products
are necessary to reduce backgrounds and maintain compatibility with
un--prescaled event triggers.  This topic is discussed in detail in
Chapter~\ref{ch:selections}.} applied on the visible decay products
preferentially selects events where the visible decay products carry the
majority of the energy of the original \taul, reducing the amount of true
missing energy in the event.  

\section{The Secondary Vertex Fit}
A novel algorithm is presented in the following, which succeeds in finding a
physical solution for every event.  As an additional benefit, the new algorithm
is found to improve the di--tau invariant mass resolution, making it easier to
separate the Higgs signal from the $\ZTT$ background.

The novel Secondary Vertex fit (SVfit) algorithm for di--tau invariant mass
reconstruction that we present in the following utilizes a likelihood
maximization to fit a $\TT$ invariant mass hypothesis for each
event.  The likelihood is composed of separate terms which represent probability
densities of:
\begin{itemize}
\item tau decay kinematics
\item matching between the momenta of neutrinos produced in the tau decays and
  the reconstructed missing transverse momentum
\item a regularization ``\pt--balance'' term which accounts for the effects on the di--tau invariant mass
 of acceptance cuts on the visible tau decay products
\item the compatibility of tau decay parameters with the position of reconstructed tracks
 and the known tau lifetime of $c\tau = 87~\micron$~\cite{PDG}.
\end{itemize}
The likelihood is maximized as function of a set of parameters which fully describe the tau decay.

\section{Parametrization of Tau Decays}
\label{sec:svParameterization}

The decay of a tau with visible four--momentum $p_{vis}$ measured in the CMS
detector (``laboratory'') frame can be parametrized by three variables.  The
invisible (neutrino) momentum is fully determined by these parameters.

The ``opening--angle'' $\theta$ is defined as the angle between the boost
direction of the tau lepton and the momentum vector of the visible decay
products in the rest frame of the tau.  The azimuthal angle of the tau in the
lab frame is denoted as $\overline{\phi}$ (we denote quantities defined in the
laboratory frame by a overline).  A local coordinate system is defined such that the
$\bar{z}$--direction lies along the visible momentum and $\overline{\phi} = 0$ lies
in the plane spanned by the momentum vector of the visible decay products and
the proton beam direction. The third parameter, $m_{\nu\nu}$, denotes the
invariant mass of the invisible momentum system.

Given $\theta$, $\overline{\phi}$ and $m_{\nu\nu}$, the energy and direction of the
tau lepton can be computed by means of the following equations: The energy of
the visible decay products in the rest frame of the tau lepton is related to the
invariant mass of the neutrino system by:
\begin{equation}
E^{vis} = \frac{m_\tau^2 + \mvis^2 - \mnus^2}{2m_\tau} 
\label{eq:restFrameMomentumRelation}
\end{equation}
Note that for hadronic decays, $\mnus$ is a constant of value zero, as only a
single neutrino is produced.  Consequently, the magnitude of $P^{vis}$ depends
on the reconstructed mass of the visible decay products only and is a constant
during the SVfit.

The opening angle $\overline{\theta}$ between the tau lepton direction and the
visible momentum vector in the laboratory frame is determined by the rest frame
quantities via the (Lorentz invariant) component of the visible momentum
perpendicular to the tau lepton direction:
\begin{align}
p^{vis}_{\perp} &= \overline p^{vis}_{\perp} \nonumber \\
\Rightarrow \sin \overline \theta &= \frac{p^{vis} \sin \theta}{\overline p^{vis}} 
\label{eq:labFrameOpeningAngle} 
\end{align}

Substituting the parameters $\mnus$ and $\theta$ into
Equations~\ref{eq:restFrameMomentumRelation} and~\ref{eq:labFrameOpeningAngle},
the energy of the tau is obtained by solving for the boost factor $\gamma$ in
the Lorentz transformation between the tau rest frame and laboratory frame of the
visible momentum component parallel to the tau direction:
\begin{align}
\overline{p}^{vis} \cos \overline{\theta} &= \gamma \beta E^{vis} + \gamma p^{vis} \cos\theta \nonumber \\ 
\Rightarrow \gamma &= \frac{E^{vis}[{\left( E^{vis} \right)^2 + \left( \overline{p}^{vis} \cos \overline{\theta} \right)^2 
 - \left( p^{vis} \cos \theta \right)^2}]^{1/2} - p^{vis} \cos \theta \overline{p}^{vis} \cos \overline{\theta}}{\left( E^{vis} \right)^2 
 - \left( p^{vis} \cos \theta \right)^2}, \nonumber \\
E^{\tau} &= \gamma m_\tau \nonumber
%\label{eq:LorentzRelationship}
\end{align}

The energy of the tau lepton in the laboratory frame as function of the measured
visible momentum depends on two of the three parameters only - the rest frame
opening angle $\theta$ and the invariant mass $\mnus$ of the neutrino system.
The direction of the tau lepton momentum vector is not fully determined by
$\theta$ and $\mnus$, but is constrained to lie on the surface of a cone of
opening angle $\overline{\theta}$ (given by Equation~\ref{eq:labFrameOpeningAngle}),
the axis of which is given by the visible momentum vector.  The tau lepton
four--vector is fully determined by the addition of the third parameter
$\overline{\phi}$, which describes the azimuthal angle of the tau lepton with respect
to the visible momentum vector.  The spatial coordinate system used is
illustrated in Figure~\ref{fig:svFitDecayParDiagram}.

\begin{figure}[t]
\begin{center}
\includegraphics*[width=72mm]{svfit_chapter/figures/decay_parameters_graphic.pdf}
\caption[Coordinate system of the SVfit]{Illustration of the coordinate system
used by the SVfit to describe the decays of tau leptons.}
\label{fig:svFitDecayParDiagram}
\end{center}
\end{figure} 

\section{Likelihood for Tau Decays}

The probability density functions for the tau decay kinematics are taken from
the kinematics review of the PDG~\cite{PDG}.  The likelihood is proportional to
the phase--space volume for two--body ($\tau \to \tau_{had} \nu$) and
three--body ($\tau \to e \nu\nu$ and $\tau \to \mu \nu\nu$) decays.  For
two--body decays the likelihood depends on the decay angle $\theta$ only:
\begin{equation*}
\mathrm{d}\Gamma \propto |\mathcal{M}|^2 \sin \theta d \theta
\label{eq:pdfKineHadronic}
\end{equation*}
For three--body decays, the likelihood depends on the invariant mass of the
neutrino system also:
\begin{equation}
\mathrm{d}\Gamma \propto |\mathcal{M}|^2 
\frac{((\mtau^2 - (\mnus + \mvis)^2)(\mtau^2 - (\mnus - \mvis)^2))^{1/2}}
{2 \mtau} 
\mnus
\mathrm{d}\mnus
\sin \theta d\theta 
\label{eq:pdfKineLeptonic}
\end{equation}
In the present implementation of the SVfit algorithm, the matrix element is
assumed to be constant, so that the likelihood depends on the phase--space
volume of the decay only~\footnote{The full matrix elements for tau decays may
be added in the future, including terms for the polarization of the tau lepton
pair, which is different in Higgs and $Z$ decays~\cite{tauDecayPolarization}.
}.


\subsection{Likelihood for Reconstructed \MET}

Momentum conservation in the plane perpendicular to the beam axis implies that
the vectorial sum of the momenta of all neutrinos produced in the decay of the
tau lepton pair matches the reconstructed missing transverse momentum.
Differences are possible due to the experimental resolution and finite $\pt$
of particles escaping detection in beam direction at high $\left| \eta \right|$.

The \MET resolution is measured in $Z \to \mu^{+} \mu^{-}$ events selected in
the $7$~\TeV data collected by CMS in 2010.  Corrections are applied to the
distribution of \MET in the Monte Carlo simulated events to match the resolution
measured in data.  The uncertainty on this correction factor is taken as a
``shape systematic.''  The treatment of this correction and its corresponding
uncertainty are described in Chapters~\ref{ch:corrections}
and~\ref{ch:systematics}.  The momentum vectors of reconstructed \MET and
neutrino momenta given by the fit parameters are projected in direction parallel
and perpendicular to the direction of the $\tau^{+} \tau^{-}$ momentum vector.
For both components, a Gaussian probability function is assumed.  The width and
mean values of the Gaussian in parallel (``$\parallel$'') and perpendicular
(``$\perp$'') direction are:
\begin{align}
\sigma_\parallel &= \max \left( 7.54 \left( 1 - 0.00542 \cdot q_{T} \right), 5. \right) \nonumber\\
\mu_\parallel &= -0.96 \nonumber \\
\sigma_\perp &= \max \left( 6.85 \left( 1 - 0.00547 \cdot q_{T} \right), 5. \right) \nonumber \\
\mu_\perp &= 0.0, \nonumber
%\label{eq:METparams}
\end{align}
where $q_{T}$ denotes the transverse momentum of the tau lepton pair.


\subsection{Likelihood for Tau Transverse Momentum Balance}
\label{sec:ptBalance}

The tau lepton transverse momentum balance likelihood term represents the
probability density $p \left( \pt^{\tau} | M_{\tau\tau} \right)$ of the tau transverse
momentum, given the condition that the tau is produced in the decay of a resonance of
mass $M_{\tau\tau}$.  The likelihood is constructed by parametrizing the shape
of the tau lepton $\pt$ distribution in simulated Higgs $\to \tau^{+}
\tau^{-}$ events as a function of the Higgs mass.  The functional form of the
parametrization is taken to be the sum of two terms.  The first term, denoted by
$p^{*} \left( \pt | M \right)$, is derived by assuming an isotropic two--body
decay, that is
\begin{equation*}
\mathrm{d}p^{*} \propto \sin\theta \mathrm{d}\theta.
\end{equation*}
Performing a variable transformation from $\theta$ to $\pt \sim \frac{M}{2}
\sin\theta$, we obtain
\begin{align}
p^{*} \left( \pt | M \right) &= \frac{\mathrm{d}p}{\mathrm{d}\pt} = \frac{\mathrm{d}p}{\mathrm{d}\cos\theta}
\left| \frac{\mathrm{d}\cos\theta}{\mathrm{d}\pt} \right| \nonumber \\
&\propto \left| \frac{\mathrm{d}}{\mathrm{d}\pt} \sqrt{1 - \left(2 \frac{\pt}{M} \right)^2} \right| \nonumber \\
&= \frac{1}{\sqrt{\left( \frac{M}{2 \pt} \right)^2 - 1}}. 
\label{eq:ptBalanceTerm1}
\end{align}
The first term of the $\pt$--balance likelihood is taken as the convolution of
Equation~\ref{eq:ptBalanceTerm1} with a Gaussian of width $s$.  The second term
is taken to be a Gamma distribution with scale parameter $\theta$ and shape
parameter~$k$, in order to account for tails in the $\pt$ distribution of the
tau lepton pair.  The complete functional form is thus given by
\begin{equation}
p \left( \pt | M \right) \propto \int_0^{\frac{M}{2}} \! p^* \left( \pt' | M \right) 
 e^{-\frac{\left( \pt - \pt' \right)^2}{2 s^2}} \; \mathrm{d}\pt' + a \Gamma \left( \pt, k, \theta \right).
\label{eq:ptBalanceLikelihood}
\end{equation}
Numerical values of the parameters s, $\theta$ and k are determined by fitting
the PDF in Equation~\ref{eq:ptBalanceLikelihood} to the tau lepton $\pt$
distribution in simulated Higgs $\to\TT$ events.  The relative weight a of the
two terms is also determined in the fit.  Replacing the integrand in
Equation~\ref{eq:ptBalanceLikelihood} by its Taylor expansion so that the
integration can be carried out analytically, keeping polynomial terms up to
fifth order, and assuming the fit parameters to depend at most linearly on the
Higgs mass, we obtain the following numerical values for the parameters:
\begin{align}
s &= 1.8 + 0.018 \cdot M_{\tau\tau} \nonumber \\
k &= 2.2 + 0.0364 \cdot M_{\tau\tau} \nonumber \\
\theta &= 6.74 + 0.02 \cdot M_{\tau\tau} \nonumber \\
a &= 0.48 - 0.0007 \cdot M_{\tau\tau}. \nonumber 
\end{align}

The motivation for the $\pt$--balance likelihood is to add a
``regularization'' term which compensates for the effect of $\pt$ cuts applied
on the visible decay products of the two tau leptons.  In particular for tau
lepton pairs produced in decays of resonances of low mass, the visible $\pt$
cuts significantly affect the distribution of the visible momentum fraction
\mbox{$x = E_{vis}/{E_{\tau}}$}.  The effect is illustrated in
Figures~\ref{fig:ptBalancePtVisCuts}
and~\ref{fig:ptBalancePtVisCutsCompareMasses}.  If no attempt would be made to
compensate for this effect,
Equations~\ref{eq:pdfKineHadronic} and~\ref{eq:pdfKineLeptonic} would yield
likelihood values that are too high at low $x$, resulting in the SVfit to
underestimate the energy of visible decay products (overestimate the energy of
neutrinos) produced in the tau decay, resulting in a significant tail of the
reconstructed mass distribution in the high mass region.  The $\tau^{+}
\tau^{-}$ invariant mass distribution reconstructed with and without the
$\pt$--balance likelihood term is shown in
Figure~\ref{fig:ptBalanceImprovedMassResolution}.  A significant improvement in
resolution and in particular a significant reduction of the non--Gaussian tail
in the region of high masses is seen.

\begin{figure}[t]
\begin{center}
\includegraphics*[height=0.75\textwidth, angle=90]{svfit_chapter/figures/pt_balance_effect.pdf} \caption[Effect of
\pt--balance term on SVfit performance]{\captiontext Distribution of di--tau
invariant mass reconstructed by the SVfit algorithm in simulated Higgs events
with \mbox{$\ma = 130~\GeVcc$}.  The SVfit algorithm is run in two
configurations, with~(blue) and without~(red) the $\pt$--balance likelihood term
included in the fit. 
  %The addition of the $\pt$ balance term improves the resolution and reduces
  %the non--Gaussian tail.
} \label{fig:ptBalanceImprovedMassResolution}
\end{center}
\end{figure} 

\begin{figure}[t]
\begin{center}
\includegraphics*[width=0.49\textwidth]{svfit_chapter/figures/scuplting_A130_muon.pdf}
\includegraphics*[width=0.49\textwidth]{svfit_chapter/figures/scuplting_A130_tau.pdf}
\caption[Effect of the visible \pt requirements on muon and hadronic $\tau$
decays]{\captiontext Normalized distributions of the fraction of total tau decay
energy carried by the muon (left) and hadronic constituents (right) in simulated
Higgs events with \mbox{$\ma = 130~\GeVcc$}.  The distribution is shown before
(blue) and after (red) the requirement on the $\pt$ of the visible decay
products described in Chapter~\ref{ch:selections}.  }
\label{fig:ptBalancePtVisCuts}
\end{center}
\end{figure} 

\begin{figure}[t]
\begin{center}
\includegraphics*[width=0.49\textwidth]{svfit_chapter/figures/scuplting_Ztautau_powheg_muon.pdf}
\includegraphics*[width=0.49\textwidth]{svfit_chapter/figures/scuplting_A200_muon.pdf}
\caption[Effect of the visible \pt requirements for $Z$ and Higgs
events]{\captiontext Normalized distributions of the fraction of total tau decay
energy carried by the muon in simulated $Z\to\TT$~(left) and Higgs events with
\mbox{$\ma = 200~\GeVcc$}~(right).   The distribution is shown before (blue) and
after (red) the requirement that the $\pt$ of the muon be greater than
15~\GeVc.} \label{fig:ptBalancePtVisCutsCompareMasses}
\end{center}
\end{figure} 

\subsection{Secondary Vertex Information} 

The parametrization of the tau decay kinematics described in
section~\ref{sec:svParameterization} can be extended to describe the production
and decay of the tau.  As the flight direction of the tau is already fully
determined by the parameters $\theta$, $\overline{\phi}$ and $m_{\nu\nu}$, the
position of the secondary (decay) vertex is hence fully determined by addition
of a single parameter for the flight distance, $r$.  The tau lifetime $c\tau =
87$~$\mu$m is large enough to allow the displacement of the tau decay vertex
from the primary event vertex to be resolved by the CMS tracking detector.  The
resolution provided by the CMS tracking detector is utilized to improve the
resolution on the $\tau^{+} \tau^{-}$ invariant mass reconstructed by the SVfit
algorithm.  The likelihood term based on the secondary vertex information is
based on the compatibility of the decay vertex position with the reconstructed
tracks of charged tau decay products.  Perhaps surprisingly, it turns out that
the flight distance parameter $R$ is sufficiently constrained even for tau
decays into a single charged hadron, electron or muon.

The parameter $R$ can be constrained further by a term which represents the
probability for a tau lepton of momentum $P$ to travel a distance $d$ before
decaying:
\begin{equation*}
p \left( d | P \right) = \frac{m_\tau}{P c\tau}e^{-\frac{m_\tau d}{P c\tau}}
\end{equation*}

The likelihood terms for the secondary vertex fit have been implemented in the
SVfit algorithm.  In the analysis presented in this note, the decay vertex
information is not used, however, because of systematic effects arising from
tracker (mis--)alignment which are not yet fully understood.

\section{Performance}

The tau pair mass reconstructed by the Secondary Vertex fit (``SVfit mass'')
provides the observable with the largest separation between signal Higgs events
and the dominant \ZTT background.  The mean of the SVfit mass is located at the
true mass of the di--tau pair.  The SVfit algorithm has a higher acceptance and
better resolution than the Collinear Approximation algorithm.  The SVfit always
finds a physical solution, improving on the efficiency of the collinear
approximation by a factor of two.  Additionally, it has a much better
resolution.  The collinear approximation reconstructed mass distribution has a
large tail at high mass due to events with poorly measured \MET\@.  The shape of
the SVfit distribution is nearly Gaussian.  The comparison is illustrated in
Figure~\ref{fig:SVversusCollinear}.  Previous searches for Higgs bosons decaying
to tau leptons~\cite{CDFMSSMHiggs} have in general used the ``visible mass'' as the
observable used to search for new resonances.  The SVfit method has the obvious
difference that it reconstructs the ``full'' tau pair mass, which is the most
natural observable corresponding to a particle decaying to tau leptons.  In
addition, the relative resolution\footnote{We define this metric of performance
as the variance of a distribution divided by its mean.}
of the SV fit is superior to that of the visible mass.  This feature is
illustrated in Figure~\ref{fig:SVversusVis}.  In Figure~\ref{fig:SVversusVis},
the visible mass distribution is scaled by an arbitrary number such that the
scaled mean of the distribution matches the true mass of the tau pair (and the
SVfit mass).  The width of the SVfit distribution is smaller than that of
the scaled visible mass distribution, indicating better performance.  The
increase in relative resolution allows a ``bump,'' due to the presence of signal
events, to be more easily distinguished from the \ZTT background.  This
increases the power of the search for the new signal.

\begin{figure}[t]
\begin{center}
\includegraphics*[width=0.49\textwidth]{svfit_chapter/figures/sv_fit_approval_plots/sv_vs_coll_ZTT_normed.pdf}
\includegraphics*[width=0.49\textwidth]{svfit_chapter/figures/sv_fit_approval_plots/sv_vs_coll_A120.pdf}
\caption[Comparison of SVfit with the Collinear Approximation
algorithm]{Comparison of the reconstructed tau pair mass spectrum in
\ZTT~(left) and MSSM \HTT{120}~(right) events after the selections described in
chapter~\ref{ch:selections}.  The mass spectrum reconstructed by the Secondary
Vertex fit is shown in blue, the result of the collinear approximation algorithm
is given in green.  In the left plot, both distributions are normalized to
unity, illustrating the improvement in resolution (shape) provided by the SVfit.
In the right plot, the distributions are normalized to an (arbitrary)
luminosity, illustrating the loss of events that occurs due to unphysical
solutions in the application of the collinear approximation.}

\label{fig:SVversusCollinear}
\end{center}
\end{figure} 

\begin{figure}[t]
\begin{center}
\includegraphics*[width=0.49\textwidth]{svfit_chapter/figures/sv_fit_approval_plots/sv_vs_vis_A120.pdf}
\includegraphics*[width=0.49\textwidth]{svfit_chapter/figures/sv_fit_approval_plots/sv_vs_vis_A300.pdf}
\caption[Comparison of SVfit with the visible mass observable]{Comparison of the
invariant mass of the muon and \tjet (the ``visible mass'') with the full \TT
mass reconstructed by the SVfit. The spectrum is shown for two simulated MSSM
Higgs samples, with $\ma = 120\GeVcc$~(left), and $\ma = 200\GeVcc$~(right).  To
illustrate that relative resolution of the SVfit is superior to that of the
visible mass, the visible mass is also shown scaled up such that the mean of the
two distributions are identical.} \label{fig:SVversusVis}
\end{center}
\end{figure} 

\ifx\master\undefined% AUTOCOMPILE
% Allows for individual chapters to be compiled.

% Usage:
% \ifx\master\undefined% AUTOCOMPILE
% Allows for individual chapters to be compiled.

% Usage:
% \ifx\master\undefined% AUTOCOMPILE
% Allows for individual chapters to be compiled.

% Usage:
% \ifx\master\undefined\input{../settings/autocompile}\fi
% (place at start and end of chapter file)

\ifx\noprelim\undefined
    % first time included
    % input preamble files
    \input{../settings/phdsetup}

    \begin{document}
    \def\noprelim{}
\else
    % already included once
    % input post files

    \singlespacing
    \bibliographystyle{../bibliography/expanded}
    \bibliography{../bibliography/references}

    \end{document}
\fi\fi
% (place at start and end of chapter file)

\ifx\noprelim\undefined
    % first time included
    % input preamble files
    % [ USER VARIABLES ]

\def\PHDTITLE {Extensions of the Theory of Computational Mechanics}
\def\PHDAUTHOR{Evan Klose Friis}
\def\PHDSCHOOL{University of California, Davis}

\def\PHDMONTH {June}
\def\PHDYEAR  {2011}
\def\PHDDEPT {Physics}

\def\BSSCHOOL {University of California at San Diego}
\def\BSYEAR   {2005}

\def\PHDCOMMITTEEA{Professor John Conway}
\def\PHDCOMMITTEEB{Professor Robin Erbacher}
\def\PHDCOMMITTEEC{Professor Mani Tripathi}

% [ GLOBAL SETUP ]

\documentclass[letterpaper,oneside,11pt]{report}

\usepackage{calc}
\usepackage{breakcites}
\usepackage[newcommands]{ragged2e}

\usepackage[pdftex]{graphicx}
\usepackage{epstopdf}

%\usepackage{tikz}
%\usetikzlibrary{positioning} % [right=of ...]
%\usetikzlibrary{fit} % [fit= ...]

%\pgfdeclarelayer{background layer}
%\pgfdeclarelayer{foreground layer}
%\pgfsetlayers{background layer,main,foreground layer}

%\newenvironment{wrap}{\noindent\begin{minipage}[t]{\linewidth}\vspace{-0.5\normalbaselineskip}\centering}{\vspace{0.5\normalbaselineskip}\end{minipage}}

%% [Venn diagram environment]
%\newenvironment{venn2}
%{\begin{tikzpicture} [every pin/.style={text=black, text opacity=1.0, pin distance=0.5cm, pin edge={black!60, semithick}},
%% define a new style 'venn'
%venn/.style={circle, draw=black!60, semithick, minimum size = 4cm}]
%
%% create circle and give it external (pin) label
%\node[venn] (X) at (-1,0) [pin={150:$H[X]$}] {};
%\node[venn] (Y) at (1,0) [pin={30:$H[Y]$}] {};
%
%% place labels of the atoms by hand
%\node at (-1.9,0) {$H[X|Y]$};
%\node at (1.9,0) {$H[Y|X]$};
%\node at (0,0) {$I[X;Y]$};}
%{\end{tikzpicture}}

%\newcommand{\wrapmath}[1]{\begin{wrap}\begin{tikzpicture}[every node/.style={inner ysep=0ex, inner xsep=0em}]\node[] {$\displaystyle\begin{aligned} #1\end{aligned}$};\end{tikzpicture}\end{wrap}}

\renewenvironment{abstract}{\chapter*{Abstract}}{}
\renewcommand{\bibname}{Bibliography}
\renewcommand{\contentsname}{Table of Contents}

\makeatletter
\renewcommand{\@biblabel}[1]{\textsc{#1}}
\makeatother

% [ FONT SETTINGS ]

\usepackage[tbtags, intlimits, namelimits]{amsmath}
\usepackage[adobe-utopia]{mathdesign}

\DeclareSymbolFont{pazomath}{OMS}{zplm}{m}{n}
\DeclareSymbolFontAlphabet{\mathcal}{pazomath}
\SetMathAlphabet\mathcal{bold}{OMS}{zplm}{b}{n}

\SetSymbolFont{largesymbols}{normal}{OMX}{zplm}{m}{n}
\SetSymbolFont{largesymbols}{bold}{OMX}{zplm}{m}{n}
\SetSymbolFont{symbols}{normal}{OMS}{zplm}{m}{n}
\SetSymbolFont{symbols}{bold}{OMS}{zplm}{b}{n}

\renewcommand{\sfdefault}{phv}
\renewcommand{\ttdefault}{fvm}

\widowpenalty 8000
\clubpenalty  8000

% [ PAGE LAYOUT ]
\usepackage{geometry}
\geometry{lmargin = 1.5in}
\geometry{rmargin = 1.0in}
\geometry{tmargin = 1.0in}
\geometry{bmargin = 1.0in}

% [ PDF SETTINGS ]

\usepackage[final]{hyperref}
\hypersetup{breaklinks  = true}
\hypersetup{colorlinks  = true}
\hypersetup{linktocpage = false}
\hypersetup{linkcolor   = blue}
\hypersetup{citecolor   = green}
\hypersetup{urlcolor    = black}
\hypersetup{plainpages  = false}
\hypersetup{pageanchor  = true}
\hypersetup{pdfauthor   = {\PHDAUTHOR}}
\hypersetup{pdftitle    = {\PHDTITLE}}
\hypersetup{pdfsubject  = {Dissertation, \PHDSCHOOL}}
\urlstyle{same}

% [ LETTER SPACING ]

\usepackage[final]{microtype}
\microtypesetup{protrusion=compatibility}
\microtypesetup{expansion=false}

\newcommand{\upper}[1]{\MakeUppercase{#1}}
\let\lsscshape\scshape

\ifcase\pdfoutput\else\microtypesetup{letterspace=15}
\renewcommand{\scshape}{\lsscshape\lsstyle}
\renewcommand{\upper}[1]{\textls[50]{\MakeUppercase{#1}}}\fi

% [ LINE SPACING ]

\usepackage[doublespacing]{setspace}
\renewcommand{\displayskipstretch}{0.0}

\setlength{\parskip   }{0em}
\setlength{\parindent }{2em}

% [ TABLE FORMATTING ]

\usepackage{booktabs}
\setlength{\heavyrulewidth}{1.5\arrayrulewidth}
\setlength{\lightrulewidth}{1.0\arrayrulewidth}
\setlength{\doublerulesep }{2.0\arrayrulewidth}

% [ SECTION FORMATTING ]

\usepackage[largestsep,nobottomtitles*]{titlesec}
\renewcommand{\bottomtitlespace}{0.75in}

\titleformat{\chapter}[display]{\bfseries\huge\singlespacing}{\filleft\textsc{\LARGE \chaptertitlename\ \thechapter}}{-0.2ex}{\titlerule[3pt]\vspace{0.2ex}}[]

\titleformat{\section}{\LARGE}{\S\thesection\hspace{0.5em}}{0ex}{}
\titleformat{\subsection}{\Large}{\S\thesubsection\hspace{0.5em}}{0ex}{}
\titleformat{\subsubsection}{\large}{\thesubsubsection\hspace{0.5em}}{0ex}{}

\titlespacing*{\chapter}{0em}{6ex}{4ex plus 2ex minus 0ex}
\titlespacing*{\section}{0em}{2ex plus 3ex minus 1ex}{0.5ex plus 0.5ex minus 0.5ex}
\titlespacing*{\subsection}{0ex}{2ex plus 3ex minus 1ex}{0ex}
\titlespacing*{\subsubsection}{0ex}{2ex plus 0ex minus 1ex}{0ex}

% [ HEADER SETTINGS ]

\usepackage{fancyhdr}

\setlength{\headheight}{\normalbaselineskip}
\setlength{\footskip  }{0.5in}
\setlength{\headsep   }{0.5in-\headheight}

\fancyheadoffset[R]{0.5in}
\renewcommand{\headrulewidth}{0pt}
\renewcommand{\footrulewidth}{0pt}

\newcommand{\pagebox}{\parbox[r][\headheight][t]{0.5in}{\hspace\fill\thepage}}

\newcommand{\prelimheaders}{\ifx\prelim\undefined\renewcommand{\thepage}{\textit{\roman{page}}}\fancypagestyle{plain}{\fancyhf{}\fancyfoot[L]{\makebox[\textwidth-0.5in]{\thepage}}}\pagestyle{plain}\def\prelim{}\fi}

\newcommand{\normalheaders}{\renewcommand{\thepage}{\arabic{page}}\fancypagestyle{plain}{\fancyhf{}\fancyhead[R]{\pagebox}}\pagestyle{plain}}

\normalheaders{}

% [ CUSTOM COMMANDS ]

\newcommand{\signaturebox}[1]{\multicolumn{1}{p{4in}}{\vspace{3ex}}\\\midrule #1\\}

%\input{../includes/cmechabbrev}

% [some math stuff - maybe stick in sep file]
\usepackage{amsthm}
\usepackage{amscd}
\theoremstyle{plain}    \newtheorem{Lem}{Lemma}
\theoremstyle{plain}    \newtheorem*{ProLem}{Proof}
\theoremstyle{plain} 	\newtheorem{Cor}{Corollary}
\theoremstyle{plain} 	\newtheorem*{ProCor}{Proof}
\theoremstyle{plain} 	\newtheorem{The}{Theorem}
\theoremstyle{plain} 	\newtheorem*{ProThe}{Proof}
\theoremstyle{plain} 	\newtheorem{Prop}{Proposition}
\theoremstyle{plain} 	\newtheorem*{ProProp}{Proof}
\theoremstyle{plain} 	\newtheorem*{Conj}{Conjecture}
\theoremstyle{plain}	\newtheorem*{Rem}{Remark}
\theoremstyle{plain}	\newtheorem*{Def}{Definition} 
\theoremstyle{plain}	\newtheorem*{Not}{Notation}

% [uniform figure scaling - maybe this is not a good idea]
\def\figscale{.7}
\def\lscale{1.0}

% [FIX ME! - red makes it easier to spot]
\newcommand{\FIX}[1]{\textbf{\textcolor{red}{#1}}}


    \begin{document}
    \def\noprelim{}
\else
    % already included once
    % input post files

    \singlespacing
    \bibliographystyle{../bibliography/expanded}
    \bibliography{../bibliography/references}

    \end{document}
\fi\fi
% (place at start and end of chapter file)

\ifx\noprelim\undefined
    % first time included
    % input preamble files
    % [ USER VARIABLES ]

\def\PHDTITLE {Extensions of the Theory of Computational Mechanics}
\def\PHDAUTHOR{Evan Klose Friis}
\def\PHDSCHOOL{University of California, Davis}

\def\PHDMONTH {June}
\def\PHDYEAR  {2011}
\def\PHDDEPT {Physics}

\def\BSSCHOOL {University of California at San Diego}
\def\BSYEAR   {2005}

\def\PHDCOMMITTEEA{Professor John Conway}
\def\PHDCOMMITTEEB{Professor Robin Erbacher}
\def\PHDCOMMITTEEC{Professor Mani Tripathi}

% [ GLOBAL SETUP ]

\documentclass[letterpaper,oneside,11pt]{report}

\usepackage{calc}
\usepackage{breakcites}
\usepackage[newcommands]{ragged2e}

\usepackage[pdftex]{graphicx}
\usepackage{epstopdf}

%\usepackage{tikz}
%\usetikzlibrary{positioning} % [right=of ...]
%\usetikzlibrary{fit} % [fit= ...]

%\pgfdeclarelayer{background layer}
%\pgfdeclarelayer{foreground layer}
%\pgfsetlayers{background layer,main,foreground layer}

%\newenvironment{wrap}{\noindent\begin{minipage}[t]{\linewidth}\vspace{-0.5\normalbaselineskip}\centering}{\vspace{0.5\normalbaselineskip}\end{minipage}}

%% [Venn diagram environment]
%\newenvironment{venn2}
%{\begin{tikzpicture} [every pin/.style={text=black, text opacity=1.0, pin distance=0.5cm, pin edge={black!60, semithick}},
%% define a new style 'venn'
%venn/.style={circle, draw=black!60, semithick, minimum size = 4cm}]
%
%% create circle and give it external (pin) label
%\node[venn] (X) at (-1,0) [pin={150:$H[X]$}] {};
%\node[venn] (Y) at (1,0) [pin={30:$H[Y]$}] {};
%
%% place labels of the atoms by hand
%\node at (-1.9,0) {$H[X|Y]$};
%\node at (1.9,0) {$H[Y|X]$};
%\node at (0,0) {$I[X;Y]$};}
%{\end{tikzpicture}}

%\newcommand{\wrapmath}[1]{\begin{wrap}\begin{tikzpicture}[every node/.style={inner ysep=0ex, inner xsep=0em}]\node[] {$\displaystyle\begin{aligned} #1\end{aligned}$};\end{tikzpicture}\end{wrap}}

\renewenvironment{abstract}{\chapter*{Abstract}}{}
\renewcommand{\bibname}{Bibliography}
\renewcommand{\contentsname}{Table of Contents}

\makeatletter
\renewcommand{\@biblabel}[1]{\textsc{#1}}
\makeatother

% [ FONT SETTINGS ]

\usepackage[tbtags, intlimits, namelimits]{amsmath}
\usepackage[adobe-utopia]{mathdesign}

\DeclareSymbolFont{pazomath}{OMS}{zplm}{m}{n}
\DeclareSymbolFontAlphabet{\mathcal}{pazomath}
\SetMathAlphabet\mathcal{bold}{OMS}{zplm}{b}{n}

\SetSymbolFont{largesymbols}{normal}{OMX}{zplm}{m}{n}
\SetSymbolFont{largesymbols}{bold}{OMX}{zplm}{m}{n}
\SetSymbolFont{symbols}{normal}{OMS}{zplm}{m}{n}
\SetSymbolFont{symbols}{bold}{OMS}{zplm}{b}{n}

\renewcommand{\sfdefault}{phv}
\renewcommand{\ttdefault}{fvm}

\widowpenalty 8000
\clubpenalty  8000

% [ PAGE LAYOUT ]
\usepackage{geometry}
\geometry{lmargin = 1.5in}
\geometry{rmargin = 1.0in}
\geometry{tmargin = 1.0in}
\geometry{bmargin = 1.0in}

% [ PDF SETTINGS ]

\usepackage[final]{hyperref}
\hypersetup{breaklinks  = true}
\hypersetup{colorlinks  = true}
\hypersetup{linktocpage = false}
\hypersetup{linkcolor   = blue}
\hypersetup{citecolor   = green}
\hypersetup{urlcolor    = black}
\hypersetup{plainpages  = false}
\hypersetup{pageanchor  = true}
\hypersetup{pdfauthor   = {\PHDAUTHOR}}
\hypersetup{pdftitle    = {\PHDTITLE}}
\hypersetup{pdfsubject  = {Dissertation, \PHDSCHOOL}}
\urlstyle{same}

% [ LETTER SPACING ]

\usepackage[final]{microtype}
\microtypesetup{protrusion=compatibility}
\microtypesetup{expansion=false}

\newcommand{\upper}[1]{\MakeUppercase{#1}}
\let\lsscshape\scshape

\ifcase\pdfoutput\else\microtypesetup{letterspace=15}
\renewcommand{\scshape}{\lsscshape\lsstyle}
\renewcommand{\upper}[1]{\textls[50]{\MakeUppercase{#1}}}\fi

% [ LINE SPACING ]

\usepackage[doublespacing]{setspace}
\renewcommand{\displayskipstretch}{0.0}

\setlength{\parskip   }{0em}
\setlength{\parindent }{2em}

% [ TABLE FORMATTING ]

\usepackage{booktabs}
\setlength{\heavyrulewidth}{1.5\arrayrulewidth}
\setlength{\lightrulewidth}{1.0\arrayrulewidth}
\setlength{\doublerulesep }{2.0\arrayrulewidth}

% [ SECTION FORMATTING ]

\usepackage[largestsep,nobottomtitles*]{titlesec}
\renewcommand{\bottomtitlespace}{0.75in}

\titleformat{\chapter}[display]{\bfseries\huge\singlespacing}{\filleft\textsc{\LARGE \chaptertitlename\ \thechapter}}{-0.2ex}{\titlerule[3pt]\vspace{0.2ex}}[]

\titleformat{\section}{\LARGE}{\S\thesection\hspace{0.5em}}{0ex}{}
\titleformat{\subsection}{\Large}{\S\thesubsection\hspace{0.5em}}{0ex}{}
\titleformat{\subsubsection}{\large}{\thesubsubsection\hspace{0.5em}}{0ex}{}

\titlespacing*{\chapter}{0em}{6ex}{4ex plus 2ex minus 0ex}
\titlespacing*{\section}{0em}{2ex plus 3ex minus 1ex}{0.5ex plus 0.5ex minus 0.5ex}
\titlespacing*{\subsection}{0ex}{2ex plus 3ex minus 1ex}{0ex}
\titlespacing*{\subsubsection}{0ex}{2ex plus 0ex minus 1ex}{0ex}

% [ HEADER SETTINGS ]

\usepackage{fancyhdr}

\setlength{\headheight}{\normalbaselineskip}
\setlength{\footskip  }{0.5in}
\setlength{\headsep   }{0.5in-\headheight}

\fancyheadoffset[R]{0.5in}
\renewcommand{\headrulewidth}{0pt}
\renewcommand{\footrulewidth}{0pt}

\newcommand{\pagebox}{\parbox[r][\headheight][t]{0.5in}{\hspace\fill\thepage}}

\newcommand{\prelimheaders}{\ifx\prelim\undefined\renewcommand{\thepage}{\textit{\roman{page}}}\fancypagestyle{plain}{\fancyhf{}\fancyfoot[L]{\makebox[\textwidth-0.5in]{\thepage}}}\pagestyle{plain}\def\prelim{}\fi}

\newcommand{\normalheaders}{\renewcommand{\thepage}{\arabic{page}}\fancypagestyle{plain}{\fancyhf{}\fancyhead[R]{\pagebox}}\pagestyle{plain}}

\normalheaders{}

% [ CUSTOM COMMANDS ]

\newcommand{\signaturebox}[1]{\multicolumn{1}{p{4in}}{\vspace{3ex}}\\\midrule #1\\}

%%%% macros fro standard references
%\eqref provided by amsmath
\newcommand{\figref}[1]{Fig.~\ref{#1}}
\newcommand{\tableref}[1]{Table~\ref{#1}}
\newcommand{\refcite}[1]{Ref.~\cite{#1}}

% Abbreviations from CMPPSS:

\newcommand{\eM}     {\mbox{$\epsilon$-machine}}
\newcommand{\eMs}    {\mbox{$\epsilon$-machines}}
\newcommand{\EM}     {\mbox{$\epsilon$-Machine}}
\newcommand{\EMs}    {\mbox{$\epsilon$-Machines}}
\newcommand{\eT}     {\mbox{$\epsilon$-transducer}}
\newcommand{\eTs}    {\mbox{$\epsilon$-transducers}}
\newcommand{\ET}     {\mbox{$\epsilon$-Transducer}}
\newcommand{\ETs}    {\mbox{$\epsilon$-Transducers}}

% Processes and sequences

\newcommand{\Process}{\mathcal{P}}

\newcommand{\ProbMach}{\Prob_{\mathrm{M}}}
\newcommand{\Lmax}   { {L_{\mathrm{max}}}}
\newcommand{\MeasAlphabet}	{\mathcal{A}}
% Original
%\newcommand{\MeasSymbol}   { {S} }
%\newcommand{\meassymbol}   { {s} }
% New symbol
\newcommand{\MeasSymbol}   { {X} }
\newcommand{\meassymbol}   { {x} }
\newcommand{\BiInfinity}	{ \overleftrightarrow {\MeasSymbol} }
\newcommand{\biinfinity}	{ \overleftrightarrow {\meassymbol} }
\newcommand{\Past}	{ \overleftarrow {\MeasSymbol} }
\newcommand{\past}	{ {\overleftarrow {\meassymbol}} }
\newcommand{\pastprime}	{ {\past}^{\prime}}
\newcommand{\Future}	{ \overrightarrow{\MeasSymbol} }
\newcommand{\future}	{ \overrightarrow{\meassymbol} }
\newcommand{\futureprime}	{ {\future}^{\prime}}
\newcommand{\PastPrime}	{ {\Past}^{\prime}}
\newcommand{\FuturePrime}	{ {\overrightarrow{\meassymbol}}^\prime }
\newcommand{\PastDblPrime}	{ {\overleftarrow{\meassymbol}}^{\prime\prime} }
\newcommand{\FutureDblPrime}	{ {\overrightarrow{\meassymbol}}^{\prime\prime} }
\newcommand{\pastL}	{ {\overleftarrow {\meassymbol}}{}^L }
\newcommand{\PastL}	{ {\overleftarrow {\MeasSymbol}}{}^L }
\newcommand{\PastLt}	{ {\overleftarrow {\MeasSymbol}}_t^L }
\newcommand{\PastLLessOne}	{ {\overleftarrow {\MeasSymbol}}^{L-1} }
\newcommand{\futureL}	{ {\overrightarrow{\meassymbol}}{}^L }
\newcommand{\FutureL}	{ {\overrightarrow{\MeasSymbol}}{}^L }
\newcommand{\FutureLt}	{ {\overrightarrow{\MeasSymbol}}_t^L }
\newcommand{\FutureLLessOne}	{ {\overrightarrow{\MeasSymbol}}^{L-1} }
\newcommand{\pastLprime}	{ {\overleftarrow {\meassymbol}}^{L^\prime} }
\newcommand{\futureLprime}	{ {\overrightarrow{\meassymbol}}^{L^\prime} }
\newcommand{\AllPasts}	{ { \overleftarrow {\rm {\bf \MeasSymbol}} } }
\newcommand{\AllFutures}	{ \overrightarrow {\rm {\bf \MeasSymbol}} }
\newcommand{\FutureSet}	{ \overrightarrow{\bf \MeasSymbol}}

% Causal states and epsilon-machines
\newcommand{\CausalState}	{ \mathcal{S} }
\newcommand{\CausalStatePrime}	{ {\CausalState}^{\prime}}
\newcommand{\causalstate}	{ \sigma }
\newcommand{\CausalStateSet}	{ \boldsymbol{\CausalState} }
\newcommand{\AlternateState}	{ \mathcal{R} }
\newcommand{\AlternateStatePrime}	{ {\cal R}^{\prime} }
\newcommand{\alternatestate}	{ \rho }
\newcommand{\alternatestateprime}	{ {\rho^{\prime}} }
\newcommand{\AlternateStateSet}	{ \boldsymbol{\AlternateState} }
\newcommand{\PrescientState}	{ \widehat{\AlternateState} }
\newcommand{\prescientstate}	{ \widehat{\alternatestate} }
\newcommand{\PrescientStateSet}	{ \boldsymbol{\PrescientState}}
\newcommand{\CausalEquivalence}	{ {\sim}_{\epsilon} }
\newcommand{\CausalEquivalenceNot}	{ {\not \sim}_{\epsilon}}

\newcommand{\NonCausalEquivalence}	{ {\sim}_{\eta} }
\newcommand{\NextObservable}	{ {\overrightarrow {\MeasSymbol}}^1 }
\newcommand{\LastObservable}	{ {\overleftarrow {\MeasSymbol}}^1 }
%\newcommand{\Prob}		{ {\rm P}}
\newcommand{\Prob}      {\Pr} % use standard command
\newcommand{\ProbAnd}	{ {,\;} }
\newcommand{\LLimit}	{ {L \rightarrow \infty}}
\newcommand{\Cmu}		{C_\mu}
\newcommand{\hmu}		{h_\mu}
\newcommand{\EE}		{{\bf E}}
\newcommand{\Measurable}{{\bf \mu}}

% Process Crypticity
\newcommand{\PC}		{\chi}
\newcommand{\FuturePC}		{\PC^+}
\newcommand{\PastPC}		{\PC^-}
% Causal Irreversibility
\newcommand{\CI}		{\Xi}
\newcommand{\ReverseMap}	{r}
\newcommand{\ForwardMap}	{f}

% Abbreviations from IB:
% None that aren't already in CMPPSS

% Abbreviations from Extensive Estimation:
\newcommand{\EstCausalState}	{\widehat{\CausalState}}
\newcommand{\estcausalstate}	{\widehat{\causalstate}}
\newcommand{\EstCausalStateSet}	{\boldsymbol{\EstCausalState}}
\newcommand{\EstCausalFunc}	{\widehat{\epsilon}}
\newcommand{\EstCmu}		{\widehat{\Cmu}}
\newcommand{\PastLOne}	{{\Past}^{L+1}}
\newcommand{\pastLOne}	{{\past}^{L+1}}

% Abbreviations from $\epsilon$-Transducers:
\newcommand{\InAlphabet}	{ \mathcal{A}}
\newcommand{\insymbol}		{ a}
\newcommand{\OutAlphabet}	{ \mathcal{B}}
\newcommand{\outsymbol}		{ b}
\newcommand{\InputSimple}	{ X}
\newcommand{\inputsimple}	{ x}
\newcommand{\BottleneckVar}	{\tilde{\InputSimple}}
\newcommand{\bottleneckvar}	{\tilde{\inputsimple}}
\newcommand{\InputSpace}	{ \mathbf{\InputSimple}}
\newcommand{\InputBi}	{ \overleftrightarrow {\InputSimple} }
\newcommand{\inputbi}	{ \overleftrightarrow {\inputsimple} }
\newcommand{\InputPast}	{ \overleftarrow {\InputSimple} }
\newcommand{\inputpast}	{ \overleftarrow {\inputsimple} }
\newcommand{\InputFuture}	{ \overrightarrow {\InputSimple} }
\newcommand{\inputfuture}	{ \overrightarrow {\inputsimple} }
\newcommand{\NextInput}	{ {{\InputFuture}^{1}}}
\newcommand{\NextOutput}	{ {\OutputFuture}^{1}}
\newcommand{\OutputSimple}	{ Y}
\newcommand{\outputsimple}	{ y}
\newcommand{\OutputSpace}	{ \mathbf{\OutputSimple}}
\newcommand{\OutputBi}	{ \overleftrightarrow{\OutputSimple} }
\newcommand{\outputbi}	{ \overleftrightarrow{\outputsimple} }
\newcommand{\OutputPast}	{ \overleftarrow{\OutputSimple} }
\newcommand{\outputpast}	{ \overleftarrow{\outputsimple} }
\newcommand{\OutputFuture}	{ \overrightarrow{\OutputSimple} }
\newcommand{\outputfuture}	{ \overrightarrow{\outputsimple} }
\newcommand{\OutputL}	{ {\OutputFuture}^L}
\newcommand{\outputL}	{ {\outputfuture}^L}
\newcommand{\InputLLessOne}	{ {\InputFuture}^{L-1}}
\newcommand{\inputLlessone}	{ {\inputufutre}^{L-1}}
\newcommand{\OutputPastLLessOne}	{{\OutputPast}^{L-1}_{-1}}
\newcommand{\outputpastLlessone}	{{\outputpast}^{L-1}}
\newcommand{\OutputPastLessOne}	{{\OutputPast}_{-1}}
\newcommand{\outputpastlessone}	{{\outputpast}_{-1}}
\newcommand{\OutputPastL}	{{\OutputPast}^{L}}
\newcommand{\OutputLPlusOne}	{ {\OutputFuture}^{L+1}}
\newcommand{\outputLplusone}	{ {\outputfutre}^{L+1}}
\newcommand{\InputPastL}	{{\InputPast}^{L}}
\newcommand{\inputpastL}	{{\inputpast}^{L}}
\newcommand{\JointPast}	{{(\InputPast,\OutputPast)}}
\newcommand{\jointpast}	{{(\inputpast,\outputpast)}}
\newcommand{\jointpastone}	{{(\inputpast_1,\outputpast_1)}}
\newcommand{\jointpasttwo}	{{(\inputpast_2,\outputpast_2)}}
\newcommand{\jointpastprime} {{({\inputpast}^{\prime},{\outputpast}^{\prime})}}
\newcommand{\NextJoint}	{{(\NextInput,\NextOutput)}}
\newcommand{\nextjoint}	{{(\insymbol,\outsymbol)}}
\newcommand{\AllInputPasts}	{ { \overleftarrow {\rm \InputSpace}}}
\newcommand{\AllOutputPasts}	{ {\overleftarrow {\rm \OutputSpace}}}
\newcommand{\DetCausalState}	{ {{\cal S}_D }}
\newcommand{\detcausalstate}	{ {{\sigma}_D} }
\newcommand{\DetCausalStateSet}	{ \boldsymbol{{\CausalState}_D}}
\newcommand{\DetCausalEquivalence}	{ {\sim}_{{\epsilon}_{D}}}
\newcommand{\PrescientEquivalence}	{ {\sim}_{\widehat{\eta}}}
\newcommand{\FeedbackCausalState}	{ \mathcal{F}}
\newcommand{\feedbackcausalstate}	{ \phi}
\newcommand{\FeedbackCausalStateSet}	{ \mathbf{\FeedbackCausalState}}
\newcommand{\JointCausalState}		{ \mathcal{J}}
\newcommand{\JointCausalStateSet}	{ \mathbf{\JointCausalState}}
\newcommand{\UtilityFunctional}	{ {\mathcal{L}}}
\newcommand{\NatureState}	{ {\Omega}}
\newcommand{\naturestate}	{ {\omega}}
\newcommand{\NatureStateSpace}	{ {\mathbf{\NatureState}}}
\newcommand{\AnAction}	{ {A}}
\newcommand{\anaction}	{ {a}}
\newcommand{\ActionSpace}	{ {\mathbf{\AnAction}}}

% Abbreviations from RURO:
\newcommand{\InfoGain}[2] { \mathcal{D} \left( {#1} || {#2} \right) }

% Abbreviations from Upper Bound:
\newcommand{\lcm}	{{\rm lcm}}
% Double-check that this isn't in the math set already!

% Abbreviations from Emergence in Space
\newcommand{\ProcessAlphabet}	{\MeasAlphabet}
\newcommand{\ProbEst}			{ {\widehat{\Prob}_N}}
\newcommand{\STRegion}			{ {\mathrm K}}
\newcommand{\STRegionVariable}		{ K}
\newcommand{\stregionvariable}		{ k}
\newcommand{\GlobalPast}		{ \overleftarrow{G}} 
\newcommand{\globalpast}		{ \overleftarrow{g}} 
\newcommand{\GlobalFuture}		{ \overrightarrow{G}}
\newcommand{\globalfuture}		{ \overrightarrow{g}}
\newcommand{\GlobalState}		{ \mathcal{G}}
\newcommand{\globalstate}		{ \gamma}
\newcommand{\GlobalStateSet}		{ {\mathbf \GlobalState}}
\newcommand{\LocalPast}			{ \overleftarrow{L}} 
\newcommand{\localpast}			{ \overleftarrow{l}}
\newcommand{\AllLocalPasts}		{ \mathbf{\LocalPast}}
\newcommand{\LocalPastRegion}		{ \overleftarrow{\mathrm L}}
\newcommand{\LocalFuture}		{ \overrightarrow{L}}
\newcommand{\localfuture}		{ \overrightarrow{l}}
\newcommand{\LocalFutureRegion}		{ \overrightarrow{\mathrm L}}
\newcommand{\LocalState}		{ \mathcal{L}}
\newcommand{\localstate}		{ \lambda}
\newcommand{\LocalStateSet}		{ {\mathbf \LocalState}}
\newcommand{\PatchPast}			{ \overleftarrow{P}}
\newcommand{\patchpast}			{ \overleftarrow{p}}
\newcommand{\PatchPastRegion}		{ \overleftarrow{\mathrm P}}
\newcommand{\PatchFuture}		{ \overrightarrow{P}}
\newcommand{\patchfuture}		{ \overrightarrow{p}}
\newcommand{\PatchFutureRegion}		{ \overrightarrow{\mathrm P}}
\newcommand{\PatchState}		{ \mathcal{P}}
\newcommand{\patchstate}		{ \pi}
\newcommand{\PatchStateSet}		{ {\mathbf \PatchState}}
\newcommand{\LocalStatesInPatch}	{\vec{\LocalState}}
\newcommand{\localstatesinpatch}	{\vec{\localstate}}
\newcommand{\PointInstantX}		{ {\mathbf x}}
% Galles's original LaTeX for the cond. indep. symbol follows:
\newcommand{\compos}{\mbox{$~\underline{~\parallel~}~$}}
\newcommand{\ncompos}{\not\hspace{-.15in}\compos}
\newcommand{\indep}			{ \rotatebox{90}{$\models$}}
\newcommand{\nindep}	{\not\hspace{-.05in}\indep}
\newcommand{\LocalEE}	{{\EE}^{loc}}
\newcommand{\EEDensity}	{\overline{\LocalEE}}
\newcommand{\LocalCmu}	{{\Cmu}^{loc}}
\newcommand{\CmuDensity}	{\overline{\LocalCmu}}

%%%%%%%%%%% added by sasa
\newcommand{\FinPast}[1]	{ \overleftarrow {\MeasSymbol} \stackrel{{#1}}{}}
\newcommand{\finpast}[1]  	{ \overleftarrow {\meassymbol}  \stackrel{{#1}}{}}
\newcommand{\FinFuture}[1]		{ \overrightarrow{\MeasSymbol} \stackrel{{#1}}{}}
\newcommand{\finfuture}[1]		{ \overrightarrow{\meassymbol} \stackrel{{#1}}{}}

\newcommand{\Partition}	{ \AlternateState }
\newcommand{\partitionstate}	{ \alternatestate }
\newcommand{\PartitionSet}	{ \AlternateStateSet }
\newcommand{\Fdet}   { F_{\rm det} }

\newcommand{\Dkl}[2] { D_{\rm KL} \left( {#1} || {#2} \right) }

\newcommand{\Period}	{p}

% To take into account time direction
\newcommand{\forward}{+}
\newcommand{\reverse}{-}
%\newcommand{\forwardreverse}{\:\!\diamond} % \pm
\newcommand{\forwardreverse}{\pm} % \pm
\newcommand{\FutureProcess}	{ {\Process}^{\forward} }
\newcommand{\PastProcess}	{ {\Process}^{\reverse} }
\newcommand{\FutureCausalState}	{ {\CausalState}^{\forward} }
\newcommand{\futurecausalstate}	{ \sigma^{\forward} }
\newcommand{\altfuturecausalstate}	{ \sigma^{\forward\prime} }
\newcommand{\PastCausalState}	{ {\CausalState}^{\reverse} }
\newcommand{\pastcausalstate}	{ \sigma^{\reverse} }
\newcommand{\BiCausalState}		{ {\CausalState}^{\forwardreverse} }
\newcommand{\bicausalstate}		{ {\sigma}^{\forwardreverse} }
\newcommand{\FutureCausalStateSet}	{ {\CausalStateSet}^{\forward} }
\newcommand{\PastCausalStateSet}	{ {\CausalStateSet}^{\reverse} }
\newcommand{\BiCausalStateSet}	{ {\CausalStateSet}^{\forwardreverse} }
\newcommand{\eMachine}	{ M }
\newcommand{\FutureEM}	{ {\eMachine}^{\forward} }
\newcommand{\PastEM}	{ {\eMachine}^{\reverse} }
\newcommand{\BiEM}		{ {\eMachine}^{\forwardreverse} }
\newcommand{\BiEquiv}	{ {\sim}^{\forwardreverse} }
\newcommand{\Futurehmu}	{ h_\mu^{\forward} }
\newcommand{\Pasthmu}	{ h_\mu^{\reverse} }
\newcommand{\FutureCmu}	{ C_\mu^{\forward} }
\newcommand{\PastCmu}	{ C_\mu^{\reverse} }
\newcommand{\BiCmu}		{ C_\mu^{\forwardreverse} }
\newcommand{\FutureEps}	{ \epsilon^{\forward} }
\newcommand{\PastEps}	{ \epsilon^{\reverse} }
\newcommand{\BiEps}	{ \epsilon^{\forwardreverse} }
\newcommand{\FutureSim}	{ \sim^{\forward} }
\newcommand{\PastSim}	{ \sim^{\reverse} }
% Used arrows for awhile, more or less confusing?
%\newcommand{\FutureCausalState}	{ \overrightarrow{\CausalState} }
%\newcommand{\PastCausalState}	{ \overleftarrow{\CausalState} }
%\newcommand{\eMachine}	{ M }
%\newcommand{\FutureEM}	{ \overrightarrow{\eMachine} }
%\newcommand{\PastEM}	{ \overleftarrow{\eMachine} }
%\newcommand{\FutureCmu}	{ \overrightarrow{\Cmu} }
%\newcommand{\PastCmu}	{ \overleftarrow{\Cmu} }

%% time-reversing and mixed state presentation operators
\newcommand{\TR}{\mathcal{T}}
\newcommand{\MSP}{\mathcal{U}}
\newcommand{\one}{\mathbf{1}}

%% (cje)
%% Provide a command \ifpm which is true when \pm 
%% is meant to be understood as "+ or -". This is
%% different from the usage in TBA.
\newif\ifpm 
\edef\tempa{\forwardreverse}
\edef\tempb{\pm}
\ifx\tempa\tempb
   \pmfalse
\else
   \pmtrue  
\fi





% [some math stuff - maybe stick in sep file]
\usepackage{amsthm}
\usepackage{amscd}
\theoremstyle{plain}    \newtheorem{Lem}{Lemma}
\theoremstyle{plain}    \newtheorem*{ProLem}{Proof}
\theoremstyle{plain} 	\newtheorem{Cor}{Corollary}
\theoremstyle{plain} 	\newtheorem*{ProCor}{Proof}
\theoremstyle{plain} 	\newtheorem{The}{Theorem}
\theoremstyle{plain} 	\newtheorem*{ProThe}{Proof}
\theoremstyle{plain} 	\newtheorem{Prop}{Proposition}
\theoremstyle{plain} 	\newtheorem*{ProProp}{Proof}
\theoremstyle{plain} 	\newtheorem*{Conj}{Conjecture}
\theoremstyle{plain}	\newtheorem*{Rem}{Remark}
\theoremstyle{plain}	\newtheorem*{Def}{Definition} 
\theoremstyle{plain}	\newtheorem*{Not}{Notation}

% [uniform figure scaling - maybe this is not a good idea]
\def\figscale{.7}
\def\lscale{1.0}

% [FIX ME! - red makes it easier to spot]
\newcommand{\FIX}[1]{\textbf{\textcolor{red}{#1}}}


    \begin{document}
    \def\noprelim{}
\else
    % already included once
    % input post files

    \singlespacing
    \bibliographystyle{../bibliography/expanded}
    \bibliography{../bibliography/references}

    \end{document}
\fi\fi
