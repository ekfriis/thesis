\ifx\master\undefined\input{settings/autocompile}\fi

\chapter{Systematics} \label{ch:systematics} Proper determination of systematic
uncertainties is one of the most challenging and important components in
performing a correct measurement.  A systematic uncertainty is the effect of the
uncertainty of some ancillary measurement (or assumption) that is used in the
computation of the final result.  An instructive example of how a systematic
uncertainty can affect the final result is a counting experiment measuring the cross
section of some signal particle in the presence of background.  The formula for
the cross section times the branching fraction is
\begin{equation}
  \sigma \times BR = \frac{N_{sig}}{\mathcal{L} \cdot \mathcal {A} \cdot
  \epsilon} = \frac{N_{obs}-N_{bkg}}{\mathcal{L} \cdot \mathcal {A} \cdot
  \epsilon}, \label{eq:CrossSectionEquation}
\end{equation}
where $N_{obs}$ is the number of events observed in data, $N_{bkg}$ is the
estimated number of background events in the observed data sample, $\mathcal{L}$
is the integrated luminosity, and $\mathcal{A} \cdot \epsilon$ is the acceptance
times efficiency of the signal.  All of the quantities in
Equation~\ref{eq:CrossSectionEquation} (with the exception of the observed
count $N_{obs}$) have some uncertainty which will effect the final measurement.
Consider a situation where the expected number of background events is
determined by fitting some sideband spectrum, and the fitted result has some
error $\delta N_{bkg}$.  The total relative effect of this error can
be obtained by error propagation
\begin{equation}
  \frac{\delta (\sigma \times BR)}{\sigma \times BR} = \frac{\partial(\sigma
  \times BR)}{\partial N_{bkg}}  \frac{1}{\sigma \times BR} \delta N_{bkg} =
  \frac{-\delta N_{bkg}} {N_{obs}-N_{bkg}}.
  \label{eq:CrossSectionEquationBkgSystematicError}
\end{equation}
It is interesting to examine
Equation~\ref{eq:CrossSectionEquationBkgSystematicError} in two scenarios.  In
the limit that $N_{obs}$ is large compared to $N_{bkg}$, the effect of the error
on the background estimate $\delta N_{bkg}$ does not affect the final result.
In a scenario when the data is dominated by background events, the relative
error on the signal measurement due to the background estimation approaches
infinity.  The sensitivity of a measurement to a systematic uncertainty on a
parameter depends on the context in which that parameter is used.

Experimental systematic uncertainties relevant for MSSM Higgs $\to \TT$ signal
extraction presented in this thesis are classified in three categories:
normalization uncertainties on the signal, normalization uncertainties on
background contributions, and shape uncertainties.  Normalization uncertainties
on the signal are due to lepton reconstruction, identification, isolation and
trigger efficiencies.  These terms are equivalent to the efficiency $\epsilon$
and acceptance terms $\mathcal{A}$ of
Equation~\ref{eq:CrossSectionEquationBkgSystematicError} and affect the expected
yield of MSSM Higgs $\to \TT$ signal and of $Z \to \TT$ background events.  They
do not affect the \emph{shapes} of visible and ``full'' invariant mass
distributions which are used to extract the MSSM Higgs $\to \TT$ signal
contribution in the analyzed dataset.  Uncertainties on the shapes of the
distributions are described by ``morphing'' systematics.  These are are due to
uncertainties on the momentum/energy scale of identified electrons, muons, tau
and other jets in the event.  The measurement of missing transverse energy
represents another source of uncertainty specific to the ``full'' mass
reconstruction by the SVfit algorithm.  The ``morphing'' systematics affect the
shapes of signal as well as background contributions.  Normalization
uncertainties on background contributions are estimated from the level of
agreement between data and Monte Carlo simulation in background dominated
control regions.

\section{Signal normalization uncertainties}

The signal normalization uncertainties are due to imperfect knowledge of how
improperly modeled effects in the detector affect our ``acceptance'' model, or
the probability that a given signal event will pass one of the selections
(detailed in Chapter~\ref{ch:selections}).  The general procedure to quantify
these uncertainties is to measure the effect in some control region in both the
data and Monte Carlo.  The ratio of data to Monte Carlo then gives a correction
factor which is applied to the simulation.  An uncertainty on the measurement of
the effect in control region (in data, simulation, or both) is then taken as the
systematic uncertainties.  The signal normalization uncertainties affecting this
analysis on muon trigger, reconstruction, identification and isolation
efficiencies are taken from the tag and probe analysis of $Z \to \MM$ events
presented in section~\ref{sec:ZmumuTagAndProbe}.  The uncertainty on the tau
reconstruction and identification efficiency is taken to be $23\%$.  The
dependency of the Higgs signal extraction on the tau identification efficiency
has been studied, the result being that uncertainties on the tau identification
efficiency affect the limit on cross--section times branching ratio for MSSM
Higgs $\to \TT$ production by a few percent only.  An uncertainty of $11\%$ is
attributed to the luminosity measurement.

\section{Background normalization uncertainties}

Uncertainties on the normalization of background processes are obtained from the
study of background enriched control regions presented in
Chapter~\ref{ch:backgrounds}.  The main non--$Z \to \TT$ background to
the analysis is due to QCD multi--jet and $W$ + jets events.  These backgrounds
are produced copiously enough for the backgrounds to be studied in control
regions dominated by a single background process with a purity exceeding $90\%$
and an event statistics exceeding the expected contribution of that background
to the analysis by more than one order of magnitude.  Both backgrounds are found
to be well modeled by the Monte Carlo simulation.  An uncertainty of $10\%$ is
attributed to the contribution of QCD and $W$ + jet backgrounds to the analysis.
The cross--section for $\ttbar + jets$ production makes it difficult to select a
high purity sample of $\ttbar + jet$ events of high event statistics.  From the
study of the $19$ events selected in the $\ttbar + jets$ background enriched
control sample we assume an uncertainty on the $\ttbar + jets$ background
contribution in the analysis of $30\%$.  The $Z \to \MM$ background has been
studied with large statistical precision in two separate control regions,
dominated by events in which the reconstructed tau--jet candidate is either due
to a misidentified quark or gluon jet or due to a misidentified muon.  Good
agreement between data and Monte Carlo simulation is found in both cases.
Sizeable uncertainties on the $Z \to \MM$ background contribution arise due to
the extrapolation from the background enriched control regions to the data
sample considered in the analysis, however: the contribution of $Z \to \MM$
background events to the analysis is due to events in which one of the two muons
produced in the $Z$ decay either escapes detection or fakes the signature of a
hadronic tau decay.  Both cases may be difficult to model precisely in the Monte
Carlo simulation.  The non--observation of a $Z$ mass peak in the mu + tau
visible mass distribution studied with the fake--rate method on the other hand
sets a limit on possible contributions from $Z \to \MM$ background events.
Conservatively, we assume an uncertainty of $100\%$ on both types of $Z \to \MM$
background contributions.


\section{Shape uncertainties}

Shape uncertainties on the distributions of visible and ``full'' invariant mass
reconstructed by the SVfit algorithm are estimated by varying the electron
energy and muon momentum scale, the energy scale of tau--jets and other jets in
the event and varying the missing transverse energy in Monte Carlo simulated
events.  After each variation the complete event is rereconstructed and passed
through the event selection.  Shifted visible and ``full'' invariant mass shapes
are obtained for each variation from the events passing all event selection
criteria.  The difference between shifted shapes and the ``nominal'' shapes
obtained from Monte Carlo simulated events with no variation of energy or
momentum scale or of the missing transverse energy applied is then taken as
shape uncertainty.

The uncertainty on the muon momentum scale is taken from the analysis known
di--muon resonances~\cite{CMS_AN_2010-059} and found to have a very small effect
only.  The uncertainty on the jet energy scale is determined from an analysis of
the \pt balance between photons and jets in $\gamma$ + jets
events~\cite{CMS-PAS-JME-10-010}.  The jet energy scale uncertainties determined
by the JetMET group are applied to tau--jets as well as other jets in the event.
The tau energy scale correction factor is currently taken to be 1.0 with an
uncertainty of 3\%.  The QCD jet energy scale has been measured to within 3\%
uncertainty.  In the future, the energy scale of the tau is expected to be
determined to a much better precision, as the neutral hadronic activity of a
hadronic tau decay is expected to be zero. The jet energy scale of 3\% can be
confidently considered~\cite{CMS-PAS-TAU-11-001} an upper limit, and is used in
this analysis as the tau energy scale uncertainty.

The modelling of missing transverse energy in different types of background
events has been studied in the background enriched control regions described in
Chapter~\ref{ch:backgrounds}.  No significant deviations between data and
Monte Carlo simulation have been found ({\it cf.}\ control plots in the
appendix).  Uncertainties due to missing transverse energy are estimated by
varying parameters of $Z$--recoil corrections within the uncertainties obtained
when fitting the $Z$--recoil correction parameters in simulated $Z \to \MM$
events versus $Z \to \MM$ events selected in data.


\section{Theory uncertainties}

The signal and background normalization as well as the shape uncertainties are
all experimental uncertainties in nature.  Additional theoretical uncertainties
arise from imprecise knowledge of parton--distribution functions (PDFs) and of
the exact dependency of signal cross--sections and branching ratios on
$tan\beta$ and $m_A$.

The uncertainties on the signal acceptance due to PDF uncertainties are
estimated using tools developed by the EWK
group~\cite{CMS_EWK_pdfUncertaintyTools}.  The acceptance is computed with
respect to MSSM Higgs $\to \TT$ decays that have electrons of $P_{T}^{e} >
15$~\GeV and $\left| \eta_{e} \right| < 2.1$, muons of $P_{T}^{\mu} > 15$~\GeV
and $\left| \eta_{\mu} \right| < 2.1$, jets produced in hadronic tau decays with
visible $P_{T}^{vis} > 20$~\GeV and $\left| \eta_{vis} \right| < 2.3$ on
generator level, depending on the analysis channel considered.  Acceptance
values are computed for the central value and $44$ eigenvectors of the CTEQ66
PDF set~\cite{CTEQpdfSet}.  The systematic uncertainty on the signal acceptance
is computed following the PDF4LHC
recommendations~\cite{pdfAccSys01,pdfAccSys02}.


The effect of Monte Carlo normalization, shape and theory uncertainties on the
signal efficiency times acceptance is summarized in
table~\ref{tab:ExpUncertainties}.

\begin{table}[t]
\begin{center}
\tablesize
\begin{tabular}{|l|c|}
\hline
Source & Effect \\
\hline
\hline
\multicolumn{2}{|c|}{Normalization uncertainties} \\
\hline
Trigger                         & $0.981 \pm 0.006$ \\
Muon identification             & $1.001 \pm 0.001$ \\
Muon isolation                  & $0.984 \pm 0.006$ \\
Tau--jet identification         & $1.00  \pm 0.30$ \\
\hline
\hline
\multicolumn{2}{|c|}{Shape uncertainties} \\
\hline
Muon momentum scale             & $\ll 1\%$ \\
Tau--jet energy scale           & $1 - 4\%^{1}$ \\
Jet energy scale (JES)          & $< 1\%^{2}$ \\
$\MET$ ($Z$--recoil correction) & $1\%$ \\
\hline
\hline
\multicolumn{2}{|c|}{Theory uncertainties} \\
\hline
PDF & $2\%^{3}$ \\
\hline
\end{tabular}
\end{center}
$^{1}$ decreasing with $m_{A}$ \\
$^{2}$ number quoted for $gg \to A/H$ and $b\bar{b} \to A/H$ sample as a whole; \\
\hspace{5mm} in the subsample of events with b--tagged jets the effect of the JES uncertainty is $4\%$ \\
$^{3}$ with small dependence on $m_{A}$ \\
\begin{center}
\caption{\captiontext
         Effect of normalization uncertainties on the $gg \to A/H$ and $b\bar{b} \to A/H$ signal efficiency times acceptance.}
\label{tab:ExpUncertainties}
\end{center}
\end{table}


\ifx\master\undefined\input{settings/autocompile}\fi
