\ifx\master\undefined% AUTOCOMPILE
% Allows for individual chapters to be compiled.

% Usage:
% \ifx\master\undefined% AUTOCOMPILE
% Allows for individual chapters to be compiled.

% Usage:
% \ifx\master\undefined% AUTOCOMPILE
% Allows for individual chapters to be compiled.

% Usage:
% \ifx\master\undefined\input{../settings/autocompile}\fi
% (place at start and end of chapter file)

\ifx\noprelim\undefined
    % first time included
    % input preamble files
    \input{../settings/phdsetup}

    \begin{document}
    \def\noprelim{}
\else
    % already included once
    % input post files

    \singlespacing
    \bibliographystyle{../bibliography/expanded}
    \bibliography{../bibliography/references}

    \end{document}
\fi\fi
% (place at start and end of chapter file)

\ifx\noprelim\undefined
    % first time included
    % input preamble files
    % [ USER VARIABLES ]

\def\PHDTITLE {Extensions of the Theory of Computational Mechanics}
\def\PHDAUTHOR{Evan Klose Friis}
\def\PHDSCHOOL{University of California, Davis}

\def\PHDMONTH {June}
\def\PHDYEAR  {2011}
\def\PHDDEPT {Physics}

\def\BSSCHOOL {University of California at San Diego}
\def\BSYEAR   {2005}

\def\PHDCOMMITTEEA{Professor John Conway}
\def\PHDCOMMITTEEB{Professor Robin Erbacher}
\def\PHDCOMMITTEEC{Professor Mani Tripathi}

% [ GLOBAL SETUP ]

\documentclass[letterpaper,oneside,11pt]{report}

\usepackage{calc}
\usepackage{breakcites}
\usepackage[newcommands]{ragged2e}

\usepackage[pdftex]{graphicx}
\usepackage{epstopdf}

%\usepackage{tikz}
%\usetikzlibrary{positioning} % [right=of ...]
%\usetikzlibrary{fit} % [fit= ...]

%\pgfdeclarelayer{background layer}
%\pgfdeclarelayer{foreground layer}
%\pgfsetlayers{background layer,main,foreground layer}

%\newenvironment{wrap}{\noindent\begin{minipage}[t]{\linewidth}\vspace{-0.5\normalbaselineskip}\centering}{\vspace{0.5\normalbaselineskip}\end{minipage}}

%% [Venn diagram environment]
%\newenvironment{venn2}
%{\begin{tikzpicture} [every pin/.style={text=black, text opacity=1.0, pin distance=0.5cm, pin edge={black!60, semithick}},
%% define a new style 'venn'
%venn/.style={circle, draw=black!60, semithick, minimum size = 4cm}]
%
%% create circle and give it external (pin) label
%\node[venn] (X) at (-1,0) [pin={150:$H[X]$}] {};
%\node[venn] (Y) at (1,0) [pin={30:$H[Y]$}] {};
%
%% place labels of the atoms by hand
%\node at (-1.9,0) {$H[X|Y]$};
%\node at (1.9,0) {$H[Y|X]$};
%\node at (0,0) {$I[X;Y]$};}
%{\end{tikzpicture}}

%\newcommand{\wrapmath}[1]{\begin{wrap}\begin{tikzpicture}[every node/.style={inner ysep=0ex, inner xsep=0em}]\node[] {$\displaystyle\begin{aligned} #1\end{aligned}$};\end{tikzpicture}\end{wrap}}

\renewenvironment{abstract}{\chapter*{Abstract}}{}
\renewcommand{\bibname}{Bibliography}
\renewcommand{\contentsname}{Table of Contents}

\makeatletter
\renewcommand{\@biblabel}[1]{\textsc{#1}}
\makeatother

% [ FONT SETTINGS ]

\usepackage[tbtags, intlimits, namelimits]{amsmath}
\usepackage[adobe-utopia]{mathdesign}

\DeclareSymbolFont{pazomath}{OMS}{zplm}{m}{n}
\DeclareSymbolFontAlphabet{\mathcal}{pazomath}
\SetMathAlphabet\mathcal{bold}{OMS}{zplm}{b}{n}

\SetSymbolFont{largesymbols}{normal}{OMX}{zplm}{m}{n}
\SetSymbolFont{largesymbols}{bold}{OMX}{zplm}{m}{n}
\SetSymbolFont{symbols}{normal}{OMS}{zplm}{m}{n}
\SetSymbolFont{symbols}{bold}{OMS}{zplm}{b}{n}

\renewcommand{\sfdefault}{phv}
\renewcommand{\ttdefault}{fvm}

\widowpenalty 8000
\clubpenalty  8000

% [ PAGE LAYOUT ]
\usepackage{geometry}
\geometry{lmargin = 1.5in}
\geometry{rmargin = 1.0in}
\geometry{tmargin = 1.0in}
\geometry{bmargin = 1.0in}

% [ PDF SETTINGS ]

\usepackage[final]{hyperref}
\hypersetup{breaklinks  = true}
\hypersetup{colorlinks  = true}
\hypersetup{linktocpage = false}
\hypersetup{linkcolor   = blue}
\hypersetup{citecolor   = green}
\hypersetup{urlcolor    = black}
\hypersetup{plainpages  = false}
\hypersetup{pageanchor  = true}
\hypersetup{pdfauthor   = {\PHDAUTHOR}}
\hypersetup{pdftitle    = {\PHDTITLE}}
\hypersetup{pdfsubject  = {Dissertation, \PHDSCHOOL}}
\urlstyle{same}

% [ LETTER SPACING ]

\usepackage[final]{microtype}
\microtypesetup{protrusion=compatibility}
\microtypesetup{expansion=false}

\newcommand{\upper}[1]{\MakeUppercase{#1}}
\let\lsscshape\scshape

\ifcase\pdfoutput\else\microtypesetup{letterspace=15}
\renewcommand{\scshape}{\lsscshape\lsstyle}
\renewcommand{\upper}[1]{\textls[50]{\MakeUppercase{#1}}}\fi

% [ LINE SPACING ]

\usepackage[doublespacing]{setspace}
\renewcommand{\displayskipstretch}{0.0}

\setlength{\parskip   }{0em}
\setlength{\parindent }{2em}

% [ TABLE FORMATTING ]

\usepackage{booktabs}
\setlength{\heavyrulewidth}{1.5\arrayrulewidth}
\setlength{\lightrulewidth}{1.0\arrayrulewidth}
\setlength{\doublerulesep }{2.0\arrayrulewidth}

% [ SECTION FORMATTING ]

\usepackage[largestsep,nobottomtitles*]{titlesec}
\renewcommand{\bottomtitlespace}{0.75in}

\titleformat{\chapter}[display]{\bfseries\huge\singlespacing}{\filleft\textsc{\LARGE \chaptertitlename\ \thechapter}}{-0.2ex}{\titlerule[3pt]\vspace{0.2ex}}[]

\titleformat{\section}{\LARGE}{\S\thesection\hspace{0.5em}}{0ex}{}
\titleformat{\subsection}{\Large}{\S\thesubsection\hspace{0.5em}}{0ex}{}
\titleformat{\subsubsection}{\large}{\thesubsubsection\hspace{0.5em}}{0ex}{}

\titlespacing*{\chapter}{0em}{6ex}{4ex plus 2ex minus 0ex}
\titlespacing*{\section}{0em}{2ex plus 3ex minus 1ex}{0.5ex plus 0.5ex minus 0.5ex}
\titlespacing*{\subsection}{0ex}{2ex plus 3ex minus 1ex}{0ex}
\titlespacing*{\subsubsection}{0ex}{2ex plus 0ex minus 1ex}{0ex}

% [ HEADER SETTINGS ]

\usepackage{fancyhdr}

\setlength{\headheight}{\normalbaselineskip}
\setlength{\footskip  }{0.5in}
\setlength{\headsep   }{0.5in-\headheight}

\fancyheadoffset[R]{0.5in}
\renewcommand{\headrulewidth}{0pt}
\renewcommand{\footrulewidth}{0pt}

\newcommand{\pagebox}{\parbox[r][\headheight][t]{0.5in}{\hspace\fill\thepage}}

\newcommand{\prelimheaders}{\ifx\prelim\undefined\renewcommand{\thepage}{\textit{\roman{page}}}\fancypagestyle{plain}{\fancyhf{}\fancyfoot[L]{\makebox[\textwidth-0.5in]{\thepage}}}\pagestyle{plain}\def\prelim{}\fi}

\newcommand{\normalheaders}{\renewcommand{\thepage}{\arabic{page}}\fancypagestyle{plain}{\fancyhf{}\fancyhead[R]{\pagebox}}\pagestyle{plain}}

\normalheaders{}

% [ CUSTOM COMMANDS ]

\newcommand{\signaturebox}[1]{\multicolumn{1}{p{4in}}{\vspace{3ex}}\\\midrule #1\\}

%\input{../includes/cmechabbrev}

% [some math stuff - maybe stick in sep file]
\usepackage{amsthm}
\usepackage{amscd}
\theoremstyle{plain}    \newtheorem{Lem}{Lemma}
\theoremstyle{plain}    \newtheorem*{ProLem}{Proof}
\theoremstyle{plain} 	\newtheorem{Cor}{Corollary}
\theoremstyle{plain} 	\newtheorem*{ProCor}{Proof}
\theoremstyle{plain} 	\newtheorem{The}{Theorem}
\theoremstyle{plain} 	\newtheorem*{ProThe}{Proof}
\theoremstyle{plain} 	\newtheorem{Prop}{Proposition}
\theoremstyle{plain} 	\newtheorem*{ProProp}{Proof}
\theoremstyle{plain} 	\newtheorem*{Conj}{Conjecture}
\theoremstyle{plain}	\newtheorem*{Rem}{Remark}
\theoremstyle{plain}	\newtheorem*{Def}{Definition} 
\theoremstyle{plain}	\newtheorem*{Not}{Notation}

% [uniform figure scaling - maybe this is not a good idea]
\def\figscale{.7}
\def\lscale{1.0}

% [FIX ME! - red makes it easier to spot]
\newcommand{\FIX}[1]{\textbf{\textcolor{red}{#1}}}


    \begin{document}
    \def\noprelim{}
\else
    % already included once
    % input post files

    \singlespacing
    \bibliographystyle{../bibliography/expanded}
    \bibliography{../bibliography/references}

    \end{document}
\fi\fi
% (place at start and end of chapter file)

\ifx\noprelim\undefined
    % first time included
    % input preamble files
    % [ USER VARIABLES ]

\def\PHDTITLE {Extensions of the Theory of Computational Mechanics}
\def\PHDAUTHOR{Evan Klose Friis}
\def\PHDSCHOOL{University of California, Davis}

\def\PHDMONTH {June}
\def\PHDYEAR  {2011}
\def\PHDDEPT {Physics}

\def\BSSCHOOL {University of California at San Diego}
\def\BSYEAR   {2005}

\def\PHDCOMMITTEEA{Professor John Conway}
\def\PHDCOMMITTEEB{Professor Robin Erbacher}
\def\PHDCOMMITTEEC{Professor Mani Tripathi}

% [ GLOBAL SETUP ]

\documentclass[letterpaper,oneside,11pt]{report}

\usepackage{calc}
\usepackage{breakcites}
\usepackage[newcommands]{ragged2e}

\usepackage[pdftex]{graphicx}
\usepackage{epstopdf}

%\usepackage{tikz}
%\usetikzlibrary{positioning} % [right=of ...]
%\usetikzlibrary{fit} % [fit= ...]

%\pgfdeclarelayer{background layer}
%\pgfdeclarelayer{foreground layer}
%\pgfsetlayers{background layer,main,foreground layer}

%\newenvironment{wrap}{\noindent\begin{minipage}[t]{\linewidth}\vspace{-0.5\normalbaselineskip}\centering}{\vspace{0.5\normalbaselineskip}\end{minipage}}

%% [Venn diagram environment]
%\newenvironment{venn2}
%{\begin{tikzpicture} [every pin/.style={text=black, text opacity=1.0, pin distance=0.5cm, pin edge={black!60, semithick}},
%% define a new style 'venn'
%venn/.style={circle, draw=black!60, semithick, minimum size = 4cm}]
%
%% create circle and give it external (pin) label
%\node[venn] (X) at (-1,0) [pin={150:$H[X]$}] {};
%\node[venn] (Y) at (1,0) [pin={30:$H[Y]$}] {};
%
%% place labels of the atoms by hand
%\node at (-1.9,0) {$H[X|Y]$};
%\node at (1.9,0) {$H[Y|X]$};
%\node at (0,0) {$I[X;Y]$};}
%{\end{tikzpicture}}

%\newcommand{\wrapmath}[1]{\begin{wrap}\begin{tikzpicture}[every node/.style={inner ysep=0ex, inner xsep=0em}]\node[] {$\displaystyle\begin{aligned} #1\end{aligned}$};\end{tikzpicture}\end{wrap}}

\renewenvironment{abstract}{\chapter*{Abstract}}{}
\renewcommand{\bibname}{Bibliography}
\renewcommand{\contentsname}{Table of Contents}

\makeatletter
\renewcommand{\@biblabel}[1]{\textsc{#1}}
\makeatother

% [ FONT SETTINGS ]

\usepackage[tbtags, intlimits, namelimits]{amsmath}
\usepackage[adobe-utopia]{mathdesign}

\DeclareSymbolFont{pazomath}{OMS}{zplm}{m}{n}
\DeclareSymbolFontAlphabet{\mathcal}{pazomath}
\SetMathAlphabet\mathcal{bold}{OMS}{zplm}{b}{n}

\SetSymbolFont{largesymbols}{normal}{OMX}{zplm}{m}{n}
\SetSymbolFont{largesymbols}{bold}{OMX}{zplm}{m}{n}
\SetSymbolFont{symbols}{normal}{OMS}{zplm}{m}{n}
\SetSymbolFont{symbols}{bold}{OMS}{zplm}{b}{n}

\renewcommand{\sfdefault}{phv}
\renewcommand{\ttdefault}{fvm}

\widowpenalty 8000
\clubpenalty  8000

% [ PAGE LAYOUT ]
\usepackage{geometry}
\geometry{lmargin = 1.5in}
\geometry{rmargin = 1.0in}
\geometry{tmargin = 1.0in}
\geometry{bmargin = 1.0in}

% [ PDF SETTINGS ]

\usepackage[final]{hyperref}
\hypersetup{breaklinks  = true}
\hypersetup{colorlinks  = true}
\hypersetup{linktocpage = false}
\hypersetup{linkcolor   = blue}
\hypersetup{citecolor   = green}
\hypersetup{urlcolor    = black}
\hypersetup{plainpages  = false}
\hypersetup{pageanchor  = true}
\hypersetup{pdfauthor   = {\PHDAUTHOR}}
\hypersetup{pdftitle    = {\PHDTITLE}}
\hypersetup{pdfsubject  = {Dissertation, \PHDSCHOOL}}
\urlstyle{same}

% [ LETTER SPACING ]

\usepackage[final]{microtype}
\microtypesetup{protrusion=compatibility}
\microtypesetup{expansion=false}

\newcommand{\upper}[1]{\MakeUppercase{#1}}
\let\lsscshape\scshape

\ifcase\pdfoutput\else\microtypesetup{letterspace=15}
\renewcommand{\scshape}{\lsscshape\lsstyle}
\renewcommand{\upper}[1]{\textls[50]{\MakeUppercase{#1}}}\fi

% [ LINE SPACING ]

\usepackage[doublespacing]{setspace}
\renewcommand{\displayskipstretch}{0.0}

\setlength{\parskip   }{0em}
\setlength{\parindent }{2em}

% [ TABLE FORMATTING ]

\usepackage{booktabs}
\setlength{\heavyrulewidth}{1.5\arrayrulewidth}
\setlength{\lightrulewidth}{1.0\arrayrulewidth}
\setlength{\doublerulesep }{2.0\arrayrulewidth}

% [ SECTION FORMATTING ]

\usepackage[largestsep,nobottomtitles*]{titlesec}
\renewcommand{\bottomtitlespace}{0.75in}

\titleformat{\chapter}[display]{\bfseries\huge\singlespacing}{\filleft\textsc{\LARGE \chaptertitlename\ \thechapter}}{-0.2ex}{\titlerule[3pt]\vspace{0.2ex}}[]

\titleformat{\section}{\LARGE}{\S\thesection\hspace{0.5em}}{0ex}{}
\titleformat{\subsection}{\Large}{\S\thesubsection\hspace{0.5em}}{0ex}{}
\titleformat{\subsubsection}{\large}{\thesubsubsection\hspace{0.5em}}{0ex}{}

\titlespacing*{\chapter}{0em}{6ex}{4ex plus 2ex minus 0ex}
\titlespacing*{\section}{0em}{2ex plus 3ex minus 1ex}{0.5ex plus 0.5ex minus 0.5ex}
\titlespacing*{\subsection}{0ex}{2ex plus 3ex minus 1ex}{0ex}
\titlespacing*{\subsubsection}{0ex}{2ex plus 0ex minus 1ex}{0ex}

% [ HEADER SETTINGS ]

\usepackage{fancyhdr}

\setlength{\headheight}{\normalbaselineskip}
\setlength{\footskip  }{0.5in}
\setlength{\headsep   }{0.5in-\headheight}

\fancyheadoffset[R]{0.5in}
\renewcommand{\headrulewidth}{0pt}
\renewcommand{\footrulewidth}{0pt}

\newcommand{\pagebox}{\parbox[r][\headheight][t]{0.5in}{\hspace\fill\thepage}}

\newcommand{\prelimheaders}{\ifx\prelim\undefined\renewcommand{\thepage}{\textit{\roman{page}}}\fancypagestyle{plain}{\fancyhf{}\fancyfoot[L]{\makebox[\textwidth-0.5in]{\thepage}}}\pagestyle{plain}\def\prelim{}\fi}

\newcommand{\normalheaders}{\renewcommand{\thepage}{\arabic{page}}\fancypagestyle{plain}{\fancyhf{}\fancyhead[R]{\pagebox}}\pagestyle{plain}}

\normalheaders{}

% [ CUSTOM COMMANDS ]

\newcommand{\signaturebox}[1]{\multicolumn{1}{p{4in}}{\vspace{3ex}}\\\midrule #1\\}

%%%% macros fro standard references
%\eqref provided by amsmath
\newcommand{\figref}[1]{Fig.~\ref{#1}}
\newcommand{\tableref}[1]{Table~\ref{#1}}
\newcommand{\refcite}[1]{Ref.~\cite{#1}}

% Abbreviations from CMPPSS:

\newcommand{\eM}     {\mbox{$\epsilon$-machine}}
\newcommand{\eMs}    {\mbox{$\epsilon$-machines}}
\newcommand{\EM}     {\mbox{$\epsilon$-Machine}}
\newcommand{\EMs}    {\mbox{$\epsilon$-Machines}}
\newcommand{\eT}     {\mbox{$\epsilon$-transducer}}
\newcommand{\eTs}    {\mbox{$\epsilon$-transducers}}
\newcommand{\ET}     {\mbox{$\epsilon$-Transducer}}
\newcommand{\ETs}    {\mbox{$\epsilon$-Transducers}}

% Processes and sequences

\newcommand{\Process}{\mathcal{P}}

\newcommand{\ProbMach}{\Prob_{\mathrm{M}}}
\newcommand{\Lmax}   { {L_{\mathrm{max}}}}
\newcommand{\MeasAlphabet}	{\mathcal{A}}
% Original
%\newcommand{\MeasSymbol}   { {S} }
%\newcommand{\meassymbol}   { {s} }
% New symbol
\newcommand{\MeasSymbol}   { {X} }
\newcommand{\meassymbol}   { {x} }
\newcommand{\BiInfinity}	{ \overleftrightarrow {\MeasSymbol} }
\newcommand{\biinfinity}	{ \overleftrightarrow {\meassymbol} }
\newcommand{\Past}	{ \overleftarrow {\MeasSymbol} }
\newcommand{\past}	{ {\overleftarrow {\meassymbol}} }
\newcommand{\pastprime}	{ {\past}^{\prime}}
\newcommand{\Future}	{ \overrightarrow{\MeasSymbol} }
\newcommand{\future}	{ \overrightarrow{\meassymbol} }
\newcommand{\futureprime}	{ {\future}^{\prime}}
\newcommand{\PastPrime}	{ {\Past}^{\prime}}
\newcommand{\FuturePrime}	{ {\overrightarrow{\meassymbol}}^\prime }
\newcommand{\PastDblPrime}	{ {\overleftarrow{\meassymbol}}^{\prime\prime} }
\newcommand{\FutureDblPrime}	{ {\overrightarrow{\meassymbol}}^{\prime\prime} }
\newcommand{\pastL}	{ {\overleftarrow {\meassymbol}}{}^L }
\newcommand{\PastL}	{ {\overleftarrow {\MeasSymbol}}{}^L }
\newcommand{\PastLt}	{ {\overleftarrow {\MeasSymbol}}_t^L }
\newcommand{\PastLLessOne}	{ {\overleftarrow {\MeasSymbol}}^{L-1} }
\newcommand{\futureL}	{ {\overrightarrow{\meassymbol}}{}^L }
\newcommand{\FutureL}	{ {\overrightarrow{\MeasSymbol}}{}^L }
\newcommand{\FutureLt}	{ {\overrightarrow{\MeasSymbol}}_t^L }
\newcommand{\FutureLLessOne}	{ {\overrightarrow{\MeasSymbol}}^{L-1} }
\newcommand{\pastLprime}	{ {\overleftarrow {\meassymbol}}^{L^\prime} }
\newcommand{\futureLprime}	{ {\overrightarrow{\meassymbol}}^{L^\prime} }
\newcommand{\AllPasts}	{ { \overleftarrow {\rm {\bf \MeasSymbol}} } }
\newcommand{\AllFutures}	{ \overrightarrow {\rm {\bf \MeasSymbol}} }
\newcommand{\FutureSet}	{ \overrightarrow{\bf \MeasSymbol}}

% Causal states and epsilon-machines
\newcommand{\CausalState}	{ \mathcal{S} }
\newcommand{\CausalStatePrime}	{ {\CausalState}^{\prime}}
\newcommand{\causalstate}	{ \sigma }
\newcommand{\CausalStateSet}	{ \boldsymbol{\CausalState} }
\newcommand{\AlternateState}	{ \mathcal{R} }
\newcommand{\AlternateStatePrime}	{ {\cal R}^{\prime} }
\newcommand{\alternatestate}	{ \rho }
\newcommand{\alternatestateprime}	{ {\rho^{\prime}} }
\newcommand{\AlternateStateSet}	{ \boldsymbol{\AlternateState} }
\newcommand{\PrescientState}	{ \widehat{\AlternateState} }
\newcommand{\prescientstate}	{ \widehat{\alternatestate} }
\newcommand{\PrescientStateSet}	{ \boldsymbol{\PrescientState}}
\newcommand{\CausalEquivalence}	{ {\sim}_{\epsilon} }
\newcommand{\CausalEquivalenceNot}	{ {\not \sim}_{\epsilon}}

\newcommand{\NonCausalEquivalence}	{ {\sim}_{\eta} }
\newcommand{\NextObservable}	{ {\overrightarrow {\MeasSymbol}}^1 }
\newcommand{\LastObservable}	{ {\overleftarrow {\MeasSymbol}}^1 }
%\newcommand{\Prob}		{ {\rm P}}
\newcommand{\Prob}      {\Pr} % use standard command
\newcommand{\ProbAnd}	{ {,\;} }
\newcommand{\LLimit}	{ {L \rightarrow \infty}}
\newcommand{\Cmu}		{C_\mu}
\newcommand{\hmu}		{h_\mu}
\newcommand{\EE}		{{\bf E}}
\newcommand{\Measurable}{{\bf \mu}}

% Process Crypticity
\newcommand{\PC}		{\chi}
\newcommand{\FuturePC}		{\PC^+}
\newcommand{\PastPC}		{\PC^-}
% Causal Irreversibility
\newcommand{\CI}		{\Xi}
\newcommand{\ReverseMap}	{r}
\newcommand{\ForwardMap}	{f}

% Abbreviations from IB:
% None that aren't already in CMPPSS

% Abbreviations from Extensive Estimation:
\newcommand{\EstCausalState}	{\widehat{\CausalState}}
\newcommand{\estcausalstate}	{\widehat{\causalstate}}
\newcommand{\EstCausalStateSet}	{\boldsymbol{\EstCausalState}}
\newcommand{\EstCausalFunc}	{\widehat{\epsilon}}
\newcommand{\EstCmu}		{\widehat{\Cmu}}
\newcommand{\PastLOne}	{{\Past}^{L+1}}
\newcommand{\pastLOne}	{{\past}^{L+1}}

% Abbreviations from $\epsilon$-Transducers:
\newcommand{\InAlphabet}	{ \mathcal{A}}
\newcommand{\insymbol}		{ a}
\newcommand{\OutAlphabet}	{ \mathcal{B}}
\newcommand{\outsymbol}		{ b}
\newcommand{\InputSimple}	{ X}
\newcommand{\inputsimple}	{ x}
\newcommand{\BottleneckVar}	{\tilde{\InputSimple}}
\newcommand{\bottleneckvar}	{\tilde{\inputsimple}}
\newcommand{\InputSpace}	{ \mathbf{\InputSimple}}
\newcommand{\InputBi}	{ \overleftrightarrow {\InputSimple} }
\newcommand{\inputbi}	{ \overleftrightarrow {\inputsimple} }
\newcommand{\InputPast}	{ \overleftarrow {\InputSimple} }
\newcommand{\inputpast}	{ \overleftarrow {\inputsimple} }
\newcommand{\InputFuture}	{ \overrightarrow {\InputSimple} }
\newcommand{\inputfuture}	{ \overrightarrow {\inputsimple} }
\newcommand{\NextInput}	{ {{\InputFuture}^{1}}}
\newcommand{\NextOutput}	{ {\OutputFuture}^{1}}
\newcommand{\OutputSimple}	{ Y}
\newcommand{\outputsimple}	{ y}
\newcommand{\OutputSpace}	{ \mathbf{\OutputSimple}}
\newcommand{\OutputBi}	{ \overleftrightarrow{\OutputSimple} }
\newcommand{\outputbi}	{ \overleftrightarrow{\outputsimple} }
\newcommand{\OutputPast}	{ \overleftarrow{\OutputSimple} }
\newcommand{\outputpast}	{ \overleftarrow{\outputsimple} }
\newcommand{\OutputFuture}	{ \overrightarrow{\OutputSimple} }
\newcommand{\outputfuture}	{ \overrightarrow{\outputsimple} }
\newcommand{\OutputL}	{ {\OutputFuture}^L}
\newcommand{\outputL}	{ {\outputfuture}^L}
\newcommand{\InputLLessOne}	{ {\InputFuture}^{L-1}}
\newcommand{\inputLlessone}	{ {\inputufutre}^{L-1}}
\newcommand{\OutputPastLLessOne}	{{\OutputPast}^{L-1}_{-1}}
\newcommand{\outputpastLlessone}	{{\outputpast}^{L-1}}
\newcommand{\OutputPastLessOne}	{{\OutputPast}_{-1}}
\newcommand{\outputpastlessone}	{{\outputpast}_{-1}}
\newcommand{\OutputPastL}	{{\OutputPast}^{L}}
\newcommand{\OutputLPlusOne}	{ {\OutputFuture}^{L+1}}
\newcommand{\outputLplusone}	{ {\outputfutre}^{L+1}}
\newcommand{\InputPastL}	{{\InputPast}^{L}}
\newcommand{\inputpastL}	{{\inputpast}^{L}}
\newcommand{\JointPast}	{{(\InputPast,\OutputPast)}}
\newcommand{\jointpast}	{{(\inputpast,\outputpast)}}
\newcommand{\jointpastone}	{{(\inputpast_1,\outputpast_1)}}
\newcommand{\jointpasttwo}	{{(\inputpast_2,\outputpast_2)}}
\newcommand{\jointpastprime} {{({\inputpast}^{\prime},{\outputpast}^{\prime})}}
\newcommand{\NextJoint}	{{(\NextInput,\NextOutput)}}
\newcommand{\nextjoint}	{{(\insymbol,\outsymbol)}}
\newcommand{\AllInputPasts}	{ { \overleftarrow {\rm \InputSpace}}}
\newcommand{\AllOutputPasts}	{ {\overleftarrow {\rm \OutputSpace}}}
\newcommand{\DetCausalState}	{ {{\cal S}_D }}
\newcommand{\detcausalstate}	{ {{\sigma}_D} }
\newcommand{\DetCausalStateSet}	{ \boldsymbol{{\CausalState}_D}}
\newcommand{\DetCausalEquivalence}	{ {\sim}_{{\epsilon}_{D}}}
\newcommand{\PrescientEquivalence}	{ {\sim}_{\widehat{\eta}}}
\newcommand{\FeedbackCausalState}	{ \mathcal{F}}
\newcommand{\feedbackcausalstate}	{ \phi}
\newcommand{\FeedbackCausalStateSet}	{ \mathbf{\FeedbackCausalState}}
\newcommand{\JointCausalState}		{ \mathcal{J}}
\newcommand{\JointCausalStateSet}	{ \mathbf{\JointCausalState}}
\newcommand{\UtilityFunctional}	{ {\mathcal{L}}}
\newcommand{\NatureState}	{ {\Omega}}
\newcommand{\naturestate}	{ {\omega}}
\newcommand{\NatureStateSpace}	{ {\mathbf{\NatureState}}}
\newcommand{\AnAction}	{ {A}}
\newcommand{\anaction}	{ {a}}
\newcommand{\ActionSpace}	{ {\mathbf{\AnAction}}}

% Abbreviations from RURO:
\newcommand{\InfoGain}[2] { \mathcal{D} \left( {#1} || {#2} \right) }

% Abbreviations from Upper Bound:
\newcommand{\lcm}	{{\rm lcm}}
% Double-check that this isn't in the math set already!

% Abbreviations from Emergence in Space
\newcommand{\ProcessAlphabet}	{\MeasAlphabet}
\newcommand{\ProbEst}			{ {\widehat{\Prob}_N}}
\newcommand{\STRegion}			{ {\mathrm K}}
\newcommand{\STRegionVariable}		{ K}
\newcommand{\stregionvariable}		{ k}
\newcommand{\GlobalPast}		{ \overleftarrow{G}} 
\newcommand{\globalpast}		{ \overleftarrow{g}} 
\newcommand{\GlobalFuture}		{ \overrightarrow{G}}
\newcommand{\globalfuture}		{ \overrightarrow{g}}
\newcommand{\GlobalState}		{ \mathcal{G}}
\newcommand{\globalstate}		{ \gamma}
\newcommand{\GlobalStateSet}		{ {\mathbf \GlobalState}}
\newcommand{\LocalPast}			{ \overleftarrow{L}} 
\newcommand{\localpast}			{ \overleftarrow{l}}
\newcommand{\AllLocalPasts}		{ \mathbf{\LocalPast}}
\newcommand{\LocalPastRegion}		{ \overleftarrow{\mathrm L}}
\newcommand{\LocalFuture}		{ \overrightarrow{L}}
\newcommand{\localfuture}		{ \overrightarrow{l}}
\newcommand{\LocalFutureRegion}		{ \overrightarrow{\mathrm L}}
\newcommand{\LocalState}		{ \mathcal{L}}
\newcommand{\localstate}		{ \lambda}
\newcommand{\LocalStateSet}		{ {\mathbf \LocalState}}
\newcommand{\PatchPast}			{ \overleftarrow{P}}
\newcommand{\patchpast}			{ \overleftarrow{p}}
\newcommand{\PatchPastRegion}		{ \overleftarrow{\mathrm P}}
\newcommand{\PatchFuture}		{ \overrightarrow{P}}
\newcommand{\patchfuture}		{ \overrightarrow{p}}
\newcommand{\PatchFutureRegion}		{ \overrightarrow{\mathrm P}}
\newcommand{\PatchState}		{ \mathcal{P}}
\newcommand{\patchstate}		{ \pi}
\newcommand{\PatchStateSet}		{ {\mathbf \PatchState}}
\newcommand{\LocalStatesInPatch}	{\vec{\LocalState}}
\newcommand{\localstatesinpatch}	{\vec{\localstate}}
\newcommand{\PointInstantX}		{ {\mathbf x}}
% Galles's original LaTeX for the cond. indep. symbol follows:
\newcommand{\compos}{\mbox{$~\underline{~\parallel~}~$}}
\newcommand{\ncompos}{\not\hspace{-.15in}\compos}
\newcommand{\indep}			{ \rotatebox{90}{$\models$}}
\newcommand{\nindep}	{\not\hspace{-.05in}\indep}
\newcommand{\LocalEE}	{{\EE}^{loc}}
\newcommand{\EEDensity}	{\overline{\LocalEE}}
\newcommand{\LocalCmu}	{{\Cmu}^{loc}}
\newcommand{\CmuDensity}	{\overline{\LocalCmu}}

%%%%%%%%%%% added by sasa
\newcommand{\FinPast}[1]	{ \overleftarrow {\MeasSymbol} \stackrel{{#1}}{}}
\newcommand{\finpast}[1]  	{ \overleftarrow {\meassymbol}  \stackrel{{#1}}{}}
\newcommand{\FinFuture}[1]		{ \overrightarrow{\MeasSymbol} \stackrel{{#1}}{}}
\newcommand{\finfuture}[1]		{ \overrightarrow{\meassymbol} \stackrel{{#1}}{}}

\newcommand{\Partition}	{ \AlternateState }
\newcommand{\partitionstate}	{ \alternatestate }
\newcommand{\PartitionSet}	{ \AlternateStateSet }
\newcommand{\Fdet}   { F_{\rm det} }

\newcommand{\Dkl}[2] { D_{\rm KL} \left( {#1} || {#2} \right) }

\newcommand{\Period}	{p}

% To take into account time direction
\newcommand{\forward}{+}
\newcommand{\reverse}{-}
%\newcommand{\forwardreverse}{\:\!\diamond} % \pm
\newcommand{\forwardreverse}{\pm} % \pm
\newcommand{\FutureProcess}	{ {\Process}^{\forward} }
\newcommand{\PastProcess}	{ {\Process}^{\reverse} }
\newcommand{\FutureCausalState}	{ {\CausalState}^{\forward} }
\newcommand{\futurecausalstate}	{ \sigma^{\forward} }
\newcommand{\altfuturecausalstate}	{ \sigma^{\forward\prime} }
\newcommand{\PastCausalState}	{ {\CausalState}^{\reverse} }
\newcommand{\pastcausalstate}	{ \sigma^{\reverse} }
\newcommand{\BiCausalState}		{ {\CausalState}^{\forwardreverse} }
\newcommand{\bicausalstate}		{ {\sigma}^{\forwardreverse} }
\newcommand{\FutureCausalStateSet}	{ {\CausalStateSet}^{\forward} }
\newcommand{\PastCausalStateSet}	{ {\CausalStateSet}^{\reverse} }
\newcommand{\BiCausalStateSet}	{ {\CausalStateSet}^{\forwardreverse} }
\newcommand{\eMachine}	{ M }
\newcommand{\FutureEM}	{ {\eMachine}^{\forward} }
\newcommand{\PastEM}	{ {\eMachine}^{\reverse} }
\newcommand{\BiEM}		{ {\eMachine}^{\forwardreverse} }
\newcommand{\BiEquiv}	{ {\sim}^{\forwardreverse} }
\newcommand{\Futurehmu}	{ h_\mu^{\forward} }
\newcommand{\Pasthmu}	{ h_\mu^{\reverse} }
\newcommand{\FutureCmu}	{ C_\mu^{\forward} }
\newcommand{\PastCmu}	{ C_\mu^{\reverse} }
\newcommand{\BiCmu}		{ C_\mu^{\forwardreverse} }
\newcommand{\FutureEps}	{ \epsilon^{\forward} }
\newcommand{\PastEps}	{ \epsilon^{\reverse} }
\newcommand{\BiEps}	{ \epsilon^{\forwardreverse} }
\newcommand{\FutureSim}	{ \sim^{\forward} }
\newcommand{\PastSim}	{ \sim^{\reverse} }
% Used arrows for awhile, more or less confusing?
%\newcommand{\FutureCausalState}	{ \overrightarrow{\CausalState} }
%\newcommand{\PastCausalState}	{ \overleftarrow{\CausalState} }
%\newcommand{\eMachine}	{ M }
%\newcommand{\FutureEM}	{ \overrightarrow{\eMachine} }
%\newcommand{\PastEM}	{ \overleftarrow{\eMachine} }
%\newcommand{\FutureCmu}	{ \overrightarrow{\Cmu} }
%\newcommand{\PastCmu}	{ \overleftarrow{\Cmu} }

%% time-reversing and mixed state presentation operators
\newcommand{\TR}{\mathcal{T}}
\newcommand{\MSP}{\mathcal{U}}
\newcommand{\one}{\mathbf{1}}

%% (cje)
%% Provide a command \ifpm which is true when \pm 
%% is meant to be understood as "+ or -". This is
%% different from the usage in TBA.
\newif\ifpm 
\edef\tempa{\forwardreverse}
\edef\tempb{\pm}
\ifx\tempa\tempb
   \pmfalse
\else
   \pmtrue  
\fi





% [some math stuff - maybe stick in sep file]
\usepackage{amsthm}
\usepackage{amscd}
\theoremstyle{plain}    \newtheorem{Lem}{Lemma}
\theoremstyle{plain}    \newtheorem*{ProLem}{Proof}
\theoremstyle{plain} 	\newtheorem{Cor}{Corollary}
\theoremstyle{plain} 	\newtheorem*{ProCor}{Proof}
\theoremstyle{plain} 	\newtheorem{The}{Theorem}
\theoremstyle{plain} 	\newtheorem*{ProThe}{Proof}
\theoremstyle{plain} 	\newtheorem{Prop}{Proposition}
\theoremstyle{plain} 	\newtheorem*{ProProp}{Proof}
\theoremstyle{plain} 	\newtheorem*{Conj}{Conjecture}
\theoremstyle{plain}	\newtheorem*{Rem}{Remark}
\theoremstyle{plain}	\newtheorem*{Def}{Definition} 
\theoremstyle{plain}	\newtheorem*{Not}{Notation}

% [uniform figure scaling - maybe this is not a good idea]
\def\figscale{.7}
\def\lscale{1.0}

% [FIX ME! - red makes it easier to spot]
\newcommand{\FIX}[1]{\textbf{\textcolor{red}{#1}}}


    \begin{document}
    \def\noprelim{}
\else
    % already included once
    % input post files

    \singlespacing
    \bibliographystyle{../bibliography/expanded}
    \bibliography{../bibliography/references}

    \end{document}
\fi\fi

\chapter{Systematics and Limit Extraction} \label{ch:systematics} 
%
In this chapter we discuss the systematic uncertainties affecting the search for
the Higgs boson and the statistical techniques used to establish an upper limit
on the Higgs~$\to\TT$ branching ratio times cross section (\xbr).  The limit can
be interpreted as the largest\footnote{At some stated level of statistical
confidence; the convention for limits in experimental high energy physics
is 95\%.} signal presence that could exist in the data and still be consistent
with the null hypothesis.  The limit on \xbr is roughly independent of the
theoretical model\footnote{Provided that the width of the Higgs bosons in the
given model is smaller than the resolution of the SVfit mass resolution.}.  In
the conclusion, we will interpret the \xbr limit result in the context of the
MSSM theory.

Proper determination of systematic uncertainties is one of the most challenging
and important components in performing the measurement correctly.  A systematic
uncertainty is the effect of the uncertainty of some ancillary measurement (or
assumption) that is used in the computation of the final result.  An instructive
example of how a systematic uncertainty can affect the final result is a
counting experiment measuring the cross section of some signal particle in the
presence of background.  The formula for the cross section times the branching
fraction is
\begin{equation}
  \sigma \times BR = \frac{N_{sig}}{\mathcal{L} \cdot \mathcal {A} \cdot
  \epsilon} = \frac{N_{obs}-N_{bkg}}{\mathcal{L} \cdot \mathcal {A} \cdot
  \epsilon}, \label{eq:CrossSectionEquation}
\end{equation}
where $N_{obs}$ is the number of events observed in data, $N_{bkg}$ is the
estimated number of background events in the observed data sample, $\mathcal{L}$
is the integrated luminosity, and $\mathcal{A} \cdot \epsilon$ is the acceptance
times efficiency of the signal.  All of the quantities in
Equation~\ref{eq:CrossSectionEquation} (with the exception of the observed
count $N_{obs}$) have some uncertainty which will effect the final measurement.
Consider a situation where the expected number of background events is
determined by fitting some sideband spectrum, and the fitted result has some
error $\delta N_{bkg}$.  The total relative effect of this error can
be obtained by error propagation
\begin{equation}
  \frac{\delta (\sigma \times BR)}{\sigma \times BR} = \frac{\partial(\sigma
  \times BR)}{\partial N_{bkg}}  \frac{1}{\sigma \times BR} \delta N_{bkg} =
  \frac{-\delta N_{bkg}} {N_{obs}-N_{bkg}}.
  \label{eq:CrossSectionEquationBkgSystematicError}
\end{equation}
It is interesting to examine
Equation~\ref{eq:CrossSectionEquationBkgSystematicError} in two scenarios.  In
the limit that $N_{obs}$ is large compared to $N_{bkg}$, the effect of the error
on the background estimate $\delta N_{bkg}$ does not affect the final result.
In contrast, in a scenario when the data is dominated by background events, the
relative error on the signal measurement due to the background estimation
approaches infinity.  The sensitivity of a measurement to a systematic
uncertainty on a parameter depends on the context in which that parameter is
used.

Experimental systematic uncertainties relevant for MSSM Higgs $\to \TT$ signal
extraction presented in this thesis are classified in three categories:
normalization uncertainties on the signal, normalization uncertainties on
background contributions, and shape uncertainties.  Normalization uncertainties
on the signal are due to lepton reconstruction, identification, isolation and
trigger efficiencies.  These terms are equivalent to the efficiency $\epsilon$
and acceptance terms $\mathcal{A}$ of
Equation~\ref{eq:CrossSectionEquationBkgSystematicError} and affect the expected
yield of MSSM Higgs $\to \TT$ signal and of $Z \to \TT$ background events.  The
uncertainties on these effects are obtained by measuring the effect in data and
simulation, according to the procedures of Chapter~\ref{ch:corrections}, and
calculating a correction factor.  The uncertainty on this correction factor is
the systematic uncertainty.  The normalization uncertainties do not affect the
\emph{shapes} of visible and ``full'' invariant mass distributions which are
used to extract the MSSM Higgs $\to \TT$ signal contribution in the analyzed
dataset.  Uncertainties on the shapes of the distributions are described by
``morphing'' systematics.  These are are due to uncertainties on the
momentum/energy scale of identified electrons, muons, tau and other jets in the
event.  As the SVfit mass reconstruction algorithm uses the missing transverse
energy, the shape of the SVfit distribution is sensitive to systematic
uncertainties on the overall scale \MET measurement.  The ``morphing''
systematics affect the shapes of signal as well as background contributions.
Normalization uncertainties on background contributions are estimated from the
level of agreement between data and Monte Carlo simulation in background
dominated control regions.

\section{Signal normalization uncertainties}

The signal normalization uncertainties are due to imperfect knowledge of how
improperly modeled effects in the detector affect our ``acceptance'' model, or
the probability that a given signal event will pass one of the selections
(detailed in Chapter~\ref{ch:selections}).  The general procedure to quantify
these uncertainties is to measure the effect in some control region in both the
data and Monte Carlo.  The ratio of data to Monte Carlo then gives a correction
factor which is applied to the simulation.  An uncertainty on the measurement of
the effect in control region (in data, simulation, or both) is then taken as the
systematic uncertainties.  The signal normalization uncertainties affecting this
analysis on muon trigger, reconstruction, identification and isolation
efficiencies are taken from the tag and probe analysis of $Z \to \MM$ events
presented in Section~\ref{sec:ZmumuTagAndProbe}.  The uncertainty on the tau
reconstruction and identification efficiency is taken to be $23\%$.  The tau
identification uncertainty measurement is discussed briefly
in~\ref{sec:HadTauIdEff}.  The dependency of the Higgs signal extraction on the
tau identification efficiency has been studied, the result being that
uncertainties on the tau identification efficiency affect the limit on
cross--section times branching ratio for MSSM Higgs $\to \TT$ production by a
few percent only.  An uncertainty of $11\%$ is attributed to the luminosity
measurement.

\section{Background normalization uncertainties}

Uncertainties on the normalization of background processes are obtained from the
study of background enriched control regions presented in
Chapter~\ref{ch:backgrounds}.  The main non--$Z \to \TT$ background to
the analysis is due to QCD multi--jet and $W$ + jets events.  These backgrounds
are produced copiously enough for the backgrounds to be studied in control
regions dominated by a single background process with a purity exceeding $90\%$
and an event statistics exceeding the expected contribution of that background
to the analysis by more than one order of magnitude.  Both backgrounds are found
to be well modeled by the Monte Carlo simulation.  An uncertainty of $10\%$ is
attributed to the contribution of QCD and $W$ + jet backgrounds to the analysis.
The cross--section for $\ttbar + jets$ production makes it difficult to select a
high purity sample of $\ttbar + jet$ events of high event statistics.  From the
study of the $19$ events selected in the $\ttbar + jets$ background enriched
control sample we assume an uncertainty on the $\ttbar + jets$ background
contribution in the analysis of $30\%$.  The $Z \to \MM$ background has been
studied with large statistical precision in two separate control regions,
dominated by events in which the reconstructed tau--jet candidate is either due
to a misidentified quark or gluon jet or due to a misidentified muon.  Good
agreement between data and Monte Carlo simulation is found in both cases.
Sizeable uncertainties on the $Z \to \MM$ background contribution arise due to
the extrapolation from the background enriched control regions to the data
sample considered in the analysis, however: the contribution of $Z \to \MM$
background events to the analysis is due to events in which one of the two muons
produced in the $Z$ decay either escapes detection or fakes the signature of a
hadronic tau decay.  Both cases may be difficult to model precisely in the Monte
Carlo simulation.  The non--observation of a $Z$ mass peak in the mu + tau
visible mass distribution studied with the fake--rate method on the other hand
sets a limit on possible contributions from $Z \to \MM$ background events.
Conservatively, we assume an uncertainty of $100\%$ on both types of $Z \to \MM$
background contributions.


\section{Shape uncertainties}

Shape uncertainties on the distributions of visible and ``full'' invariant mass
reconstructed by the SVfit algorithm are estimated by varying the electron
energy and muon momentum scale, the energy scale of tau--jets and other jets in
the event and varying the missing transverse energy in Monte Carlo simulated
events.  After each variation the complete event is re--reconstructed and passed
through the event selection.  Shifted visible and ``full'' invariant mass shapes
are obtained for each variation from the events passing all event selection
criteria.  The difference between shifted shapes and the ``nominal'' shapes
obtained from Monte Carlo simulated events with no variation of energy or
momentum scale or of the missing transverse energy applied is then taken as
shape uncertainty.

The systematic uncertainties on the muon and tau energy scales have been
provided by the muon and tau Physics Object Groups and are described in
Section~\ref{sec:MuonTauMomentumScale}.
The modelling of missing transverse energy in different types of background
events has been studied in the background enriched control regions described in
Chapter~\ref{ch:backgrounds}.  No significant deviations between data and Monte
Carlo simulation have been found ({\it cf.}\ control plots in the appendix).
Uncertainties due to missing transverse energy are estimated by varying
parameters of $Z$--recoil corrections within the uncertainties obtained when
fitting (see Section~\ref{sec:ZRecoilCorr}) the $Z$--recoil correction
parameters in simulated $Z \to \MM$ events versus $Z \to \MM$ events selected in
data.


\section{Theory uncertainties}

The signal and background normalization as well as the shape uncertainties are
all experimental uncertainties in nature.  Additional theoretical uncertainties
arise from imprecise knowledge of parton--distribution functions (PDFs) and of
the exact dependency of signal cross--sections and branching ratios on
$tan\beta$ and $m_A$.

The uncertainties on the signal acceptance due to PDF uncertainties are
estimated using tools developed by the EWK
group~\cite{CMS_EWK_pdfUncertaintyTools}.  The acceptance is computed with
respect to MSSM Higgs $\to \TT$ decays that have electrons of $P_{T}^{e} >
15$~\GeV and $\left| \eta_{e} \right| < 2.1$, muons of $P_{T}^{\mu} > 15$~\GeV
and $\left| \eta_{\mu} \right| < 2.1$, jets produced in hadronic tau decays with
visible $P_{T}^{vis} > 20$~\GeV and $\left| \eta_{vis} \right| < 2.3$ on
generator level, depending on the analysis channel considered.  Acceptance
values are computed for the central value and $44$ eigenvectors of the CTEQ66
PDF set~\cite{CTEQpdfSet}.  The systematic uncertainty on the signal acceptance
is computed following the PDF4LHC
recommendations~\cite{pdfAccSys01,pdfAccSys02}.


The effect of Monte Carlo normalization, shape and theory uncertainties on the
signal efficiency times acceptance is summarized in
table~\ref{tab:ExpUncertainties}.

\begin{table}[t]
\begin{center}
\tablesize
\begin{tabular}{|l|c|}
\hline
Source & Effect \\
\hline
\hline
\multicolumn{2}{|c|}{Normalization uncertainties} \\
\hline
Trigger                         & $0.981 \pm 0.006$ \\
Muon identification             & $1.001 \pm 0.001$ \\
Muon isolation                  & $0.984 \pm 0.006$ \\
Tau--jet identification         & $1.00  \pm 0.30$ \\
\hline
\hline
\multicolumn{2}{|c|}{Shape uncertainties} \\
\hline
Muon momentum scale             & $\ll 1\%$ \\
Tau--jet energy scale           & $1 - 4\%^{1}$ \\
Jet energy scale (JES)          & $< 1\%^{2}$ \\
$\MET$ ($Z$--recoil correction) & $1\%$ \\
\hline
\hline
\multicolumn{2}{|c|}{Theory uncertainties} \\
\hline
PDF & $2\%^{3}$ \\
\hline
\end{tabular}
\end{center}
$^{1}$ decreasing with $m_{A}$ \\
$^{2}$ number quoted for $gg \to A/H$ and $b\bar{b} \to A/H$ sample as a whole; \\
\hspace{5mm} in the subsample of events with b--tagged jets the effect of the JES uncertainty is $4\%$ \\
$^{3}$ with small dependence on $m_{A}$ \\
\begin{center}
\caption[Effect of normalization uncertainties on signal efficiency times
acceptance]{\captiontext
         Effect of normalization uncertainties on the $gg \to A/H$ and $b\bar{b} \to A/H$ signal efficiency times acceptance.}
\label{tab:ExpUncertainties}
\end{center}
\end{table}
%
\section{Limit Extraction Method}
\label{sec:statmethod}
%
Given the observed distribution of $m_{\tau\tau}$ we wish to test for
the presence of a Higgs boson signal, taking into account all
background predictions and systematic uncertainties.  To do this we
use a binned Poisson likelihood, in which we represent all sources of
systematic uncertainty by nuisance parameters.  The core of the
likelihood is simply the product of the Poisson probabilities of
observing $n_i$ events in bin $i$:
\begin{eqnarray}
  {\cal L} = \prod_{i=1}^{N_{bin}} \frac{\mu_i^{n_i} e^{-\mu_i}}{n_i!}
\end{eqnarray}
where the expected number of events in the bin is the sum of the number
of events from all sources
\begin{eqnarray}
  \mu_i = \sum_{j=1}^{N_{source}} \mu_{ji}.
\end{eqnarray}
number of expected events in a source, in turn, can be written
\begin{eqnarray}
  \mu_{ji} = L \sigma_j \epsilon_{ji}
\end{eqnarray}
where $L$ is the integrated luminosity, $\sigma_j$ is the cross section
for source $j$, and$\epsilon_{ji}$ is the efficiency for source $j$ in bin
$i$.  

To test for the presence of the signal, we examine the likelihood is a
function of our signal cross-section.  If there is a significant
signal, one can simply maximize the likelihood as a function of the
cross-section and use the usual methods to determine confidence
intervals. If there is not a significant signal, we can set upper
bounds on the signal cross-section using one of several methods, as
discussed below.

There are two types of systematic errors considered in this analysis,
and represented in the likelihood by Gaussian constrained nuisance
parameters.  In the first case, we modify our description of the
number of events in a given bin from a given source to include a
multiplicative parameter which represents the uncertainty in the
cross-section for that source.  Nominally, such parameter is unity,
but is allowed to float in the likelihood, constrained to within its
uncertainty.  This method allows one to naturally take into account
correlations between different sources, or the different channels
combined together in the analysis. Thus one might have
\begin{eqnarray}
  \mu_i = \beta_1 L \sigma_1 \epsilon_{1i} + \beta_1 \beta_2 L \sigma_1 \epsilon_{2i}
\end{eqnarray}
where we have introduced two nuisance parameters, $\beta_1$ and
$\beta_2$, where $\beta_1$ affects both sources but $\beta_2$ only
affects the second source.

The second type of nuisance parameter is used to represent systematic
errors which affect the shape (and possibly also the normalization) of
the mass distribution.  We refer to these as ``morphing parameters''
and we use that technique known as ``vertical morphing'' in which we
create templates (histograms of the efficiencies) after having shifted
the value of some parameters such as an energy scale by an amount
corresponding to a believe is one standard deviation of uncertainty in
that scale.  This technique allows one to calculate the new shape of
the mass distribution as a continuous function of the morphing
parameter, which is Gaussian constrained to zero within its
uncertainty.  We generate templates corresponding to a -1 standard
deviation shift, the nominal template, and a +1 standard deviation
shift. To get the number of predicted events in a bin, we interpolate
quadratically between these three points, and extrapolate linearly
beyond them.

The overall likelihood then, including nuisance parameters, can be written
\begin{eqnarray}
  {\cal L}(\sigma_{Higgs}) = \prod_{i=1}^{N_{bin}} \frac{\mu_i^{n_i} e^{-\mu_i}}{n_i!} \times \prod_{m=1}^{N_\beta}{\cal G}(\beta_m).
\end{eqnarray}
where here we have made it explicit that the likelihood is a function
of the signal cross-section.  To eliminate the nuisance parameters, we
use MINUIT to maximize this likelihood with respect to them.  This is
known as a profile likelihood technique.

In the absence of the signal, for even in the presence of one, we can
determine a upper $95\%$ CL bound on the cross-section of the signal using
the profile likelihood.  In one method we simply use Bayes' Theorem to
convert the likelihood to a posterior density in the signal
cross-section, and integrate to find the point below which 95\% of the
probability lies.  Though this is not strictly Bayesian, we have shown
that in complicated fits like this one the results of the profile
likelihood are identical to marginalizing the nuisance parameters.

In the other method, we find that point where the logarithm of the
profile likelihood is 1.92 units below the value of the likelihood at
zero signal cross-section. This gives similar limits to the previous
method, which tend to be more stringent when there is a negative
fluctuation or no fluctuation in the apparent signal, but less
stringent than the Bayesian limits when there is an upward fluctuation
giving an apparent signal.  The complete examination of the coverage
properties of these two methods is beyond the scope of this note. We
report the results of both prescriptions below.

In order to combine the $ggA$ and $qqA$ production modes, what we call our
signal cross-section is the sum of the cross-section times branching ratio for
both modes, assuming \mbox{$\tan\beta = 30$}.  Additionally, as discussed in
Section~\ref{sec:MSSMAndTaus}, the MSSM Higgs sector consists of two Higgs
doublets, yielding five physical Higgs bosons.  This search is sensitive to
the three neutral Higgs particles the $h^0, H^0$, and $A^0$.  The relative
contributions of the three Higgs types depends on the mass $\ma$ of the CP--odd 
Higgs.  For $\ma \leq
130~\GeVcc$, the $A^0$ and $h^0$ are approximately degenerate in mass and width.
In this region the $H^0$ has a very small relative cross section and a constant
mass of $m_{H^0} \approx 130~\GeVcc$.
For $\ma \geq 130~\GeVcc$, the $h_0$ reaches a limiting mass of $\approx
130~\GeVcc$, and the $H^0$ and $A^0$ become mass degenerate.   

\begin{table}
  \centering
  \begin{tabular}{|l|c|c|c|} 
    \hline
                & \multicolumn{3}{|c|}{Included when} \\
    Higgs State & $\ma < 130~\GeVcc$ & $\ma = 130~\GeVcc$ & $\ma > 130~\GeVcc$ \\
    \hline
    $A^0$       & yes & yes & yes \\
    $H^0$       & yes & yes & no \\
    $h^0$       & no & yes & yes \\
    \hline
  \end{tabular}
  \caption[Contributions of different MSSM Higgs boson types at different
  $\ma$.]{Logic for determining the MSSM Higgs cross section for a given mass of
  the CP--odd $A^0$ Higgs.  In some regions of parameter space, the
  contributions of one of the
  CP--event Higgs particles is ignored.}
  \label{tab:HiggsXSectionCombination}
\end{table}



In the results presented below we use nuisance parameters corresponding
to the systematic errors summarized in table~\ref{tab-sys}.

\begin{table}
  \begin{center}
    \begin{tabular}{|l|c|c|} \hline
  Source                 &       Method      &   Magnitude  \\ \hline
Tau ID/trigger           &  Multiplicative   &    20\%      \\
$Z$ cross section        &  Multiplicative   &     5\%      \\
Jet to $\tau$ fake rate  &  Multiplicative   &    20\%      \\
$\mu\to\tau$ fake rate   &  Multiplicative   &   100\%      \\
$W+$jets cross section   &  Multiplicative   &    10\%      \\
$t\bar{t}$ cross section &  Multiplicative   &    40\%      \\
integrated luminosity    &  Multiplicative   &    10\%      \\
Tau energy scale         &  Morphing         &     2\%      \\
Missing $E_T$ scale      &  Morphing         &     XX\%      \\
Muon $p_T$ scale         &   -               &   neg.       \\
EM energy scale          &   -               &   neg.       \\ \hline
    \end{tabular}
   \end{center}
  \caption{Summary of systematic uncertainties represented by nuisance 
           parameters in the likelihood, their representation method 
           and magnitudes.\label{tab-sys}}
\end{table}


\ifx\master\undefined% AUTOCOMPILE
% Allows for individual chapters to be compiled.

% Usage:
% \ifx\master\undefined% AUTOCOMPILE
% Allows for individual chapters to be compiled.

% Usage:
% \ifx\master\undefined% AUTOCOMPILE
% Allows for individual chapters to be compiled.

% Usage:
% \ifx\master\undefined\input{../settings/autocompile}\fi
% (place at start and end of chapter file)

\ifx\noprelim\undefined
    % first time included
    % input preamble files
    \input{../settings/phdsetup}

    \begin{document}
    \def\noprelim{}
\else
    % already included once
    % input post files

    \singlespacing
    \bibliographystyle{../bibliography/expanded}
    \bibliography{../bibliography/references}

    \end{document}
\fi\fi
% (place at start and end of chapter file)

\ifx\noprelim\undefined
    % first time included
    % input preamble files
    % [ USER VARIABLES ]

\def\PHDTITLE {Extensions of the Theory of Computational Mechanics}
\def\PHDAUTHOR{Evan Klose Friis}
\def\PHDSCHOOL{University of California, Davis}

\def\PHDMONTH {June}
\def\PHDYEAR  {2011}
\def\PHDDEPT {Physics}

\def\BSSCHOOL {University of California at San Diego}
\def\BSYEAR   {2005}

\def\PHDCOMMITTEEA{Professor John Conway}
\def\PHDCOMMITTEEB{Professor Robin Erbacher}
\def\PHDCOMMITTEEC{Professor Mani Tripathi}

% [ GLOBAL SETUP ]

\documentclass[letterpaper,oneside,11pt]{report}

\usepackage{calc}
\usepackage{breakcites}
\usepackage[newcommands]{ragged2e}

\usepackage[pdftex]{graphicx}
\usepackage{epstopdf}

%\usepackage{tikz}
%\usetikzlibrary{positioning} % [right=of ...]
%\usetikzlibrary{fit} % [fit= ...]

%\pgfdeclarelayer{background layer}
%\pgfdeclarelayer{foreground layer}
%\pgfsetlayers{background layer,main,foreground layer}

%\newenvironment{wrap}{\noindent\begin{minipage}[t]{\linewidth}\vspace{-0.5\normalbaselineskip}\centering}{\vspace{0.5\normalbaselineskip}\end{minipage}}

%% [Venn diagram environment]
%\newenvironment{venn2}
%{\begin{tikzpicture} [every pin/.style={text=black, text opacity=1.0, pin distance=0.5cm, pin edge={black!60, semithick}},
%% define a new style 'venn'
%venn/.style={circle, draw=black!60, semithick, minimum size = 4cm}]
%
%% create circle and give it external (pin) label
%\node[venn] (X) at (-1,0) [pin={150:$H[X]$}] {};
%\node[venn] (Y) at (1,0) [pin={30:$H[Y]$}] {};
%
%% place labels of the atoms by hand
%\node at (-1.9,0) {$H[X|Y]$};
%\node at (1.9,0) {$H[Y|X]$};
%\node at (0,0) {$I[X;Y]$};}
%{\end{tikzpicture}}

%\newcommand{\wrapmath}[1]{\begin{wrap}\begin{tikzpicture}[every node/.style={inner ysep=0ex, inner xsep=0em}]\node[] {$\displaystyle\begin{aligned} #1\end{aligned}$};\end{tikzpicture}\end{wrap}}

\renewenvironment{abstract}{\chapter*{Abstract}}{}
\renewcommand{\bibname}{Bibliography}
\renewcommand{\contentsname}{Table of Contents}

\makeatletter
\renewcommand{\@biblabel}[1]{\textsc{#1}}
\makeatother

% [ FONT SETTINGS ]

\usepackage[tbtags, intlimits, namelimits]{amsmath}
\usepackage[adobe-utopia]{mathdesign}

\DeclareSymbolFont{pazomath}{OMS}{zplm}{m}{n}
\DeclareSymbolFontAlphabet{\mathcal}{pazomath}
\SetMathAlphabet\mathcal{bold}{OMS}{zplm}{b}{n}

\SetSymbolFont{largesymbols}{normal}{OMX}{zplm}{m}{n}
\SetSymbolFont{largesymbols}{bold}{OMX}{zplm}{m}{n}
\SetSymbolFont{symbols}{normal}{OMS}{zplm}{m}{n}
\SetSymbolFont{symbols}{bold}{OMS}{zplm}{b}{n}

\renewcommand{\sfdefault}{phv}
\renewcommand{\ttdefault}{fvm}

\widowpenalty 8000
\clubpenalty  8000

% [ PAGE LAYOUT ]
\usepackage{geometry}
\geometry{lmargin = 1.5in}
\geometry{rmargin = 1.0in}
\geometry{tmargin = 1.0in}
\geometry{bmargin = 1.0in}

% [ PDF SETTINGS ]

\usepackage[final]{hyperref}
\hypersetup{breaklinks  = true}
\hypersetup{colorlinks  = true}
\hypersetup{linktocpage = false}
\hypersetup{linkcolor   = blue}
\hypersetup{citecolor   = green}
\hypersetup{urlcolor    = black}
\hypersetup{plainpages  = false}
\hypersetup{pageanchor  = true}
\hypersetup{pdfauthor   = {\PHDAUTHOR}}
\hypersetup{pdftitle    = {\PHDTITLE}}
\hypersetup{pdfsubject  = {Dissertation, \PHDSCHOOL}}
\urlstyle{same}

% [ LETTER SPACING ]

\usepackage[final]{microtype}
\microtypesetup{protrusion=compatibility}
\microtypesetup{expansion=false}

\newcommand{\upper}[1]{\MakeUppercase{#1}}
\let\lsscshape\scshape

\ifcase\pdfoutput\else\microtypesetup{letterspace=15}
\renewcommand{\scshape}{\lsscshape\lsstyle}
\renewcommand{\upper}[1]{\textls[50]{\MakeUppercase{#1}}}\fi

% [ LINE SPACING ]

\usepackage[doublespacing]{setspace}
\renewcommand{\displayskipstretch}{0.0}

\setlength{\parskip   }{0em}
\setlength{\parindent }{2em}

% [ TABLE FORMATTING ]

\usepackage{booktabs}
\setlength{\heavyrulewidth}{1.5\arrayrulewidth}
\setlength{\lightrulewidth}{1.0\arrayrulewidth}
\setlength{\doublerulesep }{2.0\arrayrulewidth}

% [ SECTION FORMATTING ]

\usepackage[largestsep,nobottomtitles*]{titlesec}
\renewcommand{\bottomtitlespace}{0.75in}

\titleformat{\chapter}[display]{\bfseries\huge\singlespacing}{\filleft\textsc{\LARGE \chaptertitlename\ \thechapter}}{-0.2ex}{\titlerule[3pt]\vspace{0.2ex}}[]

\titleformat{\section}{\LARGE}{\S\thesection\hspace{0.5em}}{0ex}{}
\titleformat{\subsection}{\Large}{\S\thesubsection\hspace{0.5em}}{0ex}{}
\titleformat{\subsubsection}{\large}{\thesubsubsection\hspace{0.5em}}{0ex}{}

\titlespacing*{\chapter}{0em}{6ex}{4ex plus 2ex minus 0ex}
\titlespacing*{\section}{0em}{2ex plus 3ex minus 1ex}{0.5ex plus 0.5ex minus 0.5ex}
\titlespacing*{\subsection}{0ex}{2ex plus 3ex minus 1ex}{0ex}
\titlespacing*{\subsubsection}{0ex}{2ex plus 0ex minus 1ex}{0ex}

% [ HEADER SETTINGS ]

\usepackage{fancyhdr}

\setlength{\headheight}{\normalbaselineskip}
\setlength{\footskip  }{0.5in}
\setlength{\headsep   }{0.5in-\headheight}

\fancyheadoffset[R]{0.5in}
\renewcommand{\headrulewidth}{0pt}
\renewcommand{\footrulewidth}{0pt}

\newcommand{\pagebox}{\parbox[r][\headheight][t]{0.5in}{\hspace\fill\thepage}}

\newcommand{\prelimheaders}{\ifx\prelim\undefined\renewcommand{\thepage}{\textit{\roman{page}}}\fancypagestyle{plain}{\fancyhf{}\fancyfoot[L]{\makebox[\textwidth-0.5in]{\thepage}}}\pagestyle{plain}\def\prelim{}\fi}

\newcommand{\normalheaders}{\renewcommand{\thepage}{\arabic{page}}\fancypagestyle{plain}{\fancyhf{}\fancyhead[R]{\pagebox}}\pagestyle{plain}}

\normalheaders{}

% [ CUSTOM COMMANDS ]

\newcommand{\signaturebox}[1]{\multicolumn{1}{p{4in}}{\vspace{3ex}}\\\midrule #1\\}

%\input{../includes/cmechabbrev}

% [some math stuff - maybe stick in sep file]
\usepackage{amsthm}
\usepackage{amscd}
\theoremstyle{plain}    \newtheorem{Lem}{Lemma}
\theoremstyle{plain}    \newtheorem*{ProLem}{Proof}
\theoremstyle{plain} 	\newtheorem{Cor}{Corollary}
\theoremstyle{plain} 	\newtheorem*{ProCor}{Proof}
\theoremstyle{plain} 	\newtheorem{The}{Theorem}
\theoremstyle{plain} 	\newtheorem*{ProThe}{Proof}
\theoremstyle{plain} 	\newtheorem{Prop}{Proposition}
\theoremstyle{plain} 	\newtheorem*{ProProp}{Proof}
\theoremstyle{plain} 	\newtheorem*{Conj}{Conjecture}
\theoremstyle{plain}	\newtheorem*{Rem}{Remark}
\theoremstyle{plain}	\newtheorem*{Def}{Definition} 
\theoremstyle{plain}	\newtheorem*{Not}{Notation}

% [uniform figure scaling - maybe this is not a good idea]
\def\figscale{.7}
\def\lscale{1.0}

% [FIX ME! - red makes it easier to spot]
\newcommand{\FIX}[1]{\textbf{\textcolor{red}{#1}}}


    \begin{document}
    \def\noprelim{}
\else
    % already included once
    % input post files

    \singlespacing
    \bibliographystyle{../bibliography/expanded}
    \bibliography{../bibliography/references}

    \end{document}
\fi\fi
% (place at start and end of chapter file)

\ifx\noprelim\undefined
    % first time included
    % input preamble files
    % [ USER VARIABLES ]

\def\PHDTITLE {Extensions of the Theory of Computational Mechanics}
\def\PHDAUTHOR{Evan Klose Friis}
\def\PHDSCHOOL{University of California, Davis}

\def\PHDMONTH {June}
\def\PHDYEAR  {2011}
\def\PHDDEPT {Physics}

\def\BSSCHOOL {University of California at San Diego}
\def\BSYEAR   {2005}

\def\PHDCOMMITTEEA{Professor John Conway}
\def\PHDCOMMITTEEB{Professor Robin Erbacher}
\def\PHDCOMMITTEEC{Professor Mani Tripathi}

% [ GLOBAL SETUP ]

\documentclass[letterpaper,oneside,11pt]{report}

\usepackage{calc}
\usepackage{breakcites}
\usepackage[newcommands]{ragged2e}

\usepackage[pdftex]{graphicx}
\usepackage{epstopdf}

%\usepackage{tikz}
%\usetikzlibrary{positioning} % [right=of ...]
%\usetikzlibrary{fit} % [fit= ...]

%\pgfdeclarelayer{background layer}
%\pgfdeclarelayer{foreground layer}
%\pgfsetlayers{background layer,main,foreground layer}

%\newenvironment{wrap}{\noindent\begin{minipage}[t]{\linewidth}\vspace{-0.5\normalbaselineskip}\centering}{\vspace{0.5\normalbaselineskip}\end{minipage}}

%% [Venn diagram environment]
%\newenvironment{venn2}
%{\begin{tikzpicture} [every pin/.style={text=black, text opacity=1.0, pin distance=0.5cm, pin edge={black!60, semithick}},
%% define a new style 'venn'
%venn/.style={circle, draw=black!60, semithick, minimum size = 4cm}]
%
%% create circle and give it external (pin) label
%\node[venn] (X) at (-1,0) [pin={150:$H[X]$}] {};
%\node[venn] (Y) at (1,0) [pin={30:$H[Y]$}] {};
%
%% place labels of the atoms by hand
%\node at (-1.9,0) {$H[X|Y]$};
%\node at (1.9,0) {$H[Y|X]$};
%\node at (0,0) {$I[X;Y]$};}
%{\end{tikzpicture}}

%\newcommand{\wrapmath}[1]{\begin{wrap}\begin{tikzpicture}[every node/.style={inner ysep=0ex, inner xsep=0em}]\node[] {$\displaystyle\begin{aligned} #1\end{aligned}$};\end{tikzpicture}\end{wrap}}

\renewenvironment{abstract}{\chapter*{Abstract}}{}
\renewcommand{\bibname}{Bibliography}
\renewcommand{\contentsname}{Table of Contents}

\makeatletter
\renewcommand{\@biblabel}[1]{\textsc{#1}}
\makeatother

% [ FONT SETTINGS ]

\usepackage[tbtags, intlimits, namelimits]{amsmath}
\usepackage[adobe-utopia]{mathdesign}

\DeclareSymbolFont{pazomath}{OMS}{zplm}{m}{n}
\DeclareSymbolFontAlphabet{\mathcal}{pazomath}
\SetMathAlphabet\mathcal{bold}{OMS}{zplm}{b}{n}

\SetSymbolFont{largesymbols}{normal}{OMX}{zplm}{m}{n}
\SetSymbolFont{largesymbols}{bold}{OMX}{zplm}{m}{n}
\SetSymbolFont{symbols}{normal}{OMS}{zplm}{m}{n}
\SetSymbolFont{symbols}{bold}{OMS}{zplm}{b}{n}

\renewcommand{\sfdefault}{phv}
\renewcommand{\ttdefault}{fvm}

\widowpenalty 8000
\clubpenalty  8000

% [ PAGE LAYOUT ]
\usepackage{geometry}
\geometry{lmargin = 1.5in}
\geometry{rmargin = 1.0in}
\geometry{tmargin = 1.0in}
\geometry{bmargin = 1.0in}

% [ PDF SETTINGS ]

\usepackage[final]{hyperref}
\hypersetup{breaklinks  = true}
\hypersetup{colorlinks  = true}
\hypersetup{linktocpage = false}
\hypersetup{linkcolor   = blue}
\hypersetup{citecolor   = green}
\hypersetup{urlcolor    = black}
\hypersetup{plainpages  = false}
\hypersetup{pageanchor  = true}
\hypersetup{pdfauthor   = {\PHDAUTHOR}}
\hypersetup{pdftitle    = {\PHDTITLE}}
\hypersetup{pdfsubject  = {Dissertation, \PHDSCHOOL}}
\urlstyle{same}

% [ LETTER SPACING ]

\usepackage[final]{microtype}
\microtypesetup{protrusion=compatibility}
\microtypesetup{expansion=false}

\newcommand{\upper}[1]{\MakeUppercase{#1}}
\let\lsscshape\scshape

\ifcase\pdfoutput\else\microtypesetup{letterspace=15}
\renewcommand{\scshape}{\lsscshape\lsstyle}
\renewcommand{\upper}[1]{\textls[50]{\MakeUppercase{#1}}}\fi

% [ LINE SPACING ]

\usepackage[doublespacing]{setspace}
\renewcommand{\displayskipstretch}{0.0}

\setlength{\parskip   }{0em}
\setlength{\parindent }{2em}

% [ TABLE FORMATTING ]

\usepackage{booktabs}
\setlength{\heavyrulewidth}{1.5\arrayrulewidth}
\setlength{\lightrulewidth}{1.0\arrayrulewidth}
\setlength{\doublerulesep }{2.0\arrayrulewidth}

% [ SECTION FORMATTING ]

\usepackage[largestsep,nobottomtitles*]{titlesec}
\renewcommand{\bottomtitlespace}{0.75in}

\titleformat{\chapter}[display]{\bfseries\huge\singlespacing}{\filleft\textsc{\LARGE \chaptertitlename\ \thechapter}}{-0.2ex}{\titlerule[3pt]\vspace{0.2ex}}[]

\titleformat{\section}{\LARGE}{\S\thesection\hspace{0.5em}}{0ex}{}
\titleformat{\subsection}{\Large}{\S\thesubsection\hspace{0.5em}}{0ex}{}
\titleformat{\subsubsection}{\large}{\thesubsubsection\hspace{0.5em}}{0ex}{}

\titlespacing*{\chapter}{0em}{6ex}{4ex plus 2ex minus 0ex}
\titlespacing*{\section}{0em}{2ex plus 3ex minus 1ex}{0.5ex plus 0.5ex minus 0.5ex}
\titlespacing*{\subsection}{0ex}{2ex plus 3ex minus 1ex}{0ex}
\titlespacing*{\subsubsection}{0ex}{2ex plus 0ex minus 1ex}{0ex}

% [ HEADER SETTINGS ]

\usepackage{fancyhdr}

\setlength{\headheight}{\normalbaselineskip}
\setlength{\footskip  }{0.5in}
\setlength{\headsep   }{0.5in-\headheight}

\fancyheadoffset[R]{0.5in}
\renewcommand{\headrulewidth}{0pt}
\renewcommand{\footrulewidth}{0pt}

\newcommand{\pagebox}{\parbox[r][\headheight][t]{0.5in}{\hspace\fill\thepage}}

\newcommand{\prelimheaders}{\ifx\prelim\undefined\renewcommand{\thepage}{\textit{\roman{page}}}\fancypagestyle{plain}{\fancyhf{}\fancyfoot[L]{\makebox[\textwidth-0.5in]{\thepage}}}\pagestyle{plain}\def\prelim{}\fi}

\newcommand{\normalheaders}{\renewcommand{\thepage}{\arabic{page}}\fancypagestyle{plain}{\fancyhf{}\fancyhead[R]{\pagebox}}\pagestyle{plain}}

\normalheaders{}

% [ CUSTOM COMMANDS ]

\newcommand{\signaturebox}[1]{\multicolumn{1}{p{4in}}{\vspace{3ex}}\\\midrule #1\\}

%%%% macros fro standard references
%\eqref provided by amsmath
\newcommand{\figref}[1]{Fig.~\ref{#1}}
\newcommand{\tableref}[1]{Table~\ref{#1}}
\newcommand{\refcite}[1]{Ref.~\cite{#1}}

% Abbreviations from CMPPSS:

\newcommand{\eM}     {\mbox{$\epsilon$-machine}}
\newcommand{\eMs}    {\mbox{$\epsilon$-machines}}
\newcommand{\EM}     {\mbox{$\epsilon$-Machine}}
\newcommand{\EMs}    {\mbox{$\epsilon$-Machines}}
\newcommand{\eT}     {\mbox{$\epsilon$-transducer}}
\newcommand{\eTs}    {\mbox{$\epsilon$-transducers}}
\newcommand{\ET}     {\mbox{$\epsilon$-Transducer}}
\newcommand{\ETs}    {\mbox{$\epsilon$-Transducers}}

% Processes and sequences

\newcommand{\Process}{\mathcal{P}}

\newcommand{\ProbMach}{\Prob_{\mathrm{M}}}
\newcommand{\Lmax}   { {L_{\mathrm{max}}}}
\newcommand{\MeasAlphabet}	{\mathcal{A}}
% Original
%\newcommand{\MeasSymbol}   { {S} }
%\newcommand{\meassymbol}   { {s} }
% New symbol
\newcommand{\MeasSymbol}   { {X} }
\newcommand{\meassymbol}   { {x} }
\newcommand{\BiInfinity}	{ \overleftrightarrow {\MeasSymbol} }
\newcommand{\biinfinity}	{ \overleftrightarrow {\meassymbol} }
\newcommand{\Past}	{ \overleftarrow {\MeasSymbol} }
\newcommand{\past}	{ {\overleftarrow {\meassymbol}} }
\newcommand{\pastprime}	{ {\past}^{\prime}}
\newcommand{\Future}	{ \overrightarrow{\MeasSymbol} }
\newcommand{\future}	{ \overrightarrow{\meassymbol} }
\newcommand{\futureprime}	{ {\future}^{\prime}}
\newcommand{\PastPrime}	{ {\Past}^{\prime}}
\newcommand{\FuturePrime}	{ {\overrightarrow{\meassymbol}}^\prime }
\newcommand{\PastDblPrime}	{ {\overleftarrow{\meassymbol}}^{\prime\prime} }
\newcommand{\FutureDblPrime}	{ {\overrightarrow{\meassymbol}}^{\prime\prime} }
\newcommand{\pastL}	{ {\overleftarrow {\meassymbol}}{}^L }
\newcommand{\PastL}	{ {\overleftarrow {\MeasSymbol}}{}^L }
\newcommand{\PastLt}	{ {\overleftarrow {\MeasSymbol}}_t^L }
\newcommand{\PastLLessOne}	{ {\overleftarrow {\MeasSymbol}}^{L-1} }
\newcommand{\futureL}	{ {\overrightarrow{\meassymbol}}{}^L }
\newcommand{\FutureL}	{ {\overrightarrow{\MeasSymbol}}{}^L }
\newcommand{\FutureLt}	{ {\overrightarrow{\MeasSymbol}}_t^L }
\newcommand{\FutureLLessOne}	{ {\overrightarrow{\MeasSymbol}}^{L-1} }
\newcommand{\pastLprime}	{ {\overleftarrow {\meassymbol}}^{L^\prime} }
\newcommand{\futureLprime}	{ {\overrightarrow{\meassymbol}}^{L^\prime} }
\newcommand{\AllPasts}	{ { \overleftarrow {\rm {\bf \MeasSymbol}} } }
\newcommand{\AllFutures}	{ \overrightarrow {\rm {\bf \MeasSymbol}} }
\newcommand{\FutureSet}	{ \overrightarrow{\bf \MeasSymbol}}

% Causal states and epsilon-machines
\newcommand{\CausalState}	{ \mathcal{S} }
\newcommand{\CausalStatePrime}	{ {\CausalState}^{\prime}}
\newcommand{\causalstate}	{ \sigma }
\newcommand{\CausalStateSet}	{ \boldsymbol{\CausalState} }
\newcommand{\AlternateState}	{ \mathcal{R} }
\newcommand{\AlternateStatePrime}	{ {\cal R}^{\prime} }
\newcommand{\alternatestate}	{ \rho }
\newcommand{\alternatestateprime}	{ {\rho^{\prime}} }
\newcommand{\AlternateStateSet}	{ \boldsymbol{\AlternateState} }
\newcommand{\PrescientState}	{ \widehat{\AlternateState} }
\newcommand{\prescientstate}	{ \widehat{\alternatestate} }
\newcommand{\PrescientStateSet}	{ \boldsymbol{\PrescientState}}
\newcommand{\CausalEquivalence}	{ {\sim}_{\epsilon} }
\newcommand{\CausalEquivalenceNot}	{ {\not \sim}_{\epsilon}}

\newcommand{\NonCausalEquivalence}	{ {\sim}_{\eta} }
\newcommand{\NextObservable}	{ {\overrightarrow {\MeasSymbol}}^1 }
\newcommand{\LastObservable}	{ {\overleftarrow {\MeasSymbol}}^1 }
%\newcommand{\Prob}		{ {\rm P}}
\newcommand{\Prob}      {\Pr} % use standard command
\newcommand{\ProbAnd}	{ {,\;} }
\newcommand{\LLimit}	{ {L \rightarrow \infty}}
\newcommand{\Cmu}		{C_\mu}
\newcommand{\hmu}		{h_\mu}
\newcommand{\EE}		{{\bf E}}
\newcommand{\Measurable}{{\bf \mu}}

% Process Crypticity
\newcommand{\PC}		{\chi}
\newcommand{\FuturePC}		{\PC^+}
\newcommand{\PastPC}		{\PC^-}
% Causal Irreversibility
\newcommand{\CI}		{\Xi}
\newcommand{\ReverseMap}	{r}
\newcommand{\ForwardMap}	{f}

% Abbreviations from IB:
% None that aren't already in CMPPSS

% Abbreviations from Extensive Estimation:
\newcommand{\EstCausalState}	{\widehat{\CausalState}}
\newcommand{\estcausalstate}	{\widehat{\causalstate}}
\newcommand{\EstCausalStateSet}	{\boldsymbol{\EstCausalState}}
\newcommand{\EstCausalFunc}	{\widehat{\epsilon}}
\newcommand{\EstCmu}		{\widehat{\Cmu}}
\newcommand{\PastLOne}	{{\Past}^{L+1}}
\newcommand{\pastLOne}	{{\past}^{L+1}}

% Abbreviations from $\epsilon$-Transducers:
\newcommand{\InAlphabet}	{ \mathcal{A}}
\newcommand{\insymbol}		{ a}
\newcommand{\OutAlphabet}	{ \mathcal{B}}
\newcommand{\outsymbol}		{ b}
\newcommand{\InputSimple}	{ X}
\newcommand{\inputsimple}	{ x}
\newcommand{\BottleneckVar}	{\tilde{\InputSimple}}
\newcommand{\bottleneckvar}	{\tilde{\inputsimple}}
\newcommand{\InputSpace}	{ \mathbf{\InputSimple}}
\newcommand{\InputBi}	{ \overleftrightarrow {\InputSimple} }
\newcommand{\inputbi}	{ \overleftrightarrow {\inputsimple} }
\newcommand{\InputPast}	{ \overleftarrow {\InputSimple} }
\newcommand{\inputpast}	{ \overleftarrow {\inputsimple} }
\newcommand{\InputFuture}	{ \overrightarrow {\InputSimple} }
\newcommand{\inputfuture}	{ \overrightarrow {\inputsimple} }
\newcommand{\NextInput}	{ {{\InputFuture}^{1}}}
\newcommand{\NextOutput}	{ {\OutputFuture}^{1}}
\newcommand{\OutputSimple}	{ Y}
\newcommand{\outputsimple}	{ y}
\newcommand{\OutputSpace}	{ \mathbf{\OutputSimple}}
\newcommand{\OutputBi}	{ \overleftrightarrow{\OutputSimple} }
\newcommand{\outputbi}	{ \overleftrightarrow{\outputsimple} }
\newcommand{\OutputPast}	{ \overleftarrow{\OutputSimple} }
\newcommand{\outputpast}	{ \overleftarrow{\outputsimple} }
\newcommand{\OutputFuture}	{ \overrightarrow{\OutputSimple} }
\newcommand{\outputfuture}	{ \overrightarrow{\outputsimple} }
\newcommand{\OutputL}	{ {\OutputFuture}^L}
\newcommand{\outputL}	{ {\outputfuture}^L}
\newcommand{\InputLLessOne}	{ {\InputFuture}^{L-1}}
\newcommand{\inputLlessone}	{ {\inputufutre}^{L-1}}
\newcommand{\OutputPastLLessOne}	{{\OutputPast}^{L-1}_{-1}}
\newcommand{\outputpastLlessone}	{{\outputpast}^{L-1}}
\newcommand{\OutputPastLessOne}	{{\OutputPast}_{-1}}
\newcommand{\outputpastlessone}	{{\outputpast}_{-1}}
\newcommand{\OutputPastL}	{{\OutputPast}^{L}}
\newcommand{\OutputLPlusOne}	{ {\OutputFuture}^{L+1}}
\newcommand{\outputLplusone}	{ {\outputfutre}^{L+1}}
\newcommand{\InputPastL}	{{\InputPast}^{L}}
\newcommand{\inputpastL}	{{\inputpast}^{L}}
\newcommand{\JointPast}	{{(\InputPast,\OutputPast)}}
\newcommand{\jointpast}	{{(\inputpast,\outputpast)}}
\newcommand{\jointpastone}	{{(\inputpast_1,\outputpast_1)}}
\newcommand{\jointpasttwo}	{{(\inputpast_2,\outputpast_2)}}
\newcommand{\jointpastprime} {{({\inputpast}^{\prime},{\outputpast}^{\prime})}}
\newcommand{\NextJoint}	{{(\NextInput,\NextOutput)}}
\newcommand{\nextjoint}	{{(\insymbol,\outsymbol)}}
\newcommand{\AllInputPasts}	{ { \overleftarrow {\rm \InputSpace}}}
\newcommand{\AllOutputPasts}	{ {\overleftarrow {\rm \OutputSpace}}}
\newcommand{\DetCausalState}	{ {{\cal S}_D }}
\newcommand{\detcausalstate}	{ {{\sigma}_D} }
\newcommand{\DetCausalStateSet}	{ \boldsymbol{{\CausalState}_D}}
\newcommand{\DetCausalEquivalence}	{ {\sim}_{{\epsilon}_{D}}}
\newcommand{\PrescientEquivalence}	{ {\sim}_{\widehat{\eta}}}
\newcommand{\FeedbackCausalState}	{ \mathcal{F}}
\newcommand{\feedbackcausalstate}	{ \phi}
\newcommand{\FeedbackCausalStateSet}	{ \mathbf{\FeedbackCausalState}}
\newcommand{\JointCausalState}		{ \mathcal{J}}
\newcommand{\JointCausalStateSet}	{ \mathbf{\JointCausalState}}
\newcommand{\UtilityFunctional}	{ {\mathcal{L}}}
\newcommand{\NatureState}	{ {\Omega}}
\newcommand{\naturestate}	{ {\omega}}
\newcommand{\NatureStateSpace}	{ {\mathbf{\NatureState}}}
\newcommand{\AnAction}	{ {A}}
\newcommand{\anaction}	{ {a}}
\newcommand{\ActionSpace}	{ {\mathbf{\AnAction}}}

% Abbreviations from RURO:
\newcommand{\InfoGain}[2] { \mathcal{D} \left( {#1} || {#2} \right) }

% Abbreviations from Upper Bound:
\newcommand{\lcm}	{{\rm lcm}}
% Double-check that this isn't in the math set already!

% Abbreviations from Emergence in Space
\newcommand{\ProcessAlphabet}	{\MeasAlphabet}
\newcommand{\ProbEst}			{ {\widehat{\Prob}_N}}
\newcommand{\STRegion}			{ {\mathrm K}}
\newcommand{\STRegionVariable}		{ K}
\newcommand{\stregionvariable}		{ k}
\newcommand{\GlobalPast}		{ \overleftarrow{G}} 
\newcommand{\globalpast}		{ \overleftarrow{g}} 
\newcommand{\GlobalFuture}		{ \overrightarrow{G}}
\newcommand{\globalfuture}		{ \overrightarrow{g}}
\newcommand{\GlobalState}		{ \mathcal{G}}
\newcommand{\globalstate}		{ \gamma}
\newcommand{\GlobalStateSet}		{ {\mathbf \GlobalState}}
\newcommand{\LocalPast}			{ \overleftarrow{L}} 
\newcommand{\localpast}			{ \overleftarrow{l}}
\newcommand{\AllLocalPasts}		{ \mathbf{\LocalPast}}
\newcommand{\LocalPastRegion}		{ \overleftarrow{\mathrm L}}
\newcommand{\LocalFuture}		{ \overrightarrow{L}}
\newcommand{\localfuture}		{ \overrightarrow{l}}
\newcommand{\LocalFutureRegion}		{ \overrightarrow{\mathrm L}}
\newcommand{\LocalState}		{ \mathcal{L}}
\newcommand{\localstate}		{ \lambda}
\newcommand{\LocalStateSet}		{ {\mathbf \LocalState}}
\newcommand{\PatchPast}			{ \overleftarrow{P}}
\newcommand{\patchpast}			{ \overleftarrow{p}}
\newcommand{\PatchPastRegion}		{ \overleftarrow{\mathrm P}}
\newcommand{\PatchFuture}		{ \overrightarrow{P}}
\newcommand{\patchfuture}		{ \overrightarrow{p}}
\newcommand{\PatchFutureRegion}		{ \overrightarrow{\mathrm P}}
\newcommand{\PatchState}		{ \mathcal{P}}
\newcommand{\patchstate}		{ \pi}
\newcommand{\PatchStateSet}		{ {\mathbf \PatchState}}
\newcommand{\LocalStatesInPatch}	{\vec{\LocalState}}
\newcommand{\localstatesinpatch}	{\vec{\localstate}}
\newcommand{\PointInstantX}		{ {\mathbf x}}
% Galles's original LaTeX for the cond. indep. symbol follows:
\newcommand{\compos}{\mbox{$~\underline{~\parallel~}~$}}
\newcommand{\ncompos}{\not\hspace{-.15in}\compos}
\newcommand{\indep}			{ \rotatebox{90}{$\models$}}
\newcommand{\nindep}	{\not\hspace{-.05in}\indep}
\newcommand{\LocalEE}	{{\EE}^{loc}}
\newcommand{\EEDensity}	{\overline{\LocalEE}}
\newcommand{\LocalCmu}	{{\Cmu}^{loc}}
\newcommand{\CmuDensity}	{\overline{\LocalCmu}}

%%%%%%%%%%% added by sasa
\newcommand{\FinPast}[1]	{ \overleftarrow {\MeasSymbol} \stackrel{{#1}}{}}
\newcommand{\finpast}[1]  	{ \overleftarrow {\meassymbol}  \stackrel{{#1}}{}}
\newcommand{\FinFuture}[1]		{ \overrightarrow{\MeasSymbol} \stackrel{{#1}}{}}
\newcommand{\finfuture}[1]		{ \overrightarrow{\meassymbol} \stackrel{{#1}}{}}

\newcommand{\Partition}	{ \AlternateState }
\newcommand{\partitionstate}	{ \alternatestate }
\newcommand{\PartitionSet}	{ \AlternateStateSet }
\newcommand{\Fdet}   { F_{\rm det} }

\newcommand{\Dkl}[2] { D_{\rm KL} \left( {#1} || {#2} \right) }

\newcommand{\Period}	{p}

% To take into account time direction
\newcommand{\forward}{+}
\newcommand{\reverse}{-}
%\newcommand{\forwardreverse}{\:\!\diamond} % \pm
\newcommand{\forwardreverse}{\pm} % \pm
\newcommand{\FutureProcess}	{ {\Process}^{\forward} }
\newcommand{\PastProcess}	{ {\Process}^{\reverse} }
\newcommand{\FutureCausalState}	{ {\CausalState}^{\forward} }
\newcommand{\futurecausalstate}	{ \sigma^{\forward} }
\newcommand{\altfuturecausalstate}	{ \sigma^{\forward\prime} }
\newcommand{\PastCausalState}	{ {\CausalState}^{\reverse} }
\newcommand{\pastcausalstate}	{ \sigma^{\reverse} }
\newcommand{\BiCausalState}		{ {\CausalState}^{\forwardreverse} }
\newcommand{\bicausalstate}		{ {\sigma}^{\forwardreverse} }
\newcommand{\FutureCausalStateSet}	{ {\CausalStateSet}^{\forward} }
\newcommand{\PastCausalStateSet}	{ {\CausalStateSet}^{\reverse} }
\newcommand{\BiCausalStateSet}	{ {\CausalStateSet}^{\forwardreverse} }
\newcommand{\eMachine}	{ M }
\newcommand{\FutureEM}	{ {\eMachine}^{\forward} }
\newcommand{\PastEM}	{ {\eMachine}^{\reverse} }
\newcommand{\BiEM}		{ {\eMachine}^{\forwardreverse} }
\newcommand{\BiEquiv}	{ {\sim}^{\forwardreverse} }
\newcommand{\Futurehmu}	{ h_\mu^{\forward} }
\newcommand{\Pasthmu}	{ h_\mu^{\reverse} }
\newcommand{\FutureCmu}	{ C_\mu^{\forward} }
\newcommand{\PastCmu}	{ C_\mu^{\reverse} }
\newcommand{\BiCmu}		{ C_\mu^{\forwardreverse} }
\newcommand{\FutureEps}	{ \epsilon^{\forward} }
\newcommand{\PastEps}	{ \epsilon^{\reverse} }
\newcommand{\BiEps}	{ \epsilon^{\forwardreverse} }
\newcommand{\FutureSim}	{ \sim^{\forward} }
\newcommand{\PastSim}	{ \sim^{\reverse} }
% Used arrows for awhile, more or less confusing?
%\newcommand{\FutureCausalState}	{ \overrightarrow{\CausalState} }
%\newcommand{\PastCausalState}	{ \overleftarrow{\CausalState} }
%\newcommand{\eMachine}	{ M }
%\newcommand{\FutureEM}	{ \overrightarrow{\eMachine} }
%\newcommand{\PastEM}	{ \overleftarrow{\eMachine} }
%\newcommand{\FutureCmu}	{ \overrightarrow{\Cmu} }
%\newcommand{\PastCmu}	{ \overleftarrow{\Cmu} }

%% time-reversing and mixed state presentation operators
\newcommand{\TR}{\mathcal{T}}
\newcommand{\MSP}{\mathcal{U}}
\newcommand{\one}{\mathbf{1}}

%% (cje)
%% Provide a command \ifpm which is true when \pm 
%% is meant to be understood as "+ or -". This is
%% different from the usage in TBA.
\newif\ifpm 
\edef\tempa{\forwardreverse}
\edef\tempb{\pm}
\ifx\tempa\tempb
   \pmfalse
\else
   \pmtrue  
\fi





% [some math stuff - maybe stick in sep file]
\usepackage{amsthm}
\usepackage{amscd}
\theoremstyle{plain}    \newtheorem{Lem}{Lemma}
\theoremstyle{plain}    \newtheorem*{ProLem}{Proof}
\theoremstyle{plain} 	\newtheorem{Cor}{Corollary}
\theoremstyle{plain} 	\newtheorem*{ProCor}{Proof}
\theoremstyle{plain} 	\newtheorem{The}{Theorem}
\theoremstyle{plain} 	\newtheorem*{ProThe}{Proof}
\theoremstyle{plain} 	\newtheorem{Prop}{Proposition}
\theoremstyle{plain} 	\newtheorem*{ProProp}{Proof}
\theoremstyle{plain} 	\newtheorem*{Conj}{Conjecture}
\theoremstyle{plain}	\newtheorem*{Rem}{Remark}
\theoremstyle{plain}	\newtheorem*{Def}{Definition} 
\theoremstyle{plain}	\newtheorem*{Not}{Notation}

% [uniform figure scaling - maybe this is not a good idea]
\def\figscale{.7}
\def\lscale{1.0}

% [FIX ME! - red makes it easier to spot]
\newcommand{\FIX}[1]{\textbf{\textcolor{red}{#1}}}


    \begin{document}
    \def\noprelim{}
\else
    % already included once
    % input post files

    \singlespacing
    \bibliographystyle{../bibliography/expanded}
    \bibliography{../bibliography/references}

    \end{document}
\fi\fi
