\ifx\master\undefined\input{settings/autocompile}\fi
\chapter{Results}
\label{ch:results}

In the 36~\pbinv of 7~\TeV data collected by CMS in 2010, the analysis selection
criteria described in Chapter~\ref{ch:selections} selects a total of 413 events.
Fifteen of these events pass the additional requirement of a central jet passing
the $b$--jet discriminator.  The expected yields from each of the background
sources, computed by the Template method (Section~\ref{sec:template}) and
verified by the Fake--rate method (Section~\ref{sec:fakerate}) are shown in
Table~\ref{tab:ExpResultsLooseAHtoMuTau}.  The total expectation from the
Standard Model backgrounds is 437, of which 11.3 are expected to pass the
$b$--tag criteria.  The overall yield seen in the data approximately 5\% lower
than expected at the nominal values of all systematic uncertainties.  This
difference is well within the statistical variance of the observed data (20.3
events), as well as the uncertainty on the integrated luminosity\footnote{The
uncertainty on the CMS luminosity measurement was 11\% at the time this analysis
was performed. The measurement was later improved, and at the time of this
writing the uncertainty on CMS 2010 integrated luminosity is 4\%. The improved
luminosity measurement is not expected to change the results of this analysis
significantly.} \fixme{need cite} and the tau identification uncertainty.
\begin{table}[t]
\begin{center}
\tablesize
\begin{tabular}{|l|c|c|}
\hline
Process & Events without $b$--tag & Events with $b$--tag \\
\hline
\ttbarpJets & 0.6 & 2.26 \\
\WpJets & 62.9 & 0.51 \\
\ZMM & 3.8 & 0.04 \\
QCD & 124.0 & 5.07 \\
$(qq)Z\rightarrow\TT$ & 234.2 & 3.46 \\
\hline
Standard Model sum & 425.5 & 11.34 \\
\hline
\hline
Data & 398 & 15 \\
\hline
\end{tabular}
\caption[Final analysis yields and background expectations]{Number of Higgs $\to
\TT \to \mu + \tau_{had}$ candidate events passing the selection criteria
described in Chapter~\ref{ch:selections}.} \label{tab:ExpResultsLooseAHtoMuTau}
\end{center}
\end{table}

The distributions of the visible mass and SVfit mass in the final selected
events are shown in Figure~\ref{fig:AHtoMuTauPlotsLoose}.  Excellent
agreement is seen between the shapes of the distributions.   Control plots
showing the transverse
momenta of the muon and visible hadronic tau in the final analysis selection are
shown in \fixme{GET PLOTS!} Figure~XXXXX.
%
\begin{figure}[t]
\begin{center}
\includegraphics*[width=75mm]{results_chapter/figures/results_loose/plotAHtoMuTauOS_woBtag_finalSamplePlots_mVisible_linear.pdf}
\includegraphics*[width=75mm]{results_chapter/figures/results_loose/plotAHtoMuTauOS_woBtag_finalSamplePlots_mVisible_log.pdf}
\includegraphics*[width=75mm]{results_chapter/figures/results_loose/plotAHtoMuTauOS_woBtag_finalSamplePlots_mSVmethod_linear.pdf}
\includegraphics*[width=75mm]{results_chapter/figures/results_loose/plotAHtoMuTauOS_woBtag_finalSamplePlots_mSVmethod_log.pdf}
\caption[Distributions of final selected events]{Distribution of visible (top)
and ``full'' \TT invariant mass reconstructed by the SVfit algorithm (bottom) in
Higgs $\to \TT \to \mu + \tau_{had}$ candidate events passing the selection
criteria described in Chapter~\ref{ch:selections}.  The distributions are shown
in linear (logarithmic) scale on the left (right).}
\label{fig:AHtoMuTauPlotsLoose}
\end{center}
\end{figure} 

The expected yields from an MSSM Higgs boson signal for \mbox{$\tan\beta = 30$}
are summarized in \label{tab:SignalExpResultsLooseAHtoMuTau}.  The contributions
from the gluon fusion production mode and the associated $b$--quark production
modes are given separately.  The yields are divided into the exclusive
categories of events containing a $b$--tagged jet and those without.  For a
Higgs mass of $\ma = 160~\GeVcc$, a total of 17 events are expected at
$\tan\beta = 30$.
%
\begin{table}[t]
\begin{center}
\tablesize
\begin{tabular}{|l|c|c|}
\hline
Process & Events without $b$--tag & Events with $b$--tag \\
\hline
\hline
\multicolumn{3}{|c|}{Gluon fusion production} \\
\hline
A90 & 37.21 & 0.86 \\
A100 & 27.40 & 0.40 \\
A120 & 14.39 & 0.14 \\
A130 & 11.81 & 0.18 \\
A160 & 4.46 & 0.09 \\
%A180 & 2.71 & 0.04 \\  % not - jobs failed, bad stat.
A200 & 1.51 & 0.03 \\
A250 & 0.47 & 0.01 \\
A300 & 0.15 & 0.0 \\
A350 & 0.06 & 0.44 \\
\hline
\multicolumn{3}{|c|}{Associated $b$--quark production} \\
\hline
bbA90 & 33.07 &  5.50 \\
bbA100 & 30.18 &  4.77 \\
bbA120 & 21.91 & 4.02 \\
bbA130 & 18.34 & 3.35 \\ 
bbA160 & 10.35 &  2.10 \\
%bbA180 & 7.23  &  1.69 \\
bbA200 & 4.85  &  1.29 \\
bbA250 & 2.11 &  0.55 \\
bbA300 & 0.97 &  0.26 \\
bbA350 & 0.41 & 0.13 \\
\hline
\end{tabular}
\caption[Expected signal yields at \mbox{$\tan \beta = 30$}]{Number of Higgs
signal event expected to pass the selection criteria described in
Section~\ref{ch:selections}.  The expected signal yield is given for MSSM
parameter \mbox{$\tan \beta = 30$}, using the cross sections provided by the LHC
Higgs Cross Section working group.} \label{tab:SignalExpResultsLooseAHtoMuTau}
\end{center}
\end{table}
\fixme{Get cite}

We compute upper limits on the cross section times the branching ratio using the 
Bayesian and DLL methods described in Section~\label{sec:statmethod}.  The 
\begin{table}
  \begin{center}
    \begin{tabular}{|c|c|c|c|c|c|}
    \hline
    \multicolumn{6}{|c|}{Bayesian Limit} \\
    \hline
   mass &-2$\sigma$&-1$\sigma$&   median &+1$\sigma$&+2$\sigma$\\  \hline
     90 &  338.451 &  426.345  &  603.758 &  864.279 &  995.085 \\
    100 &  139.186 &  201.021  &  284.872 &  392.952 &  526.509 \\
    130 &   23.652 &   33.018  &   45.208 &   67.276 &   99.547 \\
    160 &   12.596 &   16.516  &   23.052 &   33.980 &   49.437 \\
    200 &    7.731 &   10.091  &   14.179 &   20.334 &   28.955 \\
    250 &    5.194 &    6.753  &    9.568 &   13.277 &   18.317 \\
    350 &    3.586 &    4.548  &    6.020 &    8.192 &   11.552 \\  \hline
    \end{tabular}
    \begin{tabular}{|c|c|c|c|c|c|}
    \hline
    \multicolumn{6}{|c|}{Delta Log Likelihood Limit} \\
    \hline
   mass &-2$\sigma$&-1$\sigma$&   median &+1$\sigma$&+2$\sigma$\\  \hline
     90 &  207.584 &  287.853  &  483.422 &  817.838 &  995.397 \\
    100 &   91.429 &  140.344  &  231.893 &  384.402 &  622.317 \\
    130 &   16.239 &   25.512  &   41.171 &   74.202 &  141.748 \\
    160 &    9.157 &   13.164  &   21.658 &   38.859 &   68.062 \\
    200 &    5.320 &    7.757  &   13.443 &   23.028 &   40.261 \\
    250 &    3.499 &    4.899  &    8.410 &   14.182 &   24.621 \\
    350 &    2.298 &    3.065  &    4.733 &    7.920 &   15.243 \\  \hline
    \end{tabular}
    \caption[Expected 95\% CL $\sigma \times \text{BR}$ upper limits] {Expected
    95\% CL upper limit bands using the Bayesian prescription (top), and Delta
    Log Likelihood (bottom)}
    \label{tab-exp-limit-TaNC}
  \end{center}
\end{table}

In order to combine the $ggA$ and $qqA$ production modes, what we call our
signal cross-section is the sum of the cross-section times branching ratio for
both modes, assuming \mbox{$\tan\beta = 30$}.  Additionally, as discussed in
Section~\ref{sec:MSSMAndTaus}, the MSSM Higgs sector consists of two Higgs
doublets, yielding five physical Higgs bosons.  This search is sensitive to
the three neutral Higgs particles the $h^0, H^0$, and $A^0$.  The relative
contributions of the three Higgs types depends on the mass $\ma$ of the CP--odd 
Higgs.  For $\ma \leq
130~\GeVcc$, the $A^0$ and $h^0$ are approximately degenerate in mass and width.
In this region the $H^0$ has a very small relative cross section and a constant
mass of $m_{H^0} \approx 130~\GeVcc$.
For $\ma \geq 130~\GeVcc$, the $h_0$ reaches a limiting mass of $\approx
130~\GeVcc$, and the $H^0$ and $A^0$ become mass degenerate.   

\begin{table}
  \centering
  \begin{tabular}{l|ccc} 
                & \multicolumn{3}{|c}{Included when} \\
    Higgs State & $\ma < 130~\GeVcc$ & $\ma = 130~\GeVcc$ & $\ma > 130~\GeVcc$ \\
    \hline
    $A^0$       & yes & yes & yes \\
    $H^0$       & yes & yes & no \\
    $h^0$       & no & yes & yes \\
  \end{tabular}
  \caption[Contributions of different MSSM Higgs boson types at different
  $\ma$.]{Logic for determining the MSSM Higgs cross section for a given mass of
  the CP--odd $A^0$ Higgs.  In some regions of parameter space, the
  contributions of one of the
  CP--event Higgs particles is ignored.}
  \label{tab:HiggsXSectionCombination}
\end{table}


\ifx\master\undefined\input{settings/autocompile}\fi
