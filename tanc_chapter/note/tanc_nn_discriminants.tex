Each neural network corresponds to a different decay mode topology and as such
each network uses different observables as inputs.  However, many of the input
observables are used in multiple neural nets.  The superset of all observables
is listed and defined below. Table~\ref{tab:nn_var_table} maps the input
observables to their associated neural networks.  
%The signal and background
%distributions of the input observables for tau--candidates in the training
%sample are shown in appendix~\ref{sec:input_descriptions}.  
In three prong
decays, the definition of the ``main track'' is important.  The main track
corresponds to the track with charge opposite to that of the total charge of the
three tracks.  This distinction is made to facilitate the use of the ``Dalitz''
observables, allowing identification of intermediate resonances in three--body
decays.  This is motivated by the fact that the three prong decays of the tau
generally proceed through $\tau^{-} \to a1^{-} \nu_\tau \rightarrow
\pi^{-} \rho^0 \nu_\tau \to \pi^{-} \pi^{+} \pi^{-} \nu_\tau$; the
oppositely charged track can always be identified with the $\rho^0$ decay.


\begin{description}
    
  \item[ChargedOutlierAngleN] \hfill \\
  $\Delta R$ between the Nth charged object (ordered by \pt) in the isolation region
    and the tau--candidate momentum axis. If the number of
    isolation region objects is less than N, the input is set at one.

  \item[ChargedOutlierPtN] \hfill \\
  Transverse momentum of the Nth charged object in the isolation region.  If the number of
    isolation region objects is less than N, the input is set at zero.

  \item[DalitzN] \hfill \\
   Invariant mass of four vector sum of the ``main track'' and the Nth signal
    region object. 

  \item[Eta] \hfill \\
  Pseudo-rapidity of the signal region objects. 

  \item[InvariantMassOfSignal] \hfill \\
  Invariant mass of the composite object formed by the signal region constituents.

  \item[MainTrackAngle] \hfill \\
  $\Delta R$ between the ``main track'' and the composite four--vector formed by the 
    signal region constituents.

  \item[MainTrackPt] \hfill \\
  Transverse momentum of the ``main track.'' 

  \item[OutlierNCharged] \hfill \\
  Number of charged objects in the isolation region.

  \item[OutlierSumPt] \hfill \\
  Sum of the transverse momentum of objects in the isolation region.

  \item[PiZeroAngleN] \hfill \\
  $\Delta R$ between the Nth $\pi^0$ object in the signal region (ordered by \pt) and
    the tau--candidate momentum axis.

  \item[PiZeroPtN] \hfill \\
  Transverse momentum of the Nth $\pi^0$ object in the signal region.

  \item[TrackAngleN] \hfill \\
  $\Delta R$ between the Nth charged object in the signal region (ordered by \pt) and
    the tau--candidate momentum axis, exclusive of the main track.

  \item[TrackPtN] \hfill \\
  Transverse momentum of the Nth charged object in the signal region, exclusive of the 
    main track.

\end{description}
    


\begin{table}[h]
   \centering
\begin{tabular}{l|c|c|c|c|c|}
\multirow{2}{*}{Input observable} & \multicolumn{5}{c}{Neural network} \\
 & $\pi^{-}\nu_\tau$ & $\pi^{-}\pi^0\nu_\tau$ & $\pi^{-}\pi^0\pi^0\nu_\tau$ & $\pi^{-}\pi^{+}\pi^{-}\nu_\tau$ & $\pi^{-}\pi^{+}\pi^{-}\pi^0\nu_\tau$\\
\hline
ChargedOutlierAngle1&$\bullet$ & $\bullet$ & $\bullet$ & $\bullet$ & $\bullet$\\
ChargedOutlierAngle2&$\bullet$ & $\bullet$ & $\bullet$ & $\bullet$ & $\bullet$\\
ChargedOutlierPt1&$\bullet$ & $\bullet$ & $\bullet$ & $\bullet$ & $\bullet$\\
ChargedOutlierPt2&$\bullet$ & $\bullet$ & $\bullet$ & $\bullet$ & $\bullet$\\
ChargedOutlierPt3&$\bullet$ & $\bullet$ & $\bullet$ & $\bullet$ & $\bullet$\\
ChargedOutlierPt4&$\bullet$ & $\bullet$ & $\bullet$ & $\bullet$ & $\bullet$\\
Dalitz1& &  & $\bullet$ & $\bullet$ & $\bullet$\\
Dalitz2& &  & $\bullet$ & $\bullet$ & $\bullet$\\
Eta&$\bullet$ & $\bullet$ & $\bullet$ & $\bullet$ & $\bullet$\\
InvariantMassOfSignal& & $\bullet$ & $\bullet$ & $\bullet$ & $\bullet$\\
MainTrackAngle& & $\bullet$ & $\bullet$ & $\bullet$ & $\bullet$\\
MainTrackPt&$\bullet$ & $\bullet$ & $\bullet$ & $\bullet$ & $\bullet$\\
OutlierNCharged&$\bullet$ & $\bullet$ & $\bullet$ & $\bullet$ & $\bullet$\\
OutlierSumPt&$\bullet$ & $\bullet$ & $\bullet$ & $\bullet$ & $\bullet$\\
PiZeroAngle1& & $\bullet$ & $\bullet$ &  & $\bullet$\\
PiZeroAngle2& &  & $\bullet$ &  & \\
PiZeroPt1& & $\bullet$ & $\bullet$ &  & $\bullet$\\
PiZeroPt2& &  & $\bullet$ &  & \\
TrackAngle1& &  &  & $\bullet$ & $\bullet$\\
TrackAngle2& &  &  & $\bullet$ & $\bullet$\\
TrackPt1& &  &  & $\bullet$ & $\bullet$\\
TrackPt2& &  &  & $\bullet$ & $\bullet$\\
\end{tabular}
\caption[Variables used in the different TaNC neural networks]{Input observables
used for each of the neural networks implemented by the Tau Neural Classifier.
The columns represents the neural networks associated to various decay modes and
the rows represent the superset of input observables (see
section~\ref{sec:tanc_nn_discriminants}) used in the neural networks.  A dot in
a given row and column indicates that the observable in that row is used in the
neural network corresponding to that column.  } \label{tab:nn_var_table}

\end{table}

