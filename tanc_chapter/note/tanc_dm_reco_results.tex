The performance of the decay mode reconstruction can be measured for
tau--candidates that are matched to generator level hadronically decaying tau
leptons by examining the correlation of the reconstructed decay mode to the true
decay mode determined from the Monte Carlo generator level information.
Figure~\ref{fig:dmResolution} compares the decay mode reconstruction performance
of a naive approach where the decay mode is determined by simply counting the
number of photons to the performance of the photon merging and filtering
approach described in section~\ref{sec:decay_mode_reco}.  The correlation for
the merging and filtering algorithm is much more diagonal, indicating higher
performance.  The performance is additonally presented for comparison in tabular form in
table~\ref{tab:dmResolutionStandard} (merging and filtering approach) and
table~\ref{tab:dmResolutionNoNothing} (naive approach).

The performance of the decay mode reconstruction is dependent on the transverse
momentum and $\eta$ of the tau--candidate and is shown in
figure~\ref{fig:dmKinematics}.  The \pt dependence is largely due to
threshold effects; high multiplicity decay modes are suppressed at low
transverse momentum as the constituents are below the minimum \pt quality
requirements.  In the forward region, nuclear interactions and conversions from
the increased material budget enhances modes containing $\pi^0$ mesons.


\begin{table}[htp]
   \centering
   \begin{tabular}{l|cccccc}

\multirow{2}{*}{True decay mode} & \multicolumn{6}{c}{Reconstructed Decay Mode}\\

 & $\pi^{-}\nu_\tau$ & $\pi^{-}\pi^0\nu_\tau$ & $\pi^{-}\pi^0\pi^0\nu_\tau$ & $\pi^{-}\pi^{+}\pi^{-}\nu_\tau$ & $\pi^{-}\pi^{+}\pi^{-}\pi^0\nu_\tau$& Other \\
\hline
$\pi^{-}\nu_\tau$ & 14.8\% &1.6\% &0.4\% &0.1\% &0.0\% & 0.7\% \\
$\pi^{-}\pi^0\nu_\tau$ & 6.0\% &17.1\% &9.0\% &0.1\% &0.1\% & 5.5\% \\
$\pi^{-}\pi^0\pi^0\nu_\tau$ & 0.9\% &3.8\% &4.2\% &0.0\% &0.1\% & 5.9\% \\
$\pi^{-}\pi^{+}\pi^{-}\nu_\tau$ & 0.8\% &0.3\% &0.1\% &9.7\% &1.6\% & 6.2\% \\
$\pi^{-}\pi^{+}\pi^{-}\pi^0\nu_\tau$ & 0.1\% &0.2\% &0.1\% &1.7\% &2.7\% & 4.5\% \\

\end{tabular}
\label{tab:dmResolutionNoNothing} \caption[Decay mode performance -- naive
reconstruction]{Decay mode correlation table for the selected dominant decay
modes for the naive approach.  The percentage in a given row and column
indicates the fraction of hadronic tau decays from
$Z\rightarrow\tau^{+}\tau^{-}$ events that are matched to a generator level
decay mode given by the row and are reconstructed with the decay mode given by
the column.  Entries in the "Other" column are immediately tagged as background.
}
\end{table}


\begin{table}[htp]
   \centering
   \begin{tabular}{l|cccccc}

\multirow{2}{*}{True decay mode} & \multicolumn{6}{c}{Reconstructed Decay Mode}\\

 & $\pi^{-}\nu_\tau$ & $\pi^{-}\pi^0\nu_\tau$ & $\pi^{-}\pi^0\pi^0\nu_\tau$ & $\pi^{-}\pi^{+}\pi^{-}\nu_\tau$ & $\pi^{-}\pi^{+}\pi^{-}\pi^0\nu_\tau$& Other \\
\hline
$\pi^{-}\nu_\tau$ & 16.2\% &1.0\% &0.1\% &0.1\% &0.0\% & 0.3\% \\
$\pi^{-}\pi^0\nu_\tau$ & 10.7\% &21.4\% &3.6\% &0.2\% &0.1\% & 1.9\% \\
$\pi^{-}\pi^0\pi^0\nu_\tau$ & 1.8\% &7.1\% &4.4\% &0.1\% &0.0\% & 1.5\% \\
$\pi^{-}\pi^{+}\pi^{-}\nu_\tau$ & 0.9\% &0.2\% &0.0\% &11.5\% &0.6\% & 5.4\% \\
$\pi^{-}\pi^{+}\pi^{-}\pi^0\nu_\tau$ & 0.1\% &0.3\% &0.0\% &3.2\% &2.9\% & 2.7\% \\

\end{tabular}
\label{tab:dmResolutionStandard} \caption[Decay mode performance -- TaNC
reconstruction]{Decay mode correlation table for the selected dominant decay
modes for the merging and filtering approach.  The percentage in a given row and
column indicates the fraction of hadronic tau decays from
$Z\rightarrow\tau^{+}\tau^{-}$ events that are matched to a generator level
decay mode given by the row and are reconstructed with the decay mode given by
the column.  Entries in the "Other" column are immediately tagged as background.
}
\end{table}


\begin{figure}[thbp]
   \setlength{\unitlength}{1mm}
   \begin{center}
      \begin{picture}(150, 75)(0,0)
         \put(0.5, 0) {\mbox{\includegraphics*[height=73mm]{tanc_chapter/figures/decayModeMergerAndFilter/dmResolutionStandard.pdf}}}
         % place holder
         \put(75.0, 0) {\mbox{\includegraphics*[height=73mm]{tanc_chapter/figures/decayModeMergerAndFilter/dmResolutionNoNothing.pdf}}}
      \end{picture}
   \caption[Tau decay mode reconstruction performance]{Correlations between
   reconstructed tau decay mode and true tau decay mode for hadronic tau decays
   in $Z \rightarrow \tau^{+}\tau^{-}$ events.  The correlation when no photon
   merging or filtering is applied is shown on the right, and the correlation
   for the algorithm described in section~\ref{sec:decay_mode_reco} is on the
   right.  The horizontal and vertical axis are the decay mode indices of the
   true and reconstructed decay mode, respectively.  The decay mode index
   $N_{DM}$ is defined as $N_{DM} = (N_{\pi^{\pm}} - 1)\cdot5 + N_{\pi^0}$.  The
   area of the box in each cell is proportional to the fraction of
   tau--candidates that were reconstructed with the decay mode indicated on the
   vertical axis for the true tau decay on the horizontal axis.  The performance
   of a decay mode reconstruction algorithm can be determined by the spread of
   the reconstructed number of $\pi^0$ mesons about the true number (the
   diagonal entries) determined from the generator level Monte Carlo
   information.  If the reconstruction was perfect, the correlation would be
   exactly diagonal.
   %The vertical slice corresponding to each true tau decay mode is normalized;
   %the cells in each column sum to one.
   } \label{fig:dmResolution}
   \end{center}
\end{figure}

\begin{figure}[thbp]
   \setlength{\unitlength}{1mm}
   \begin{center}
      \begin{picture}(150, 140)(0,0)
         \put(0.5, 0) {\mbox{\includegraphics*[height=70mm]{tanc_chapter/figures/dmVsPt.pdf}}}
         \put(0.5, 70) {\mbox{\includegraphics*[height=70mm]{tanc_chapter/figures/dmVsEta.pdf}}}
      \end{picture}
   \caption[Kinematic dependence of decay mode reconstruction]{Kinematic
   dependence of reconstructed decay mode for tau--candidates from
   $Z\rightarrow\tau^{+}\tau^{-}$ (left) and QCD di--jets (right) versus
   transverse momentum (top) and pseudo--rapidity (bottom).  Each curve is the
   probability for a tau--candidate to be reconstructed with the associated
   decay mode after the leading pion and decay mode preselection has been
   applied.  } \label{fig:dmKinematics}
   \end{center}
\end{figure}
