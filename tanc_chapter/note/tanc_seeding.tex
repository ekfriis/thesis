Tau candidates in all CMS tau identification algorithms are seeded by
reconstructed jets.  After particle-flow reconstruction, particle candidates
are clustered into collections by the iterativeCone5 jet algorithm. The
efficiencies for jet seed reconstruction are described in~\cite{CMS-PAS-PFT-08-001} and are near
unity for both signal and background.  Quality cuts (see
~\ref{table:pftau_q_cuts} are are then applied to the constituents in the jets.
The remaining particles are then considered a ``tau candidate.''  

\begin{table}
   \centering
   \caption{Quality cuts on tau seed jet constituents} 
   \begin{tabular}{l r}
      Particle flow charged hadron candidates & \\
      \hline
      Minimum transverse momentum & 0.5 GeV/$c$ \\
      Minimum tracker hits & 3 \\
      Maximum tracker $\chi^2$ & 100 \\
      Maximum track transverse impact parameter & 0.03cm \\
      \hline \hline
      Particle flow gamma candidates & \\
      \hline
      Minimum transverse energy & 0.5 GeV \\
   \end{tabular}
\end{table}

The constituents of tau candidate seed jets are then divided into two
collections, called ``signal'' and ``isolation'' collections as the method at
this stage is identical to the process for defining the isolation annulus.
