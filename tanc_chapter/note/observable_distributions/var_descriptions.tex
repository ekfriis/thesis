
\begin{description}
    
  \item[ChargedOutlierAngleN] \hfill \\
  $\Delta R$ between the Nth charged object (ordered by \pt) in the isolation region
    and the tau--candidate momentum axis. If the number of
    isolation region objects is less than N, the input is set at one.

  \item[ChargedOutlierPtN] \hfill \\
  Transverse momentum of the Nth charged object in the isolation region.  If the number of
    isolation region objects is less than N, the input is set at zero.

  \item[DalitzN] \hfill \\
   Invariant mass of four vector sum of the ``main track'' and the Nth signal
    region object. 

  \item[Eta] \hfill \\
  Pseudo-rapidity of the signal region objects. 

  \item[InvariantMassOfSignal] \hfill \\
  Invariant mass of the composite object formed by the signal region constituents.

  \item[MainTrackAngle] \hfill \\
  $\Delta R$ between the ``main track'' and the composite four--vector formed by the 
    signal region constituents.

  \item[MainTrackPt] \hfill \\
  Transverse momentum of the ``main track.'' 

  \item[OutlierNCharged] \hfill \\
  Number of charged objects in the isolation region.

  \item[OutlierSumPt] \hfill \\
  Sum of the transverse momentum of objects in the isolation region.

  \item[PiZeroAngleN] \hfill \\
  $\Delta R$ between the Nth $\pi^0$ object in the signal region (ordered by \pt) and
    the tau--candidate momentum axis.

  \item[PiZeroPtN] \hfill \\
  Transverse momentum of the Nth $\pi^0$ object in the signal region.

  \item[TrackAngleN] \hfill \\
  $\Delta R$ between the Nth charged object in the signal region (ordered by \pt) and
    the tau--candidate momentum axis, exclusive of the main track.

  \item[TrackPtN] \hfill \\
  Transverse momentum of the Nth charged object in the signal region, exclusive of the 
    main track.

\end{description}
    
