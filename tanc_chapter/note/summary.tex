The Tau Neural classifier introduces two complimentary new techniques for tau
lepton physics at CMS: reconstruction of the hadronic tau decay mode and
discrimination from quark and gluon jets using neural networks.  The decay mode
reconstruction strategy presented in Section~\ref{sec:decay_mode_reco}
significantly improves the determination of the decay mode. This information has
the potential to be useful in studies of tau polarization and background
estimation.

The Tau Neural classifier tau identification algorithm significantly improves
tau discrimination performance compared to isolation--based
approaches~\cite{CMS-PAS-PFT-08-001} used in previous CMS analyses.
Figure~\ref{fig:finalPerfCurve} compares the performance of the ``shrinking
cone'' isolation tau--identification algorithm~\cite{CMS-PAS-PFT-08-001} to the
performance of the TaNC for a scan of requirements on the transformed neural
network output.  The signal efficiency and QCD di--jet fake rate versus
tau candidate transverse momentum and pseudo--rapidity for the four benchmark
points and the isolation based tau identification are show in
Figure~\ref{fig:kinematicPerformance}.  For tau candidates with transverse
momentum between 20 and 50~\GeVc, the TaNC operating cut can be chosen such that
the two methods have identical signal efficiency; at this point the TaNC
algorithm reduces the background fake rate by an additional factor of 3.9.  This
reduction in background will directly improve the significance of searches for
new physics using tau leptons at CMS.

\begin{figure}[thbp]
   \setlength{\unitlength}{1mm}
   \begin{center}
      \begin{picture}(150, 150)(0,0)
         \put(0.5, 0.5)
         {\mbox{\includegraphics*[height=150mm]{tanc_chapter/figures/20_pt_50_perf_curve_from_5_pt_200_transform_plain_test_wrt_classic.pdf}}}
      \end{picture}
   \caption[Tau Neural Classifier performance comparison]{Performance curve (red) of the TaNC tau identification for various
   requirements on the output transformed according to
   Equation~\ref{eq:TransformCut}.  The horizontal axis is the efficiency for
   true taus with transverse momentum between 20 and 50~\GeVc to satisfy the tau
   identification requirements.  The vertical axis gives the rate at which QCD
   di--jets with generator--level transverse momentum between 20 and 50 \GeVc
   are incorrectly identified as taus.  The performance point for the same
   tau candidates using the isolation based tau--identification~\cite{CMS-PAS-PFT-08-001}
   used in many previous CMS analyses is indicated by the black star in the
   figure.  An additional requirement that the signal cone contain one or three
   charged hadrons (typical in a final physics analysis) has been applied to the
   isolation based tau--identification to ensure a conservative comparison.  }
   \label{fig:finalPerfCurve}
   \end{center}
\end{figure}


\begin{figure}[thbp]
   \setlength{\unitlength}{1mm}
   \begin{center}
      \begin{picture}(150, 150)(0,0)
         \put(2.5, 75)
         {\mbox{\includegraphics*[height=70mm]{tanc_chapter/figures/eff_background_pt.pdf}}}
         \put(75,  75)
         {\mbox{\includegraphics*[height=70mm]{tanc_chapter/figures/eff_background_eta.pdf}}}
         \put(2.5, 0)
         {\mbox{\includegraphics*[height=70mm]{tanc_chapter/figures/eff_signal_pt.pdf}}}
         \put(75, 0)
         {\mbox{\includegraphics*[height=70mm]{tanc_chapter/figures/eff_signal_eta.pdf}}}
      \end{picture}
   \caption[Tau Neural Classifier kinematic performance]{Comparison of the
   identification efficiency for hadronic tau decays from $Z \to \tau^{+}
   \tau^{-}$ decays (bottom row) and the misidentification rate for QCD di--jets
   (top row) versus tau candidate transverse momentum (left) and
   pseudo-rapidity (right) for different tau identification algorithms.  The
   efficiency (fake--rate) in a given bin is defined as the quotient of the
   number of true tau hadronic decays (generator level jets) in that bin that
   are matched to a reconstructed tau candidate that passes the identification
   algorithm divided by the number of true tau hadronic decays (generator level
   jets) in that bin.  In the low transverse momentum region both the number of
   tau candidates in the denominator and the algorithm acceptance vary rapidly
   with respect to $\pt$ for both signal and background; a minimum transverse
   momentum requirement of 20 \GeVc is applied to the pseudorapidity plots to
   facilitate interpretation of the plots.  } \label{fig:kinematicPerformance}
   \end{center}
\end{figure}

