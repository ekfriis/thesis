High tau identification performance is important for the discovery potential of
many possible new physics signals at the Compact Muon Solenoid (CMS).  The
Standard Model background rates from true tau leptons are typically the same
order of magnitude as the expected signal rate in many searches for new
physics.  The challenge of doing physics with taus is driven by the rate at
which objects are incorrectly tagged as taus.  In particular, quark and gluon
jets have a significantly higher production cross-section and events where
these objects are incorrectly identified as tau leptons can dominate the
backgrounds of searches for new physics using taus.  Efficient identification
of hadronic tau decays and low misidentification rate for quarks and gluons
is thus essential to maximize the significance of searches for new physics at
CMS.

%New physics signals may be discovered through tau lepton hadronic decay channels
%in early CMS data.  The tau lepton plays a paticularly important role in
%searches for Higgs bosons.  In the Minimal Supersymmetric Model (MSSM), the
%production cross--section is enhanced by the parameter $\tan\beta$.  The
%coupling of the MSSM Higgs to the tau lepton is also enchaced. \fixme(finish
%this)

%The tau plays a paticularly important role in the search for Higgs
%boson particle.  In the Standard Model (SM), the Higgs boson couplings to fermions
%are proportional to the fermion mass, which enhances the $H \rightarrow \tau^{+}
%\tau^{-}$ branching ratio relative to other leptonic decay modes.  For SM Higgs
%masses below the $W^{+}W^{-}$ and $ZZ$ production threshold, the SM Higgs decays
%to tau lepton pairs approximately 10\% of the time.  The significance of the tau
%is enhanced in the Minimal Supersymmetric Model (MSSM), where the MSSM Higgs
%coupling to the tau is enhanced by a factor of $\tan\beta$.

Tau leptons are unique in that they are the only type of leptons which are heavy
enough to decay to hadrons.  The hadronic decays compose approximately 65\% of
all tau decays, the remainder being split nearly evenly between $\tau^{-}
\rightarrow \mu^{-} \bar \nu_\mu \nu_\tau$ and $\tau^{-} \rightarrow e^{-} \bar
\nu_e \nu_\tau$.  The hadronic decays are typically composed of one or three charged
pions and zero to two neutral pions.  The neutral pions decay almost
instantaneously to pairs of photons.

In this note, we will describe a technique to identify hadronic tau decays.  Tau
decays to electrons and muons are difficult to distinguish from electrons and
muons produced in $pp$ collisions.  Analyses that use exclusively
non--hadronically decaying taus typically require that the leptonic ($e,\mu$)
decays be of opposite flavor.  The discrimination of hadronic tau decays from
electrons and muons is described in~\cite{CMS-PAS-PFT-08-001}. With the Tau Neural
Classifier, we aim to improve the discrimination of true hadronic tau decays
from quark and gluon jets using a neural network approach.

%associated with a collimated jet containing either one or three tracks
%reconstructed in the pixel and silicon strip tracker, plus a low number of
%neutral electromagnetic showers reconstructed in the calorimeter.
