Photons are merged into composite $\pi^0$ candidates by examining the invariant
mass of all possible pairs of photons in the signal region.  Only $\pi^0$
candidates (photon pairs) with a composite invariant mass less than 0.2 GeV/c
are considered. The combination of the high granularity of the CMS ECAL and the
particle flow algorithm provide excellent energy and angular resolution for
photons; the $\pi^0$ mass peak is readily visible in the invariant mass spectrum
of signal photon pairs (see figure~\ref{fig:imDiPhotonsForTrueDM1}).  The
$\pi^0$ candidates that satisfy the invariant mass requirement are ranked by the
difference between the composite invariant mass of the photon pair and the
invariant mass of the $\pi^0$ meson given by the PDG~\cite{PDG}. The best pairs
are then tagged as $\pi^0$ mesons, removing lower--ranking candidate $\pi^0$s as
necessary to ensure that no photon is included in more than one $\pi^0$ meson.

\begin{figure}[thbp]
   \begin{center}
         \includegraphics*[height=70mm]{tanc_chapter/figures/decayModeMergerAndFilter/invariantMassOfDiPhotonsForDM1.pdf}
         \caption{Invariant mass of the photon pair for reconstructed
         tau--candidates with two reconstructed photons in the signal region
         that are matched to generator level $\tau^{-} \to \pi^{-}\pi^0\nu_\tau$
         decays. }
   \end{center}
   \label{fig:imDiPhotonsForTrueDM1}
\end{figure}

