Photons from the underlying event and other reconstruction effects cause the
number of reconstructed photons to be greater than the true number of photons
expected from a given hadronic tau decay.  Photons that have not been merged into
a $\pi^0$ meson candidate are recursively filtered by requiring that the
fraction of the transverse momentum carried by the lowest \pt photon be
greater than 10\% with respect to the entire (tracks, $\pi^0$ candidates, and
photons) tau--candidate. In the case that a photon is not merged but meets the
minimum momentum fraction requirement, it is considered a $\pi^0$
candidate.  This requirement removes extraneous photons, while minimizing the
removal of single photons that correspond to a true $\pi^0$ meson
(see Figure~\ref{fig:photonFiltering}). A mass hypothesis with the
nominal~\cite{PDG} value of the $\pi^0$ is applied to all $\pi^0$ candidates.
All objects that fail the filtering requirements are moved to the isolation
collection.

\begin{figure}[thbp]
  \setlength{\unitlength}{1mm}
  \begin{center}
    \includegraphics*[width=0.49\textwidth]{tanc_chapter/figures/decayModeMergerAndFilter/gammaPtFractionSinglePhotonForDM0.pdf}
    \includegraphics*[width=0.49\textwidth]{tanc_chapter/figures/decayModeMergerAndFilter/gammaPtFractionSinglePhotonForDM1.pdf}
    \caption[Neutral energy fraction in visible $\tau$ decays]{Fraction of total
    $\tau$-candidate transverse momenta carried by the photon for reconstructed
    taus containing a single photons for two benchmark cases.  On the left, the
    reconstructed tau--candidate is matched to generator level \mbox{$\tau^{-}
    \rightarrow \pi^{-}\nu_\tau$} decays, for which no photon is expected.  On
    the right, the reconstructed tau--candidate is matched to generator level
    \mbox{$\tau^{-} \rightarrow \pi^{-}\pi^0\nu_\tau$} decays and the photon is
    expected to correspond to a true $\pi^0$ meson.  The requirement on the
    \pt fraction of the lowest $\pt$ photon improves the purity of the decay
    mode reconstruction.  } \label{fig:photonFiltering}
  \end{center}
\end{figure}

