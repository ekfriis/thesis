The TaNC algorithm has been optimized for the early stages of LHC operation.
The quality selections applied by the decay mode reconstruction algorithm will
need to be re--tuned and the neural networks retrained after each significant
change in instantaneous luminosity to remove the effects of multiple
interactions in the same bunch crossing, and after collision energy changes due
to the increased energy of the multiple scattering events.  

The dependence of the algorithm on the robustness of Monte
Carlo simulations can be minimized by using real events as the background
training sample.  As the QCD di--jet production rate is many orders of magnitude
larger than that of tau leptons, it is relatively straightforward to obtain a
pure background training sample with early CMS datasets.

The relatively long lifetime of the tau lepton ($c\tau = 87\mu$m) permits the
possibility that using displaced decay vertex information could provide
discrimination power against quark and gluon jets, which are expected to be
produced at the primary vertex.  A feasibility study is planned to determine if
adding vertex related observables to the neural networks could improve the
performance of the TaNC.
