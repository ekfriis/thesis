The major task in reconstructing the decay mode of the tau is determining the
number of $\pi^0$ mesons produced in the decay.  A $\pi^0$ meson decays almost
instantaneously to a pair of photons.  The photon objects are reconstructed using the
particle flow algorithm~\cite{CMS-PAS-PFT-09-001}. The initial collection of
photon objects considered to be $\pi^0$ candidates are the photons in the signal
cone described by using the ``shrinking--cone'' tau algorithm, described
in~\cite{CMS-PAS-PFT-08-001}.  

The reconstruction of photons from $\pi^0$ decays present in the signal cone is
complicated by a number of factors.  To suppress calorimeter noise and underlying
event photons, all photons with minimum transverse energy less than 0.5 GeV are
removed from the signal cone, which removes some signal photons.  Photons
produced in secondary interactions, pile-up events, and electromagnetic showers
produced by signal photons that convert to electron--positron pairs can
contaminate the signal cone with extra low transverse energy photons.  Highly
boosted $\pi^0$ mesons may decay into a pair of photons with a small opening
angle, resulting in two overlapping showers in the ECAL being reconstructed as
one photon.  The $\pi^0$ meson content of the tau--candidate is reconstructed in
two stages.  First, photon pairs are merged together into candidate $\pi^0$
mesons.  The remaining un--merged photons are then subject to a quality
requirement.
