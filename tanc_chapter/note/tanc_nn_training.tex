The samples used to train the TaNC neural networks are typical of the signals
and backgrounds found in common physics analyses using taus.  The signal--type
training sample is composed of reconstructed tau--candidates that are matched to
generator level hadronic tau decays coming from simulated \ZTT events.  The
background training sample consists of reconstructed tau--candidates in
simulated QCD $2\rightarrow2$ hard scattering events.  The QCD $\pt$ spectrum is
steeply falling, and to obtain sufficient statistics across a broad range of
$\pt$ the sample is split into different $\hat \pt$ bins.  Each binned QCD
sample imposes a generator level cut on the transverse momentum of the hard
interaction.  During the evaluation of discrimination performance the QCD
samples are weighted according to their respective integrated luminosities to
remove any effect of the binning.

The signal and background samples are split into five subsamples corresponding
to each reconstructed decay mode.  An additional selection is applied to each
subsample by requiring a ``leading pion'': either a charged hadron or gamma
candidate with transverse momentum greater than 5~\GeVc.  A large number of QCD
training events is required as both the leading pion selection and the
requirement that the decay mode match one of the dominant modes given in
Table~\ref{tab:decay_modes} are effective discriminants.  For each subsample,
80\% of the signal and background tau--candidates are used for training the
neural networks, with half (40\%) used as a
validation sample used to ensure the neural network is not over--trained. The
number of signal and background entries used for training and validation in each
decay mode subsample is given in Table~\ref{tab:trainingEvents}.

%Chained 100 signal files.
%Chained 208 background files.
%Total signal entries: 874266
%Total background entries: 9526176
%Pruning non-relevant entries.
%After pruning, 584895 signal and 644315 background entries remain.
%**********************************************************************************
%*********************************** Summary **************************************
%**********************************************************************************
%*     NumEvents with weight > 0 (Total NumEvents)                                *
%*--------------------------------------------------------------------------------*
%*shrinkingConePFTauDecayModeProducer   ThreeProngNoPiZero: Signal:      53257(53271)            Background:155793(155841)
%*shrinkingConePFTauDecayModeProducer  ThreeProngOnePiZero: Signal:      13340(13342)            Background:135871(135942)
%*shrinkingConePFTauDecayModeProducer    OneProngTwoPiZero: Signal:      34780(34799)            Background:51181(51337)
%*shrinkingConePFTauDecayModeProducer    OneProngOnePiZero: Signal:      136464(138171)          Background:137739(139592)
%*shrinkingConePFTauDecayModeProducer     OneProngNoPiZero: Signal:      300951(345312)          Background:144204(161603)

\begin{table}
   \centering
   \begin{tabular}{lcc}
      %\multirow{2}{*}{}                         & \multicolumn{2}{c}{Events}    \\
                                                & Signal        & Background    \\
      \hline
      Total number of tau--candidates           & 874266        & 9526176       \\
      Tau--candidates passing preselection      & 584895        & 644315        \\
      Tau--candidates with $W(\pt,\eta)>0$      & 538792        & 488917        \\
      \hline
      Decay Mode                        & \multicolumn{2}{c}{Training Events}   \\
      \hline
      $\pi^{-}$                         & 300951   & 144204                     \\
      $\pi^{-}\pi^0$                    & 135464   & 137739                     \\
      $\pi^{-}\pi^0\pi^0$               & 34780    & 51181                      \\
      $\pi^{-}\pi^{-}\pi^{+}$           & 53247    & 155793                     \\
      $\pi^{-}\pi^{-}\pi^{+}\pi^0$      & 13340    & 135871                     \\
   \end{tabular}
   \label{tab:trainingEvents} \caption[Neural network training event
   statistics]{Number of events used for neural network training and validation
   for each selected decay mode.}
\end{table}

The remaining 20\% of the signal and background samples are
reserved as a statistically independent sample to evaluate the performance of
the neural nets after the training is completed.  The TaNC uses the Multi--layer
Perceptron (MLP)
neural network implementation provided by the TMVA software package, described
in~\cite{TMVA}.  The MLP classifier is a feed-forward artificial neural
network. There are two layers of hidden nodes and a single node in the output
layer.  The hyperbolic tangent function is used for the neuron activation
function.  

The neural networks used in the TaNC have two hidden layers and single node in
the output layers.  The number of nodes in the first and second hidden layers
are chosen to be $N+1$ and $2N+1$, respectively, where $N$ is the number of
input observables for that neural network.  According to the Kolmogorov's
theorem~\cite{Kolmogorov}, any continuous function $g(x)$ defined on a vector
space of dimension $d$ spanned by $x$ can be represented by
\begin{equation}
   g(x) = \sum_{j=1}^{j=2d+1} \Phi_j \left(\sum_{i=1}^{d} \phi_i(x) \right)
   \label{eq:Kolmogorov}
\end{equation}
for suitably chosen functions for $\Phi_j$ and $\phi_j$.  As the form of
Equation~\ref{eq:Kolmogorov} is similar to the topology of a two hidden--layer
neural network, Kolmogorov's theorem suggests that \emph{any} classification
problem can be solved with a neural network with two hidden layers containing
the appropriate number of nodes.

The neural network is trained for 500 epochs. At ten epoch intervals, the
neural network error is computed using the validation sample to check for
over--training (see Figure~\ref{fig:overTrainCheck}). The neural network error
$E$ is defined~\cite{TMVA} as
\begin{equation}
   E = \frac{1}{2} \sum_{i=1}^N (y_{ANN,i} - \hat y_i)^2
   \label{eq:NNerrorFunc}
   %note - not right for weighted dists?
\end{equation}
where $N$ is the number of training events, $y_{ANN,i}$ is the neural network output
for the $i$th training event, and $y_i$ is the desired (-1 for background, 1 for signal) output
the $i$th event. No evidence  of over--training is observed.

\begin{figure}[thbp]
   \setlength{\unitlength}{1mm}
   \begin{center}
      \begin{picture}(150, 195)(0,0)
         \put(0.5, 130)
         {\mbox{\includegraphics*[height=60mm]{tanc_chapter/figures/overtrainCheck_OneProngNoPiZero.pdf}}}
         \put(65,  130)
         {\mbox{\includegraphics*[height=60mm]{tanc_chapter/figures/overtrainCheck_OneProngOnePiZero.pdf}}}
         \put(0.5, 65) 
         {\mbox{\includegraphics*[height=60mm]{tanc_chapter/figures/overtrainCheck_OneProngTwoPiZero.pdf}}}
         \put(65, 65) 
         {\mbox{\includegraphics*[height=60mm]{tanc_chapter/figures/overtrainCheck_ThreeProngNoPiZero.pdf}}}
         \put(33, 0) 
         {\mbox{\includegraphics*[height=60mm]{tanc_chapter/figures/overtrainCheck_ThreeProngOnePiZero.pdf}}}
      \end{picture}
   \caption[Neural network over--training validation plots]{Neural network
   classification error for training (solid red) and testing (dashed blue)
   samples at ten epoch intervals over the 500 training epochs for each decay
   mode neural network.  The vertical axis represents the classification error,
   defined by equation~\ref{eq:NNerrorFunc}.  N.B. that the choice of hyperbolic
   tangent for neuron activation functions results in the desired outputs for
   signal and background to be 1 and -1, respectively.  This results in the
   computed neural network error being larger by a factor of four than the case
   where the desired outputs are (0, 1).  Classifier over--training would be
   evidenced by divergence of the classification error of the training and
   testing samples, indicating that the neural net was optimizing about
   statistical fluctuations in the training sample.  }
   \label{fig:overTrainCheck}
   \end{center}
\end{figure}


The neural networks use as input observables the transverse momentum and $\eta$
of the tau--candidates.  These observables are included as their correlations
with other observables can increase the separation power of the ensemble of
observables.  For example, the opening angle in $\Delta R$ for signal
tau--candidates is inversely related to the transverse momentum, while for
background events the correlation is very small~\cite{DavisTau}. In the
training signal and background samples, there is significant discrimination
power in the $\pt$ spectrum.   However, it is desirable to eliminate any
systematic dependence of the neural network output on $\pt$ and $\eta$, as in
practice the TaNC will be presented with tau--candidates whose $\pt-\eta$
spectrum will be analysis dependent. The dependence on $\pt$ and $\eta$ is
removed by applying a $\pt$ and $\eta$ dependent weight to the tau--candidates
when training the neural nets.  

The weights are defined such that in any region in the vector space spanned by
$\pt$ and $\eta$ where the signal sample and background sample probability
density functions are different, the sample with higher probability density is
weighted such that the samples have identical $\pt-\eta$ probability
distributions.  This removes regions of $\pt-\eta$ space where the training
sample is exclusively signal or background.  The weights are computed according to
\begin{align*}
   W(\pt, \eta) &=  {\rm less}(p_{sig}(\pt, \eta), p_{bkg}(\pt, \eta))\\
   w_{sig}(\pt, \eta) &=  W(\pt, \eta)/p_{sig}(\pt, \eta) \\
   w_{bkg}(\pt, \eta) &=  W(\pt, \eta)/p_{bkg}(\pt, \eta) 
\end{align*}
where \mbox{$p_{sig}(\pt,\eta)$} and \mbox{$p_{bkg}(\pt,\eta)$} are the probability densities of
the signal and background samples after the ``leading pion'' and dominant decay mode
selections. Figure~\ref{fig:nnTrainingWeights} shows the signal and background
training $\pt$ distributions before and after the weighting is applied.


\begin{figure}[thbp]
\setlength{\unitlength}{1mm}
\begin{center}
\includegraphics*[width=0.49\textwidth]{tanc_chapter/figures/training_weights_unweighted.pdf}
\includegraphics*[width=0.49\textwidth]{tanc_chapter/figures/training_weights_weighted.pdf}
%\put(-5.5, 112.5){\small (a)}
%\put(72.0, 112.5){\small (b)}
%\put(-5.5, 54.5){\small (c)}
%\put(72.0, 54.5){\small (d)}
\caption[Kinematic weighting of training sample]{Transverse momentum spectrum of
signal and background tau--candidates used in neural net training before (left)
and after (right) the application of \mbox{$\pt-\eta$} dependent weight function.
Application of the weights lowers the training significance of tau--candidates
in regions of \mbox{$\pt-\eta$} phase space where either the signal or background
samples has an excess of events. } \label{fig:nnTrainingWeights}
\end{center}
\end{figure} 

