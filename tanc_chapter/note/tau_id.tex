The tau identification strategies used in previously published CMS analyses are
fully described in~\cite{PFT08001}.  A summary of the basic methods and
strategies is given here. There are two primary methods for selecting objects
used to reconstruct tau leptons.  The CaloTau algorithm uses tracks
reconstructed by the tracker and clusters of hits in the electromagnetic and
hadronic calorimeter.  The other method (PFTau) uses objects reconstructed by
the CMS particle flow algorithm, which is described in~\cite{PFT09}.  The
particle flow algorithm provides a global and unique description of every
particle (charged hadron, photon, electron, etc.) in the event; measurements
from sub--detectors are combined according to their measured resolutions to
improve energy and angular resolution and reduce double counting.  The
strategies described in this paper use the particle flow objects.

Both methods typically use an ``leading object'' and an isolation requirement to reject
quark and gluon jet background.  Quark and gluon jets are less collimated and
have a higher constituent multiplicity and softer constituent $p_T$ spectrum
than a hadronic tau decay of the same transverse momentum.  The ``leading
track'' requirement is applied by requiring a relatively high momentum object
near the center of the jet; typically a charged track with transverse momentum
greater than 5 GeV/c within $\Delta R < 0.1$ about the center of the jet axis.
The isolation requirement exploits the collimation of true taus by defining an
isolation annulus about the kinematic center of the jet and requiring no
detector activity about a threshold in that annulus.  This approach yields a
misidentification rate of approximately 1\% for QCD backgrounds and a hadronic
tau identification efficiency of approximately 50\%~\cite{PFT08001}.
