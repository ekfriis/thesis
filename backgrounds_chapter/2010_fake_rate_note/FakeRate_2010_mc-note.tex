%%%%%%%%%%%%%%%%%%%%%%%%%%%%%%%%%%%%%%%%%%%%%%%%%%%%%%%%%%%%%%%%%%%%
%
%  Common definitions
%
%  N.B. use of \providecommand rather than \newcommand means
%       that a definition is ignored if already specified
%
%                                              L. Taylor 18 Feb 2005
%%%%%%%%%%%%%%%%%%%%%%%%%%%%%%%%%%%%%%%%%%%%%%%%%%%%%%%%%%%%%%%%%%%%

% Some shorthand
% turn off italics
\newcommand {\etal}{\mbox{et al.}\xspace} %et al. - no preceding comma
\newcommand {\ie}{\mbox{i.e.}\xspace}     %i.e.
\newcommand {\eg}{\mbox{e.g.}\xspace}     %e.g.
\newcommand {\etc}{\mbox{etc.}\xspace}     %etc.
\newcommand {\vs}{\mbox{\sl vs.}\xspace}      %vs.
\newcommand {\mdash}{\ensuremath{\mathrm{-}}} % for use within formulas

% some terms whose definition we may change
\newcommand {\Lone}{Level-1\xspace} % Level-1 or L1 ?
\newcommand {\Ltwo}{Level-2\xspace}
\newcommand {\Lthree}{Level-3\xspace}

% Some software programs (alphabetized)
\providecommand{\ACERMC} {\textsc{AcerMC}\xspace}
\providecommand{\ALPGEN} {{\textsc{alpgen}}\xspace}
\providecommand{\CHARYBDIS} {{\textsc{charybdis}}\xspace}
\providecommand{\CMKIN} {\textsc{cmkin}\xspace}
\providecommand{\CMSIM} {{\textsc{cmsim}}\xspace}
\providecommand{\CMSSW} {{\textsc{cmssw}}\xspace}
\providecommand{\COBRA} {{\textsc{cobra}}\xspace}
\providecommand{\COCOA} {{\textsc{cocoa}}\xspace}
\providecommand{\COMPHEP} {\textsc{CompHEP}\xspace}
\providecommand{\EVTGEN} {{\textsc{evtgen}}\xspace}
\providecommand{\FAMOS} {{\textsc{famos}}\xspace}
\providecommand{\GARCON} {\textsc{garcon}\xspace}
\providecommand{\GARFIELD} {{\textsc{garfield}}\xspace}
\providecommand{\GEANE} {{\textsc{geane}}\xspace}
\providecommand{\GEANTfour} {{\textsc{geant4}}\xspace}
\providecommand{\GEANTthree} {{\textsc{geant3}}\xspace}
\providecommand{\GEANT} {{\textsc{geant}}\xspace}
\providecommand{\HDECAY} {\textsc{hdecay}\xspace}
\providecommand{\HERWIG} {{\textsc{herwig}}\xspace}
\providecommand{\HIGLU} {{\textsc{higlu}}\xspace}
\providecommand{\HIJING} {{\textsc{hijing}}\xspace}
\providecommand{\IGUANA} {\textsc{iguana}\xspace}
\providecommand{\ISAJET} {{\textsc{isajet}}\xspace}
\providecommand{\ISAPYTHIA} {{\textsc{isapythia}}\xspace}
\providecommand{\ISASUGRA} {{\textsc{isasugra}}\xspace}
\providecommand{\ISASUSY} {{\textsc{isasusy}}\xspace}
\providecommand{\ISAWIG} {{\textsc{isawig}}\xspace}
\providecommand{\MADGRAPH} {\textsc{MadGraph}\xspace}
\providecommand{\MCATNLO} {\textsc{mc@nlo}\xspace}
\providecommand{\MCFM} {\textsc{mcfm}\xspace}
\providecommand{\MILLEPEDE} {{\textsc{millepede}}\xspace}
\providecommand{\ORCA} {{\textsc{orca}}\xspace}
\providecommand{\OSCAR} {{\textsc{oscar}}\xspace}
\providecommand{\PHOTOS} {\textsc{photos}\xspace}
\providecommand{\PROSPINO} {\textsc{prospino}\xspace}
\providecommand{\PYTHIA} {{\textsc{pythia}}\xspace}
\providecommand{\SHERPA} {{\textsc{sherpa}}\xspace}
\providecommand{\TAUOLA} {\textsc{tauola}\xspace}
\providecommand{\TOPREX} {\textsc{TopReX}\xspace}
\providecommand{\XDAQ} {{\textsc{xdaq}}\xspace}


%  Experiments
\newcommand {\DZERO}{D\O\xspace}     %etc.


% Measurements and units...

\newcommand{\de}{\ensuremath{^\circ}}
\newcommand{\ten}[1]{\ensuremath{\times \text{10}^\text{#1}}}
\newcommand{\unit}[1]{\ensuremath{\text{\,#1}}\xspace}
\newcommand{\mum}{\ensuremath{\,\mu\text{m}}\xspace}
\newcommand{\micron}{\ensuremath{\,\mu\text{m}}\xspace}
\newcommand{\cm}{\ensuremath{\,\text{cm}}\xspace}
\newcommand{\mm}{\ensuremath{\,\text{mm}}\xspace}
\newcommand{\mus}{\ensuremath{\,\mu\text{s}}\xspace}
\newcommand{\keV}{\ensuremath{\,\text{ke\hspace{-.08em}V}}\xspace}
\newcommand{\MeV}{\ensuremath{\,\text{Me\hspace{-.08em}V}}\xspace}
\newcommand{\GeV}{\ensuremath{\,\text{Ge\hspace{-.08em}V}}\xspace}
\newcommand{\TeV}{\ensuremath{\,\text{Te\hspace{-.08em}V}}\xspace}
\newcommand{\PeV}{\ensuremath{\,\text{Pe\hspace{-.08em}V}}\xspace}
\newcommand{\keVc}{\ensuremath{{\,\text{ke\hspace{-.08em}V\hspace{-0.16em}/\hspace{-0.08em}}c}}\xspace}
\newcommand{\MeVc}{\ensuremath{{\,\text{Me\hspace{-.08em}V\hspace{-0.16em}/\hspace{-0.08em}}c}}\xspace}
\newcommand{\GeVc}{\ensuremath{{\,\text{Ge\hspace{-.08em}V\hspace{-0.16em}/\hspace{-0.08em}}c}}\xspace}
\newcommand{\TeVc}{\ensuremath{{\,\text{Te\hspace{-.08em}V\hspace{-0.16em}/\hspace{-0.08em}}c}}\xspace}
\newcommand{\keVcc}{\ensuremath{{\,\text{ke\hspace{-.08em}V\hspace{-0.16em}/\hspace{-0.08em}}c^\text{2}}}\xspace}
\newcommand{\MeVcc}{\ensuremath{{\,\text{Me\hspace{-.08em}V\hspace{-0.16em}/\hspace{-0.08em}}c^\text{2}}}\xspace}
\newcommand{\GeVcc}{\ensuremath{{\,\text{Ge\hspace{-.08em}V\hspace{-0.16em}/\hspace{-0.08em}}c^\text{2}}}\xspace}
\newcommand{\TeVcc}{\ensuremath{{\,\text{Te\hspace{-.08em}V\hspace{-0.16em}/\hspace{-0.08em}}c^\text{2}}}\xspace}

\newcommand{\pbinv} {\mbox{\ensuremath{\,\text{pb}^\text{$-$1}}}\xspace}
\newcommand{\fbinv} {\mbox{\ensuremath{\,\text{fb}^\text{$-$1}}}\xspace}
\newcommand{\nbinv} {\mbox{\ensuremath{\,\text{nb}^\text{$-$1}}}\xspace}
\newcommand{\percms}{\ensuremath{\,\text{cm}^\text{$-$2}\,\text{s}^\text{$-$1}}\xspace}
\newcommand{\lumi}{\ensuremath{\mathcal{L}}\xspace}
\newcommand{\Lumi}{\ensuremath{\mathcal{L}}\xspace}%both upper and lower
%
% Need a convention here:
\newcommand{\LvLow}  {\ensuremath{\mathcal{L}=\text{10}^\text{32}\,\text{cm}^\text{$-$2}\,\text{s}^\text{$-$1}}\xspace}
\newcommand{\LLow}   {\ensuremath{\mathcal{L}=\text{10}^\text{33}\,\text{cm}^\text{$-$2}\,\text{s}^\text{$-$1}}\xspace}
\newcommand{\lowlumi}{\ensuremath{\mathcal{L}=\text{2}\times \text{10}^\text{33}\,\text{cm}^\text{$-$2}\,\text{s}^\text{$-$1}}\xspace}
\newcommand{\LMed}   {\ensuremath{\mathcal{L}=\text{2}\times \text{10}^\text{33}\,\text{cm}^\text{$-$2}\,\text{s}^\text{$-$1}}\xspace}
\newcommand{\LHigh}  {\ensuremath{\mathcal{L}=\text{10}^\text{34}\,\text{cm}^\text{$-$2}\,\text{s}^\text{$-$1}}\xspace}
\newcommand{\hilumi} {\ensuremath{\mathcal{L}=\text{10}^\text{34}\,\text{cm}^\text{$-$2}\,\text{s}^\text{$-$1}}\xspace}

% Some usual physics terms


% SM (still to be classified)

\newcommand{\kt}{\ensuremath{k_{\mathrm{T}}}\xspace}
\newcommand{\BC}{\ensuremath{{B_{\mathrm{c}}}}\xspace}
\newcommand{\bbarc}{\ensuremath{{\overline{b}c}}\xspace}
\newcommand{\bbbar}{\ensuremath{{b\overline{b}}}\xspace}
\newcommand{\ccbar}{\ensuremath{{c\overline{c}}}\xspace}
\newcommand{\JPsi}{\ensuremath{{J}\hspace{-.08em}/\hspace{-.14em}\psi}\xspace}
\newcommand{\bspsiphi}{\ensuremath{B_s \to \JPsi\, \phi}\xspace}
%\newcommand{\ttbar}{\ensuremath{{t\overline{t}}}\xspace}
\newcommand{\AFB}{\ensuremath{A_\text{FB}}\xspace}
\newcommand{\EE}{\ensuremath{e^+e^-}\xspace}
\newcommand{\MM}{\ensuremath{\mu^+\mu^-}\xspace}
\newcommand{\TT}{\ensuremath{\tau^+\tau^-}\xspace}
\newcommand{\wangle}{\ensuremath{\sin^{2}\theta_{\text{eff}}^\text{lept}(M^2_\mathrm{Z})}\xspace}
\newcommand{\ttbar}{\ensuremath{{t\overline{t}}}\xspace}
\newcommand{\stat}{\ensuremath{\,\text{(stat.)}}\xspace}
\newcommand{\syst}{\ensuremath{\,\text{(syst.)}}\xspace}
% these moved to similar defs
%\newcommand{\Etmiss}{\ensuremath{E_{\mathrm{T}\!{\rm miss}}}}
%\newcommand{\VEtmiss}{\ensuremath{{\vec E}_{\mathrm{T}\!{\rm miss}}}}

%%%  E-gamma definitions
\newcommand{\HGG}{\ensuremath{\mathrm{H}\to\gamma\gamma}}
\newcommand{\gev}{\GeV}
\newcommand{\GAMJET}{\ensuremath{\gamma + \text{jet}}}
\newcommand{\PPTOJETS}{\ensuremath{\mathrm{pp}\to\text{jets}}}
\newcommand{\PPTOGG}{\ensuremath{\mathrm{pp}\to\gamma\gamma}}
\newcommand{\PPTOGAMJET}{\ensuremath{\mathrm{pp}\to\gamma +
\mathrm{jet}
}}
\newcommand{\MH}{\ensuremath{\mathrm{M_{\mathrm{H}}}}}
\newcommand{\RNINE}{\ensuremath{\mathrm{R}_\mathrm{9}}}
\newcommand{\DR}{\ensuremath{\Delta\mathrm{R}}}



% Physics symbols ...

\newcommand{\PT}{\ensuremath{p_{\mathrm{T}}}\xspace}
\newcommand{\pt}{\ensuremath{p_{\mathrm{T}}}\xspace}
\newcommand{\ET}{\ensuremath{E_{\mathrm{T}}}\xspace}
\newcommand{\HT}{\ensuremath{H_{\mathrm{T}}}\xspace}
\newcommand{\et}{\ensuremath{E_{\mathrm{T}}}\xspace}
\providecommand{\Em}{\ensuremath{E\hspace{-0.6em}/}\xspace}
\providecommand{\Pm}{\ensuremath{p\hspace{-0.5em}/}\xspace}
\providecommand{\PTm}{\ensuremath{{p}_\mathrm{T}\hspace{-1.02em}/}\xspace}
\providecommand{\PTslash}{\ensuremath{{p}_\mathrm{T}\hspace{-1.02em}/}\xspace}
\newcommand{\ETm}{\ensuremath{E_{\mathrm{T}}^{\text{miss}}}\xspace}
\providecommand{\ETslash}{\ensuremath{E_{\mathrm{T}}\hspace{-1.1em}/}\xspace}
\newcommand{\MET}{\ensuremath{E_{\mathrm{T}}^{\text{miss}}}\xspace}
\newcommand{\ETmiss}{\ensuremath{E_{\mathrm{T}}^{\text{miss}}}\xspace}
\newcommand{\VEtmiss}{\ensuremath{{\vec E}_{\mathrm{T}}^{\text{miss}}}\xspace}

% roman face derivative
\providecommand{\dd}[2]{\ensuremath{\frac{\mathrm{d} #1}{\mathrm{d} #2}}}
% Particle names which will track the italic/non-italic face convention
\providecommand{\zp}{\ensuremath{\mathrm{Z}^\prime}\xspace}
\providecommand{\JPsi}{\ensuremath{\mathrm{J}\hspace{-.08em}/\hspace{-.14em}\psi}\xspace}
\providecommand{\Z}{\ensuremath{\mathrm{Z}}\xspace}
\providecommand{\ttbar}{\ensuremath{\mathrm{t}\overline{\mathrm{t}}}\xspace}
% Extensions for missing names in PENNAMES
\newcommand{\cPgn}{\ensuremath{\nu}}
\newcommand{\cPJgy}{\JPsi}
\newcommand{\cPZ}{\Z}
\newcommand{\cPZpr}{\zp}
%%%%%%
% From Albert
%

\newcommand{\ga}{\ensuremath{\gtrsim}}
\newcommand{\la}{\ensuremath{\lesssim}}
%\def\ga{\mathrel{\rlap{\raise.6ex\hbox{$>$}}{\lower.6ex\hbox{$\sim$}}}}
%\def\la{\mathrel{\rlap{\raise.6ex\hbox{$<$}}{\lower.6ex\hbox{$\sim$}}}}
%
\newcommand{\swsq}{\ensuremath{\sin^2\theta_W}\xspace}
\newcommand{\cwsq}{\ensuremath{\cos^2\theta_W}\xspace}
\newcommand{\tanb}{\ensuremath{\tan\beta}\xspace}
\newcommand{\tanbsq}{\ensuremath{\tan^{2}\beta}\xspace}
\newcommand{\sidb}{\ensuremath{\sin 2\beta}\xspace}
\newcommand{\alpS}{\ensuremath{\alpha_S}\xspace}
\newcommand{\alpt}{\ensuremath{\tilde{\alpha}}\xspace}

\newcommand{\QL}{\ensuremath{Q_L}\xspace}
\newcommand{\sQ}{\ensuremath{\tilde{Q}}\xspace}
\newcommand{\sQL}{\ensuremath{\tilde{Q}_L}\xspace}
\newcommand{\ULC}{\ensuremath{U_L^C}\xspace}
\newcommand{\sUC}{\ensuremath{\tilde{U}^C}\xspace}
\newcommand{\sULC}{\ensuremath{\tilde{U}_L^C}\xspace}
\newcommand{\DLC}{\ensuremath{D_L^C}\xspace}
\newcommand{\sDC}{\ensuremath{\tilde{D}^C}\xspace}
\newcommand{\sDLC}{\ensuremath{\tilde{D}_L^C}\xspace}
\newcommand{\LL}{\ensuremath{L_L}\xspace}
\newcommand{\sL}{\ensuremath{\tilde{L}}\xspace}
\newcommand{\sLL}{\ensuremath{\tilde{L}_L}\xspace}
\newcommand{\ELC}{\ensuremath{E_L^C}\xspace}
\newcommand{\sEC}{\ensuremath{\tilde{E}^C}\xspace}
\newcommand{\sELC}{\ensuremath{\tilde{E}_L^C}\xspace}
\newcommand{\sEL}{\ensuremath{\tilde{E}_L}\xspace}
\newcommand{\sER}{\ensuremath{\tilde{E}_R}\xspace}
\newcommand{\sFer}{\ensuremath{\tilde{f}}\xspace}
\newcommand{\sQua}{\ensuremath{\tilde{q}}\xspace}
\newcommand{\sUp}{\ensuremath{\tilde{u}}\xspace}
\newcommand{\suL}{\ensuremath{\tilde{u}_L}\xspace}
\newcommand{\suR}{\ensuremath{\tilde{u}_R}\xspace}
\newcommand{\sDw}{\ensuremath{\tilde{d}}\xspace}
\newcommand{\sdL}{\ensuremath{\tilde{d}_L}\xspace}
\newcommand{\sdR}{\ensuremath{\tilde{d}_R}\xspace}
\newcommand{\sTop}{\ensuremath{\tilde{t}}\xspace}
\newcommand{\stL}{\ensuremath{\tilde{t}_L}\xspace}
\newcommand{\stR}{\ensuremath{\tilde{t}_R}\xspace}
\newcommand{\stone}{\ensuremath{\tilde{t}_1}\xspace}
\newcommand{\sttwo}{\ensuremath{\tilde{t}_2}\xspace}
\newcommand{\sBot}{\ensuremath{\tilde{b}}\xspace}
\newcommand{\sbL}{\ensuremath{\tilde{b}_L}\xspace}
\newcommand{\sbR}{\ensuremath{\tilde{b}_R}\xspace}
\newcommand{\sbone}{\ensuremath{\tilde{b}_1}\xspace}
\newcommand{\sbtwo}{\ensuremath{\tilde{b}_2}\xspace}
\newcommand{\sLep}{\ensuremath{\tilde{l}}\xspace}
\newcommand{\sLepC}{\ensuremath{\tilde{l}^C}\xspace}
\newcommand{\sEl}{\ensuremath{\tilde{e}}\xspace}
\newcommand{\sElC}{\ensuremath{\tilde{e}^C}\xspace}
\newcommand{\seL}{\ensuremath{\tilde{e}_L}\xspace}
\newcommand{\seR}{\ensuremath{\tilde{e}_R}\xspace}
\newcommand{\snL}{\ensuremath{\tilde{\nu}_L}\xspace}
\newcommand{\sMu}{\ensuremath{\tilde{\mu}}\xspace}
\newcommand{\sNu}{\ensuremath{\tilde{\nu}}\xspace}
\newcommand{\sTau}{\ensuremath{\tilde{\tau}}\xspace}
\newcommand{\Glu}{\ensuremath{g}\xspace}
\newcommand{\sGlu}{\ensuremath{\tilde{g}}\xspace}
\newcommand{\Wpm}{\ensuremath{W^{\pm}}\xspace}
\newcommand{\sWpm}{\ensuremath{\tilde{W}^{\pm}}\xspace}
\newcommand{\Wz}{\ensuremath{W^{0}}\xspace}
\newcommand{\sWz}{\ensuremath{\tilde{W}^{0}}\xspace}
\newcommand{\sWino}{\ensuremath{\tilde{W}}\xspace}
\newcommand{\Bz}{\ensuremath{B^{0}}\xspace}
\newcommand{\sBz}{\ensuremath{\tilde{B}^{0}}\xspace}
\newcommand{\sBino}{\ensuremath{\tilde{B}}\xspace}
\newcommand{\Zz}{\ensuremath{Z^{0}}\xspace}
\newcommand{\sZino}{\ensuremath{\tilde{Z}^{0}}\xspace}
\newcommand{\sGam}{\ensuremath{\tilde{\gamma}}\xspace}
\newcommand{\chiz}{\ensuremath{\tilde{\chi}^{0}}\xspace}
\newcommand{\chip}{\ensuremath{\tilde{\chi}^{+}}\xspace}
\newcommand{\chim}{\ensuremath{\tilde{\chi}^{-}}\xspace}
\newcommand{\chipm}{\ensuremath{\tilde{\chi}^{\pm}}\xspace}
\newcommand{\Hone}{\ensuremath{H_{d}}\xspace}
\newcommand{\sHone}{\ensuremath{\tilde{H}_{d}}\xspace}
\newcommand{\Htwo}{\ensuremath{H_{u}}\xspace}
\newcommand{\sHtwo}{\ensuremath{\tilde{H}_{u}}\xspace}
\newcommand{\sHig}{\ensuremath{\tilde{H}}\xspace}
\newcommand{\sHa}{\ensuremath{\tilde{H}_{a}}\xspace}
\newcommand{\sHb}{\ensuremath{\tilde{H}_{b}}\xspace}
\newcommand{\sHpm}{\ensuremath{\tilde{H}^{\pm}}\xspace}
\newcommand{\hz}{\ensuremath{h^{0}}\xspace}
\newcommand{\Hz}{\ensuremath{H^{0}}\xspace}
\newcommand{\Az}{\ensuremath{A^{0}}\xspace}
\newcommand{\Hpm}{\ensuremath{H^{\pm}}\xspace}
\newcommand{\sGra}{\ensuremath{\tilde{G}}\xspace}
%
\newcommand{\mtil}{\ensuremath{\tilde{m}}\xspace}
%
\newcommand{\rpv}{\ensuremath{\rlap{\kern.2em/}R}\xspace}
\newcommand{\LLE}{\ensuremath{LL\bar{E}}\xspace}
\newcommand{\LQD}{\ensuremath{LQ\bar{D}}\xspace}
\newcommand{\UDD}{\ensuremath{\overline{UDD}}\xspace}
\newcommand{\Lam}{\ensuremath{\lambda}\xspace}
\newcommand{\Lamp}{\ensuremath{\lambda'}\xspace}
\newcommand{\Lampp}{\ensuremath{\lambda''}\xspace}
%
\newcommand{\spinbd}[2]{\ensuremath{\bar{#1}_{\dot{#2}}}\xspace}

\newcommand{\MD}{\ensuremath{{M_\mathrm{D}}}\xspace}% ED mass
\newcommand{\Mpl}{\ensuremath{{M_\mathrm{Pl}}}\xspace}% Planck mass
\newcommand{\Rinv} {\ensuremath{{R}^{-1}}\xspace}



%%%%%%%%%%%%%%%%%%%%%%%%%%%%%%%%%%%%%%%%%%%%%%%%%%%%%%%%%%%%%%%%%%%%
%
% Hyphenations (only need to add here if you get a nasty word break)
%
\hyphenation{en-viron-men-tal}%    just an example


%%%%%%%%%%%%%%%%%%%%%%%%%%%%%%
% custom definitions
%%%%%%%%%%%%%%%%%%%%%%%%%%%%%%
\newcommand{\tablesize}{\small}
%\newcommand{\captiontext}{\small \it}
%\newcommand{\captiontext}{\it}
\newcommand{\captiontext}{}

\cmsNoteHeader{AN--XX/2010}

\title{Estimation of Background contributions to Tau analyses via Fake--Rate technique}

\address[ucdavis]{
 \\
  Department of Physics \\
  University of California \\ 
  One Shields Avenue \\
  Davis, CA 95616--8677
}

\author[ucdavis]{J.~Conway}
\author[ucdavis]{E.~K.~Friis}
\author[ucdavis]{C.~Veelken}

% please supply the date in yyyy/mm/dd format. Today has been
% redefined to do so, but it should be fixed as of the final release date.
\date{\today}

\abstract{
  The fake--rate technique provides a data--driven way of estimating background contributions
  in physics analyses involving tau leptons in the final state.
  The technique is applicable in any analysis in which dominant background contributions 
  are expected to arise from the misidentification of quark or gluon jets as hadronic tau decays.
  In this note, we provide a detailed description of the method and demonstrate its applicability
  to estimate background contributions using the analysis of 
  $Z \rightarrow \tau^{+} \tau^{-} \rightarrow \mu + \tau\mbox{--jet}$ by way of example.
}

\hypersetup{
  pdfauthor={John Conway, Evan Klose Friis, Christian Veelken},
  pdftitle={CMS Physics Analysis Summary}, 
  pdfsubject={CMS},
  pdfkeywords={CMS, physics, software, background estimation}
}

\begin{document}

\maketitle 

\section{Introduction}

Many analyses of interest for discovering new physics phenomena at the LHC are based on the capability 
of the detectors to reconstruct and identify tau leptons.
In about two thirds of the cases tau leptons decay hadronically,
typically into either one or three charged mesons (predominantly $\pi^{+}$,
$\pi^{-}$) plus zero to two neutral pions.
The experimental challenge in reconstructing and identifying hadronic tau decays
is to discriminate efficiently between tau--jets on the one hand and quark/gluon jets on the other hand.
Details of the algorithm used to identify and reconstruct tau lepton hadronic decays can be found in~\cite{PFlowTauReco}.

Since quark and gluon jets are produced with cross--sections which are in general some orders of magnitude larger
than the cross--sections with which tau leptons are produced, significant background contributions to physics analyses
involving tau leptons in the final state may arise from the misidentification of quark and gluon jets as tau--jets,
even if the probabilities of such misidentification (``fake--rate'') per jet 
is on the 10$^{-2}$ to 10$^{-3}$ level~\cite{PFlowTauReco, TaNCnote}.

In this note, we describe how knowledge of the probabilities with which quark and gluon jets get misidentified as tau--jets
may be utilized to obtain an estimate of background contributions in physics analyses.
As an illustrative example and in order to demonstrate the precision achievable with the method,
we present results of applying the fake--rate technique to estimate
the contributions of QCD, $W$+jets, $t\bar{t}$+jets and
$Z \rightarrow \mu^{+} \mu^{-}$ backgrounds in the measurement of the $Z \rightarrow \tau^{+} \tau^{-}$ cross--section,
in the channel $Z \rightarrow \tau^{+} \tau^{-} \rightarrow \mu + \tau\mbox{-jet}$.
Details of the analysis can be found in reference~\cite{EWKtauNote}.

The results described in this note were obtained from Monte Carlo simulations 
of the $Z \rightarrow \tau^{+} \tau^{-}$ signal and different background processes
for a centre--of--mass energy of $\sqrt{s} = 7$~TeV.
Analysis of the $\sqrt{s} = 900$~GeV data recorded in 2009~\cite{refFakeRate2009dataResults}
indicate that the probabilities for quark and gluon jets 
to fake the signatures of hadronic tau decays are well modeled by the Monte Carlo simulation.
Once data--samples of sufficient event statistics are available at collision energies of $\sqrt{s} = 7$~TeV,
fake--rates at the higher centre--of--mass energy will be measured in data
and the values obtained from data will henceforth be used for the purpose of estimating background contributions
via the fake--rate technique.


\section{Parametrization of fake--rates}
\label{secFakeRateParametrization}

\begin{figure}[t]
\setlength{\unitlength}{1mm}
\begin{center}
\begin{picture}(150,112)(0,0)
\put(10.5, 60){\mbox{\includegraphics*[height=52mm]{figures/AN2008_043_fig20left.pdf}}}
\put(86.0, 60){\mbox{\includegraphics*[height=52mm]{figures/AN2008_043_fig20right.pdf}}}
\put(10.5, 2){\mbox{\includegraphics*[height=52mm]{figures/AN2008_043_fig22left.pdf}}}
\put(86.0, 2){\mbox{\includegraphics*[height=52mm]{figures/AN2008_043_fig22right.pdf}}}
%\put(-5.5, 112.5){\small (a)}
%\put(72.0, 112.5){\small (b)}
%\put(-5.5, 54.5){\small (c)}
%\put(72.0, 54.5){\small (d)}
\end{picture}
\caption{\captiontext Cumulative efficiencies (left) and fake--rates (right) of successively applied tau identification cuts 
         of the ``shrinking signal cone'' particle--flow based tau identification algorithm described in~\cite{PFlowTauReco}
         as function of $P_{T}^{jet}$ (top) and $\eta_{jet}$ (bottom) of tau--jet candidates.
	 The efficiencies/fake--rates for the complete set of tau identification criteria 
         are represented by the blue (downwards facing) triangles.}
\label{figPFTauReco_EfficienciesAndFakeRates}
\end{center}
\end{figure} 

Efficiencies and fake--rates of the tau identification algorithm based 
on requiring no tracks of $P_{T} > 1$~GeV and ECAL energy deposits of $P_{T} > 1.5$~GeV
reconstructed within an ``isolation cone'' of size $dR_{iso} = 0.5$ 
and outside of a ``shrinking signal cone'' of size $dR_{sig} = 5.0 / E_{T}$ 
as it is used in the $Z \rightarrow \tau^{+} \tau^{-} \rightarrow \mu + \tau\mbox{-jet}$ analysis 
are displayed in figure~\ref{figPFTauReco_EfficienciesAndFakeRates}.
In order to account for the visible $P_{T}$ and $\eta$ dependence,
we parametrize the fake--rates in bins of transverse momentum and pseudo--rapidity.
As we will show in section~\ref{secFakeRateApplication},
the parametrization of the fake--rates by $P_{T}$ and $\eta$ makes it possible 
to not only estimate the total number of background events contributing to physics analyses,
but to model the distributions of kinematic observables with a precision
that is sufficient to extract information on the background shape.

We add a third quantity, the $E_{T}$-weighted jet--width $R_{jet}$, to the
parametrization in order to account for differences between the fake--rates of
quark and gluon jets.  The jet width is defined as 
\begin{equation}
R_{jet} = \sqrt{E \left( \eta^2 \right) + E \left( \phi^2 \right)}
\end{equation}
where $E \left( \eta^2 \right)$ ($E \left( \phi^2 \right)$) is the second $\eta$ ($\phi$) moment of the jet constituents, weighted by
constituent transverse energy. Analyses performed by the CDF
collaboration~\cite{CDFtypeFakeRateMethod} found that systematic uncertainties
on background estimates obtained from the fake--rate method are reduced in case
differences between quark and gluon jets get accounted for in this way.

Efficiencies and fake--rates are then obtained by counting the fraction of tau--jet candidates
passing all tau identification cuts and discriminators
in a given bin of $P_{T}^{jet}$, $\eta_{jet}$ and $R_{jet}$:
\begin{equation}
P_{fr} \left( P_{T}^{jet}, \eta_{jet}, R_{jet} \right) := 
  \frac{N_{jets} \left( P_{T}^{jet}, \eta_{jet}, R_{jet} \vert \mbox{all tau id. cuts and discriminators passed} \right)}
       {N_{jets} \left( P_{T}^{jet}, \eta_{jet}, R_{jet} \vert \mbox{preselection passed} \right)}
\label{eqBgEstFakeRate_fr}
\end{equation}
The preselection in the denominator of equation~\ref{eqBgEstFakeRate_fr} in general refers to $P_{T}$ and $\eta$ cuts,
which are applied with thresholds matching those applied on the final analysis level,
but may as well include loose tau identification criteria (which may be applied e.g. already during event skimming).

Different sets of fake--rates are determined for the highest $P_{T}$ and for the second highest $P_{T}$ jet in QCD di--jet events,
for jets in a QCD event sample enriched by the contribution of heavy quarks and gluons 
by requiring the presence of a muon reconstructed in the final state,
and for jets in ``electroweak'' events selected by requiring a $W$ boson in the final state.

Tau identification efficiencies are obtained from a $Z \rightarrow \tau^{+} \tau^{-}$ Monte Carlo sample.


\section{Application to background estimation}
\label{secFakeRateApplication}

Knowledge of the tau identification efficiencies and fake--rates as function of the parameters $P_{T}^{jet}$, $\eta_{jet}$ and $R_{jet}$ 
as defined by equation~\ref{eqBgEstFakeRate_fr} is utilized to obtain an estimate for the contributions of background processes
to physics analyses involving tau lepton hadronic decays in the final state.
The basic idea is to replace tau identification cuts and discriminators by appropriately chosen weights.

Application of the fake--rate technique consists of two stages.
The first stage consists of loosening the tau identification cuts and discriminators and applying only
the preselection requirements defined by the denominator of equation~\ref{eqBgEstFakeRate_fr},
in order to obtain an event sample dominated by contributions of background processes,
which are expected to increase by the inverse of the (average) fake--rate, typically by a factor $\mathcal{O} \left( 100 \right)$.
In the second stage, weights are applied to all events in the background dominated control sample,
according to the probabilities $P_{fr} \left( P_{T}^{jet}, \eta_{jet}, R_{jet} \right)$ for jets 
to fake the signature of a hadronic tau decay.
After application of the weights, an estimate for the total number of background events
passing the tau identification cuts and discriminators and thus contributing to the final analysis sample is obtained.

The fake--rate technique works best if all background contributions to the analysis 
arise from misidentification of quark and gluon jets as hadronic tau decays. 
Corrections to the estimate obtained from the fake--rate technique  
are needed in case of background processes contributing to the final analysis sample
which either produce genuine tau leptons in the final state (e.g. $t\bar{t}$+ jets)
or in which tau--jet candidates are due to misidentified electrons or muons
(e.g. $Z \rightarrow \mu^{+} \mu^{-}$, $Z \rightarrow e^{+} e^{-}$),
as the latter may fake signatures of hadronic tau decays with very different probabilities than quark and gluon jets.

In the ``simple'' fake--rate method described in more detail in the next section, 
the corrections are taken from Monte Carlo simulations.
Corrections based on Monte Carlo are needed also to compensate for signal contributions to the background dominated control sample.

An alternative to Monte Carlo based corrections is to utilize additional information 
contained in the background dominated control sample.
The modified version is described in section~\ref{secBgEstFakeRate_frCDFtypeWeights}.
It has been used to estimate background contributions in searches for Higgs boson production
with subsequent decays into tau lepton pairs performed by the CDF collaboration in TeVatron run $II$ data~\cite{CDFtypeFakeRateMethod}.
We will refer to the modified version as ``CDF--type'' method in the following.


\subsection{``Simple'' weight method}

In the ``simple'' method all tau--jet candidates within the background dominated event sample are weighted
by the probabilities of quark and gluon jets to fake the signature of a hadronic tau decay:
\begin{equation}
w_{jet}^{simple} \left( P_{T}^{jet}, \eta_{jet}, R_{jet} \right) := P_{fr} \left( P_{T}^{jet}, \eta_{jet}, R_{jet} \right)
\label{eqBgEstFakeRate_frSimpleJetWeight}
\end{equation}
These weights are applied to all jets in the background dominated control sample which pass the preselection defined by 
the denominator of equation~\ref{eqBgEstFakeRate_fr}.
Note that the weights defined by equation~\ref{eqBgEstFakeRate_frSimpleJetWeight} can be used to estimate
the contributions of background processes to distributions of tau--jet related observables.
They cannot be used as event weights.

In order to compare distributions of event level quantities or per--particle quantities for 
particles of types different from tau leptons decaying hadronically, event weights need to be defined.
Neglecting the small fraction of background events in which multiple tau--jet candidates pass the complete set 
of all tau identification cuts and discriminators, 
event weights can be computed by summing up the per--jet weights defined by equation~\ref{eqBgEstFakeRate_frSimpleJetWeight}
over all tau--jet candidates in the event which pass the preselection:
\begin{equation}
W_{event}^{simple} := \Sigma w_{jet}^{simple}
\label{eqBgEstFakeRate_frSimpleEventWeight}
\end{equation}

A bit of care is needed in case one wants to compare distributions of observables related to ``composite particles'' 
the multiplicity of which depends on the multiplicity of tau--jet candidates in the event 
(e.g. combinations of muon + tau--jet pairs in case of the $Z \rightarrow \tau^{+} \tau^{-} \rightarrow \mu + \tau\mbox{--jet}$ analysis).
Per--particle weights need to be computed for such ``composite particles'', 
depending on $P_{T}^{jet}$, $\eta_{jet}$, $R_{jet}$ of its tau--jet candidate constituent, 
according to:
\begin{equation}
w_{comp-part}^{simple} \left( P_{T}^{jet}, \eta_{jet}, R_{jet} \right) := 
  w_{jet}^{simple} \left( P_{T}^{jet}, \eta_{jet}, R_{jet} \right)
\label{eqBgEstFakeRate_frSimpleCompositeParticleWeight}
\end{equation}

\begin{figure}[t]
\setlength{\unitlength}{1mm}
\begin{center}
\begin{picture}(150,112)(0,0)
\put(0.5, 60){\mbox{\includegraphics*[height=52mm, viewport=23 25 525 404]{figures/plotBgEstFakeRateZtoMuTau_WplusJets_frSimpleMuonPt.pdf}}}
\put(78.0, 60){\mbox{\includegraphics*[height=52mm, viewport=23 25 525 404]{figures/plotBgEstFakeRateZtoMuTau_QCD_frSimpleMuonPt.pdf}}}
\put(0.5, 2){\mbox{\includegraphics*[height=52mm, viewport=23 25 525 404]{figures/plotBgEstFakeRateZtoMuTau_TTplusJets_frSimpleMuonPt.pdf}}}
\put(78.0, 2){\mbox{\includegraphics*[height=52mm, viewport=23 25 525 404]{figures/plotBgEstFakeRateZtoMuTau_Zmumu_frSimpleMuonPt.pdf}}}
%\put(-5.5, 112.5){\small (a)}
%\put(72.0, 112.5){\small (b)}
%\put(-5.5, 54.5){\small (c)}
%\put(72.0, 54.5){\small (d)}
\end{picture}
\caption{\captiontext Distributions of muon transverse momentum in
         $W$+jets (top left), QCD (top right), $t\bar{t}$+jets (bottom left) 
         and $Z \rightarrow \mu^{+} \mu^{-}$ (bottom right) background events which pass all selection criteria 
         of the $Z \rightarrow \tau^{+} \tau^{-} \rightarrow \mu + \tau\mbox{-jet}$ cross--section analysis
	 compared to the estimate obtained from the fake--rate technique, 
         computed according to equation~\ref{eqBgEstFakeRate_frSimpleEventWeight}.
	 The expected contribution of background processes is indicated by points.
	 Lines of different colors represent the estimates obtained by applying 
         fake--rate weights determined for different compositions of light quark, heavy quark and gluon jets,
         as described in section~\ref{secFakeRateParametrization}.
	 The maximum (minimum) estimate is interpreted as upper (lower) bound.
	 The difference between the bounds is taken as systematic uncertainty on the estimate 
         obtained from the ``simple'' fake--rate method and is represented by the gray shaded area.}
\label{figBgEstFakeRate_frSimpleResults_muonPt}
\end{center}
\end{figure} 

\begin{figure}[t]
\setlength{\unitlength}{1mm}
\begin{center}
\begin{picture}(150,112)(0,0)
\put(0.5, 60){\mbox{\includegraphics*[height=52mm, viewport=23 25 525 404]{figures/plotBgEstFakeRateZtoMuTau_WplusJets_frSimpleTauJetPt.pdf}}}
\put(78.0, 60){\mbox{\includegraphics*[height=52mm, viewport=23 25 525 404]{figures/plotBgEstFakeRateZtoMuTau_QCD_frSimpleTauJetPt.pdf}}}
\put(0.5, 2){\mbox{\includegraphics*[height=52mm, viewport=23 25 525 404]{figures/plotBgEstFakeRateZtoMuTau_TTplusJets_frSimpleTauJetPt.pdf}}}
\put(78.0, 2){\mbox{\includegraphics*[height=52mm, viewport=23 25 525 404]{figures/plotBgEstFakeRateZtoMuTau_Zmumu_frSimpleTauJetPt.pdf}}}
%\put(-5.5, 112.5){\small (a)}
%\put(72.0, 112.5){\small (b)}
%\put(-5.5, 54.5){\small (c)}
%\put(72.0, 54.5){\small (d)}
\end{picture}
\caption{\captiontext Distributions of transverse momenta of the tau--jet candidates in
         $W$+jets (top left), QCD (top right), $t\bar{t}$+jets (bottom left) 
         and $Z \rightarrow \mu^{+} \mu^{-}$ (bottom right) background events which pass all selection criteria 
         of the $Z \rightarrow \tau^{+} \tau^{-} \rightarrow \mu + \tau\mbox{-jet}$ cross--section analysis
	 compared to the estimate obtained from the fake--rate technique, 
         computed according to equation~\ref{eqBgEstFakeRate_frSimpleJetWeight}.
	 The expected contribution of background processes is indicated by points.
	 Lines of different colors represent the estimates obtained by applying 
         fake--rate weights determined for different compositions of light quark, heavy quark and gluon jets,
         as described in section~\ref{secFakeRateParametrization}.
	 The maximum (minimum) estimate is interpreted as upper (lower) bound.
	 The difference between the bounds is taken as systematic uncertainty on the estimate 
         obtained from the ``simple'' fake--rate method and is represented by the gray shaded area.}
\label{figBgEstFakeRate_frSimpleResults_tauJetPt}
\end{center}
\end{figure} 

\begin{figure}[t]
\setlength{\unitlength}{1mm}
\begin{center}
\begin{picture}(150,112)(0,0)
\put(0.5, 60){\mbox{\includegraphics*[height=52mm, viewport=23 25 525 404]{figures/plotBgEstFakeRateZtoMuTau_WplusJets_frSimpleMvisible.pdf}}}
\put(78.0, 60){\mbox{\includegraphics*[height=52mm, viewport=23 25 525 404]{figures/plotBgEstFakeRateZtoMuTau_QCD_frSimpleMvisible.pdf}}}
\put(0.5, 2){\mbox{\includegraphics*[height=52mm, viewport=23 25 525 404]{figures/plotBgEstFakeRateZtoMuTau_TTplusJets_frSimpleMvisible.pdf}}}
\put(78.0, 2){\mbox{\includegraphics*[height=52mm, viewport=23 25 525 404]{figures/plotBgEstFakeRateZtoMuTau_Zmumu_frSimpleMvisible.pdf}}}
%\put(-5.5, 112.5){\small (a)}
%\put(72.0, 112.5){\small (b)}
%\put(-5.5, 54.5){\small (c)}
%\put(72.0, 54.5){\small (d)}
\end{picture}
\caption{\captiontext Distributions of the visible invariant mass of muon plus tau--jet in
	 $W$+jets (top left), QCD (top right), $t\bar{t}$+jets (bottom left) 
         and $Z \rightarrow \mu^{+} \mu^{-}$ (bottom right) background events which pass all selection criteria 
         of the $Z \rightarrow \tau^{+} \tau^{-} \rightarrow \mu + \tau\mbox{-jet}$ cross--section analysis
	 compared to the estimate obtained from the fake--rate technique, 
         computed according to equation~\ref{eqBgEstFakeRate_frSimpleCompositeParticleWeight}.
	 The expected contribution of background processes is indicated by points.
	 Lines of different colors represent the estimates obtained by applying 
         fake--rate weights determined for different compositions of light quark, heavy quark and gluon jets,
         as described in section~\ref{secFakeRateParametrization}.
	 The maximum (minimum) estimate is interpreted as upper (lower) bound.
	 The difference between the bounds is taken as systematic uncertainty on the estimate 
         obtained from the ``simple'' fake--rate method and is represented by the gray shaded area.}
\label{figBgEstFakeRate_frSimpleResults_mVisible}
\end{center}
\end{figure} 


\begin{table}[t]
\begin{center}
\tablesize
\begin{tabular}{|l|c|c|c|c|c|c|}
\hline
\multirow{2}{22mm}{Background}  &             & \multicolumn{4}{c|}{Estimate obtained by applying weights of type:} & Average \\
%\hhline{|~|~|-|-|-|-|}
%\hline
\multirow{2}{18mm}{Process}     & Expectation & QCD       & QCD        & QCD             & \multirow{2}{18mm}{$W$ + jets} & fake--rate \\
                                &             & lead. jet & second jet & $\mu$--enriched &                                & estimate \\
\hline
\hline
$W$+jets                        & $163.0 \pm  7.1$ & $157.2 \pm  2.8$ & $140.9 \pm  2.7$ & $129.9 \pm  2.5$ & $177.9 \pm  3.2$ & $151.5^{+26.6}_{-21.8}$ \\
QCD                             & $246.4 \pm 31.8$ & $269.2 \pm 14.0$ & $246.5 \pm 14.3$ & $219.7 \pm 11.8$ & $300.8 \pm 15.2$ & $259.1^{+44.9}_{-41.7}$ \\
$t\bar{t}$+jets                 & $ 12.2 \pm  0.6$ & $ 14.3 \pm  0.3$ & $ 12.6 \pm  0.3$ & $ 11.6 \pm  0.3$ & $ 16.5 \pm  0.3$ & $ 13.8^{+2.7}_{-2.2}$ \\
$Z \rightarrow \mu^{+} \mu^{-}$ & $ 68.6 \pm  2.9$ & $ 58.2 \pm  1.3$ & $ 51.2 \pm  1.2$ & $ 48.5 \pm  1.1$ & $ 65.8 \pm  1.4$ & $ 55.9^{+10.0}_{-7.5}$ \\
\hline
$\Sigma$~Background             & $490.4 \pm 32.7$ & $499.9 \pm 14.4$ & $451.2 \pm 14.6$ & $409.7 \pm 12.1$ & $561.1 \pm 15.6$ & $480.2^{+82.7}_{-71.9}$ \\     
\hline
\hline
$Z \rightarrow \tau^{+} \tau^{-}$ & $-$ & $284.3 \pm 3.7$ & $269.0 \pm 3.9$ & $256.5 \pm 3.3$ & $325.3 \pm 4.2$ & $283.3^{+42.2}_{-27.1}$ \\
\hline
\end{tabular}
\end{center}
\begin{center}
\caption{\captiontext Number of events from $W$+jets, QCD, $t\bar{t}$+jets
         and $Z \rightarrow \mu^{+} \mu^{-}$ background processes
	 expected to pass all selection criteria of the 
         $Z \rightarrow \tau^{+} \tau^{-} \rightarrow \mu + \tau\mbox{-jet}$ cross--section analysis
         compared to the estimates obtained by weighting events in the background dominated control sample 
         with the ``simple'' fake--rate weights
         defined by equation~\ref{eqBgEstFakeRate_frSimpleEventWeight}.}
\label{tabBgEstFakeRate_frSimpleResults}
\end{center}
\end{table}

\begin{figure}[t]
\setlength{\unitlength}{1mm}
\begin{center}
\begin{picture}(150,54)(0,0)
\put(-7.5, 2){\mbox{\includegraphics*[height=49mm, viewport=20 33 528 498]{figures/plotBgEstFakeRateZtoMuTau_smBackgroundSum_frSimpleMuonPt.pdf}}}
\put(48.0, 2){\mbox{\includegraphics*[height=49mm, viewport=20 33 528 498]{figures/plotBgEstFakeRateZtoMuTau_smBackgroundSum_frSimpleTauJetPt.pdf}}}
\put(103.5, 2){\mbox{\includegraphics*[height=49mm, viewport=20 33 528 498]{figures/plotBgEstFakeRateZtoMuTau_smBackgroundSum_frSimpleMvisible.pdf}}}
\put(-9.5, 54.5){\small (a)}
\put(46.0, 54.5){\small (b)}
\put(101.5, 54.5){\small (c)}
\end{picture}
\caption{\captiontext Distributions of muon transverse momentum (a),
	 transverse momenta of the tau--jet candidates (b)
         and of the visible invariant mass of muon plus tau--jet (c)
	 for the sum of Standard Model background processes 
         $W$+jets (top left), QCD, $t\bar{t}$+jets and $Z \rightarrow \mu^{+} \mu^{-}$ 
         passing all selection criteria 
         of the $Z \rightarrow \tau^{+} \tau^{-} \rightarrow \mu + \tau\mbox{-jet}$ cross--section analysis
	 compared to the estimate obtained from the fake--rate technique, 
         computed according to equations~\ref{eqBgEstFakeRate_frSimpleJetWeight},~\ref{eqBgEstFakeRate_frSimpleEventWeight} 
         and~\ref{eqBgEstFakeRate_frSimpleCompositeParticleWeight}.
	 The expected contribution is indicated by points.
	 Lines of different colors represent the estimates obtained by applying 
         fake--rate weights determined for different compositions of light quark, heavy quark and gluon jets,
         as described in section~\ref{secFakeRateParametrization}.
	 The maximum (minimum) estimate is interpreted as upper (lower) bound.
	 The difference between the bounds is taken as systematic uncertainty on the estimate 
         obtained from the ``simple'' fake--rate method and is represented by the gray shaded area.}
\label{figBgEstFakeRate_frSimpleResults_smBackgroundSum}
\end{center}
\end{figure} 

Results of applying the ``simple'' fake--rate weights defined by 
equations~\ref{eqBgEstFakeRate_frSimpleJetWeight},~\ref{eqBgEstFakeRate_frSimpleEventWeight} 
and~\ref{eqBgEstFakeRate_frSimpleCompositeParticleWeight}
to the $Z \rightarrow \tau^{+} \tau^{-} \rightarrow \mu + \tau\mbox{-jet}$ cross--section analysis
are displayed in figure~\ref{figBgEstFakeRate_frSimpleResults_muonPt},~\ref{figBgEstFakeRate_frSimpleResults_tauJetPt} 
and~\ref{figBgEstFakeRate_frSimpleResults_mVisible},
showing distributions of the muon transverse momentum,
the transverse momentum of the tau--jet candidates and the visible invariant mass of muon plus tau--jet ``composite particles''
respectively.
The total number of background events estimated by equation~\ref{eqBgEstFakeRate_frSimpleEventWeight} 
is compared to the expected number of QCD, $W$+jets, $t\bar{t}$+jets and $Z \rightarrow \mu^{+} \mu^{-}$ events
which pass all event selection criteria of the $Z \rightarrow \tau^{+} \tau^{-} \rightarrow \mu + \tau\mbox{-jet}$ analysis
in table~\ref{tabBgEstFakeRate_frSimpleResults}.
The estimates obtained from the fake--rate method are in good agreement with the contributions
expected for QCD and $W$+jets background processes.
Perhaps surprisingly, the fake--rate technique also predicts the contributions of the
$t\bar{t}$+jets and $Z \rightarrow \mu^{+} \mu^{-}$ backgrounds rather precisely. 
The explanation for the agreement between estimates obtained from the fake--rate method and the expected contributions
of $t\bar{t}$+jets and $Z \rightarrow \mu^{+} \mu^{-}$ backgrounds
is illustrated in figure~\ref{figBgEstFakeRate_frSimpleMatchingGenParticlePDGid}.
In the figure, it can be seen that most tau--jet candidates in 
$t\bar{t}$+jets and $Z \rightarrow \mu^{+} \mu^{-}$ events passing all selection criteria of the final
$Z \rightarrow \tau^{+} \tau^{-} \rightarrow \mu + \tau\mbox{-jet}$ cross--section analysis
do in fact arise from misidentification of quark and gluon jets as hadronic tau lepton decays.
$Z \rightarrow \mu^{+} \mu^{-}$ background contributions arising from misidentification of muons
are visible in two bins around the nominal $Z$ mass in figure~\ref{figBgEstFakeRate_frSimpleResults_mVisible} only.
Contributions from genuine tau leptons in $t\bar{t}$+jets background events are small 
and within the variations covered by differences between fake--rate weights determined for 
varying compositions of light quark, heavy quark and gluon jets as described in section~\ref{secFakeRateParametrization}.

\begin{figure}[t]
\setlength{\unitlength}{1mm}
\begin{center}
\begin{picture}(150,54)(0,0)
\put(78.0, 2){\mbox{\includegraphics*[height=52mm, viewport=14 25 525 404]{figures/finalSamplePlotZtoMuTau_pdgIdGenParticleMatchingTau.pdf}}}
\put(0.5, 2){\mbox{\includegraphics*[height=52mm, viewport=14 25 525 404]{figures/finalSamplePlotZtoMuTau_pdgIdGenParticleMatchingMuon.pdf}}}
%\put(-5.5, 54.5){\small (a)}
%\put(72.0, 54.5){\small (b)}
\end{picture}
\caption{\captiontext Type of generator level particle which matches muons (left) and tau--jet candidates (right)
         reconstructed in signal and different types of background events 
	 passing all selection criteria of the 
         $Z \rightarrow \tau^{+} \tau^{-} \rightarrow \mu + \tau\mbox{-jet}$ cross--section analysis.        
         Particle species are encoded according to the definition of the particle data group~\cite{refPDGid}:
         numbers $1$-$5$ represent quarks, a value of $11$ ($13$) represents electrons (muons),
	 $15$ represents tau leptons (all tau decay modes included) and $22$ gluons.
         Events for which unambiguous matching of reconstructed to generator level particles failed
	 are represented by entries in bins $0$ and $24$.}
\label{figBgEstFakeRate_frSimpleMatchingGenParticlePDGid}
\end{center}
\end{figure} 

Different estimates are obtained for the fake--rate probabilities 
determined for the highest and second highest $P_{T}$ jet in QCD di--jet events,
jets in a muon enriched QCD sample and jets in $W$+jets events.
The arithmetic average of the four estimates together with the difference between
the computed average and the minimum/maximum value is given in table~\ref{tabBgEstFakeRate_frSimpleResults}.

We take the average value as ``best'' estimate of the background contribution 
and the difference between the average and the minimum/maximum estimate as its systematic uncertainty.
We obtain a value of $\mathcal{O} \left( 15 \% \right)$ for the systematic uncertainty
and find that the true sum of QCD, $W$+jets, $t\bar{t}$+jets and $Z \rightarrow \mu^{+} \mu^{-}$ background contributions
agrees well with the ``best'' estimate obtained by the fake--rate method
within the systematic uncertainty.

Note that the estimate for the sum of background contributions which one obtains
in case one applies the ``simple'' fake--rate weights defined by equation~\ref{eqBgEstFakeRate_frSimpleEventWeight}
to a background dominated control sample selected in data is likely to overestimate the true value
of background contributions by a significant amount.
The reason is that contributions of the $Z \rightarrow \tau^{+} \tau^{-}$ signal are non--negligible.
In fact, signal contributions to the background dominated control sample are expected to be $14.9 \%$
and since the per--jet weights computed by equation~\ref{eqBgEstFakeRate_frSimpleJetWeight}
are larger on average in signal than in background events,
the signal contribution increases by the weighting and amounts to $37.1 \%$
of the sum of event weights computed by equation~\ref{eqBgEstFakeRate_frSimpleEventWeight}
and given in table~\ref{tabBgEstFakeRate_frSimpleResults}.

The contribution of the $Z \rightarrow \tau^{+} \tau^{-}$ signal
needs to be determined by Monte Carlo simulation
and subtracted from the estimate obtained by applying the ``simple'' fake--rate method to data,
in order to get an unbiased estimate of the true background contributions.


\subsection{``CDF--type'' weights}
\label{secBgEstFakeRate_frCDFtypeWeights}

%The ``simple'' method described in the last section does actually overestimate the sum of background contributions 
%to the final analysis by some amount.
%The reason is that signal events present in the background dominated control
%sample (selected by not applying any tau identification cuts or discriminators)
%contribute to the estimate.
%One way of handling the signal contribution is to subtract from the estimate obtained from the ``simple'' method
%a correction term which is to be estimated from Monte Carlo.
%Another way is to correct for the signal contribution by adjusting the weights,
%based on information that is in the analyzed data sample only,
%avoiding the need to rely on Monte Carlo based corrections.
%The idea of adjusting the weights in such a way that signal contributions cancel
%while an unbiased estimate of the sum of background contributions is maintained 
%is the basis of the ``CDF--type'' fake--rate method.

Instead of subtracting from the estimate obtained for the sum of background contributions
a correction determined by Monte Carlo simulation,
the signal contribution to the background dominated event sample selected in data can be corrected for
by adjusting the weights, based solely on information contained in the analyzed data sample,
hence avoiding the need to rely on Monte Carlo based corrections.

In the ``CDF--type'' method, additional information, namely whether or not tau--jet candidates
pass or fail the tau identification cuts and discriminators, is drawn from the data.
The desired cancellation of signal contributions is achieved by assigning negative weights to those tau--jet candidates 
which pass all tau identification cuts and discriminators, 
i.e. to a fair fraction of genuine hadronic tau decays, 
but to a small fraction of quark and gluon jets only.
The small reduction of the background estimate by negative weights assigned to quark and gluon jets 
is accounted for by a small increase of the positive weights assigned to those tau--jet candidates 
for which at least one of the tau identification cuts or discriminators fails.
In this way, an unbiased estimate of the background contribution is maintained.

To be specific, the ``CDF--type'' weights assigned to tau--jet candidates are computed as
(for a derivation of the formula, see section~\ref{secDerrivationCDFtypeFakeRateWeights} of the appendix):
\begin{equation}
w_{jet}^{CDF} \left( P_{T}^{jet}, \eta_{jet}, R_{jet} \right) 
:= 
\begin{cases} 
   \frac{P_{fr} \left( P_{T}^{jet}, \eta_{jet}, R_{jet} \right) \cdot 
         \varepsilon \left( P_{T}^{jet}, \eta_{jet}, R_{jet} \right)}
        {\varepsilon \left( P_{T}^{jet}, \eta_{jet}, R_{jet} \right) - P_{fr} \left( P_{T}^{jet}, \eta_{jet}, R_{jet} \right)}
      \mbox{ if all tau id. cuts and discriminators passed} \\
   \frac{P_{fr} \left( P_{T}^{jet}, \eta_{jet}, R_{jet} \right) \cdot 
         \left( 1 - \varepsilon \left( P_{T}^{jet}, \eta_{jet}, R_{jet} \right) \right)}
        {\varepsilon \left( P_{T}^{jet}, \eta_{jet}, R_{jet} \right) - P_{fr} \left( P_{T}^{jet}, \eta_{jet}, R_{jet} \right)}
      %\mbox{ if at least one tau id. cut or discriminator failed}
      \mbox{ otherwise}
\end{cases}
\label{eqBgEstFakeRate_frCDFtypeJetWeight}
\end{equation}

Event weights and the weights assigned to ``composite particles'' 
are computed in the same way as for the ``simple'' weights,
based on the weights assigned to the tau--jet candidates:
\begin{eqnarray}
W_{event}^{CDF} & := & \Sigma w_{jet}^{CDF} \nonumber \\
w_{comp-part}^{CDF} \left( P_{T}^{jet}, \eta_{jet}, R_{jet} \right) & := & 
  w_{jet}^{CDF} \left( P_{T}^{jet}, \eta_{jet}, R_{jet} \right),
\label{eqBgEstFakeRate_frCDFtypeEvent_and_CompositeParticleWeight}
\end{eqnarray}
where the sums extend over all jets in the background dominated control sample which pass the preselection defined by 
the denominator of equation~\ref{eqBgEstFakeRate_fr}.

The effect of the negative weights to compensate the positive weights
in case the ``CDF--type'' fake--rate method is applied 
to signal events containing genuine hadronic tau decays 
is shown in table~\ref{tabBgEstFakeRate_frCDFtypeResults}.
% illustrated in figure~\ref{figBgEstFakeRate_frCDFtypeResults_mVisibleSignal}.
As expected, positive and negative weights do indeed cancel
in the statistical average.

%\begin{figure}[t]
%\setlength{\unitlength}{1mm}
%\begin{center}
%\begin{picture}(150,54)(0,0)
%\put(0.5, 2){\mbox{\includegraphics*[height=52mm, viewport=23 25 525 404]{figures/plotBgEstFakeRateZtoMuTau_Ztautau_frSimpleMvisible.pdf}}}
%\put(78.0, 2){\mbox{\includegraphics*[height=52mm, viewport=23 25 525 404]{figures/plotBgEstFakeRateZtoMuTau_Ztautau_frCDFmVisible.pdf}}}
%\end{picture}
%\caption{\captiontext Distributions of visible invariant mass of muon plus tau--jet in
%	  $Z \rightarrow \tau^{+} \tau^{-}$ signal events weighted by ``simple'' weights 
%         computed according to equation~\ref{eqBgEstFakeRate_frSimpleCompositeParticleWeight} (left)
%         and ``CDF--type'' weights 
%         computed according to equation~\ref{eqBgEstFakeRate_frCDFtypeEvent_and_CompositeParticleWeight} (right).
%	  The signal contribution to the background estimate computed by the ``simple'' method is non--negligible 
%         and needs to be corrected for.
%	  The ``CDF--type'' weights achieve a statistical cancellation of positive and negative weights,
%         such that the total signal contribution averages to zero, avoiding the need for Monte Carlo based corrections.}
%\label{figBgEstFakeRate_frCDFtypeResults_mVisibleSignal}
%\end{center}
%\end{figure} 

Figures~\ref{figBgEstFakeRate_frCDFtypeResults_muonPt},~\ref{figBgEstFakeRate_frCDFtypeResults_tauJetPt} 
and~\ref{figBgEstFakeRate_frCDFtypeResults_mVisible}
demonstrate that an unbiased estimate of the background contribution by the ``CDF--type'' weights is maintained.
Overall, the estimates obtained are in good agreement with the contributions
expected for different background processes, indicating that the adjustment of negative and positive weights 
works as expected for the background as well.
%The agreement between expected contributions and the estimates obtained from the ``CDF--type'' method
%is summarized in terms of the total number of events in table~\ref{tabBgEstFakeRate_frCDFtypeResults}.

Results obtained by the ``CDF--type'' fake--rate method
are summarized in table~\ref{tabBgEstFakeRate_frCDFtypeResults},
in which the total number of background events 
estimated by equation~\ref{eqBgEstFakeRate_frCDFtypeEvent_and_CompositeParticleWeight}
is compared to the true background contributions.
The ``best'' estimate of the background contribution obtained from the ``CDF--type'' method
is again taken as the arithmetic average of the estimates
obtained by applying the fake--rate probabilities for the highest and second highest $P_{T}$ jet
in QCD di--jet events, jets in a muon enriched QCD sample and jets in $W$+jets events.
Systematic uncertainties are taken from the difference between the computed average value
and the minimum/maximum estimate.
We obtain a value of $\mathcal{O} \left( 15 \mbox{--} 20 \% \right)$ for the systematic uncertainty
of the ``CDF--type'' method, slightly higher than the systematic uncertainty obtained for the ``simple'' method.
The small increase of systematic uncertainties is in agreement with our expectation
for fluctuations of the jet--weights in case weights of negative and positive sign are used.

\begin{figure}[t]
\setlength{\unitlength}{1mm}
\begin{center}
\begin{picture}(150,112)(0,0)
\put(0.5, 60){\mbox{\includegraphics*[height=52mm, viewport=23 25 525 404]{figures/plotBgEstFakeRateZtoMuTau_WplusJets_frCDFmuonPt.pdf}}}
\put(78.0, 60){\mbox{\includegraphics*[height=52mm, viewport=23 25 525 404]{figures/plotBgEstFakeRateZtoMuTau_QCD_frCDFmuonPt.pdf}}}
\put(0.5, 2){\mbox{\includegraphics*[height=52mm, viewport=23 25 525 404]{figures/plotBgEstFakeRateZtoMuTau_TTplusJets_frCDFmuonPt.pdf}}}
\put(78.0, 2){\mbox{\includegraphics*[height=52mm, viewport=23 25 525 404]{figures/plotBgEstFakeRateZtoMuTau_Zmumu_frCDFmuonPt.pdf}}}
%\put(-5.5, 112.5){\small (a)}
%\put(72.0, 112.5){\small (b)}
%\put(-5.5, 54.5){\small (c)}
%\put(72.0, 54.5){\small (d)}
\end{picture}
\caption{\captiontext Distributions of muon transverse momentum in
	 $W$+jets (top left), QCD (top right), $t\bar{t}$+jets (bottom left) 
         and $Z \rightarrow \mu^{+} \mu^{-}$ (bottom right) background events which pass all selection criteria 
         of the $Z \rightarrow \tau^{+} \tau^{-} \rightarrow \mu + \tau\mbox{-jet}$ cross--section analysis
	 compared to the estimate obtained from the fake--rate technique, 
         computed according to equation~\ref{eqBgEstFakeRate_frCDFtypeEvent_and_CompositeParticleWeight}.
	 The expected contribution of background processes is indicated by points.
	 Lines of different colors represent the estimates obtained by applying 
         fake--rate weights determined for different compositions of light quark, heavy quark and gluon jets,
         as described in section~\ref{secFakeRateParametrization}.
	 The maximum (minimum) estimate is interpreted as upper (lower) bound.
	 The difference between the bounds is taken as systematic uncertainty on the estimate 
         obtained from the ``CDF--type'' fake--rate method and is represented by the gray shaded area.}
\label{figBgEstFakeRate_frCDFtypeResults_muonPt}
\end{center}
\end{figure} 

\begin{figure}[t]
\setlength{\unitlength}{1mm}
\begin{center}
\begin{picture}(150,112)(0,0)
\put(0.5, 60){\mbox{\includegraphics*[height=52mm, viewport=23 25 525 404]{figures/plotBgEstFakeRateZtoMuTau_WplusJets_frCDFtauJetPt.pdf}}}
\put(78.0, 60){\mbox{\includegraphics*[height=52mm, viewport=23 25 525 404]{figures/plotBgEstFakeRateZtoMuTau_QCD_frCDFtauJetPt.pdf}}}
\put(0.5, 2){\mbox{\includegraphics*[height=52mm, viewport=23 25 525 404]{figures/plotBgEstFakeRateZtoMuTau_TTplusJets_frCDFtauJetPt.pdf}}}
\put(78.0, 2){\mbox{\includegraphics*[height=52mm, viewport=23 25 525 404]{figures/plotBgEstFakeRateZtoMuTau_Zmumu_frCDFtauJetPt.pdf}}}
%\put(-5.5, 112.5){\small (a)}
%\put(72.0, 112.5){\small (b)}
%\put(-5.5, 54.5){\small (c)}
%\put(72.0, 54.5){\small (d)}
\end{picture}
\caption{\captiontext Distributions of transverse momenta of the tau--jet candidates in
	 $W$+jets (top left), QCD (top right), $t\bar{t}$+jets (bottom left) 
         and $Z \rightarrow \mu^{+} \mu^{-}$ (bottom right) background events which pass all selection criteria 
         of the $Z \rightarrow \tau^{+} \tau^{-} \rightarrow \mu + \tau\mbox{-jet}$ cross--section analysis
	 compared to the estimate obtained from the fake--rate technique, 
         computed according to equation~\ref{eqBgEstFakeRate_frCDFtypeJetWeight}.
	 The expected contribution of background processes is indicated by points.
	 Lines of different colors represent the estimates obtained by applying 
         fake--rate weights determined for different compositions of light quark, heavy quark and gluon jets,
         as described in section~\ref{secFakeRateParametrization}.
	 The maximum (minimum) estimate is interpreted as upper (lower) bound.
	 The difference between the bounds is taken as systematic uncertainty on the estimate 
         obtained from the ``CDF--type'' fake--rate method and is represented by the gray shaded area.}
\label{figBgEstFakeRate_frCDFtypeResults_tauJetPt}
\end{center}
\end{figure} 

\begin{figure}[t]
\setlength{\unitlength}{1mm}
\begin{center}
\begin{picture}(150,112)(0,0)
\put(0.5, 60){\mbox{\includegraphics*[height=52mm, viewport=23 25 525 404]{figures/plotBgEstFakeRateZtoMuTau_WplusJets_frCDFmVisible.pdf}}}
\put(78.0, 60){\mbox{\includegraphics*[height=52mm, viewport=23 25 525 404]{figures/plotBgEstFakeRateZtoMuTau_QCD_frCDFmVisible.pdf}}}
\put(0.5, 2){\mbox{\includegraphics*[height=52mm, viewport=23 25 525 404]{figures/plotBgEstFakeRateZtoMuTau_TTplusJets_frCDFmVisible.pdf}}}
\put(78.0, 2){\mbox{\includegraphics*[height=52mm, viewport=23 25 525 404]{figures/plotBgEstFakeRateZtoMuTau_Zmumu_frCDFmVisible.pdf}}}
%\put(-5.5, 112.5){\small (a)}
%\put(72.0, 112.5){\small (b)}
%\put(-5.5, 54.5){\small (c)}
%\put(72.0, 54.5){\small (d)}
\end{picture}
\caption{\captiontext Distributions of the visible invariant mass of muon plus tau--jet in
         $W$+jets (top left), QCD (top right), $t\bar{t}$+jets (bottom left) 
         and $Z \rightarrow \mu^{+} \mu^{-}$ (bottom right) background events which pass all selection criteria 
         of the $Z \rightarrow \tau^{+} \tau^{-} \rightarrow \mu + \tau\mbox{-jet}$ cross--section analysis
	 compared to the estimate obtained from the fake--rate technique, 
         computed according to equation~\ref{eqBgEstFakeRate_frCDFtypeEvent_and_CompositeParticleWeight}.
	 The expected contribution of background processes is indicated by points.
	 Lines of different colors represent the estimates obtained by applying 
         fake--rate weights determined for different compositions of light quark, heavy quark and gluon jets,
         as described in section~\ref{secFakeRateParametrization}.
	 The maximum (minimum) estimate is interpreted as upper (lower) bound.
	 The difference between the bounds is taken as systematic uncertainty on the estimate 
         obtained from the ``CDF--type'' fake--rate method and is represented by the gray shaded area.}
\label{figBgEstFakeRate_frCDFtypeResults_mVisible}
\end{center}
\end{figure} 

\begin{table}[t]
\begin{center}
\tablesize
\begin{tabular}{|l|c|c|c|c|c|c|}
\hline
\multirow{2}{22mm}{Background}  &             & \multicolumn{4}{c|}{Estimate obtained by applying weights of type:} & Average \\
%\hhline{|~|~|-|-|-|-|}
%\hline
\multirow{2}{18mm}{Process}     & Expectation & QCD       & QCD        & QCD             & \multirow{2}{18mm}{$W$ + jets} & fake--rate \\
                                &             & lead. jet & second jet & $\mu$--enriched &                          & estimate \\
\hline
\hline
$W$+jets                        & $163.0 \pm  7.1$ & $163.2 \pm  3.8$ & $140.6 \pm  3.4$ & $128.0 \pm  3.1$ & $188.3 \pm  4.2$ & $155.0^{+33.6}_{-27.3}$ \\
QCD                             & $246.4 \pm 31.8$ & $300.5 \pm 19.5$ & $266.1 \pm 19.0$ & $236.0 \pm 16.4$ & $335.1 \pm 20.4$ & $284.4^{+55.5}_{-52.0}$ \\
$t\bar{t}$+jets                 & $ 12.2 \pm  0.6$ & $ 13.1 \pm  0.3$ & $ 11.5 \pm  0.3$ & $ 10.2 \pm  0.3$ & $ 15.4 \pm  0.4$ & $ 12.6^{+2.8}_{-2.4}$ \\
$Z \rightarrow \mu^{+} \mu^{-}$ & $ 68.6 \pm  2.9$ & $ 52.7 \pm  1.4$ & $ 46.7 \pm  1.4$ & $ 41.9 \pm  1.2$ & $ 60.3 \pm  1.6$ & $ 50.4^{+10.1}_{-8.6}$ \\
\hline
$\Sigma$~Background             & $490.4 \pm 32.7$ & $529.5 \pm 19.9$ & $464.9 \pm 19.3$ & $416.1 \pm 16.8$ & $599.1 \pm 20.9$ & $502.4^{+99.4}_{-88.4}$ \\ 
\hline
\hline
$Z \rightarrow \tau^{+} \tau^{-}$ & $-$ & $0.3 \pm 2.4$ & $-10.6 \pm 2.5$ & $3.8 \pm 2.0$ & $-10.8 \pm 2.8$ & $-4.3^{+8.4}_{-7.2}$ \\
\hline
\end{tabular}
\end{center}
\begin{center}
\caption{\captiontext Number of events from $W$+jets, QCD, $t\bar{t}$+jets
         and $Z \rightarrow \mu^{+} \mu^{-}$ background processes
	 expected to pass all selection criteria of the 
         $Z \rightarrow \tau^{+} \tau^{-} \rightarrow \mu + \tau\mbox{-jet}$ cross--section analysis         
         compared to the estimates obtained by weighting events in the background dominated control sample 
         with the ``CDF--type'' fake--rate weights
         defined by equation~\ref{eqBgEstFakeRate_frCDFtypeEvent_and_CompositeParticleWeight}.}
\label{tabBgEstFakeRate_frCDFtypeResults}
\end{center}
\end{table}

\begin{figure}[t]
\setlength{\unitlength}{1mm}
\begin{center}
\begin{picture}(150,54)(0,0)
\put(-7.5, 2){\mbox{\includegraphics*[height=49mm, viewport=20 33 528 498]{figures/plotBgEstFakeRateZtoMuTau_smBackgroundSum_frCDFmuonPt.pdf}}}
\put(48.0, 2){\mbox{\includegraphics*[height=49mm, viewport=20 33 528 498]{figures/plotBgEstFakeRateZtoMuTau_smBackgroundSum_frCDFtauJetPt.pdf}}}
\put(103.5, 2){\mbox{\includegraphics*[height=49mm, viewport=20 33 528 498]{figures/plotBgEstFakeRateZtoMuTau_smBackgroundSum_frCDFmVisible.pdf}}}
\put(-9.5, 54.5){\small (a)}
\put(46.0, 54.5){\small (b)}
\put(101.5, 54.5){\small (c)}
\end{picture}
\caption{\captiontext Distributions of muon transverse momentum (a),
	 transverse momenta of the tau--jet candidates (b)
         and of the visible invariant mass of muon plus tau--jet (c)
	 for the sum of Standard Model background processes 
         $W$+jets (top left), QCD, $t\bar{t}$+jets and $Z \rightarrow \mu^{+} \mu^{-}$ 
         passing all selection criteria 
         of the $Z \rightarrow \tau^{+} \tau^{-} \rightarrow \mu + \tau\mbox{-jet}$ cross--section analysis
	 compared to the estimate obtained from the fake--rate technique, 
         computed according to equations~\ref{eqBgEstFakeRate_frCDFtypeJetWeight}
	 and~\ref{eqBgEstFakeRate_frCDFtypeEvent_and_CompositeParticleWeight}. 
	 The expected contribution is indicated by points.
	 Lines of different colors represent the estimates obtained by applying 
         fake--rate weights determined for different compositions of light quark, heavy quark and gluon jets,
         as described in section~\ref{secFakeRateParametrization}.
	 The maximum (minimum) estimate is interpreted as upper (lower) bound.
	 The difference between the bounds is taken as systematic uncertainty on the estimate 
         obtained from the ``CDF--type'' fake--rate method and is represented by the gray shaded area.}
\label{figBgEstFakeRate_frCDFtypeResults_smBackgroundSum}
\end{center}
\end{figure} 


\section{Summary}

Two different methods for estimating background contributions to the final event sample of physics analyses 
involving tau leptons in the final state have been presented.
The background estimates obtained from the ``simple'' and ``CDF--type'' methods are found to be in good agreement
with the expectation within the estimated systematic uncertainties of $\mathcal{O} \left( 15 \% \right)$.
The ``simple'' method relies on corrections for signal contributions which need to be determined by the Monte Carlo simulation,
while the ``CDF--type'' method utilizes additional information drawn from the data 
and does not need Monte Carlo based corrections.


\section{Appendix}

\subsection{Derivation of ``CDF--type'' fake--rate weights}
\label{secDerrivationCDFtypeFakeRateWeights}

The basic idea of the ``CDF--type'' weights is to assign negative (positive) weights to tau--jet candidates
passing all tau identification cuts and discriminators (failing at least one cut or discriminator),
such that signal contributions of genuine hadronic tau decays to the background dominated control sample 
on average cancel after application of the weights, while providing an unbiased estimate 
of the contribution of background processes arising from misidentification of quark and gluon jets.

For the derivation of equation~\ref{eqBgEstFakeRate_frCDFtypeJetWeight} for the ``CDF--type'' weights assigned to tau--jet candidates,
we will use the following notation:
Let $n_{\tau}$ ($n_{QCD}$) denote the total number of tau--jets (quark and gluon jets) in a certain bin of
transverse momentum $P_{T}^{jet}$, pseudo--rapidity $\eta_{jet}$ and jet--width $R_{jet}$ and
$n_{\tau}^{sel}$ ($n_{QCD}^{sel}$) denote the number of tau--jets (quark and gluon jets) in that bin 
which pass all tau identification cuts and discriminators.

By definition of the tau identification 
efficiency $\varepsilon := \varepsilon \left( P_{T}^{jet}, \eta_{jet}, R_{jet} \right)$ 
and 
fake--rate $f := f \left( P_{T}^{jet}, \eta_{jet}, R_{jet} \right)$:
\begin{eqnarray}
n_{\tau}^{sel} & = & \varepsilon \cdot n_{\tau} \nonumber \\
n_{QCD}^{sel} & = & f \cdot n_{QCD}.
\label{eqBgEstFakeRate_eff_and_frDef}
\end{eqnarray}

Depending on whether or not a given tau--jet candidate passes all tau identification cuts and discriminators or not,
we will assign a weight of value $w_{passed}$ or $w_{failed}$ to it.

The values of the weights $w_{passed}$ and $w_{failed}$ 
shall be adjusted such that they provide an unbiased estimate of the background contribution:
\begin{equation}
w_{passed} \cdot f \cdot n_{QCD} + w_{failed} \cdot \left( 1 - f \right) \cdot n_{QCD} \equiv n_{QCD}^{sel} = f \cdot n_{QCD}
\label{eqBgEstFakeRate_QCD}
\end{equation}
while averaging to zero for genuine hadronic tau decays:
\begin{equation*}
w_{passed} \cdot \varepsilon \cdot n_{\tau} + w_{failed} \cdot \left( 1 - \varepsilon \right) \cdot n_{\tau} \equiv 0.
\label{eqBgEstFakeRate_tau}
\end{equation*}
The latter equation yields the relation:
\begin{equation}
w_{passed} = -\frac{1 - \varepsilon}{\varepsilon} \cdot w_{failed},
\label{eqBgEstFakeRate_weightRelation}
\end{equation}
associating the two types of weights.
By inserting relation~\ref{eqBgEstFakeRate_weightRelation} into equation~\ref{eqBgEstFakeRate_QCD} we obtain:
\begin{eqnarray*}
& & -\frac{1 - \varepsilon}{\varepsilon} \cdot w_{failed} \cdot f \cdot n_{QCD} + w_{failed} \cdot \left( 1 - f \right) \cdot n_{QCD} 
 = f \cdot n_{QCD} \\
& \Rightarrow & \left( \frac{-f + \varepsilon \cdot f + \varepsilon - f \cdot \varepsilon}{\varepsilon} \right) \cdot w_{failed} = f \\
& \Rightarrow & w_{failed} = \frac{f \cdot \varepsilon}{\varepsilon - f} 
\end{eqnarray*}
and 
\begin{equation}
w_{passed} = -\frac{f \cdot \left( 1 - \varepsilon \right)}{\varepsilon - f}
\end{equation}
which matches exactly equation~\ref{eqBgEstFakeRate_frCDFtypeJetWeight} 
for the ``CDF--type'' weights applied to tau--jet candidates given in section~\ref{secBgEstFakeRate_frCDFtypeWeights}.

\subsection{Usage of the fake--rate weights in CMSSW analyses}

In this section we aim to provide a recipe how to use the ``simple'' fake--rate
weights defined by equations~\ref{eqBgEstFakeRate_frSimpleJetWeight},
~\ref{eqBgEstFakeRate_frSimpleEventWeight}
and~\ref{eqBgEstFakeRate_frSimpleCompositeParticleWeight} 
and the ``CDF--type'' fake--rate weights defined by 
equations~\ref{eqBgEstFakeRate_frCDFtypeJetWeight}
and~\ref{eqBgEstFakeRate_frCDFtypeEvent_and_CompositeParticleWeight} 
in CMSSW analyses.

The probabilities defined by equation~\ref{eqBgEstFakeRate_fr} for quark
and gluon jets to fake the signature of a hadronic tau decay may depend to a
large extent upon the sample preselection, exact tau identification cuts, and discriminator
requirements applied on the analysis level. We therefore include a prescription 
for modifying the configuration to redetermine these probabilities, either from simulated
Monte Carlo samples or from data.

In general, the usage of the fake--rate weights described in this note requires
five steps to be taken:
\begin{itemize}
\item production of Ntuples containing tau--jet candidates
\item determination of fake--rate probabilities
\item production of either ``simple'' or ``CDF--type'' weights 
\item addition of weights to pat::Tau 
\item application of weights in the analysis code
\end{itemize}
We will describe the individual steps in detail in the following.

We aimed to make the recipe for how to use the fake--rate weights in CMSSW
analyses as much as possible independent of a specific CMSSW software version.
We expect, however, that the technical details may depend (at least to some
extent) on the actual software version used, and for this reason we would like
to mention that the recipe given in the following has been tested with release
CMSSW\_3\_3\_6\_patch5.  Additional packages which we have used to obtain the
results presented in this note are listed in
table~\ref{tabBgEstFakeRate_cmsswTags}.

\begin{table}[t]
\begin{center}
\tablesize
\begin{tabular}{|l|l|}
\hline
Package & Version \\
\hline
RecoTauTag/Configuration    & V00--20--00 \\
RecoTauTag/RecoTau          & V00--20--00 \\
RecoTauTag/TauAnalysisTools & V00--04--00 \\
RecoTauTag/TauTagTools      & V00--20--00 \\
\hline
\end{tabular}
\end{center}
\begin{center}
\caption{\captiontext CMSSW software packages used in addition to release CMSSW\_3\_3\_6\_patch5.}
\label{tabBgEstFakeRate_cmsswTags}
\end{center}
\end{table}

\subsubsection{Ntuple production}
\label{secAppendixNtupleProduction}

The fake--rate technique workflow begins with the production of Ntuple files
which contain information about the selected sample of reconstructed tau--jet candidates.
Each Ntuple entry represents one tau--jet candidate.
The entry contains information about $P_{T}$, $\vert\eta\vert$ and $R_{jet}$ of the tau--jet candidates,
as well as of all individual tau discriminators.
The observables used for parametrization and the discriminators associated to each tau--jet candidate
are defined in \newline
\verb|RecoTauTag/TauAnalysisTools/python/tools/ntupleDefinitions.py|.  

The Ntuple files are produced by configuration files stored in \newline
\verb|RecoTauTag/TauAnalysisTools/test/fakeRate|.  
The Ntuple production is performed on the grid via CRAB,
the CRAB jobs being configured to return the Ntuple data files to the user.
The DBS names of RECO/AOD files produced for $\sqrt{s} = 7$~TeV centre--of--mass energy with CMSSW\_3\_1\_2
(Summer'09 Monte Carlo production)
and of the CRAB crab configuration files 
for producing the Ntuples which we used to determine 
fake--rate probabilities and tau identification efficiencies 
are listed in table~\ref{tabBgEstFakeRate_NtupleProdCfgFiles}.

\begin{table}[t]
\begin{center}
\tablesize
\begin{tabular}{|lc|}
\hline
Sample (Steering file) & Events \\
\hline
\hline
% QCD Samples
\multicolumn{2}{|l|}{QCD Dijet (separately for highest and second highest $P_{T}$ jets; {\tt multicrab.cfg})} \\
{\tt /QCDDiJet\_Pt20to30/Summer09-MC\_31X\_V3\_7TeV\_AODSIM-v1/AODSIM}   & 1008752 \\
{\tt /QCDDiJet\_Pt30to50/Summer09-MC\_31X\_V3\_7TeV\_AODSIM-v1/AODSIM}   & 1010858 \\
{\tt /QCDDiJet\_Pt50to80/Summer09-MC\_31X\_V3\_7TeV\_AODSIM-v1/AODSIM}   &  108517 \\
{\tt /QCDDiJet\_Pt80to120/Summer09-MC\_31X\_V3\_7TeV\_AODSIM-v1/AODSIM}  &  581928 \\
{\tt /QCDDiJet\_Pt120to170/Summer09-MC\_31X\_V3\_7TeV\_AODSIM-v1/AODSIM} &   52062 \\
\hline
% ppmux
\multicolumn{2}{|l|}{Muon enriched QCD ({\tt crab\_ppmux.cfg})} \\
{\tt /InclusiveMu15/Summer09-MC\_31X\_V3\_7TeV\_AODSIM-v1/AODSIM} & 5261979 \\
\hline
% WJets
\multicolumn{2}{|l|}{$W$+jets ({\tt crab\_wjets.cfg})} \\
{\tt /WJets-sherpa/Summer09-MC\_31X\_V3\_7TeV\_AODSIM-v1/AODSIM} & 10980000 \\
\hline
% ztt
\multicolumn{2}{|l|}{$Z \rightarrow \tau^{+}\tau^{-}$ ({\tt crab\_ztt.cfg})} \\
{\tt /Ztautau/Summer09-MC\_31X\_V3\_7TeV\_AODSIM-v1/AODSIM } & 2099189 \\
\hline
\end{tabular}
\end{center}
\begin{center}
\caption{\captiontext Samples used to produce the Ntuples used in fake rate computation for the various channels.  
         The configuration steering files given in brackets are located in the directory 
         {\tt RecoTauTag/TauAnalysisTools/test/fakeRate}.}
\label{tabBgEstFakeRate_NtupleProdCfgFiles}
\end{center}
\end{table}

We have produced Ntuple files for QCD di--jet, muon enriched QCD and $W$+jets background samples
and for the $Z \rightarrow \tau^{+} \tau^{-}$ signal.
The Ntuples are available on castor, at the locations specified in table~\ref{tabBgEstFakeRate_NtupleCastorLocations}.

\begin{table}[t]
\begin{center}
\tablesize
\begin{tabular}{|l|c|l|}
\hline
\multirow{2}{15mm}{Sample}  & Tau--jet   & \multirow{2}{35mm}{Ntuple locations} \\
                            & candidates & \\
\hline
QCD highest $P_{T}$ jet & 688627 & {\tt V000400/DijetXX\_YY} \\
\hline
QCD second highest $P_{T}$ jet & 240004 & {\tt V000400/DijetXX\_YY} \\
\hline
Muon enriched QCD & 1219239 & {\tt V000400/InclusiveMu15} \\
\hline
$W$+jets & 286876 & {\tt V000400/WJets} \\
\hline
$Z \rightarrow \tau^{+} \tau^{-}$ & 414770 & {\tt V000400/Ztautau} \\
\hline
\end{tabular}
\end{center}
\begin{center}
\caption{\captiontext Location of Ntuple files used for determination of fake--rates 
         in QCD di--jet, muon enriched QCD and $W$+jets background 
         and for determination of tau identification efficiencies 
         in $Z \rightarrow \tau^{+} \tau^{-}$ signal events.
         The Ntuples were produced from RECO/AOD Monte Carlo sample generated and reconstructed
	 for $\sqrt{s} = 7$~TeV centre--of--mass energy with CMSSW\_3\_1\_2 and are located in 
         subdirectories of {\tt /castor/cern.ch/user/friis/TauFakeRate}.  
         For the QCD di--jet samples, the ntuples are binned by generator level $\hat P_{T}$, 
         and must be weighted to correspond to the same integrated
         luminosity when the used to produce the fake--rate histograms.} 
\label{tabBgEstFakeRate_NtupleCastorLocations}
\end{center}
\end{table}

As described in section~\ref{secFakeRateParametrization}, different sets of
fake--rate probabilities are computed for the highest $P_{T}$ (``leading'') and
second highest $P_{T}$ jet in QCD di--jet events, for jets in QCD events in
which a muon is reconstructed and for jets in ``electroweak'' events selected by
requiring a $W$ boson in the final state.  The idea of computing separate sets
of fake--rates for these samples is to ``cover'' differences in fake--rates
between light quark, heavy quark and gluon jets in this way.  We suggest to use
these different sets to compute corresponding background estimates separately
for each set, compute the ``final'' background estimate obtained from the
fake--rate method as the average of the four sets and take the difference
between the average and the minimum/maximum estimate as systematic uncertainty.


\subsubsection{Fake--rate determination}
\label{secAppendixFakeRateDetermination}

The fake--rates are stored in three--dimensional histograms (of ROOT type TH3F)
which are used as look--up tables.
The histograms are parametrized by $P_{T}$ ($x$--axis), $\vert\eta\vert$ ($y$--axis) and $R_{jet}$ ($z$--axis)
as described in section~\ref{secFakeRateParametrization}.
Fake--rate probabilities and tau identification efficiencies
are computed according to equation~\ref{eqBgEstFakeRate_fr}, separately for each bin.

Per default, the three--dimensional histograms are defined by the bin--edges:
\begin{itemize}
\item $P_{T}$ = { 20, 25, 30, 40, 60, 120 }
\item $\vert\eta\vert$ = { 0, 0.5, 1.0, 1.2, 1.5, 2.0, 2.5 }
\item $R_{jet}$ = { 0, 0.0125, 0.025, 0.05, 0.075, 0.10, 0.15, 0.20, 0.30 }
\end{itemize}
in the $P_{T}$, $\vert\eta\vert$, $R_{jet}$ parameter space.

The binning has been optimized for the $Z \rightarrow \tau^{+} \tau^{-} \rightarrow \mu + \tau\mbox{-jet}$ analysis.  
A finer binning has been used in the region $P_{T} \gtrsim 20$~GeV, since both the tau identification efficiency
and the fake--rate values considerably vary in the vicinity of the $P_{T}$
threshold.  Note that you will need to change the binning in case you are
analyzing tau leptons of significantly higher $P_{T}$.

\begin{table}[t]
\begin{center}
\tablesize
\begin{tabular}{|l|l|}
\hline
Filename & Content \\
\hline
\hline
\multicolumn{2}{|c|}{\textbf{Fake--rates}} \\
\hline
{\tt dijet\_highpt\_histograms.root} & for the highest $P_{T}$ jet in QCD di--jet events \\
{\tt dijet\_secondpt\_histograms.root} & for the second highest $P_{T}$ jet in QCD di--jet events \\
{\tt ppmux\_histograms.root} & for jets in muon enriched QCD events \\
{\tt wjets\_histograms.root} & for jets in $W$+jets events \\
\hline
\hline
\multicolumn{2}{|c|}{\textbf{Tau identification efficiencies}} \\
\hline
{\tt ztt\_histograms.root} & for $Z \rightarrow \tau^{+} \tau^{-}$ events \\
\hline
\end{tabular}
\end{center}
\begin{center}
\caption{\captiontext Location of ROOT files containing fake--rate probability (top part)
         and tau identification efficiency (bottom part) histograms.
         The ROOT files are located in subdirectories of {\tt /afs/cern.ch/user/f/friis/public/TauFakeRateMar11/}.}
\label{tabBgEstFakeRate_HistogramCastorLocations}
\end{center}
\end{table}

The fake--rate probabilities for jets in QCD di--jet, muon enriched QCD and $W$+jets background events
are stored in separate ROOT files, which are available on castor at the locations 
given in table~\ref{tabBgEstFakeRate_HistogramCastorLocations}.
Each ROOT file contains histograms for various combinations of tau identification cuts and discriminator requirements.
For each such requirement, a total of three histograms is contained in the ROOT file,
storing the number of events passing the requirements of the numerator and the denominator in equation~\ref{eqBgEstFakeRate_fr},
and the fake--rate probability and tau identification efficiency values that are
computed by dividing the numerator by the denominator histogram.
%The naming convention for the histograms is detailed in table~\ref{}.
%Per default,
For the ROOT files available on castor,
the denominator is defined by the selection $P_{T} > 20$~GeV, $\vert\eta\vert < 2.1$,
leading track $P_{T} > 5$~GeV and discriminators against electrons and muons passed.

In order to produce sets of histograms matching the tau identification criteria used in your analysis,
you need to edit the configuration files specifying the selection criteria for numerator and denominator.
The  selection criteria for the numerator are defined in \newline
\verb|RecoTauTag/TauAnalysisTools/python/fakeRate/histogramConfiguration.py|,
while the  selection criteria for the denominator are defined in \newline
\verb|RecoTauTag/TauAnalysisTools/test/fakeRate/sources.py|.
The binning, location of the Ntuple files, desired output location and the luminosity normalization
factors are also defined in the latter configuration file.
Note that the selection criteria applied in the denominator must not be looser than the requirements 
applied during event skimming.

After editing the configuration files, you can start executing the {\it cmsRun} jobs 
which produce the three--dimensional histograms.
Parallel execution of those jobs on a condor compatible batch system is supported.
The jobs are submitted by the command:
\begin{small}
\begin{verbatim} 
  cd $CMSSW_BASE/src/RecoTauTag/TauTagTools/test/fakeRate 
  condor_submit SubmitHistograms.jdl 
\end{verbatim}
\end{small}
Once all jobs have finished execution, you need to merge the output of different jobs.
Execute:
\begin{small}
\begin{verbatim}
   cd $CMSSW_BASE/src/RecoTauTag/TauTagTools/test/fakeRate 
  ./mergeHistograms.sh
\end{verbatim}
\end{small}

In the case that the user does not have access to a batch system supporting
condor, the histograms can be merged sequentially by executing:
\begin{small}
\begin{verbatim}
   cd $CMSSW_BASE/src/RecoTauTag/TauTagTools/test/fakeRate 
   python make_histograms.py
\end{verbatim}
\end{small}

Finally, in order to use the sets of histograms as look--up tables 
for production of per--jet and of event weights,
edit the configuration file \newline
\verb|RecoTauTag/TauAnalysisTools/PFTauEfficiencyAssociator_cfi.py|
and set the {\it filename} attributes of {\it efficiencySources} configuration parameters
to the location of the ROOT files produced by executing {\it mergeHistograms.sh}.
(our recommendation is to copy the ROOT files into the {\it public} subdirectory
 of your home directory that you have on \verb|/afs/cern.ch/user/..|).


\subsubsection{Weight production}
\label{secAppendixWeightProduction}

The production of the per--jet and of the event weights proceeds in two stages.
In the first stage, per--jet weights are computed according to
equations~\ref{eqBgEstFakeRate_frSimpleJetWeight} (\ref{eqBgEstFakeRate_frCDFtypeJetWeight})
and stored in the RECO/AOD event--content as pat::LookupTableRecord objects
associated to the reco::PFTau collection.  In the second stage, event
weights are computed according to
equations~\ref{eqBgEstFakeRate_frSimpleEventWeight} (\ref{eqBgEstFakeRate_frCDFtypeEvent_and_CompositeParticleWeight}).
are embedded into the pat::Tau collection.

In order to compute the per--jet weights, you first need to add:
\begin{small}
\begin{verbatim}
  from RecoTauTag.TauAnalysisTools.PFTauEfficiencyAssociator_cfi import *
  from TauAnalysis.BgEstimationTools.fakeRateJetWeightProducer_cfi.py import *
\end{verbatim}
\end{small}
to your top--level python configuration file (the one ending with {\it \_cfg.py}) 
and add the sequence {\it associateTauFakeRates} plus the module {\it bgEstFakeRateJetWeights} to your analysis path.
You then need to edit the file \newline
\verb|TauAnalysis/BgEstimationTools/python/fakeRateJetWeightProducer_cfi.py| and adapt 
the following configuration parameters to the needs of your analysis:
\begin{itemize}
\item {\it method} \\
  the string type parameter which specifies if per--jet weights are to be computed via
  either equation~\ref{eqBgEstFakeRate_frSimpleJetWeight} ("simple") 
  or equation~\ref{eqBgEstFakeRate_frCDFtypeJetWeight} ("CDF") 
\item {\it allTauJetSource} \\
  the InputTag that specifies the reco::PFTau collection stored in the RECO/AOD event--content
  for which tau identification efficiency and fake--rate probabilities are to be computed.
  Per default, tau identification efficiencies and fake--rates are computed
  for the reco::PFTau collection reconstructed by the particle--flow based tau identification algorithm
  for a ``shrinking signal cone'' of size $dR_{sig} = 5.0 / E_{T}$.
  Note that the value of this configuration parameter needs to match the collection 
  used when determining the tau identification and fake--rate probabilities as described in section~\ref{secAppendixFakeRateDetermination}
  as well as the collection used as input for pat::Tau production,
  specified in the configuration file \newline 
  \verb|PhysicsTools/PatAlgos/python/producersLayer1/tauProducer_cfi.py|
\item {\it preselTauJetSource} \\
  the InputTag which specifies the collection of tau--jet candidates passing the preselection requirements
  defined by the denominator of equation~\ref{eqBgEstFakeRate_fr}.
  The collection of preselected tau--jet candidates may either be of type 
  std::vector$<$reco::PFTau$>$ or of type std::vector$<$pat::Tau$>$
\item {\it frTypes} \\
  the configuration parameter set which specifies the different sets of fake--rate probabilities
  for which per--jet weights are to be computed.
  The parameter which you need to adapt to your analysis is the InputTag of the tau--jet discriminator collection
  which indicates whether tau--jet candidates have passed the preselection or not.
  The tau--jet discriminator collection needs to be of type reco::PFTauDiscriminator.
  The value of this discriminator is expected to be 
  $1$ if a given tau--jet candidates passes all preselection requirements
  and $0$ otherwise.
  Note that a single discriminator collection needs to contain the information
  concerning all cuts and tau identification discriminators applied during preselection.
  You may need to produce such discriminator collection prior to adding the {\it bgEstFakeRateJetWeights} module
  to your analysis module (an example for this can e.g. be found in \newline
  \verb|TauAnalysis/RecoTools/python/recoPFTauIdentification_cfi.py|)
\end{itemize}
(the other configuration parameters can be kept at their default values).

In order to compute the event weights, you need to add:
\begin{small}
\begin{verbatim}
  from TauAnalysis.BgEstimationTools.fakeRateEventWeightProducer_cfi.py import *
\end{verbatim}
\end{small}
to your top--level python configuration file and add the module {\it bgEstFakeRateEventWeights} to your analysis path.
Per default, all configuration parameters for the {\it bgEstFakeRateEventWeights} module are taken
from the parameters of the {\it bgEstFakeRateJetWeights} module which you already configured.
This guarantees consistency between the configuration parameters of the two modules and simplifies the configuration
to the extend that there is no need to edit the \newline
\verb|TauAnalysis/BgEstimationTools/python/fakeRateEventWeightProducer_cfi.py| file.

As the pat::LookupTableRecord objects are needed for production of the pat::Tau collection,
the first stage needs to proceed before the PAT--tuple production sequence,
while the second stage proceeds best after the PAT--tuple production sequence.

\subsubsection{Addition of weights to pat::Taus}

In order to add the per--jet weights produced as described in section~\ref{secAppendixWeightProduction}
to the pat::Tau objects, such that they can be used in your analysis later,
you simply need to add:
\begin{small}
\begin{verbatim}
  from PhysicsTools.PatAlgos.producersLayer1.tauProducer_cfi.py import *
  from RecoTauTag.TauAnalysisTools.PFTauEfficiencyAssociator_cfi import *
  from TauAnalysis.BgEstimationTools.fakeRateJetWeightProducer_cfi.py import *

  process.allLayer1Taus.efficiencies = cms.PSet()
  build_pat_efficiency_loader(shrinkingConeZTTEffSimAssociator, 
	                      None, process.allLayer1Taus.efficiencies)
  build_pat_efficiency_loader(shrinkingConeWJets, 
                              None, process.allLayer1Taus.efficiencies)
  build_pat_efficiency_loader(shrinkingConeMuEnrichedQCDAssociator, 
                              None, process.allLayer1Taus.efficiencies)
  build_pat_efficiency_loader(shrinkingConeDiJetHighPt, 
                              None, process.allLayer1Taus.efficiencies)
  build_pat_efficiency_loader(shrinkingConeDiJetSecondPt, 
                              None, process.allLayer1Taus.efficiencies)
  frTypes = getPSetAttributes(process.bgEstFakeRateJetWeights.frTypes)	
  for frType in frTypes: 
    frLabel = "".join(["bgEstFakeRateJetWeight", "_", frType])
    frInputTag = cms.InputTag('bgEstFakeRateJetWeights', frType)
    setattr(process.allLayer1Taus.efficiencies, frLabel, frInputTag)

  process.allLayer1Taus.addEfficiencies = cms.bool(True)
\end{verbatim}
\end{small}
to your top--level python configuration file.

\subsubsection{Application of weights in the analysis}

As described in section~\ref{secFakeRateApplication}, 
usage of the per--jet and of the event weights for the purpose of estimating background contributions
via the fake--rate technique requires to separate steps.

In the first step, you need to loosen the tau identification cuts and discriminators and apply only
the preselection requirements for which you have computed the fake--rates,
matching the denominator of equation~\ref{eqBgEstFakeRate_fr}.

In the second step, you will then fill histograms of tau--jet related observables with jet--weights,
histograms of per--particle quantities for particles of types different from hadronically decaying tau leptons with event weights
and sum the event weights in order to obtain an estimate for the total number of background events 
passing all event selection criteria applied on the final analysis level.
The following code snippet demonstrates how the event and per--jet weights can be accessed from within an EDAnalyzer:
\begin{small}
\begin{verbatim}
  edm::Handle<double> evtWeight;
  edm::InputTag evtWeightSource('bgEstFakeRateEventWeights', 'qcdMuEnriched');
  evt.getByLabel(evtWeightSource, evtWeight);

  std::cout << "event weight = " << (*evtWeight) << std::endl;

  edm::Handle<pat::TauCollection> patTaus;
  evt.getByLabel('cleanLayer1Taus', patTaus);

  for ( std::vector<pat::Tau>::const_iterator patTau = patTaus->begin(); 
        patTau != patTaus->end(); ++patTau ) {
    std::cout << "per-jet weight = " 
              << patTau->efficiency("bgEstFakeRateJetWeight_qcdMuEnriched").value() 
              << std::endl;
    std::cout << "(tau id. efficiency = " 
              << patTau->efficiency("effByStandardChainZTTsim").value() 
              << "," << " fake-rate = " 
              << patTau->efficiency("frByStandardChainMuEnrichedQCDsim").value() 
              << ")" << std::endl;
  }
\end{verbatim}
\end{small}

The names of the InputTags which you need to use in order to access the event weights
computed for different sets of fake--rate probabilities described in section~\ref{secAppendixFakeRateDetermination}
are composed as detailed in table~\ref{tabBgEstFakeRate_evtWeightLabels}.

Table~\ref{tabBgEstFakeRate_jetWeightLabels} describes the schema how the 
string type arguments which need to be passed to the pat::Tau::efficiency function 
are composed.

\begin{table}[t]
\begin{center}
\tablesize
\begin{tabular}{|l|l|}
\hline
Label & Instance \\
\hline
bgEstFakeRateEventWeights & qcdDiJetLeadJet \\
                          & qcdDiJetSecondLeadJet \\
                          & qcdMuEnriched \\
                          & WplusJets \\
\hline
\end{tabular}
\end{center}
\begin{center}
\caption{\captiontext Labeling schema for ``simple''/``CDF--type'' event weights defined by 
         equations~\ref{eqBgEstFakeRate_frSimpleEventWeight}/~\ref{eqBgEstFakeRate_frCDFtypeEvent_and_CompositeParticleWeight}.
         The left (right) column specifies the label (instance) name parameter of the InputTag
         which you need to pass to the edm::Event::getByLabel function
         in order to access the value of the weight for a given event
         (e.g. edm::InputTag("bgEstFakeRateEventWeights", "qcdMuEnriched")).}
\label{tabBgEstFakeRate_evtWeightLabels}
\end{center}
\end{table}

\begin{table}[t]
\begin{center}
\tablesize
\begin{tabular}{|l|l|l|}
\hline
Prefix & Suffix \\
\hline
bgEstFakeRateJetWeight\_ & qcdDiJetLeadJet \\
                         & qcdDiJetSecondLeadJet \\
                         & qcdMuEnriched \\
                         & WplusJets \\
\hline
\end{tabular}
\end{center}
\begin{center}
\caption{\captiontext Labeling schema ``simple''/``CDF--type'' per--jet weights defined by 
         equations~\ref{eqBgEstFakeRate_frSimpleJetWeight}/~\ref{eqBgEstFakeRate_frCDFtypeJetWeight}.
         The argument which you need to pass to the pat::Tau::efficiency function in order to access the values
         of per--jet weights is given by concatenation of the left and right columns
         (e.g. patTau$\rightarrow$("bgEstFakeRateJetWeight\_qcdMuEnriched")).}
\label{tabBgEstFakeRate_jetWeightLabels}
\end{center}
\end{table}

For debugging purposes it may be useful for you to access the ``raw''
tau identification and fake--rate probabilities taken from the TH3s,
before the values are passed to the {\it bgEstFakeRateJetWeights} module.
Per default, those probabilities are available via the pat::Tau::efficiency function.
The string type arguments which need to be passed to the pat::Tau::efficiency function 
in order to access the ``raw''
tau identification and fake--rate probabilities
are composed as described in table~\ref{tabBgEstFakeRate_rawProbabilityValues}.

\begin{table}[t]
\begin{center}
\tablesize
\begin{tabular}{|l|l|l|}
\hline
Prefix & Discriminator & Sample \\
\hline
eff & ByStandardChain & ZTTsim \\
\hline
fr  & ByStandardChain & DiJetHighPtsim \\
    &                 & DiJetSecondPtsim \\
    &                 & MuEnrichedQCDsim \\
    &                 & WJetssim \\
\hline
\end{tabular}
\end{center}
\begin{center}
\caption{\captiontext Labeling schema for tau identification efficiencies and fake--rates
         entering the computation of the per--jet weights defined by 
         equations~\ref{eqBgEstFakeRate_frSimpleJetWeight}/~\ref{eqBgEstFakeRate_frCDFtypeJetWeight}.
	 You can access the tau identification efficiency and fake--rate values
         by calling the pat::Tau::efficiency function,
         passing as argument the concatenation of the ``Prefix'' with the ``Discriminator'' and the ``Sample'' column
	 (e.g. patTau$\rightarrow$("effByStandardChainZTTsim")).}
\label{tabBgEstFakeRate_rawProbabilityValues}
\end{center}
\end{table}


\clearpage
\begin{thebibliography}{9}

\bibitem{refFakeRate2009dataResults}
  M.~Bachtis {\it et al.}, ``Particle-flow reconstruction of $0.9$~TeV and $2.36$~TeV collision events in CMS'',
  CMS AN--2010/031
\bibitem{PFlowTauReco} 
  S.~Gennai {\it et al.},``Tau reconstruction and identification with particle-flow techniques using the CMS detector at LHC'', 
  CMS AN--2008/043
\bibitem{TaNCnote}
  J.~Conway, E.~Friis, C.~Veelken, ``Tau identification using the Tau Neural Classifier Algorithm'',
%``Tau identification and decay mode reconstruction using neural networks'',
  CMS AN in preparation
\bibitem{EWKtauNote}
  S.~Bagliesi et al., ``Tau production from $Z$ decays in ${\mathrm{pp}}$ collisions at $\sqrt{s} = 7~$TeV'',
  EWK--09--015; in preparation
\bibitem{CDFtypeFakeRateMethod}
  D.~Jang, ``Search for MSSM Higgs decaying to $\tau$ pairs in $p\bar{p}$ collision at $\sqrt{s} = 1.96$~TeV at CDF'',
  PhD thesis, Rutgers University, 2006; 
  A.~Abulencia {\it et al.} (CDF Collaboration), 
  ``Search for neutral MSSM Higgs bosons decaying to tau pairs in $p\bar{p}$ collisions at $\sqrt{s} = 1.96$~TeV'',
  Phys.\ Rev.\ Lett.\ {\bf 96} (2006) 011802;
  C.~Cuenca Almenar, ``Search for the neutral MSSM Higgs bosons in the $\tau\tau$ decay channels at CDF Run II'',
  PhD thesis, Universitat de Valencia, 2008
\bibitem{refPDGid} 
  C.~Amsler {\it et al.} (Particle Data Group), Physics Letters B {\bf 667}, 1 (2008)

\end{thebibliography}

\end{document}
