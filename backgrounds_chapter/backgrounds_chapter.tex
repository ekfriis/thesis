\ifx\master\undefined\input{settings/autocompile}\fi

\newcommand{\tablesize}{\small}

\chapter{Data--Driven Background Estimation}
\label{ch:backgrounds}

\section{Introduction}
%
For the result of this analysis to be reliable, it is of paramount importance
that the backgrounds be well understood.  The CMS experiment has adopted a
policy that if possible, all background processes should be measured in a
``data--driven'' way.  By requiring that the background comes from data, biases
due to incorrectly modeling the background processes in simulation can be
eliminated.  In general, the data--driven methods also have the feature that
they are independent of the uncertainty on the integrated luminosity.  This
analysis measures the backgrounds using two complementary methods.

The template method
fits background shape templates (obtained from data) to the $M_{vis}$ spectrum
of events selected in the $Z/\gamma^{*} \rightarrow \tau^{+} \tau^{-}
\rightarrow \mu + \tau_{had}$ cross--section analysis and is described in
section~\ref{sec:template}.  
The fake--rate technique is based on applying
probabilities for quark and gluon jets to be misidentified as hadronic tau
decays to events passing all event selection criteria except the tau
identification requirements.  The probabilities with which jets fake hadronic
tau signatures are measured in data.  The fake--rate technique is detailed in
section~\ref{sec:fakerate}.

\section{Background enriched control regions}
\label{sec:controlregions}
%
The criteria applied to select events in the background enriched control regions
are based on the work described in~\cite{CMS_AN_2010-088}.  With respect to that
work, the muon isolation criteria applied to select \ZMM, \WpJets, \ttbarpJets
and QCD background enriched control samples have been changed to relative
isolation with respect to charged hadrons and neutral electromagnetic objects
reconstructed by the particle--flow algorithm.   The basic strategy to select
the enriched backgrounds is to disable, or invert, the specific selections of
Chapter~\ref{ch:selections} that were implemented to reject the corresponding
background.  The criteria applied to select events in the different background
enriched control samples of the muon + tau--jet channel are summarized in
table~\ref{tab:EventSelectionMuTauBgControlRegions}.  The goal of the background
enriched selection process is to select different background processes with high
purity.  A highly pure background control sample improves the stability of
inferences about the signal region using information in the enriched control
region.

Tau--jet candidates considered in the \ZMM sample where the
reconstructed tau--jet candidate is faked by a misidentified muon and in the
\ttbarpJets control sample are required to pass the ``loose'' TaNC
discriminator.  In the $\ZMM$ sample where the reconstructed tau--jet candidate
is faked by a misidentified quark or gluon jet, the \WpJets and the QCD
enriched control samples, tau--jet candidates are required to pass the ``very
loose'', but fail the ``loose'' TaNC discriminator.  The set of triggers used to
select events in the background enriched control samples is the same as for the
analysis (see table~\ref{tab:AHtoMuTauTriggers}).  Monte Carlo simulated events
are required to pass the HLT\_Mu9 trigger path and are weighted according to the
description in Chapter~\ref{ch:corrections}, in order to account for
the difference in efficiency between HLT\_Mu9 and the trigger paths required to
have passed in the data.

\begin{table}[t]
\begin{center}
\tablesize
\begin{tabular}{|l|c|c|c|c|c|}
\hline
\multirow{2}{17mm}{Requirement} & \multicolumn{5}{c|}{Enriched background process} \\
%\hhline{|~|-----|}
 & \multicolumn{2}{c|}{$Z \to \mu^{+} \mu^{-}$} & \multirow{2}{20mm}{$W$ + jets} & \multirow{2}{20mm}{$t\bar{t}$ + jets} & \multirow{2}{20mm}{QCD} \\
%\hhline{|~|--|~~~|}
 & Muon fake & Jet fake & & & \\
\hline
\hline
Muon rel.\ iso.   & $< 0.15$ & $< 0.1$ & $< 0.1$ & $< 0.1$ & $> 0.10$~$\&\&$~$< 0.30$ \\
Muon Track IP    & - & - & - & - & - \\
Tau TaNC discr.\  & - & $^{1}$ & $^{1}$ & medium passed & $^{1}$ \\
Tau 1$\|$3-Prong & - & - & - & - & - \\
Charge(Tau) = $\pm$1 & - & - & - & - & - \\
Tau $\mu$-Veto & inverted & applied & applied & applied & applied \\
Charge(Muon+Tau) & applied & - & - & applied & - \\                         
$M_{T}$(Muon-MET) & - & $< 40$~\GeV & - & - & $< 40$~\GeV \\
$P_{\zeta} - 1.5 \cdot P_{\zeta}^{vis}$ & $> -20$~\GeV & - & - & - & $> -20$~\GeV \\
\hline
global Muons & $< 2$ & - & $< 2$ & $< 2$ & $< 2$ \\
central Jet Veto & - & - & $^{2}$ & - & - \\
b-Tagging & - & - & - & $^{3}$ & - \\ 
\hline
\end{tabular}
\end{center}
\begin{small}
\mbox{$^{1}$ vloose passed $\&\&$ loose failed}
\mbox{$^{2}$ no Jets of $E_{T} > 20$~\GeV within $\vert \eta \vert < 2.1$ (other than the $\tau$--jet candidate)} 
\mbox{$^{3}$ min.\ two Jets of $E_{T} > 40$~\GeV, at least one of which with
$E_{T} > 60$~\GeV and} \\
\hspace{5mm} \mbox{at least of which with ``TrackCountingHighEff'' discriminator $> 2.5$}
\end{small}
\begin{center}
\caption[Criteria used to select background enriched control regions]{\captiontext 
         Criteria to select events in different background enriched control samples.
         Hyphens indicate event selection criteria which are not applied.}
\label{tab:EventSelectionMuTauBgControlRegions}
\end{center}
\end{table}

The number of events observed in the different control samples is compared to
the Monte Carlo expectation in table~\ref{tab:ResultsMuTauBgControlRegions}.
Except for the contribution of \ZMM events in which the
reconstructed tau--jet candidate is due to a misidentified quark or gluon jet,
good agreement between data and Monte Carlo simulation is observed.  Differences
observed between data and simulation will be accounted for as systematic
uncertainties.

The distributions of visible and ``full'' \TT invariant mass reconstructed by
the SVfit algorithm observed in the background enriched control regions is
compared to the Monte Carlo simulation in
figures~\ref{fig:VisMassMuTauBgControlRegions}
and~\ref{fig:SVfitMassMuTauBgControlRegions}.  The template for the \WpJets
background has been corrected for the bias on the $M_{vis}^{\mu \tau_{had}}$
shape caused by the $M_{T}^{\mu\MET} < 50$~\GeV and $P_{\zeta} - 1.5 \cdot
P_{\zeta}^{vis} > -20$~\GeV requirements applied in the cross--section analysis
via the reweighting procedure described in~\cite{CMS_AN_2010-088}.  In the
\ttbarpJets enriched control region a peak at the $Z$ mass is observed in data,
which is not modeled by the Monte Carlo samples considered.  The peak could be
due to $Z \to \mu^{+} \mu^{-}$ events produced in association with heavy quarks.
On the other hand, the contribution from \ttbarpJets events to that sample seems
to be overestimated.  The origin of the $Z$ mass peak merits further
investigations, but overall the \ttbarpJets is a negligible background
contribution.

\begin{figure}
\setlength{\unitlength}{1mm}
\begin{center}
\begin{picture}(150,170)(0,0)
\put(0.5, 118){\mbox{\includegraphics*[width=52mm, angle=90]
  {backgrounds_chapter/figures/bgEstControlZtoMuTau_ZmumuMuonMisIdEnriched_mVisible.pdf}}}
\put(78.0, 118){\mbox{\includegraphics*[width=52mm, angle=90]
  {backgrounds_chapter/figures/bgEstControlZtoMuTau_ZmumuJetMisIdEnriched_mVisible.pdf}}}
\put(0.5, 60){\mbox{\includegraphics*[width=52mm, angle=90]
  {backgrounds_chapter/figures/bgEstControlZtoMuTau_WplusJetsEnriched_mVisible.pdf}}}
\put(78.0, 60){\mbox{\includegraphics*[width=52mm, angle=90]
  {backgrounds_chapter/figures/bgEstControlZtoMuTau_TTplusJetsEnriched_mVisible.pdf}}}
\put(0.5, 2){\mbox{\includegraphics*[width=52mm, angle=90]
  {backgrounds_chapter/figures/bgEstControlZtoMuTau_QCDenriched_mVisible.pdf}}}
%\put(0.5, 60){\mbox{\includegraphics*[width=52mm, angle=90]
%  {figures/bgEstControlZtoMuTau_WplusJetsEnriched_mVisible.pdf}}}
%\put(78.0, 60){\mbox{\includegraphics*[width=52mm, angle=90]
%  {figures/bgEstControlZtoMuTau_WplusJetsEnriched_mVisible.pdf}}}
%\put(0.5, 2){\mbox{\includegraphics*[width=52mm, angle=90]
%  {figures/bgEstControlZtoMuTau_TTplusJetsEnriched_mVisible.pdf}}}
%\put(78.0, 2){\mbox{\includegraphics*[width=52mm, angle=90]
%  {figures/bgEstControlZtoMuTau_QCDenriched_mVisible.pdf}}}
\put(-5.5, 170.5){\small (a)}
\put(72.0, 170.5){\small (b)}
\put(-5.5, 112.5){\small (c)}
\put(72.0, 112.5){\small (d)}
\put(-5.5, 54.5){\small (e)}
\put(72.0, 54.5){\small (f)}
\end{picture}
\caption[Visible mass distribution of the backgrounds in the signal and control
regions]{\captiontext 
	 Distribution of visible mass of muon plus the tau--jet candidate reconstructed
         in the background enriched control samples for 
         $Z \to \mu^{+} \mu^{-}$ (a) and (b), $W$ + jets (c), $t\bar{t}$ + jets (d) and QCD multi-jet (e) backgrounds.
%         $Z \to \mu^{+} \mu^{-}$ (a) and (b), $W$ + jets before (c) 
%         and after (d) the bias correction described in~cite{CMS_AN_2010-088}
%         $t\bar{t}$ + jets (e) and QCD multi-jet (f) backgrounds.
         In (a) reconstructed tau--jet candidates are expected to be dominantly due to misidentified muons,
         while in (b) they are expected to be mostly due to misidentified misidentified quark or gluon jets.}
\label{fig:VisMassMuTauBgControlRegions}
\end{center}
\end{figure} 

\begin{figure}
\setlength{\unitlength}{1mm}
\begin{center}
\begin{picture}(150,170)(0,0)
\put(0.5, 118){\mbox{\includegraphics*[width=52mm, angle=90]
  {backgrounds_chapter/figures/bgEstControlZtoMuTau_ZmumuMuonMisIdEnriched_mSVmethod.pdf}}}
\put(78.0, 118){\mbox{\includegraphics*[width=52mm, angle=90]
  {backgrounds_chapter/figures/bgEstControlZtoMuTau_ZmumuJetMisIdEnriched_mSVmethod.pdf}}}
\put(0.5, 60){\mbox{\includegraphics*[width=52mm, angle=90]
  {backgrounds_chapter/figures/bgEstControlZtoMuTau_WplusJetsEnriched_mSVmethod.pdf}}}
\put(78.0, 60){\mbox{\includegraphics*[width=52mm, angle=90]
  {backgrounds_chapter/figures/bgEstControlZtoMuTau_TTplusJetsEnriched_mSVmethod.pdf}}}
\put(0.5, 2){\mbox{\includegraphics*[width=52mm, angle=90]
  {backgrounds_chapter/figures/bgEstControlZtoMuTau_QCDenriched_mSVmethod.pdf}}}
%\put(0.5, 60){\mbox{\includegraphics*[width=52mm, angle=90]
%  {figures/bgEstControlZtoMuTau_WplusJetsEnriched_mSVmethod.pdf}}}
%\put(78.0, 60){\mbox{\includegraphics*[width=52mm, angle=90]
%  {figures/bgEstControlZtoMuTau_WplusJetsEnriched_mSVmethod.pdf}}}
%\put(0.5, 2){\mbox{\includegraphics*[width=52mm, angle=90]
%  {figures/bgEstControlZtoMuTau_TTplusJetsEnriched_mSVmethod.pdf}}}
%\put(78.0, 2){\mbox{\includegraphics*[width=52mm, angle=90]
%  {figures/bgEstControlZtoMuTau_QCDenriched_mSVmethod.pdf}}}
\put(-5.5, 170.5){\small (a)}
\put(72.0, 170.5){\small (b)}
\put(-5.5, 112.5){\small (c)}
\put(72.0, 112.5){\small (d)}
\put(-5.5, 54.5){\small (e)}
\put(72.0, 54.5){\small (f)}
\end{picture}
\caption[SVfit mass distribution of the backgrounds in the signal and control
regions]{\captiontext 
	 Distribution of ``full'' invariant mass reconstructed by the SVfit algorithm
         in the background enriched control samples for 
         $Z \to \mu^{+} \mu^{-}$ (a) and (b), $W$ + jets (c), $t\bar{t}$ + jets (d) and QCD multi-jet (e) backgrounds.
%         $Z \to \mu^{+} \mu^{-}$ (a) and (b), $W$ + jets before (c) 
%         and after (d) the bias correction described in~cite{CMS_AN_2010-088}
%         $t\bar{t}$ + jets (e) and QCD multi-jet (f) backgrounds.
         In (a) reconstructed tau--jet candidates are expected to be dominantly due to misidentified muons,
         while in (b) they are expected to be mostly due to misidentified misidentified quark or gluon jets.}
\label{fig:SVfitMassMuTauBgControlRegions}
\end{center}
\end{figure} 



\section{The Fake--rate Method}
\label{sec:fakerate}

In this note, we describe how knowledge of the probabilities with which quark
and gluon jets get misidentified as tau--jets may be utilized to obtain an
estimate of background contributions in physics analyses.  As an illustrative
example and in order to demonstrate the precision achievable with the method, we
present results of applying the fake--rate technique to estimate the
contributions of QCD, \WpJets, \ttbarpJets and $Z \rightarrow \mu^{+}
\mu^{-}$ backgrounds in the measurement of the $Z \rightarrow \tau^{+} \tau^{-}$
cross--section, in the channel $Z \rightarrow \tau^{+} \tau^{-} \rightarrow \mu
+ \tau\mbox{-jet}$.  Details of the analysis can be found in
reference~\cite{CMS-PAS-EWK-10-002}.

The results described in this note were obtained from Monte Carlo simulations of
the $Z \rightarrow \tau^{+} \tau^{-}$ signal and different background processes
for a centre--of--mass energy of $\sqrt{s} = 7$~\TeV.  Analysis of the $\sqrt{s}
= 7$~\\TeV data recorded in 2009~\cite{CMS-PAS-PFT-09-001} indicate that
the probabilities for quark and gluon jets to fake the signatures of hadronic
tau decays are well modeled by the Monte Carlo simulation.  Once data--samples
of sufficient event statistics are available at collision energies of $\sqrt{s}
= 7$~\TeV, fake--rates at the higher centre--of--mass energy will be measured in
data and the values obtained from data will henceforth be used for the purpose
of estimating background contributions via the fake--rate technique.

\subsection{Parameterization of fake--rates}

\begin{figure}[t]
\setlength{\unitlength}{1mm}
\begin{center}
\begin{picture}(150,112)(0,0)
\put(10.5, 60){\mbox{\includegraphics*[height=52mm]{backgrounds_chapter/2010_fake_rate_note/figures/AN2008_043_fig20left.pdf}}}
\put(86.0, 60){\mbox{\includegraphics*[height=52mm]{backgrounds_chapter/2010_fake_rate_note/figures/AN2008_043_fig20right.pdf}}}
\put(10.5, 2){\mbox{\includegraphics*[height=52mm]{backgrounds_chapter/2010_fake_rate_note/figures/AN2008_043_fig22left.pdf}}}
\put(86.0, 2){\mbox{\includegraphics*[height=52mm]{backgrounds_chapter/2010_fake_rate_note/figures/AN2008_043_fig22right.pdf}}}
%\put(-5.5, 112.5){\small (a)}
%\put(72.0, 112.5){\small (b)}
%\put(-5.5, 54.5){\small (c)}
%\put(72.0, 54.5){\small (d)}
\end{picture}
\caption[$\pt$ and $\eta$ dependency of tau ID performance]{Cumulative
efficiencies (left) and fake--rates (right) of successively applied tau
identification cuts of the ``shrinking signal cone'' particle--flow based tau
identification algorithm described in~\cite{CMS-PAS-PFT-08-001} as function of
$\pt^{jet}$ (top) and $\eta^{jet}$ (bottom) of tau--jet candidates.  The
efficiencies/fake--rates for the complete set of tau identification criteria are
represented by the blue (downwards facing) triangles.}
\label{figPFTauReco_EfficienciesAndFakeRates}
\end{center}
\end{figure} 

Efficiencies and fake--rates of the tau identification algorithm based on
requiring no tracks of $\pt > 1~\GeVc$ and ECAL energy deposits of $\pt >
1.5~GeV$ reconstructed within an ``isolation cone'' of size $dR_{iso} = 0.5$ and
outside of a ``shrinking signal cone'' of size $dR_{sig} = 5.0 / E_{T}$ as it is
used in the $Z \rightarrow \tau^{+} \tau^{-} \rightarrow \mu + \tau\mbox{-jet}$
analysis are displayed in figure~\ref{figPFTauReco_EfficienciesAndFakeRates}.
In order to account for the visible \pt and $\eta$ dependence, we
parametrize the fake--rates in bins of transverse momentum and pseudo--rapidity.
As we will show in section~\ref{sec:FakeRateApplication}, the parametrization of
the fake--rates by \pt and $\eta$ makes it possible to not only estimate the
total number of background events contributing to physics analyses, but to model
the distributions of kinematic observables with a precision that is sufficient
to extract information on the background shape.

We add a third quantity, the $E_{T}$-weighted jet--width $R_{jet}$, to the
parametrization in order to account for differences between the fake--rates of
quark and gluon jets.  The jet width is defined as 
\begin{equation}
R_{jet} = \sqrt{E \left( \eta^2 \right) + E \left( \phi^2 \right)}
\end{equation}
where $E \left( \eta^2 \right)$ ($E \left( \phi^2 \right)$) is the second $\eta$
($\phi$) moment of the jet constituents, weighted by constituent transverse
energy. Analyses performed by the CDF collaboration~\cite{CDFMSSMHiggs,
CDFFakerateDJang, CDFFakerateAlmenar}
found that systematic uncertainties on background estimates obtained from the
fake--rate method are reduced in case differences between quark and gluon jets
get accounted for in this way.

\subsection{Measurement of fake--rates}

Efficiencies and fake--rates are then obtained by counting the fraction of tau--jet candidates
passing all tau identification cuts and discriminators
in a given bin of $\pt^{jet}$, $\eta_{jet}$ and $R_{jet}$:
\begin{equation}
P_{fr} \left( \pt^{jet}, \eta_{jet}, R_{jet} \right) := 
  \frac{N_{jets} \left( \pt^{jet}, \eta_{jet}, R_{jet} \vert \mbox{all tau ID cuts and discriminators passed} \right)}
       {N_{jets} \left( \pt^{jet}, \eta_{jet}, R_{jet} \vert \mbox{preselection passed} \right)}
\label{eqBgEstFakeRate_fr}
\end{equation}
The pre--selection in the denominator of equation~\ref{eqBgEstFakeRate_fr} in
general refers to \pt and $\eta$ cuts, which are applied with thresholds
matching those applied on the final analysis level, but may as well include
loose tau identification criteria (which may be applied \eg already during event
skimming).

Different sets of fake--rates are determined for the highest \pt and for the
second highest \pt jet in QCD di--jet events, for jets in a QCD event sample
enriched by the contribution of heavy quarks and gluons by requiring the
presence of a muon reconstructed in the final state, and for jets in
``electroweak'' events selected by requiring a $W$ boson in the final state.

\subsection{Application of Fake--rates}
\label{sec:FakeRateApplication}

Knowledge of the tau identification efficiencies and fake--rates as function of
the parameters $\pt^{jet}$, $\eta_{jet}$ and $R_{jet}$ as defined by
equation~\ref{eqBgEstFakeRate_fr} is utilized to obtain an estimate for the
contributions of background processes to physics analyses involving tau lepton
hadronic decays in the final state.  The basic idea is to replace tau
identification cuts and discriminators by appropriately chosen weights.

Application of the fake--rate technique consists of two stages.  The first stage
consists of loosening the tau identification cuts and discriminators and
applying only the preselection requirements defined by the denominator of
equation~\ref{eqBgEstFakeRate_fr}, in order to obtain an event sample dominated
by contributions of background processes, which are expected to increase by the
inverse of the (average) fake--rate, typically by a factor $\mathcal{O} \left(
100 \right)$.  In the second stage, weights are applied to all events in the
background dominated control sample, according to the probabilities $P_{fr}
\left( \pt^{jet}, \eta_{jet}, R_{jet} \right)$ for jets to fake the signature
of a hadronic tau decay.  After application of the weights, an estimate for the
total number of background events passing the tau identification cuts and
discriminators and thus contributing to the final analysis sample is obtained.

The fake--rate technique works best if all background contributions to the
analysis arise from misidentification of quark and gluon jets as hadronic tau
decays.  Corrections to the estimate obtained from the fake--rate technique  are
needed in case of background processes contributing to the final analysis sample
which either produce genuine tau leptons in the final state (e.g. $t\bar{t}$+
jets) or in which tau--jet candidates are due to misidentified electrons or
muons (e.g. $Z \rightarrow \mu^{+} \mu^{-}$, $Z \rightarrow e^{+} e^{-}$), as
the latter may fake signatures of hadronic tau decays with very different
probabilities than quark and gluon jets.

In the ``simple'' fake--rate method described in more detail in the next
section, the corrections are taken from Monte Carlo simulations.  Corrections
based on Monte Carlo are needed also to compensate for signal contributions to
the background dominated control sample.

An alternative to Monte Carlo based corrections is to utilize additional
information contained in the background dominated control sample.  The modified
version is described in section~\ref{secBgEstFakeRate_frCDFtypeWeights}.  It has
been used to estimate background contributions in searches for Higgs boson
production with subsequent decays into tau lepton pairs performed by the CDF
collaboration in TeVatron run $II$ data~\cite{CDFMSSMHiggs,
CDFFakerateDJang, CDFFakerateAlmenar}.  We will
refer to the modified version as ``CDF--type'' method in the following.

\subsubsection{``Simple'' weight method}

In the ``simple'' method all tau--jet candidates within the background dominated
event sample are weighted by the probabilities of quark and gluon jets to fake
the signature of a hadronic tau decay:
\begin{equation}
w_{jet}^{simple} \left( \pt^{jet}, \eta_{jet}, R_{jet} \right) := P_{fr} \left( \pt^{jet}, \eta_{jet}, R_{jet} \right)
\label{eqBgEstFakeRate_frSimpleJetWeight}
\end{equation}
These weights are applied to all jets in the background dominated control sample
which pass the preselection defined by the denominator of
equation~\ref{eqBgEstFakeRate_fr}.  Note that the weights defined by
equation~\ref{eqBgEstFakeRate_frSimpleJetWeight} can be used to estimate the
contributions of background processes to distributions of tau--jet related
observables.  They cannot be used as event weights.

In order to compare distributions of event level quantities or per--particle
quantities for particles of types different from tau leptons decaying
hadronically, event weights need to be defined.  Neglecting the small fraction
of background events in which multiple tau--jet candidates pass the complete set
of all tau identification cuts and discriminators, event weights can be computed
by summing up the per--jet weights defined by
equation~\ref{eqBgEstFakeRate_frSimpleJetWeight} over all tau--jet candidates in
the event which pass the preselection:
\begin{equation}
W_{event}^{simple} := \Sigma w_{jet}^{simple}
\label{eqBgEstFakeRate_frSimpleEventWeight}
\end{equation}

A bit of care is needed in case one wants to compare distributions of
observables related to ``composite particles'' the multiplicity of which depends
on the multiplicity of tau--jet candidates in the event (\eg combinations of
muon + tau--jet pairs in case of the $Z \rightarrow \tau^{+} \tau^{-}
\rightarrow \mu + \tau\mbox{--jet}$ analysis).  Per--particle weights need to be
computed for such ``composite particles'', depending on $\pt^{jet}$,
$\eta_{jet}$, $R_{jet}$ of its tau--jet candidate constituent, according to:
\begin{equation}
w_{comp-part}^{simple} \left( \pt^{jet}, \eta_{jet}, R_{jet} \right) := 
  w_{jet}^{simple} \left( \pt^{jet}, \eta_{jet}, R_{jet} \right)
\label{eqBgEstFakeRate_frSimpleCompositeParticleWeight}
\end{equation}

Different estimates are obtained for the fake--rate probabilities determined for
the highest and second highest $\pt$ jet in QCD di--jet events, jets in a muon
enriched QCD sample and jets in $W$+jets events.  
The arithmetic average of the
four estimates together with the difference between the computed average and the
minimum/maximum value is given in table~\ref{tabBgEstFakeRate_frSimpleResults}.

We take the average value as ``best'' estimate of the background contribution
and the difference between the average and the minimum/maximum estimate as its
systematic uncertainty.  We obtain a value of $\mathcal{O} \left( 15 \% \right)$
for the systematic uncertainty and find that the true sum of QCD, $W$+jets,
$t\bar{t}$+jets and $Z \rightarrow \mu^{+} \mu^{-}$ background contributions
agrees well with the ``best'' estimate obtained by the fake--rate method within
the systematic uncertainty.

Note that the estimate for the sum of background contributions which one obtains
in case one applies the ``simple'' fake--rate weights defined by
equation~\ref{eqBgEstFakeRate_frSimpleEventWeight} to a background dominated
control sample selected in data is likely to overestimate the true value of
background contributions by a significant amount.  The reason is that
contributions of the $Z \rightarrow \tau^{+} \tau^{-}$ signal are
non--negligible.  In fact, signal contributions to the background dominated
control sample are expected to be $14.9 \%$ and since the per--jet weights
computed by equation~\ref{eqBgEstFakeRate_frSimpleJetWeight} are larger on
average in signal than in background events, the signal contribution increases
by the weighting and amounts to $37.1 \%$ of the sum of event weights computed
by equation~\ref{eqBgEstFakeRate_frSimpleEventWeight} and given in
table~\ref{tabBgEstFakeRate_frSimpleResults}.

The contribution of the $Z \rightarrow \tau^{+} \tau^{-}$ signal needs to be
determined by Monte Carlo simulation and subtracted from the estimate obtained
by applying the ``simple'' fake--rate method to data, in order to get an
unbiased estimate of the true background contributions.

\subsubsection{``CDF--type'' weights}
\label{secBgEstFakeRate_frCDFtypeWeights}

%The ``simple'' method described in the last section does actually overestimate the sum of background contributions 
%to the final analysis by some amount.
%The reason is that signal events present in the background dominated control
%sample (selected by not applying any tau identification cuts or discriminators)
%contribute to the estimate.
%One way of handling the signal contribution is to subtract from the estimate obtained from the ``simple'' method
%a correction term which is to be estimated from Monte Carlo.
%Another way is to correct for the signal contribution by adjusting the weights,
%based on information that is in the analyzed data sample only,
%avoiding the need to rely on Monte Carlo based corrections.
%The idea of adjusting the weights in such a way that signal contributions cancel
%while an unbiased estimate of the sum of background contributions is maintained 
%is the basis of the ``CDF--type'' fake--rate method.

Instead of subtracting from the estimate obtained for the sum of background
contributions a correction determined by Monte Carlo simulation, the signal
contribution to the background dominated event sample selected in data can be
corrected for by adjusting the weights, based solely on information contained in
the analyzed data sample, hence avoiding the need to rely on Monte Carlo based
corrections.

In the ``CDF--type'' method, additional information, namely whether or not
tau--jet candidates pass or fail the tau identification cuts and discriminators,
is drawn from the data.  The desired cancellation of signal contributions is
achieved by assigning negative weights to those tau--jet candidates which pass
all tau identification cuts and discriminators, \ie to a fair fraction of
genuine hadronic tau decays, but to a small fraction of quark and gluon jets
only.  The small reduction of the background estimate by negative weights
assigned to quark and gluon jets is accounted for by a small increase of the
positive weights assigned to those tau--jet candidates for which at least one of
the tau identification cuts or discriminators fails.  In this way, an unbiased
estimate of the background contribution is maintained.

To be specific, the ``CDF--type'' weights assigned to tau--jet candidates are computed as:
\begin{equation}
w_{jet}^{CDF} \left( \pt^{jet}, \eta_{jet}, R_{jet} \right) 
:= 
\begin{cases} 
   \frac{P_{fr} \left( \pt^{jet}, \eta_{jet}, R_{jet} \right) \cdot 
         \varepsilon \left( \pt^{jet}, \eta_{jet}, R_{jet} \right)}
        {\varepsilon \left( \pt^{jet}, \eta_{jet}, R_{jet} \right) - P_{fr} \left( \pt^{jet}, \eta_{jet}, R_{jet} \right)}
      \mbox{if all tau ID cuts and discriminators passed} \\
   \frac{P_{fr} \left( \pt^{jet}, \eta_{jet}, R_{jet} \right) \cdot 
         \left( 1 - \varepsilon \left( \pt^{jet}, \eta_{jet}, R_{jet} \right) \right)}
        {\varepsilon \left( \pt^{jet}, \eta_{jet}, R_{jet} \right) - P_{fr} \left( \pt^{jet}, \eta_{jet}, R_{jet} \right)}
      %\mbox{ if at least one tau id. cut or discriminator failed}
      \mbox{otherwise}
\end{cases}
\label{eqBgEstFakeRate_frCDFtypeJetWeight}
\end{equation}

The basic idea of the ``CDF--type'' weights is to assign negative (positive)
weights to tau--jet candidates passing all tau identification cuts and
discriminators (failing at least one cut or discriminator), such that signal
contributions of genuine hadronic tau decays to the background dominated control
sample on average cancel after application of the weights, while providing an
unbiased estimate of the contribution of background processes arising from
misidentification of quark and gluon jets.

For the derivation of equation~\ref{eqBgEstFakeRate_frCDFtypeJetWeight} for the
``CDF--type'' weights assigned to tau--jet candidates, we will use the following
notation: Let $n_{\tau}$ ($n_{QCD}$) denote the total number of tau--jets (quark
and gluon jets) in a certain bin of transverse momentum $\pt^{jet}$,
pseudo--rapidity $\eta_{jet}$ and jet--width $R_{jet}$ and $n_{\tau}^{sel}$
($n_{QCD}^{sel}$) denote the number of tau--jets (quark and gluon jets) in that
bin which pass all tau identification cuts and discriminators.

By definition of the tau identification efficiency $\varepsilon := \varepsilon
\left( \pt^{jet}, \eta_{jet}, R_{jet} \right)$ and fake--rate $f := f \left(
\pt^{jet}, \eta_{jet}, R_{jet} \right)$:
\begin{eqnarray}
n_{\tau}^{sel} & = & \varepsilon \cdot n_{\tau} \nonumber \\
n_{QCD}^{sel} & = & f \cdot n_{QCD}.
\label{eqBgEstFakeRate_eff_and_frDef}
\end{eqnarray}

Depending on whether or not a given tau--jet candidate passes all tau
identification cuts and discriminators or not, we will assign a weight of value
$w_{passed}$ or $w_{failed}$ to it.

The values of the weights $w_{passed}$ and $w_{failed}$ shall be adjusted such
that they provide an unbiased estimate of the background contribution:
\begin{equation}
w_{passed} \cdot f \cdot n_{QCD} + w_{failed} \cdot \left( 1 - f \right) \cdot n_{QCD} \equiv n_{QCD}^{sel} = f \cdot n_{QCD}
\label{eqBgEstFakeRate_QCD}
\end{equation}
while averaging to zero for genuine hadronic tau decays:
\begin{equation*}
w_{passed} \cdot \varepsilon \cdot n_{\tau} + w_{failed} \cdot \left( 1 - \varepsilon \right) \cdot n_{\tau} \equiv 0.
\label{eqBgEstFakeRate_tau}
\end{equation*}
The latter equation yields the relation:
\begin{equation}
w_{passed} = -\frac{1 - \varepsilon}{\varepsilon} \cdot w_{failed},
\label{eqBgEstFakeRate_weightRelation}
\end{equation}
associating the two types of weights.  By inserting
relation~\ref{eqBgEstFakeRate_weightRelation} into
equation~\ref{eqBgEstFakeRate_QCD} we obtain:
\begin{eqnarray*}
& & -\frac{1 - \varepsilon}{\varepsilon} \cdot w_{failed} \cdot f \cdot n_{QCD} + w_{failed} \cdot \left( 1 - f \right) \cdot n_{QCD} 
 = f \cdot n_{QCD} \\
& \Rightarrow & \left( \frac{-f + \varepsilon \cdot f + \varepsilon - f \cdot \varepsilon}{\varepsilon} \right) \cdot w_{failed} = f \\
& \Rightarrow & w_{failed} = \frac{f \cdot \varepsilon}{\varepsilon - f} 
\end{eqnarray*}
and 
\begin{equation}
w_{passed} = -\frac{f \cdot \left( 1 - \varepsilon \right)}{\varepsilon - f}
\end{equation}
which matches exactly equation~\ref{eqBgEstFakeRate_frCDFtypeJetWeight} 
for the ``CDF--type'' weights applied to tau--jet candidates given in section~\ref{secBgEstFakeRate_frCDFtypeWeights}.


Event weights and the weights assigned to ``composite particles'' 
are computed in the same way as for the ``simple'' weights,
based on the weights assigned to the tau--jet candidates:
\begin{eqnarray}
W_{event}^{CDF} & := & \Sigma w_{jet}^{CDF} \nonumber \\
w_{comp-part}^{CDF} \left( \pt^{jet}, \eta_{jet}, R_{jet} \right) & := & 
  w_{jet}^{CDF} \left( \pt^{jet}, \eta_{jet}, R_{jet} \right),
\label{eqBgEstFakeRate_frCDFtypeEvent_and_CompositeParticleWeight}
\end{eqnarray}
where the sums extend over all jets in the background dominated control sample
which pass the preselection defined by the denominator of
equation~\ref{eqBgEstFakeRate_fr}.

The effect of the negative weights to compensate the positive weights
in case the ``CDF--type'' fake--rate method is applied 
to signal events containing genuine hadronic tau decays 
is shown in table~\ref{tabBgEstFakeRate_frCDFtypeResults}.
% illustrated in figure~\ref{figBgEstFakeRate_frCDFtypeResults_mVisibleSignal}.
As expected, positive and negative weights do indeed cancel
in the statistical average.

%\begin{figure}[t]
%\setlength{\unitlength}{1mm}
%\begin{center}
%\begin{picture}(150,54)(0,0)
%\put(0.5, 2){\mbox{\includegraphics*[height=52mm, viewport=23 25 525 404]{figures/plotBgEstFakeRateZtoMuTau_Ztautau_frSimpleMvisible.pdf}}}
%\put(78.0, 2){\mbox{\includegraphics*[height=52mm, viewport=23 25 525 404]{figures/plotBgEstFakeRateZtoMuTau_Ztautau_frCDFmVisible.pdf}}}
%\end{picture}
%\caption{\captiontext Distributions of visible invariant mass of muon plus tau--jet in
%	  $Z \rightarrow \tau^{+} \tau^{-}$ signal events weighted by ``simple'' weights 
%         computed according to equation~\ref{eqBgEstFakeRate_frSimpleCompositeParticleWeight} (left)
%         and ``CDF--type'' weights 
%         computed according to equation~\ref{eqBgEstFakeRate_frCDFtypeEvent_and_CompositeParticleWeight} (right).
%	  The signal contribution to the background estimate computed by the ``simple'' method is non--negligible 
%         and needs to be corrected for.
%	  The ``CDF--type'' weights achieve a statistical cancellation of positive and negative weights,
%         such that the total signal contribution averages to zero, avoiding the need for Monte Carlo based corrections.}
%\label{figBgEstFakeRate_frCDFtypeResults_mVisibleSignal}
%\end{center}
%\end{figure} 

Figures~\ref{figBgEstFakeRate_frCDFtypeResults_muonPt},~\ref{figBgEstFakeRate_frCDFtypeResults_tauJetPt} 
and~\ref{figBgEstFakeRate_frCDFtypeResults_mVisible}
demonstrate that an unbiased estimate of the background contribution by the ``CDF--type'' weights is maintained.
Overall, the estimates obtained are in good agreement with the contributions
expected for different background processes, indicating that the adjustment of negative and positive weights 
works as expected for the background as well.
%The agreement between expected contributions and the estimates obtained from the ``CDF--type'' method
%is summarized in terms of the total number of events in table~\ref{tabBgEstFakeRate_frCDFtypeResults}.

Results obtained by the ``CDF--type'' fake--rate method are summarized in
table~\ref{tabBgEstFakeRate_frCDFtypeResults}, in which the total number of
background events estimated by
equation~\ref{eqBgEstFakeRate_frCDFtypeEvent_and_CompositeParticleWeight} is
compared to the true background contributions.  The ``best'' estimate of the
background contribution obtained from the ``CDF--type'' method is again taken as
the arithmetic average of the estimates obtained by applying the fake--rate
probabilities for the highest and second highest $\pt$ jet in QCD di--jet
events, jets in a muon enriched QCD sample and jets in $W$+jets events.
Systematic uncertainties are taken from the difference between the computed
average value and the minimum/maximum estimate.  We obtain a value of
$\mathcal{O} \left( 15 \mbox{--} 20 \% \right)$ for the systematic uncertainty
of the ``CDF--type'' method, slightly higher than the systematic uncertainty
obtained for the ``simple'' method.  The small increase of systematic
uncertainties is in agreement with our expectation for fluctuations of the
jet--weights in case weights of negative and positive sign are used.

\subsection{k-Nearest Neighbor fake--rate calculation}
For the fake--rate method to give correct results, care must be taken that the
measured fake--rate is well defined in all of the regions of phase space where
it will be used.  Previous implementations~\cite{CMS_AN_2010-074} of the
fake--rate method in CMS implemented the fake--rate parameterization by binning
the numerator (tau ID passed) and denominator (tau ID passed and failed)
distributions in the three dimensions of the parameterizations. This method has
the disadvantage that the determination of the optimal binning is extremely
difficult to determine. Furthermore, any bins with no entries in the denominator
distribution caused the fake--rate to be undefined in those regions.

To overcome these problems, the fake--rate parameterization is implemented by
adapting a multivariate technique known as a $k-$Nearest~Neighbor classifier
(\kNN).  A \kNN classifier is typically used to classify events operates by
populating (``training'') an $n-$dimensional space with signal and background
events.  The probability for a given point $x$ in the space to be
``signal--like'' is determined by finding the $k$ nearest neighbors and
computing the ratio
\begin{equation}
  p_{sig} = \frac{n_{sig}}{n_{sig} + n_{bkg}},
  \label{eq:KNNEquation}
\end{equation}
where $n_{sig}$,$n_{bkg}$ are the observed number of signal and background events,
respectively. By construction, $k = n_{sig} + n_{bkg}$.  The principle of
operation is illustrated in Figure~\ref{fig:KNN}
\begin{figure}
  \centering
  \includegraphics[width=0.6\textwidth]{backgrounds_chapter/figures/knn_3d_s13_b7_x02.pdf}
  \caption[$k-$Nearest Neighbor classifier example]{Example of the operation of
  a \kNN classifier.  The closest $k=50$ neighbors (those inside the circle) to a
  test point (indicated by the star marker) are selected. The probability that
  the star marker is a signal event is given the number of signal neighbors (blue
  markers) in the circle divided by $k$. Image credit:~\cite{TMVA}}
  \label{fig:KNN}
\end{figure}

The classification feature of a \kNN can be trivially adapted to parameterize
the fake--rate such that it is defined everywhere.  Examining the form of
Equation~\ref{eq:KNNEquation}, if one replaces $n_{sig}$ with $n_{passed}$ and
$n_{bkg}$ with $n_{failed}$, the equation is equivalent to the tau--fake rate.
We thus train the \kNN with tau--candidates which pass the tau--ID as signal
events and those which fail as background events. The resulting classifier is a
function which returns the expected fake--rate for any point in the space of the
parameterization.  The choice of $k$ must be optimized.  When $k$ is low, the
small number of neighbors causes large counting fluctuations in the fake rate.
If $k$ is too large, the \kNN effectively averages over a large area of the
space of the variables\footnote{In the limit $k\to\inf$, the \kNN output reduces to a
single number.  In this extreme case, all information about the dependence of
the fake--rate on the variables is lost.}.  For the training statistics
available in the 2010 data, $k=20$ is found to be the optimal choice.

\subsection{Results of Background Estimation}
%
An independent estimate of the background contributions is obtained by applying
the fake--rate method described in~\cite{CMS_AN_2010-074}.  Fake--rates in QCD
multi--jet events (light quark enriched sample), QCD events containing muons
(heavy quark and gluon enriched sample) and $W$ + jet events are measured in
data~\cite{CMS-PAS-PFT-10-004,CMS-PAS-TAU-11-001} and applied to events which pass all the
event selection criteria listed in table~\ref{tab:AHtoMuTauEventSelection}
except for the requirement for tau--jet candidates to pass the ``medium'' tight
TaNC discriminator and have unit charge.

No assumption is made on the composition of $Z \to \mu^{+} \mu^{-}$, \WpJets,
$t\bar{t} + jets$ and QCD backgrounds contributing to the event sample selected
by the analysis.  Differences between fake--rates obtained for QCD multi--jet,
QCD muon enriched and $W$ + jets background events are attributed as systematic
uncertainties of the fake--rate method.  Per jet and per event weights have been
computed by the ``simple'' and ``CDF-type'' weights as described
in~\cite{CMS_AN_2010-074} and the results are found to be compatible within
statistical and systematic uncertainties.  In the following, we present results
for ``CDF-type'' weights.  The ``CDF-type'' weights have the advantage that the
background estimate obtained does not change, whether there is MSSM Higgs $\to
\tau^{+} \tau^{-}$ signal present in the data or not.

Tau identification efficiencies need to be known when using ``CDF-type''
weights.  Dedicated studies have checked the tau identification efficiencies in
data~\cite{CMS-PAS-TAU-11-001}.  Statistical and systematic uncertainties of
these studies are still sizeable at present, in the order to $20-30\%$.  No
indication has been found, however, that the Monte Carlo simulation would not
correctly model hadronic tau decays in data.  For the purpose of computing
fake--rate weights via the ``CDF-type'' method, tau identification efficiencies
are taken from the Monte Carlo simulation of hadronic tau decays in $Z \to
\tau^{+} \tau^{-}$ events.  Systematic uncertainties on the background estimate
obtained by the fake--rate method are determined by varying the tau
identification efficiencies by $\pm 30\%$ relative to the value obtained from
the Monte Carlo simulation.

The results of applying the fake--rate method to the mu + tau channel are
summarized in table~\ref{tab:MuTauFakeRateResultsOS}.  The background prediction
has been corrected for the expected contribution of $XX.X$ events from $Z \to
\mu^{+} \mu^{-}$ background events in which the reconstructed tau--jet is due to
a misidentified muon.  The obtained estimate is in good agreement with the Monte
Carlo expectation.

\begin{table}[t]
\begin{center}
\tablesize
\begin{tabular}{|l|c|c|}
\hline
Events weighted by: & Estimate \\
\hline
QCD lead jet       & $202.1^{+14.9}_{-74.8}$\\ % scaled 1.034 to account for tau code change
QCD second jet      & $198.0^{+22.8}_{-79.3}$\\% scaled 1.034 to account for tau code change
QCD $\mu$--enriched & $213.3^{+17.7}_{-82.6}$ \\
\WpJets          & $232.8^{+21.1}_{-95.0}$ \\
\hline
%$N_{bgr}$ estimate  & $223.0^{+23.9}_{-65.9}$ \\ - before Zmumu correction.
$N_{bgr}$ estimate  & $236.1^{+24.1}_{-65.9}$ \\ %- after Zmumu correction.
\hline
\end{tabular}
\end{center}
\begin{center}
\caption[Fake--rate method results]{Estimate for background contributions in mu +
tau channel obtained by weighting events passing all selection criteria listed
in table~\ref{tab:AHtoMuTauEventSelection} except for the requirement for
tau--jet candidates to pass the ``medium'' tight TaNC discriminator and have
unit charge by fake--rates measured in QCD multi--jet, QCD muon enriched and $W$
+ jets data samples.} \label{tab:MuTauFakeRateResultsOS}
\end{center}
\end{table}

As an additional cross--check of the method, a sample of events containing a
muon plus a tau--jet of like--sign charge is selected in data and compared to
the background prediction obtained by applying the fake--rate method to the
like--sign sample.  The like--sign sample is expected to be dominated by the
contributions of $W$ + jets and QCD background processes and allows to verify
the fake--rate method in a practically signal free event sample.  The background
estimate obtained by the fake--rate method is compared to the number of events
observed in the like--sign data sample in
table~\ref{tab:MuTauFakeRateResultsSS}.  The number of events expected in the
like--sign control sample from Monte Carlo simulation is indicated in the
caption.  All numbers are in good agreement.

The fake--rate method does not only allow to estimate the total number of
background events, but allows to model the distributions of background processes
as well.  The capability to model distributions is illustrated in
figure~\ref{fig:MuTauFakeRateResultsSS}, which shows good agreement between the
distributions observed in the like-sign data sample and the predictions obtained
by the fake--rate method for the distributions of muon plus tau--jet visible
mass and of the ``full'' invariant mass reconstructed by the SVfit algorithm.

\begin{table}[t]
\begin{center}
\tablesize
\begin{tabular}{|l|c|c|}
\hline
Events weighted by:     & Estimate \\
\hline
QCD lead jet           & $191.7^{+2.3}_{-17.9}$ \\% scaled 1.034 to account for tau code change
QCD second jet          & $185.1^{+6.0}_{-21.1}$ \\% scaled 1.034 to account for tau code change
QCD $\mu$--enriched     & $194.7^{+2.0}_{-20.5}$ \\
\WpJets              & $208.9^{+0.5}_{-14.4}$ \\
\hline
Fake--rate estimate     & $201.8^{+14.2}_{-18.9}$ \\
\hline
Observed                & $216$ \\
\hline
\end{tabular}
\end{center}
\begin{center}
\caption[Yields in like--sign control region]{Number of events observed in
like--sign control region compared to estimate obtained by fake--rate method.
The number of observed events as well as the number of background events
predicted by the fake--rate method is on good agreement with the Monte Carlo
expectation of $XX.X$ events for the sum of $Z \to \mu^{+} \mu^{-}$, \WpJets,
\ttbarpJets and QCD background contributions in the like--sign control region.}
\label{tab:MuTauFakeRateResultsSS}
\end{center}
\end{table}

\begin{figure}[t]
\setlength{\unitlength}{1mm}
\begin{center}
\begin{picture}(150,52)(0,0)
%\put(0.5, 2){\mbox{\includegraphics*[height=52mm, viewport=19 0 524 396]{figures/likeSignMuTau_visMass.pdf}}}
%\put(78.0, 2){\mbox{\includegraphics*[height=52mm, viewport=19 0 524 396]{figures/likeSignMuTau_SVfitMass.pdf}}}
\end{picture}
\caption[Comparison of visible mass and SVfit mass]{Distribution of visible mass
(left) and ``full'' invariant mass reconstructed by the SVfit algorithm (right)
observed in the like--sign charge control region compared to the background
estimate obtained by the fake--rate method.} \label{fig:MuTauFakeRateResultsSS}
\end{center}
\end{figure} 

\section{Template method}
\label{sec:template}

Shape templates for the $\mu + \tau_{had}$ visible mass $M_{vis}$ are obtained
from data, using a set of dedicated control regions which are chosen to select a
high purity sample of one particular background process each.  The number of
events selected in each control region and comparisons to the predictions from
Monte Carlo simulations are summarized in
Table~\ref{tab:ResultsMuTauBgControlRegions}.  The template $M_{vis}$ shapes
obtained from data in the background enriched control regions are compared to
the signal region shapes obtained by Monte Carlo simulation in
figure~\ref{fig:VisMassTemplates}.  The $M_{vis}$ spectrum observed in the
$Z/\gamma^{*} \rightarrow \tau^{+} \tau^{-}$ cross--section analysis is fitted
to the sum of these templates.  Estimates for background yields are obtained
from the normalization factor of each template, determined by the fit.  Further
details of the method can be found in~\cite{CMS_AN_2010-088}
and~\cite{CMS_AN_2011-021}. 

\begin{table}[t]
\begin{center}
\tablesize
\begin{tabular}{|l|c|c|c|c|c|c|c|c|}
\hline
\multirow{2}{17mm}{Enriched Selection} & \multicolumn{7}{c|}{Contribution from} & \multirow{2}{12mm}{Purity} \\
%\hhline{|~|-------|~|}
 & Data & $\Sigma$~SM & $Z \to \tau^{+} \tau^{-}$ & $Z \to \mu^{+} \mu^{-}$ & $W$ + jets & $t\bar{t}$ + jets & QCD & \\
\hline
\hline
$Z \to \mu^{+} \mu^{-}$ & & & & & & & & \\
\hspace{2mm} Muon fake & $15156$ & $17109.8$ & $331.6$ & $16586.6$ & $55.1$ & $80.4$ & $35.0$ & $96.9\%$ \\
\hspace{2mm} Jet fake & $85$ & $62.7$ & $2.5$ & $55.5$ & $0.5$ & $1.4$ & $2.4$ & $88.5\%$ \\
$W$ + jets & $514$ & $642.4$ & $17.9$ & $22.9$ & $581.7$ & $0.8$ & $16.7$ & $90.6\%$ \\  
$t\bar{t}$ + jets & $26$ & $39.7$ & $0.7$ & $< 0.1$ & $0.6$ & $38.4$ & $< 1.0$ & $96.7\%$ \\
QCD & $2510$ & $2571.8$ & $16.6$ & $0.8$ & $9.3$ & $1.6$ & $2543.4$ & $98.9\%$ \\
\hline
\end{tabular}
\caption{\captiontext 
         Number of events observed in the different background enriched control
         samples compared to Monte Carlo expectations.  $\Sigma$~SM denotes the
         sum of $Z \to \tau^{+} \tau^{-}$, $Z \to \mu^{+} \mu^{-}$, $W$ + jets,
         $t\bar{t}$ + jets and QCD processes.  The expected purity of each
         control sample is computed as the ratio of contribution of the enriched
         process to $\Sigma$~SM.}
\label{tab:ResultsMuTauBgControlRegions}
\end{center}
\end{table}

\begin{figure}
\setlength{\unitlength}{1mm}
\begin{center}
\begin{picture}(150,170)(0,0)
\put(0.5, 118){\mbox{\includegraphics*[width=52mm, angle=90]
  {backgrounds_chapter/figures/plotBgEstTemplateData_vs_AnalysisZtoMuTau_ZmumuMuonMisIdEnriched_visMass.pdf}}}
\put(78.0, 118){\mbox{\includegraphics*[width=52mm, angle=90]
  {backgrounds_chapter/figures/plotBgEstTemplateData_vs_AnalysisZtoMuTau_ZmumuJetMisIdEnriched_visMass.pdf}}}
\put(0.5, 60){\mbox{\includegraphics*[width=52mm, angle=90]
  {backgrounds_chapter/figures/plotBgEstTemplateData_vs_AnalysisZtoMuTau_WplusJetsEnriched_visMass.pdf}}}
\put(78.0, 60){\mbox{\includegraphics*[width=52mm, angle=90]
  {backgrounds_chapter/figures/plotBgEstTemplateData_vs_AnalysisZtoMuTau_WplusJetsEnriched_visMass_corrected.pdf}}}
\put(0.5, 2){\mbox{\includegraphics*[width=52mm, angle=90]
  {backgrounds_chapter/figures/plotBgEstTemplateData_vs_AnalysisZtoMuTau_TTplusJetsEnriched_visMass.pdf}}}
\put(78.0, 2){\mbox{\includegraphics*[width=52mm, angle=90]
  {backgrounds_chapter/figures/plotBgEstTemplateData_vs_AnalysisZtoMuTau_QCDenriched_visMass.pdf}}}
\put(-5.5, 170.5){\small (a)}
\put(72.0, 170.5){\small (b)}
\put(-5.5, 112.5){\small (c)}
\put(72.0, 112.5){\small (d)}
\put(-5.5, 54.5){\small (e)}
\put(72.0, 54.5){\small (f)}
\end{picture}
\caption{\captiontext 
         $\mu + \tau_{had}$ shape templates obtained from $Z \to \mu^{+}
         \mu^{-}$ (a) and (b), $W$ + jets before (c) and after (d) the bias
         correction explained in section~\ref{sec:template}, $t\bar{t}$ + jets
         (e) and QCD multi--jet (f) backgrounds enriched control regions
         compared to the expected distribution of the of the enriched background
         process to the signal region, predicted by Monte Carlo simulations.  In
         (a) reconstructed tau--jet candidates are expected to be dominantly due
         to misidentified muons, while in (b) they are expected to be mostly due
         to misidentified quark or gluon jets.}
\label{fig:VisMassTemplates}
\end{center}
\end{figure} 

The control regions from which the shape templates for $Z \to \mu^{+} \mu^{-}$,
$W$+ jets, \ttbar + jets and QCD background processes are obtained are described
in~\cite{CMS_AN_2011-021}.  The TaNC~\cite{CMS_AN_2010-099} discriminators used
in~\cite{CMS_AN_2011-021} are replaced by the corresponding discriminators of
the HPS algorithm~\cite{CMS_AN_2010-082}.  The $Z/\gamma^{*} \to \tau^{+}
\tau^{-}$ signal shape is obtained via the $Z/\gamma^{*} \to \mu^{+} \mu^{-}$
embedding technique~\cite{MCEmbedding}.

\begin{figure}
\setlength{\unitlength}{1mm}
\begin{center}
\includegraphics*[width=62mm, angle=90]{backgrounds_chapter/figures/fitBgEstTemplateZtoMuTau_visMass.pdf}
\caption{\captiontext 
         $M_{vis}$ distribution of events selected by the
         $Z/\gamma^{*} \rightarrow \tau^{+} \tau^{-} \rightarrow \mu + \tau_{had}$ 
         cross--section analysis
         compared to the sum of shape templates for signal and background processes
         scaled by the normalization factors determined by the fit.}
\label{fig:TemplateFitControlPlot}
\end{center}
\end{figure} 

The $\mu + \tau_{had}$ visible mass spectrum observed in the
$Z/\gamma^{*} \rightarrow \tau^{+} \tau^{-} \rightarrow \mu + \tau_{had}$ 
cross--section analysis is compared to the sum
of template shapes scaled by the normalization factors determined by the fit in Fig.~\ref{fig:TemplateFitControlPlot}.
The corresponding estimates for background contributions are summarized in Tab.~\ref{tab:BgEstTemplateMethod}.

 %QCD: normalization = 130.547 +/- 35.1123
 %TTplusJets: normalization = 8.84426 +/- 6.7345
 %WplusJets: normalization = 80.2047 +/- 24.7961
 %ZmumuJetMisId: normalization = 6.21001 +/- 5.72237
 %ZmumuMuonMisId: normalization = 8.77512e-05 +/- 7.37436
 %Ztautau: normalization = 296.405 +/- 30.7937

\begin{table}[t]
\begin{center}
\tablesize
\begin{tabular}{|l|c|c|c|c|c|c|c|c|}
\hline
Process & Estimate \\
\hline
\hline
$Z \to \mu^{+} \mu^{-}$ & \\
\hspace{2mm} Muon fake &    $5.7 \pm  6.0$ \\
\hspace{2mm} Jet fake  & $< 14.5$ \\
$W$ + jets                 &   $70.1 \pm 19.6$ \\  
$t\bar{t}$ + jets          &    $7.6 \pm  6.9$ \\
QCD                        &  $141.3 \pm 40.4$ \\
\hline
$N_{bgr}$ estimate         &  $226.5 \pm 33.1$ \\
\hline
\end{tabular}
\caption{\captiontext Estimated contributions of individual background processes
to the signal region, obtained via the template method.  As the shapes are very
similar, the normalization factors for QCD and \WpJets background processes
are anti--correlated.  As a consequence, the sum of background contributions is
determined by the fit more precisely than the individual contributions.}
\label{tab:BgEstTemplateMethod}
\end{center}
\end{table}



\ifx\master\undefined\input{settings/autocompile}\fi
